\section{Flow}
\label{flow}
\index{Flow}

\emph{Flow} nach Mihaly Csikszentmihalyi~\cite{flow} (Aussprache des Nachnamens: \glqq Chick sent me high!\grqq) definiert einen befriedigenden Zustand, in dem man komplett in eine Tätigkeit versunken ist und dabei potenziell extrem produktiv ist.

Es dauert einige Zeit, bis man in einen Flow-Zustand kommt.


\subsection{Merkmale von Flow-Zuständen}

\begin{itemize}
  \item \glqq in die Tätigkeit versunken sein\grqq
  \item fokussierte Aufmerksamkeit nur auf die Tätigkeit
  \item hohe Motivation
  \item verändertes Zeitgefühl
  \item Gefühl von Kontrolle über die Tätigkeit
  \item Gefühl von Mühelosigkeit
\end{itemize}


\subsection{Beispiele mögliche Flow-Aktivitäten}

\begin{itemize}
  \item Fahrradfahren, Autofahren
  \item Sport
  \item Computerspiele spielen
  \item lesen
  \item programmieren
  \item Rätsel lösen
  \item guter Sex
  \item musizieren
  \item Yoga
  \item Kontakt-Improvisation
  \item Schach
\end{itemize}


\subsection{Was zu Flow beiträgt}

\begin{itemize}
  \item eine Tätigkeit, die man schon gut beherrscht
  \item Anforderungsniveau: weder zu leicht noch zu schwierig
  \item eine Tätigkeit, die die volle Konzentration beansprucht
  \item klare Zielsetzung
  \item klare Rückmeldungen
  \item keine Störungen
  \item intrinsische Motivation
\end{itemize}


\subsection{Was Flow behindert}

\begin{itemize}
  \item Unterbrechungen und Ablenkungen
  \item Multitasking
  \item Unterforderung, Langeweile
  \item Überforderung
  \item unklare oder fehlende Ziele
\end{itemize}
