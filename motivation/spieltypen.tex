\subsection{Typen von Spieler\_innen}
\label{spielertypen}
\index{Spieler\_innen-Typen}
\index{Spielpräferenzen}


\subsubsection{Bartle-Taxonomie}
\label{bartle-taxonomie}
\index{Bartle-Taxonomie}

Richard Bartle~\cite{bartle-taxonomy} hat diese Typen von Spieler\_innen oder Präferenzen beim Spielen in massiven Multiplayer-Onlinerollenspielen (MMORPGs) identifiziert.
\index{MMORPG}

\paragraph{Achievement (Achievers):} Diese Menschen lieben es, Ziele und Erfolge zu erreichen.
\index{Achievers}
\paragraph{Exploration (Explorers):} Diese Menschen lieben es, die Welt zu erkunden und damit zu interagieren.
\index{Explorers}
\paragraph{Socializing (Socializers):} Diese Menschen lieben es, im Spiel mit anderen Menschen zu interagieren, ihnen zu helfen und mit ihnen zusammenzuarbeiten.
\index{Socializers}
\paragraph{Imposition (Killers):} Diese Menschen lieben es, andere Menschen zu besiegen oder ihnen zu schaden.
\index{Killers}

\emph{Killer} sind üblicherweise ein kleine Minderheit in Spielen. Sie können allerdings das Spiel für die anderen zerstören, insbesondere für die \emph{Socializers}.

Die Bartle-Taxonomie funktioniert hauptsächlich für MMORPGs und andere kompetitive Online-Spiele.


\subsubsection{Kim’s Social Action Matrix}
\label{kims-social-action-matrix}
\index{Kim’s Social Action Matrix}

Amy Jo Kim hat aufbauend auf die Bartle-Taxonomie \emph{Kim’s Social Action Matrix}~\cite{kims-social-action-matrix} definiert, die auch auch Spiele anwendbar ist, für die die Bartle-Taxonomie nicht gut funktioniert, nämlich Casual-Spiele, soziale Spiele und edukative Spiele.

Sie unterscheidet die folgenden Spielpräferenzen:

\paragraph{Compete:} mit anderen in Wettstreit treten, um die eigenen Skills auf den Prüfstand zu stellen
\index{compete}
\paragraph{Collaborate:} mit anderen zusammenarbeiten für ein größeres gemeinsames Ziel
\index{collaborate}
\paragraph{Express:} sich selbst ausdrücken, Dinge gestalten, den eigenen Avatar anpassen, Kreativität
\index{express}
\paragraph{Explore:} die Welt entdecken, Grenzen finden, die Regeln des Systems verstehen, sich Wissen aneignen, und Schlupflöcher im Spiel finden
\index{explore}

Teil der \emph{Explore}-Präferenz ist auch der Wunsch, Dinge vollständig abzuschließen (\glqq\fett{Completionist}\grqq).
\index{Completionist}
