\section{Selbstbestimmungstheorie (\emph{self-determination theory})}
\label{sdt}
\index{Selbstbestimmungstheorie}
\index{self-determination theory}
\index{SDT}

Selbstbestimmungstheorie (\emph{self-determination theory, SDT}~\cite{self-determiniation-theory} ist ein empirischer Ansatz dazu, was Menschen intrinsisch motiviert.


\subsection{Psychologische Grundbedürfnisse}
\index{Grundbedürfnisse}
\index{Bedürfnisse}

Die SDT sieht drei universelle psychologische Grundbedürfnisse, die für effektives Verhalten und für psychische Gesundheit erfüllt sein müssen:

\begin{itemize}
  \item Kompetenz
  \item Autonomie/Selbstbestimmung
  \item soziale Eingebundenheit
\end{itemize}

Die Listen, was auf das jeweilige Bedürfnis einzahlt oder dagegen arbeitet, sind unvollständig und als Beispiele gedacht.


\subsection{Kompetenz}
\index{Kompetenz}

\subsubsection{Was dies befördert}

\begin{itemize}
  \item jemanden mentorieren
  \item sinnvolle Lernpfade mit \glqq Scaffolding\grqq
  \item Fokus auf Kompetenz und Lernen statt auf harte Ziele
  \item aus Fehlern lernen
  \item Superkräfte und Stärken von Leuten nutzen
  \item konstruktives Feedback
  \item unerwartetes positives Feedback
  \item Weiterbildungsmöglichkeiten schaffen
  \item Aufgaben zum Hineinwachsen anbieten
  \item Erfolge feiern
  \item Hindernisse überwinden
\end{itemize}

\subsubsection{Was dies behindert}

\begin{itemize}
  \item Fehler bestrafen
  \item allen Leute dieselben Aufgaben geben
  \item Tadel
  \item Lob (von oben herab mit Fokus auf Beurteilung)
  \item negatives Feedback (ohne Fokus auf Wachstum)
  \item Aufgaben, für die die Leute die Fähigkeiten nicht haben (und sich auch nicht aneignen können)
\end{itemize}



\subsection{Autonomie, Selbstbestimmung}
\index{Autonomie}
\index{Selbstbestimmung}

\subsubsection{Was dies befördert}

\begin{itemize}
  \item Leute in relevante Entscheidungen einbeziehen
  \item Initiative wertschätzen und fördern
  \item Entscheidungen gut begründen
  \item Rahmenbedingungen erklären
  \item Struktur und Orientierung schaffen
  \item Arbeitsumfeld bedürfnisorientiert gestalten
  \item Leuten Prozess-Aufgaben anbieten (z.\,B.~Moderation)
  \item demokratischer/kooperativer Führungsstil
  \item in Einklang mit den eigenen Werten und Idealen arbeiten und handeln
  \item auf die eigenen Lebensziele hinarbeiten
\end{itemize}

\subsubsection{Was dies behindert}

\begin{itemize}
  \item Top-down-Management
  \item autoritärer Führungsstil \index{autoritärer Führungsstil}
  \item über die Köpfe von Leuten hinweg entscheiden
  \item kontrollierende Sprache (\glqq müssen\grqq, \glqq notwendig\grqq, \glqq verboten\grqq etc.
  \item Belohnung und Bestrafung
  \item Deadlines (vor allem willkürliche Deadlines) \index{Deadlines}
\end{itemize}



\subsection{soziale Eingebundenheit, Zugehörigkeit}
\index{soziale Eingebundenheit}
\index{soziale Zugehörigkeit}


\subsubsection{Was dies befördert}

\begin{itemize}
  \item Zusammenarbeit
  \item anderen helfen
  \item Bekanntschaften machen
  \item etwas Nettes zusammen machen
  \item jemanden mentorieren
  \item Beziehungen bewusst aufbauen und pflegen
  \item authentisch über Emotionen und Bedürfnisse sprechen
\end{itemize}

\subsubsection{Was dies behindert}

\begin{itemize}
  \item Konkurrenz
\end{itemize}
