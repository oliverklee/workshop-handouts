\chapter{Tools}

\section{PhpStorm}

\begin{itemize}
  \item Always use \fett{PhpStorm} for editing PHP files.
  \item Configure and regularly use PhpStorm’s code inspection feature and reduce the warnings to a minimum.
  \item When having a single extension as the PhpStorm project, include the TYPO3 Core source and the phpunit extension in the include path (if you do not have then in \texttt{vendor/} anyway).
  \item Use the \fett{XLIFF Support} and \fett{TYPO3 XLIFF Utility} PhpStorm plugins.
  \item Use the \fett{TYPO3 CMS} PhpStorm plugin by Cedric Ziel.
  \item If you can afford it, use the \fett{Fluid} and \fett{TypoScript} PhpStorms by sgalinski. They also include great code completion. They are worth every cent.
  \item If you can not afford the paid Fluid and TypoScript plugins, you can use the free TypoScript plugin (also by sgalinski) and the Fluid schema files by Helmut Hummel:
    \begin{itemize}
      \item \url{http://insight.helhum.io/post/85031122475/xml-schema-auto-completion-in-phpstorm}
      \item Important: There are updated schema URLs:\\\url{http://insight.helhum.io/post/130270697975/updated-fluid-schema-urls}
      \item You’ll also need to add \texttt{xmlns:f} attribute with the URL to the outmost tag(s) in your templates, layouts and partials. (This is not necessary if you
    \end{itemize}
\end{itemize}


\section{Debugging}
Use \fett{Xdebug} for debugging. Don’t use \texttt{echo}, \texttt{var\_dump}  etc. for this.


\section{Code style}

For converting old code to PSR-2, use php-cs-fixer. You can use the configuration from the Core or from oelib:

\begin{itemize}
  \item \texttt{typo3\_src/Build/.php\_cs}
  \item \texttt{EXT:oelib/Configuration/PhpCsFixer/FixerConfiguration.php}
\end{itemize}

The call to fix all files in a certain directory looks like this:

\begin{textcode}
php-cs-fixer fix --config-file Configuration/PhpCsFixer/FixerConfiguration.php .
\end{textcode}
