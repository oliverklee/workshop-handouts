\chapter{Fluid templates}

\section{What to put where}

\subsection{Business logic}

Move logic from templates into the models if it's business logic.

\subsection{JavaScript}

Don't use script tags or event handler attributes in your HTML. See the JavaScript section on page~\pageref{javascript} for solutions.

\subsection{Data for JavaScript}

Put data for JavaScript in data attributes, not in variables in script tags. This avoids XSS in that case.\footnote{\url{https://speakerdeck.com/helhum/fluid-xss}}

\subsection{Partials}

Use partials to avoid deep nesting.

Instead of using lots of arguments for a partial, use specialized partials to improve readability.


\section{Static images}

For static images that do not need to be resized, use plain IMG tags instead of an image view helper. This makes the code shorter and the rendering faster.


\section{Syntax and formatting}

For consistency with PSR-2, use 4 spaces for indentation.


\section{DOCTYPE and HTML element}
Add a DOCTYPE and HTML element to your templates and layouts to help PhpStorm. (Also, the HTML element is the place to put your \texttt{xml:ns} attribute.)

\subsection{Well-formed XML/HTML}

Make sure your Fluid templates are well-formed HTML/XML:

\begin{itemize}
  \item Always close tags (unless they are self-closing), and don't put orphaned opening or closing tags in conditions. This might result in slightly more code (which is okay).
  \item Use the inline form of view helpers if you need to use a view helper within an opening tag (e.g. to add HTML attributes or attribute values).
\end{itemize}

\subsection{Indentation}
For consistency with PSR-2, use 4 spaces for indentation.

\section{Tools}

Use and configure PhpStorm auto-formatting.

Use a converter\footnote{\url{http://www.fluid-converter.com/}} for converting view helpers in your templates to inline notation.
