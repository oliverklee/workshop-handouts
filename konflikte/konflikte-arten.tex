\section{Arten von Konflikten}
\label{konflikt-arten}
\index{Konflikt: Arten}

Kurt Lewin~\cite{gelingende-kommunikation-revisited, schneider-konflikte} unterscheidet mehrere Arten von Konflikten.


\subsection{Innere Konflikte (intrapersonale Konflikte)}
\index{innere Konflikte}
\index{intrapersonale Konflikte}

\index{Motivkonflikt}

\subsubsection{Annäherungskonflikt (Appetenzkonflikt, Appetenz-Appetenz-Konflikt)}
\index{Annäherungskonflikt}
\index{Appetenzkonflikt}
\index{Appetenz-Appetenz-Konflikt}

Beide Optionen sind gut, schließen sich aber gegenseitig aus.

\subsubsection{Ambivalenzkonflikt (Annäherungs-Vermeidungs-Konflikt, Appetenz-Aversions-Konflikt)}
\index{Ambivalenzkonflikt}
\index{Annäherungs-Vermeidungs-Konflikt}
\index{Appetenz-Aversions-Konflikt}

Das Ziel hat sowohl Vorteile als auch Nachteile.


\subsubsection{Vermeidungskonflikt}
\index{Vermeidungskonflikt}

Ich will eigentlich keine der möglichen Optionen.


\subsubsection{Identitätskonflikt}
\index{Identitätskonflikt}

Wer bin ich? Was für ein Mensch will ich sein?


\subsection{Zwischenmenschliche Konflikte (interpersonale Konflikte)}
\index{zwischenmenschliche Konflikte}
\index{interpersonale Konflikte}

\subsubsection{Rollenkonflikt}
\index{Rollenkonflikt}

Wer darf und muss was in welcher Rolle? Wer hat überhaupt welche Rolle?

\subsubsection{Verteilungskonflikt}
\index{Verteilungskonflikt}

Wer bekommt wie viel?

\subsubsection{Identitätskonflikt}
\index{Identitätskonflikt}

Wer bin ich? Was für ein Mensch will ich sein?

\subsubsection{Zielkonflikt}
\index{Zielkonflikt}

Wo wollen wir (zusammen) hin? Warum wollen wir das?

\subsubsection{Beziehungskonflikt}
\index{Beziehungskonflikt}

Wie gehen wir miteinander um? Wie stehen wir zueinander?

\subsubsection{Informationskonflikt}
\index{Informationskonflikt}

Was ist Sache? Wer weiß was? Was ist richtig, was ist falsch?


\subsection{Konflikte nach Beteiligten}
\index{Generationenkonflikte}
\index{Informationskonflikt}
\index{Arbeitskampf}
\index{ethnische Konflikte}

\begin{itemize}
  \item Familienkonflikte
  \item Generationenkonflikte
  \item Arbeitskampf
  \item Konflikte innerhalb von Gruppen
  \item Konflikte in der Schule
  \item ethnische Konflikte
  \item Konflikte zwischen Staaten
  \item \ldots
\end{itemize}


\subsection{Konflikte nach Themen}

\subsubsection{Politischer Konflikt}
\index{politischer Konflikt}

Laut~\cite{konfliktbarometer-2003} ist ein politischer Konflikt ein \glqq Interessengegensatz (Positionsdifferenz) um nationale Werte von einiger Dauer und Reichweite zwischen mindestens zwei Parteien (organisierten Gruppen, Staaten, Staatengruppen, Staatenorganisationen), die entschlossen sind, diesen zu ihren Gunsten zu entscheiden.\grqq

\subsubsection{Sozialer Konflikt}
\index{sozialer Konflikt}

Ein sozialer Konflikt ist der Kampf um Handlungsmacht oder Macht in der Gesellschaft.

\subsubsection{Psychischer Konflikt (Grundkonflikt)}
\index{psychischer Konflikt}
\index{Grundkonflikt}
\index{Freud, Sigmund}

\emph{Grundkonflikt} beschreibt einen Konflikt zwischen widersprüchlichen Bedürfnissen. Der Begriff ist ein Fachbegriff aus der Psychoanalyse und der Tiefenpsychologie und wurde von Sigmund Freud gebildet.

Wenn ein Anteil des Konfliktes unbewusst ist und nicht wahrgenommen werden kann, weil er zum Schutze der noch wenig ausgereiften Persönlichkeit verdrängt werden musste, kann der Konflikt nicht bewusst gelöst werden.

\subsubsection{Ethischer Konflikt (ethisches Dilemma)}
\index{ethischer Konflikt}
\index{ethisches Dilemma}
\index{moralischer Konflikt}

Zwei widersprüchliche moralische Anforderungen ziehen mich in unterschiedliche Richtungen.

\subsubsection{Religionskonflikt, Weltanschauungskonflikt}
\index{Religionskonflikt}
\index{Ideologie-Konflikt}
\index{Weltanschauungskonflikt}

Was sollen wir glauben? Wofür sind wir auf der Welt? Was ist der Sinn von Dingen?
