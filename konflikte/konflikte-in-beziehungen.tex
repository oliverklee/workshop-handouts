\section{Konflikte in Beziehungen}
\label{konflikte-in-beziehungen}
\index{Beziehungen}
\index{Liebesbeziehungen}

Die Prinzipien, die für Konflikte in Liebesbeziehungen gelten, lassen sich auch in anderen Kontexten anwenden.

Viele Inhalte in diesem Abschnitt stammen aus dem Buch \emph{The Intimate Enemy} \cite{intimate-enemy}. Dieses Buch geht davon aus, dass es essentiell für das Bestehen von langfristigen Beziehungen ist, Konflikte anzusprechen und auszutragen. Andere Inhalte kommen aus dem Buch \emph{The Seven Principles For Making Marriage Work} \cite{gottman-7-principles} von John Gottman. Das Konzept von HALT habe ich aus dem Podcast \emph{Multiamory} \cite{halt-multiamory} sowie aus deren Buch \cite{multiamory-book} übernommen.


\subsection{Positive Interaktionen schaffen}
\index{Interaktionen}
\index{positive Interaktionen}
\index{negative Interaktionen}

Laut John Gottman ist das Verhältnis von positiven zu negativen Interaktionen in langfristig stabilen Beziehungen \fett{5:1} (oder besser). Das heißt, dass es sehr hilfreich ist, bewusst viel Angenehmes miteinander zu erleben sowie sich gegenseitig immer mal wieder von Herzen etwas Nettes zu sagen, damit ihr dadurch den Goodwill habt, auch Konflikte durchzustehen.


\subsection{Wie man eine Beziehung zerstört}

\paragraph{Gunny-Sacking} (auf Deutsch in etwa \glqq in Jutesäcken sammeln\grqq) bezeichnet, dass ihr Probleme über eine lange Zeit still ansammelt und dann irgendwann in einem Schwung über der anderen Person auskübelt.
\index{Gunny-Sacking}

\paragraph{Hinterhalt und Überfall:} ein Konfliktgespräch beginnen, wenn die andere Person gerade auf dem Sprung ist oder gerade nach Haus gekommen ist
\index{Hinterhalt}
\index{Überfall}

\paragraph{Fahrerflucht:} ein Konfliktgespräch beginnen oder der anderen Person etwas vorwerfen, und dann direkt das Thema wechseln oder die Situation verlassen
\index{Fahrerflucht}

\paragraph{Mit einem persönlichen Angriff starten} statt mit einer neutralen Beschreibung der Ereignisses
\index{persönliche Angriffe}

\paragraph{Reparaturversuche zurückweisen} (zum Beispiel aus Rache oder Trotz)
\index{Reparaturversuche}

\paragraph{Emotional dichtmachen:} Jede funktionierende Beziehung (auch Freundschaften oder Arbeitsbeziehungen) braucht es, dass alle Beteiligten eine Verbindung dazu zulassen, was in ihnen gegenseitig lebendig ist.

\paragraph{Verachtung:} Das bedeutet, dass ihr die Person als Ganzes ablehnt und verurteilt statt nur deren Verhalten.
\index{Verachtung}

\paragraph{Verteidigung (Defensiveness)} ist, wenn ihr die Not der anderen Person als Angriff interpretiert und euch dann darauf fokussiert, dies abzuwehren und eventuell auch zurückzuschlagen.
\index{Verteidigung}
\index{Defensiveness}

\paragraph{Fluten (Flooding)} bedeutet, so viele Probleme und Vorwürfe auf einmal anzusprechen, dass sich die andere Person davon überwältig fühlt
\index{Fluten}
\index{Flooding}


\subsection{Konflikte konstruktiv austragen}

\begin{itemize}
  \item Konfliktgespräche \fett{vorher planen}, anstatt die spontan zwischendurch zu führen
  \item Gespräch \fett{zu zweit} führen statt vor Zeug\_innen
  \item auf \fett{das Miteinander fokussieren} statt aufs Gewinnen oder Rechthaben
  \item die \fett{lösbaren Probleme lösen}, und mit \fett{den anderen Problem leben lernen}
  \item das Gespräch unterbrechen, falls ihr gerade \fett{HALT} seid:
    \index{HALT}
    \begin{itemize}
      \item \fett{H}ungry: hungrig
      \item \fett{A}ngry: wütend
      \item \fett{L}onely: einsam
      \item \fett{T}ired: zu müde (oder auch betrunken, auf anderen Drogen oder krank)
    \end{itemize}
\end{itemize}
