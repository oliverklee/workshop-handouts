\section{Stärken dieser Methode}

Gewaltfreie Kommunikation (GfK) ist vor allem hilfreich bei Konflikten, bei denen es um Gefühle, (fehlende) Empathie oder die Beziehung miteinander geht.

GfK funktioniert auch gut, um Lösungen zu finden, die die Bedürfnisse aller beteiligten Personen gut erfüllen.

Sie ist hingegen nicht so stark bei Konflikten, wo es vor allem um die Lösung eines Sachkonfliktes geht, in dem Emotionen keine große Rolle spielen.
