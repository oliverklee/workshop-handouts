\documentclass[a4paper,openany,twoside,titlepage,10pt,headsepline]{scrbook}

%----------------------------------------------------------------------------------
% Typografie und Fonts
%----------------------------------------------------------------------------------

% Vektorfonts statt Pixelfonts
\usepackage[T1]{fontenc}
\usepackage{lmodern}

% Input-Encoding für UTF-8
\usepackage[utf8]{inputenc}

\newcommand{\fett}[1]{\textsf{\textbf{#1}}}

%----------------------------------------------------------------------------------
% Pakete und Parameter
%----------------------------------------------------------------------------------

% Mehrsprachigkeit für Deutsch und Englisch erlauben
\usepackage[ngerman,english]{babel}

% Mehr von der Seitenbreite nutzen
% \usepackage{a4wide}

% Grafikpaket
\usepackage{color}
\usepackage[pdftex]{graphicx}

% Absätze werden nicht eingezogen, sondern vertikal abgesetzt
\setlength{\parindent}{0mm}
\addtolength{\parskip}{0.4em}

% Bibliographieeinstellungen
\usepackage{natbib}
\bibliographystyle{alpha}

% lesbare Verweise
\usepackage[pdftex,plainpages=false,pdfpagelabels]{hyperref}

% nette URLs
\usepackage{url}

% für Boxen etc.
% \usepackage{framed}
\definecolor{shadecolor}{rgb}{0.8,0.8,0.8}

% Anführungszeichen sprachabhängig machen
% \usepackage[babel]{csquotes}


%----------------------------------------------------------------------------------
% Seitenlayout
%----------------------------------------------------------------------------------


% Seiten-Kopfzeilen und -Fußzeilen
\usepackage{scrlayer-scrpage}
\pagestyle{headings}

% Kopfzeile auf linker Seite: "1  Einführung"
\renewcommand{\chaptermark}[1]{%
\markboth{\thechapter\ \ \ \ #1}{}}

% Kopfzeile auf rechter Seite: "1.1  Basics"
\renewcommand{\sectionmark}[1]{%
\markright{\thesection\ \ \ \ #1}{}}

% Seitenlayout
\topmargin0mm
\footskip10mm % Abstand von unserem Rand zu Datum

\newlength{\fullwidth} % Seites des Textes plus Randnotizen
\setlength{\fullwidth}{\textwidth}
\addtolength{\fullwidth}{\marginparsep}
\addtolength{\fullwidth}{\marginparwidth}

% Maximal drei Ebenen nummerieren
\setcounter{secnumdepth}{3}
% Maximale Gliederungstiefe, die noch ins Inhaltsverzeichnis aufgenommen wird
\setcounter{tocdepth}{1}

% zweispaltiges Layout möglich machen
\usepackage{multicol}

% Descriptions ohne Einzug
\renewenvironment{description}[1][0pt]
{\list{}{
    \labelwidth=0pt \leftmargin=#1
     \let\makelabel\descriptionlabel
  }
}
{\endlist}

\raggedbottom


\AtBeginDocument{\selectlanguage{ngerman}}

\title{Konfliktmanagement\\für Teams und Führungskräfte}
\author{Oliver Klee\\\texttt{www.oliverklee.de}\\\texttt{seminare@oliverklee.de}}
\date{Version vom \today}

\begin{document}

\frontmatter

\maketitle

\tableofcontents


\mainmatter

\chapter{Seminar-Handwerkszeug}
\section{Regeln für den Workshop}
\label{gfk-workshopregeln}
\index{Workshopregeln}

\paragraph{Vegas-Regel:} Was wir hier persönlichen Dingen teilen, bleibt im Workshop. Wir erzählen Dinge nur anonymisiert nach außen.

\paragraph{Keine dummen Fragen:} Es gibt keine dummen Fragen. Für Fragen, die nicht gut in den Rahmen des aktuellen Themas passen, haben wir einen Themenkühlschrank.

\paragraph{Für-sich-sorgen-Regeln:} Wir alle versuchen, uns auf dem Workshop gut um uns selbst zu kümmern. Wenn wir etwas brauchen, sprechen wir es an oder sorgen selbst dafür.

\paragraph{Freiwilligkeit:} Alle Übungen im Workshop sind freiwillig. Es ist okay, bei eine Übung komplett auszusetzen oder eine Runde zu passen.

\paragraph{Gut zueinander sein:} Wir tun unser Bestes, konstruktiv miteinander umzugehen und uns gut zu behandeln.

\section{Paarinterview zum Kennenlernen}
\label{paarinterview}
\index{Paarinterview}

Nehmt euch für das Interview 15~Minuten Zeit pro Person. Wechselt selbstständig.

\subsection{Leitfragen}

\begin{itemize}
 \item Wo und wie wohnst du?
 \item Was machst du in Beruf und Ehrenamt so? Und was hast du vorher so Interessantes gemacht?
 \item Was sind ein paar Dinge, die dir im Leben zurzeit Freude bereiten?
 \item Was brauchst du (von anderen Personen oder der Umgebung), damit die Zusammenarbeit mit dir gut funktioniert?
 \item Was sollten andere Menschen über dich wissen, wenn sie mit dir zusammenarbeiten?
 \item Was machst du, um trotz der aktuellen Krisen psychisch halbwegs gesund zu bleiben?
 \item Was ist ein \emph{Guilty Pleasure}, dem du ab und an frönst?
\end{itemize}


\subsection{Bonusfragen}

Falls ihr euch schon gut kennt und noch etwas Zeit habt, könnt ihr euch mit diesen Zusatzfragen noch besser kennenlernen.

Diese Fragen kommen aus dem Spiel \emph{Gesprächsstoff XL} \cite{gespraechsstoff}.

\begin{itemize}
  \item Welche Fernsehsendung hast du nie verpasst, als du noch jünger warst?
  \item Wie meinst du, würden dich deine Freund\_innen beschreiben, wenn sie nur drei Wörter verwenden dürften?
  \item Kannst du ein Beispiel dafür nennen, wann du einmal zum richtigen Zeitpunkt am richtigen Ort warst?
  \item Was ist, von der Persönlichkeit her, der größte Unterschied zwischen dir und deinen Eltern?
  \item Was antwortest du einem Kind, das fragt, ob es einen Gott gibt?
  \item Was ist dein Lieblings-Knabberzeug (oder -Nascherei)?
  \item Erzähle von bedeutungslosem Wissen, das du hast.
  \item Kannst du etwas nennen, von dem du wünscht, früher damit angefangen zu haben?
  \item Welche Person hat dich zuletzt so richtig wütend gemacht?
  \item Wenn du wählen müsstest: Würdest du lieber einen anderen Menschen umbringen und straffrei davonkommen oder 25 Jahre für einen Mord im Gefängnis sitzen, den du nicht begangen hast?
  \item Bist du schon einmal im Kino eingeschlafen?
  \item Wenn du den Leben noch einmal leben könntest, was würdest du nicht wieder tun?
  \item Erzähle etwas über ein Ereignis der letzten Jahre, an das du dich für immer erinnern wirst.
\end{itemize}

Die Antworten auf diese Fragen sind \emph{nicht} Teil der Vorstellung vor der Gruppe.

\section{Feedback: Tipps und Tricks}
\index{Feedback}

\subsection{Was ist Feedback?}
Feedback ist für euch eine Gelegenheit, in kurzer Zeit viel über euch selbst zu lernen. Feedback ist ein Anstoß, damit ihr danach an euch arbeiten könnt (wenn ihr wollt).

Feedback heißt, dass euch jemandem einen persönlichen, subjektiven Eindruck in Bezug auf konkrete Punkte mitteilt. Da es sich um einen persönlichen Eindruck im Kopf eines einzelnen Menschen handelt, sagt Feedback nichts darüber aus, wie ihr tatsächlich wart. Es bleibt allein euch selbst überlassen, das Feedback, das ihr bekommt, für euch selbst zu einem großen Gesamtbild zusammenzusetzen.

Es kann übrigens durchaus vorkommen, dass ihr zur selben Sache von verschiedenen Personen völlig unterschiedliches (oder gar gegensätzliches) Feedback bekommt.

Es geht beim Feedback \emph{nicht} darum, euch mitzuteilen, ob ihr ein guter oder schlechter Mensch, ein guter Redner, eine schlechte Rhetorikerin oder so seid. Solche Aussagen haben für euch keinen Lerneffekt. Stattdessen schrecken sie euch ab, Neues auszuprobieren und dabei auch einmal so genannte Fehler zu machen.

Insbesondere ist Feedback keine Grundsatzdiskussion, ob das eine oder andere Verhalten generell gut oder schlecht ist. Solche Diskussionen führt ihr besser am Abend bei einem Bierchen.

\subsection{Feedback geben}
\begin{itemize}
  \item  "`ich"' statt "`man"' oder "`wir"'
  \item die \emph{eigene} Meinung sagen
  \item die andere Person direkt ansprechen: "`du/Sie"' statt "`er/sie"'
  \item eine konkrete, spezifische Beobachtung schildern
  \item nicht verallgemeinern
  \item nicht analysieren oder psychologisieren (nicht: "`du machst das nur, weil \ldots"')
  \item Feedback möglichst unmittelbar danach geben
  \item konstruktiv: nur Dinge ansprechen, die die andere Person auch ändern kann
\end{itemize}

\subsection{Feedback entgegennehmen}
\begin{itemize}
  \item vorher den Rahmen für das Feedback abstecken: Inhalt, Vortragstechnik, Schriftbild \ldots
  \item gut zuhören und ausreden lassen
  \item sich nicht rechtfertigen, verteidigen oder entschuldigen
  \item Missverständnisse klären, Hintergründe erläutern
  \item Feedback als Chance zur Weiterentwicklung sehen
\end{itemize}

\section{Das Blitzlicht}

\begin{itemize}
  \item wird nicht visualisiert
  \item jedeR spricht nur für sich
  \item keine Diskussion (Ausnahme: wichtige Verständnisfragen)
  \item nicht unterbrechen
  \item wer anfängt, fängt an
  \item kurz (ein Blitzlicht ist kein Flutlicht)
\end{itemize}
\section{Themenzentrierte Interaktion (TZI)}
\label{tzi}
\index{Themenzentrierte Interaktion}
\index{TZI}

Die Themenzentrierte Interaktion (TZI)~\cite{tzi-reiser} ist ein Handlungskonzept zur Arbeit in Gruppen. Ziele sind soziales Lernen, die Förderung persönlicher Entwicklung und Fortschritte im Thema.


\subsection{Postulate der TZI}

\subsubsection{Sei deine eigene Chairperson!}

Jede Person spricht nur für sich selbst. Jede Person hat die Verantwortung, sich um sich selbst und ihre Bedürfnisse zu kümmern und sich dafür einzusetzen.

\subsubsection{Störungen haben Vorrang!}

Wenn etwas den Prozess, das Arbeiten oder das Lernen stört, kümmern wir uns so schnell wie möglich darum. Das schließt auch Konflikte ein.

\subsubsection{Verantworte dein Tun und Lassen~-- persönlich und gesellschaftlich!}

Deine Handlungen und Worte haben Konsequenzen. Setze dich damit auseinander und übernimm Verantwortung dafür.


\subsection{Vierfaktorenmodell}

\paragraph{Ich:} Ich selbst.

\paragraph{Wir:} Die Gruppe, in der wir gerade hier sind.

\paragraph{Es:} Die Aufgabe oder das Problem.

\paragraph{Globe:} Die Rahmenbedingungen und äußeren Umstände.


\chapter{Kommunikation}
\section{Prinzipien der Kommunikation}
\label{kommunikationsprinzipien}
\index{Kommunikation}
\index{Missverständnisse}
\index{Sender-Empfänger-Modell}

\begin{itemize}
  \item Es gibt bei Kommunikation immer sendende und (mindestens) eine empfangende Partei (Sender-Empfänger-Modell nach Shannon und Weaver). Diese Rollen können in einer Interaktion öfter wechseln.
  \item Die Partei, die etwas von der anderen will, hat die Zuständigkeit dafür dafür (und das Interesse daran), dass die Kommunikation erfolgreich ist.
  \item Jede Partei hat nur auf ihre eigene Hälfte der Kommunikation direkten Einfluss.
  \item Missverständnisse passieren, und sie sind eher die Regel denn die Ausnahme~-- wir bemerken sie nur oft nicht.
\end{itemize}


\subsection{Metakommunikation}
\label{metakommunikation}
\index{Metakommunikation}

\emph{Metakommunikation} (\glqq Kommunikation über Kommunikation\grqq) bedeutet, die Kommunikation auf eine höhere Ebene zu verlagern und darüber zu reden, wie wir miteinander reden, wie wir miteinander umgehen und was uns beschäftigt.


\chapter{Konflikte}
\section{Was ist ein Konflikt?}
\label{konflikt-definition}
\index{Konflikt: Definition}

Allgemein ist Konflikt ist eine \fett{Situation}, in der verschiedene \fett{Parteien} etwas \fett{Unterschiedliches} tun oder wollen, was sie in dem Moment nicht \fett{miteinander vereinbart} bekommen.

Die Konfliktparteien können dabei zwei Menschen sein, Gruppen von Menschen, Nationen~-- oder auch mehrere Seiten eines Menschen (was dann ein innerer Konflikt wäre).

\section{Eskalationsstufen}
\label{eskalationsstufen}
\index{Eskalationsstufen}
\index{Phasen eines Konflikts}
\index{Konflikt: Phasen}

Friedrich Glas \cite{glasl-konfliktmanegement} teilt Konflikte in mehrere Eskalationsstufen oder Phasen ein:

\vspace{1em}

\begin{tabular}{ l l }
 \multirow{3}{5em}{win-win}   & 1.~Verhärtung \\
                              & 2.~Polarisierung \& Debatte \\
                              & 3.~Taten statt Worte! \\ \hline
 \multirow{3}{5em}{win-lose}  & 4.~Sorge um Image und Koalition \\
                              & 5.~Gesichtsverlust \\
                              & 6.~Drohstrategien \\ \hline
 \multirow{3}{5em}{lose-lose} & 7.~begrenzte Vernichtung \\
                              & 8.~Zersplitterung \\
                              & 9.~gemeinsam in den Abgrund \\
\end{tabular}


\subsection{1. Hauptphase (win-win)}

\subsubsection{Stufe 1~-- Verhärtung}

Konflikte beginnen mit Spannungen, z.\,B.~gelegentliches Aufeinanderprallen von Meinungen. Es ist alltäglich und wird nicht als Beginn eines Konflikts wahrgenommen. Wenn daraus doch ein Konflikt entsteht, werden die Meinungen fundamentaler. Der Konflikt könnte tiefere Ursachen haben.

\subsubsection{Stufe 2~-- Debatte, Polemik}

Ab hier überlegen sich die Konfliktparteien Strategien, um die anderen von ihren Argumenten zu überzeugen. Meinungsverschiedenheiten führen zu einem Streit. Man will die anderen unter Druck setzen. Schwarz-Weiß-Denken entsteht.

\subsubsection{Stufe 3~-- Taten statt Worte}

Die Konfliktparteien erhöhen den Druck auf die jeweils anderen, um sich oder die eigene Meinung durchzusetzen. Gespräche werden z.~,B.~abgebrochen. Es findet keine verbale Kommunikation mehr statt, und der Konflikt verschärft sich schneller. Das Mitgefühl für die \glqq anderen\grqq{} geht verloren.


\subsection{2.~Hauptphase (win-lose)}

\subsubsection{Stufe 4~-- Koalitionen, Images}

Der Konflikt verschärft sich dadurch, dass man Sympathisant\_innen für seine Sache sucht. Da man sich im Recht glaubt, kann man die Gegenseite denunzieren. Es geht nicht mehr um die Sache, sondern darum, den Konflikt zu gewinnen, damit die Gegenseite verliert.

\subsubsection{Stufe 5~-- Gesichtsverlust}

Die Gegenseite soll in ihrer Identität vernichtet werden, zum Beispiel durch Unterstellungen. Hier ist der Vertrauensverlust vollständig. Gesichtsverlust bedeutet in diesem Sinne Verlust der moralischen Glaubwürdigkeit.

\subsubsection{Stufe 6~-- Drohstrategien}

Mit Drohungen versuchen die Konfliktparteien, die Situation absolut zu kontrollieren. Sie sollen die eigene Macht veranschaulichen. Man droht z.\,B.~mit einer Forderung (\glqq 10 Millionen Euro\grqq), die durch eine Sanktion (\glqq Sonst sprenge ich Ihr Hauptgebäude in die Luft!\grqq) verschärft und durch das Sanktionspotenzial (Sprengstoff zeigen) untermauert wird. Hier entscheiden die Proportionen über die Glaubwürdigkeit der Drohung.


\subsection{3.~Hauptphase (lose-lose)}

\subsubsection{Stufe 7~-- Begrenzte Vernichtung(sschläge)}

Hier soll der Gegenseite mit allen Tricks empfindlich geschadet werden. Die Gegenseite wird nicht mehr als Mensch wahrgenommen. Ab hier wird ein begrenzter eigener Schaden schon als Gewinn angesehen, sollte der der Gegenseite größer sein.

\subsubsection{Stufe 8~-- Zersplitterung}

Das Unterstützungssystem der Gegenseite soll mit Vernichtungsaktionen zerstört werden.

\subsubsection{Stufe 9~-- Gemeinsam in den Abgrund}

Ab hier kalkuliert man die eigene Vernichtung mit ein, um die Gegenseite zu besiegen.

\section{Annahmen über Konflikte auf diesem Workshop}
\label{konflikte-annahmen}

\begin{itemize}
  \item Spannungen, Reibungen und Konflikte sind in der Zusammenarbeit unvermeidbar und nichts Schlimmes.
  \item Konflikte sind hilfreich dafür, dass wir uns gegenseitig besser kennenlernen und unsere Zusammenarbeit verbessern können.
  \item Je früher wir Spannungen, Reibungen und Konflikte ansprechen, desto einfacher und schmerzloser ist es.
  \item Konflikte nicht anzusprechen und nicht zu klären, behindert die Zusammenarbeit.
  \item Konfliktmanagement ist eine Fähigkeit, die man lernen kann.
  \item Nicht alle Konflikte mit allen Menschen lassen sich lösen. Choose your battles.
\end{itemize}

\section{Arten von Konflikten}
\label{konflikt-arten}
\index{Konflikt: Arten}

Kurt Lewin~\cite{gelingende-kommunikation-revisited, schneider-konflikte} unterscheidet mehrere Arten von Konflikten.


\subsection{Innere Konflikte (intrapersonale Konflikte)}
\index{innere Konflikte}
\index{intrapersonale Konflikte}

\index{Motivkonflikt}

\subsubsection{Annäherungskonflikt (Appetenzkonflikt, Appetenz-Appetenz-Konflikt)}
\index{Annäherungskonflikt}
\index{Appetenzkonflikt}
\index{Appetenz-Appetenz-Konflikt}

Beide Optionen sind gut, schließen sich aber gegenseitig aus.

\subsubsection{Ambivalenzkonflikt (Annäherungs-Vermeidungs-Konflikt, Appetenz-Aversions-Konflikt)}
\index{Ambivalenzkonflikt}
\index{Annäherungs-Vermeidungs-Konflikt}
\index{Appetenz-Aversions-Konflikt}

Das Ziel hat sowohl Vorteile als auch Nachteile.


\subsubsection{Vermeidungskonflikt}
\index{Vermeidungskonflikt}

Ich will eigentlich keine der möglichen Optionen.


\subsubsection{Identitätskonflikt}
\index{Identitätskonflikt}

Wer bin ich? Was für ein Mensch will ich sein?


\subsection{Zwischenmenschliche Konflikte (interpersonale Konflikte)}
\index{zwischenmenschliche Konflikte}
\index{interpersonale Konflikte}

\subsubsection{Rollenkonflikt}
\index{Rollenkonflikt}

Wer darf und muss was in welcher Rolle? Wer hat überhaupt welche Rolle?

\subsubsection{Verteilungskonflikt}
\index{Verteilungskonflikt}

Wer bekommt wie viel?

\subsubsection{Identitätskonflikt}
\index{Identitätskonflikt}

Wer bin ich? Was für ein Mensch will ich sein?

\subsubsection{Zielkonflikt}
\index{Zielkonflikt}

Wo wollen wir (zusammen) hin? Warum wollen wir das?

\subsubsection{Beziehungskonflikt}
\index{Beziehungskonflikt}

Wie gehen wir miteinander um? Wie stehen wir zueinander?

\subsubsection{Informationskonflikt}
\index{Informationskonflikt}

Was ist Sache? Wer weiß was? Was ist richtig, was ist falsch?


\subsection{Konflikte nach Beteiligten}
\index{Generationenkonflikte}
\index{Informationskonflikt}
\index{Arbeitskampf}
\index{ethnische Konflikte}

\begin{itemize}
  \item Familienkonflikte
  \item Generationenkonflikte
  \item Arbeitskampf
  \item Konflikte innerhalb von Gruppen
  \item Konflikte in der Schule
  \item ethnische Konflikte
  \item Konflikte zwischen Staaten
  \item \ldots
\end{itemize}


\subsection{Konflikte nach Themen}
\index{politische Konflikte}
\index{soziale Konflikte}
\index{ethische Konflikte}

\begin{itemize}
  \item politische Konflikte
  \item soziale Konflikte
  \item ethische Konflikte
  \item \ldots
\end{itemize}

\input{konflikte-phasen}
\section{Strategien zu Konfliktlösungen}
\label{konfliktloesungs-strategien}
\index{Konfliktlösungen}
\index{Lösungsstrategien}

Gerhard Schwarz~\cite{schwarz-konfliktmanagement} stellt folgende Strategien vor, mit Konflikten umzugehen (und sie optimalerweise zu lösen). Die Strategien sind grob geordnet von \glqq wenig bis sehr hilfreich\grqq.

\subsection{Flucht}
\index{Flucht}

Man geht sich aus dem Weg oder spricht den Konflikt nicht an. Das ist das ursprünglich-instinktive Verhalten von Menschen bei Konflikten.

\subsubsection{Vorteile}
\begin{itemize}
  \item schmerzlos
  \item kurzfristig gelöst
  \item energiesparend
\end{itemize}

\subsubsection{Nachteile}
\begin{itemize}
  \item Konflikt verschärft sich
  \item unbefriedigend
\end{itemize}


\subsection{Vernichtung}
\index{Vernichtung}

\subsubsection{Vorteile}
\begin{itemize}
  \item dauerhaft gelöst
\end{itemize}

\subsubsection{Nachteile}
\begin{itemize}
  \item nicht korrigierbar
  \item destruktiv
  \item kein Lernen, keine Weiterentwicklung
\end{itemize}


\subsection{Unterordnung, Unterwerfung}
\index{Unterordnung}
\index{Unterwerfung}

\subsubsection{Vorteile}
\begin{itemize}
  \item umkehrbar
\end{itemize}

\subsubsection{Nachteile}
\begin{itemize}
  \item die stärkere Person gewinnt und nicht die, die im Recht ist
\end{itemize}


\subsection{Delegation}
\index{Delegation}

\subsubsection{Vorteile}
\begin{itemize}
  \item endgültige Lösung
  \item für die Konfliktparteien direkt wenig Arbeit
\end{itemize}

\subsubsection{Nachteile}
\begin{itemize}
  \item wenig Identifikation mit dem Ergebnis
  \item Stille-Post-Effekt
  \item Perspektiven und Bedürfnisse können untergehen
  \item umständlich, langwierig
\end{itemize}


\subsection{Kompromiss}
\index{Kompromiss}

\subsubsection{Vorteile}
\begin{itemize}
  \item Alle können ihr Gesicht wahren.
  \item zumindest teilweise Zufriedenheit: Alle sind gleichermaßen zufrieden.
  \item geteilte Verantwortung
\end{itemize}

\subsubsection{Nachteile}
\begin{itemize}
  \item Teilverluste: Alle sind gleichermaßen unzufrieden.
  \item wenig Blick auf die tatsächlichen Bedürfnisse und Interessen
\end{itemize}


\subsection{Konsens, Konsent}
\index{Konsens}
\index{Konsent}

Die Website \emph{The Decider}\footnote{\url{https://thedecider.app/}} erklärt den Unterschied zwischen Konsens und Konsent (und einigen anderen Entscheidungsfindungsmodellen). Sie führt euch auch durch den Prozess, ein passendes Entscheidungsfindungsmodell für eine Situation zu finden.

\subsubsection{Vorteile}
\begin{itemize}
  \item hohe Akzeptanz für die Lösung
  \item dauerhafte Lösung
  \item Berücksichtigung der Bedürfnisse und Interessen
\end{itemize}

\subsubsection{Nachteile}
\begin{itemize}
  \item tendenziell langwierig und aufwändig
  \item erfordert mehr emotionale und kommunikative Kompetenz bei den Konfliktparteien
\end{itemize}

\section{Difficult Conversations}

Die Inhalt aus diesem Kapitel stammen aus dem Buch~\cite{difficult-conversations}.


\subsection{Stärken dieser Methode}

Dieser Ansatz fokussiert sich vor allem darauf, sich selbst, die andere Person und das Problem gut zu verstehen. Der Teil zu Lösungen ist hingegen eher kurz. Dadurch ist dieser Ansatz vor allem bei Problemen hilfreich, bei denen die Beziehung beschädigt ist und wo es weniger ums Lösen eines sachlichen Problems geht.


\subsection{Die Schritte}

\begin{enumerate}
  \item Vorbereitung: die drei Gespräche für sich selbst durchgehen
  \item Ziel des Gespräch überlegen: entscheiden, ob ich das Problem tatsächlich ansprechen möchte.
  \item die andere Person ansprechen und \fett{mit der dritten Geschichte beginnen}
  \item die beiden Geschichten aus der Sicht beider Personen austauschen (inklusive der Gefühle)
  \item die Probleme gemeinsam lösen
\end{enumerate}


\subsection{Die drei Gespräche}

Jedes Konfliktgespräch besteht eigentlich aus drei unterschiedlichen Gespräche:

\begin{enumerate}
  \item das \glqq Was ist passiert?\grqq-Gespräch
  \item das Gefühle-Gespräch
  \item das Gespräch zur Identität
\end{enumerate}

\subsubsection{Das \glqq Was ist passiert?\grqq-Gespräch}

Hier geht es (in den allermeisten Fällen) überhaupt nicht darum, wer \glqq Recht hat\grqq.

Die unterschiedlichen Geschichten:

\begin{itemize}
  \item Die Geschichte aus Sicht von Person 1.
  \item Die Geschichte aus Sicht von Person 2.
  \item Die Geschichte aus Sicht einer neutralen dritten Person, die nicht in die Köpfe der beiden beteiligten Personen hineinschauen kann (\glqq die dritte Geschichte\grqq).
\end{itemize}

Fragen zu der jeweils eigenen Geschichte:

\begin{itemize}
  \item Was habe ich selbst getan?
  \item Was habe ich wahrgenommen (ohne Interpretation)?
  \item Was waren meine Absichten?
  \item Welche Auswirkungen hatte das Ganze auf mich?
  \item Wie habe ich selbst zu dem Problem beigetragen?
\end{itemize}


\subsubsection{Das Gefühle-Gespräch}

Fragen dazu:

\begin{itemize}
  \item Wie habe ich mich dabei gefühlt?
  \item Wie fühle ich mich jetzt?
\end{itemize}

\subsubsection{Das Gespräch zur Identität}

Fragen dazu:

\begin{itemize}
  \item Als was für ein Mensch sehe ich mich? Und was für ein Mensch will ich sein?
  \item Wie hat das, was passiert ist, meine Identität und mein Selbstbild gefährdet?
\end{itemize}


\subsection{Strategien zur Problemlösung}

\begin{itemize}
  \item Haben wir alle Informationen? Was fehlt?
  \item Welche Ideen für eine Lösung haben wir?
  \item Wie können wir eine Lösung testen, um zu schauen, ob sie etwas taugt?
  \item Was fehlt mir persönlich noch?
  \item Was bräuchte es, damit ich einer Lösung zustimmen kann und will?
  \item Was bräuchte es, damit du einer Lösung zustimmen kannst und willst?
  \item Was wäre dein Rat?
  \item Was wären klare Standards, um eine Lösungsidee zu bewerten?
  \item Falls wir keine Lösungsidee finden, mit der wir beide einverstanden sind: Was wären die Alternativen?
\end{itemize}


\section{Das Harvard-Konzept}
\label{harvard-konzept}
\index{Harvard-Konzept}

Das \emph{Harvard-Konzept}~\cite{harvard-konzept} ist eine Methode für Verhandlungen und zur Konfliktlösung. Es ist ein Teil des \emph{Harvard Negotiation Project} der Harvard Law School.

Ziel der Methode ist eine konstruktive und friedliche Einigung mit einem Win-Win-Ergebnis. Es kam unter anderem beim Camp-David-Abkommen im Nahen Osten zum Einsatz.

\subsection{Grundsätzlicher Ansatz}

Das Ziel vom Harvard-Konzept ist, dass wir davon wegkommen, um Positionen zu feilschen, und stattdessen echt Win-Win-Lösungen zu finden versuchen.


\subsection{Prinzipien}


\subsubsection{Menschen und Probleme getrennt voneinander behandeln}

\begin{itemize}
  \item Sachprobleme als Sachprobleme behandeln
  \item wertschätzend miteinander umgehen
  \item Probleme auf der Beziehungsebene explizit ansprechen und behandeln
\end{itemize}


\subsubsection{Auf Interessen konzentrieren statt auf Positionen}

\paragraph{Positionen} sind sehr konkret, zum Beispiel: \glqq Ich will diese Tafel Schokolade haben.\grqq Positionen sind den \emph{Strategien} aus der Gewaltfreien Kommunikation ähnlich.

\paragraph{Interessen} sind das, weswegen ich eine Position vertrete, zum Beispiel für die Schokolade: \glqq Ich habe Hunger.\grqq\ oder \glqq Ich habe Lust auf etwas zu naschen.\grqq Interessen sind den \emph{Bedürfnissen} aus der Gewaltfreien Kommunikation ähnlich, aber etwas weniger universell.

Wenn wir auf die Interessen schauen, ist es viel einfacher, eine Win-Win-Lösung zu finden, die zu den Interessen aller beteiligten Parteien passt.


\subsubsection{Win-Win-Lösungsideen entwickeln}

Lösungen funktionieren dann besonders gut, wenn sie die Interessen aller Parteien erfüllen.


\subsubsection{Neutrale Beurteilungskriterien vereinbaren}

Nachdem ihr die gegenseitigen Interessen herausgefunden (und aufgeschrieben) habt, könnt ihr schauen, welche weiteren Beurteilungskriterien ihr noch wichtig findet.

Dies könnten zum Beispiel sein:

\begin{itemize}
  \item Gleichbehandlung
  \item Kosten
  \item zeitlicher Rahmen
\end{itemize}


\subsection{Ja, aber \ldots}


\subsubsection{Die \glqq Beste Alternative\grqq\ überlegen}

Überlegt euch vorher, was eure \glqq Beste Alternative\grqq\ (der \glqq Plan B\grqq) wäre für den Fall, dass ihr miteinander zu keinem Ergebnis kommt.


\subsubsection{Wenn die anderen nicht mitspielen: Verhandlungs-Judo}

\begin{itemize}
  \item Warum ist dir das wichtig?
  \item Was würde passieren, wenn wir das genau so machen, wie du es vorschlägst?
  \item Was würdest du stattdessen vorschlagen?
  \item Bitte korrigiere mich, falls ich falsch liege.
\end{itemize}

\section{Konflikte in Beziehungen}
\label{konflikte-in-beziehungen}
\index{Beziehungen}
\index{Liebesbeziehungen}

Die Prinzipien, die für Konflikte in Liebesbeziehungen gelten, lassen sich auch in anderen Kontexten anwenden.

Viele Inhalte in diesem Abschnitt stammen aus dem Buch \emph{The Intimate Enemy} \cite{intimate-enemy}. Dieses Buch geht davon aus, dass es essentiell für das Bestehen von langfristigen Beziehungen ist, Konflikte anzusprechen und auszutragen. Andere Inhalte kommen aus dem Buch \emph{The Seven Principles For Making Marriage Work} \cite{gottman-7-principles} von John Gottman. Das Konzept von HALT habe ich aus dem Podcast \emph{Multiamory} \cite{halt-multiamory} sowie aus deren Buch \cite{multiamory-book} übernommen.


\subsection{Positive Interaktionen schaffen}
\index{Interaktionen}
\index{positive Interaktionen}
\index{negative Interaktionen}

Laut John Gottman ist das Verhältnis von positiven zu negativen Interaktionen in langfristig stabilen Beziehungen \fett{5:1} (oder besser). Das heißt, dass es sehr hilfreich ist, bewusst viel Angenehmes miteinander zu erleben sowie sich gegenseitig immer mal wieder von Herzen etwas Nettes zu sagen, damit ihr dadurch den Goodwill habt, auch Konflikte durchzustehen.


\subsection{Wie man eine Beziehung zerstört}

\paragraph{Gunny-Sacking} (auf Deutsch in etwa \glqq in Jutesäcken sammeln\grqq) bezeichnet, dass ihr Probleme über eine lange Zeit still ansammelt und dann irgendwann in einem Schwung über der anderen Person auskübelt.
\index{Gunny-Sacking}

\paragraph{Hinterhalt und Überfall:} ein Konfliktgespräch beginnen, wenn die andere Person gerade auf dem Sprung ist oder gerade nach Haus gekommen ist
\index{Hinterhalt}
\index{Überfall}

\paragraph{Fahrerflucht:} ein Konfliktgespräch beginnen oder der anderen Person etwas vorwerfen, und dann direkt das Thema wechseln oder die Situation verlassen
\index{Fahrerflucht}

\paragraph{Mit einem persönlichen Angriff starten} statt mit einer neutralen Beschreibung der Ereignisses
\index{persönliche Angriffe}

\paragraph{Reparaturversuche zurückweisen} (zum Beispiel aus Rache oder Trotz)
\index{Reparaturversuche}

\paragraph{Emotional dichtmachen:} Jede funktionierende Beziehung (auch Freundschaften oder Arbeitsbeziehungen) braucht es, dass alle Beteiligten eine Verbindung dazu zulassen, was in ihnen gegenseitig lebendig ist.

\paragraph{Verachtung:} Das bedeutet, dass ihr die Person als Ganzes ablehnt und verurteilt statt nur deren Verhalten.
\index{Verachtung}

\paragraph{Verteidigung (Defensiveness)} ist, wenn ihr die Not der anderen Person als Angriff interpretiert und euch dann darauf fokussiert, dies abzuwehren und eventuell auch zurückzuschlagen.
\index{Verteidigung}
\index{Defensiveness}

\paragraph{Fluten (Flooding)} bedeutet, so viele Probleme und Vorwürfe auf einmal anzusprechen, dass sich die andere Person davon überwältig fühlt
\index{Fluten}
\index{Flooding}


\subsection{Konflikte konstruktiv austragen}

\begin{itemize}
  \item Konfliktgespräche \fett{vorher planen}, anstatt die spontan zwischendurch zu führen
  \item Gespräch \fett{zu zweit} führen statt vor Zeug\_innen
  \item auf \fett{das Miteinander fokussieren} statt aufs Gewinnen oder Rechthaben
  \item die \fett{lösbaren Probleme lösen}, und mit \fett{den anderen Problem leben lernen}
  \item das Gespräch unterbrechen, falls ihr gerade \fett{HALT} seid:
    \index{HALT}
    \begin{itemize}
      \item \fett{H}ungry: hungrig
      \item \fett{A}ngry: wütend
      \item \fett{L}onely: einsam
      \item \fett{T}ired: zu müde (oder auch betrunken, auf anderen Drogen oder krank)
    \end{itemize}
\end{itemize}

\section{Entschuldigungen}
\label{entschuldigungen}
\index{Entschuldigungen}

Die meisten Inhalte in diesem Abschnitt kommen aus dem Buch \emph{The 5 Apology Languages} \cite{apology-languages}, ergänzt durch Inhalt aus dem Podcast \emph{Familie verstehen} \cite{familie-verstehen-podcast} von Kathy Weber.


\subsection{Funktion von Entschuldigungen}

\begin{itemize}
  \item die gemeinsame Beziehung reparieren
  \item damit die Person sich besser fühlt, die sich entschuldigen möchte, indem sie sich dadurch eigene Bedürfnisse erfüllt, zum Beispiel:
    \begin{itemize}
      \item Verbindung
      \item Freundschaft
      \item Integrität (nach den eigenen Werten handeln)
      \item Zusammenarbeit
      \item Harmonie
    \end{itemize}
\end{itemize}


\subsection{Warum der Begriff \glqq Entschuldigung\grqq\ problematisch ist}

\begin{itemize}
  \item Es geht nicht darum, die schuldige Person zu finden und zu bestrafen oder zu beschämen, sondern darum, die Beziehung zu reparieren.
  \item Das Konzept von Schuld trennt Menschen, anstatt sie zu verbinden.
  \item Niemand ist gerne schuldig. Das macht es schwieriger als nötig, um Entschuldigung zu bitten.
  \item Niemand kann sich selbst entschuldigen. Ihr könnt nur die andere Person um Entschuldigung oder Verzeihung bitten, und diese kann dann die Entschuldigung annehmen (oder auch nicht).
\end{itemize}


\subsection{Elemente einer guten Entschuldigung}

Eine gut funktionierende Entschuldigung enthält einige dieser Elemente (wenn auch nicht notwendigerweise immer alle):

\paragraph{Zeigen, dass ihr wisst, was tatsächlich passiert ist.} Ansonsten kann eure Entschuldigung nicht glaubhaft sein. Oft wird aus der Situation auch schon deutlich, dass ihr das wisst~-- etwa, wenn ihr gerade jemanden auf den Fuß getreten seid.

\paragraph{Bedauern ausdrücken:} \glqq Es tut mir Leid.\grqq\ oder \glqq Ich bedauere total, dass \ldots\grqq\ Dadurch zeigt ihr, dass euch die andere Person und ihr Leid nicht gleichgültig sind.

\paragraph{Empathie ausdrücken:} \glqq Ich kann mit gut vorstellen, dass du gerade total sauer bist.\grqq\ oder \glqq Das war wahrscheinlich total unangenehm für dich.\grqq

\paragraph{Verantwortung für das eigene Handeln übernehmen:} \glqq Es war mein Fehler.\grqq\ oder \glqq Ich sehe, dass dadurch, dass ich \ldots\ getan habe, \ldots\ geschehen ist.\grqq

\paragraph{Wiedergutmachung:} \glqq Wie kann ich den Schaden reparieren?\grqq\ oder \glqq Was würde dir jetzt dazu helfen?\grqq

\paragraph{Die Wiederholungsgefahr reduzieren:} Bei Dingen, die theoretisch noch einmal geschehen könnten, könnt ihr erklären, was ihr tun werdet, damit das nicht noch einmal vorkommt.

\paragraph{Um Verzeihung bitten:} \glqq Ich bitte um Entschuldigung.\grqq\ oder \glqq Ist das so für dich okay?\grqq\ oder \glqq Mir ist es wichtig, dass es zwischen uns wieder okay ist. Ist jetzt noch etwas zwischen uns doof?\grqq


\subsection{Nicht-Entschuldigungen (Nonpologys)}
\index{Nicht-Entschuldigungen}
\index{Nonpologys}

Dies sind Aussagen, die wie Entschuldigungen klingen, aber tatsächlich keine Verantwortung für das Verhalten erkennen lassen.

Einige Merkmale und Muster:

\begin{itemize}
  \item Rechtfertigung und Erklärung des eigenen Verhaltens
  \item dem Gegenüber die Verantwortung geben (etwa für ein angebliches Missverständnis oder für dessen Gefühle)
  \item auf die Absicht fokussieren statt auf die Wirkung
  \item das eigene Verhalten oder die Auswirkung kleinreden
  \item auf andere Themen und Aspekte ablenken
  \item sich selbst als Opfer darstellen (beispielsweise der Berichterstattung)
  \item Gaslighting
\end{itemize}


\chapter{Gewaltfreie Kommunikation}
\index{Gewaltfreie Kommunikation}
\section{Annahmen in der gewaltfreien Kommunikation}

\begin{itemize}
  \item Alle Menschen sind zu Empathie fähig, und alle Menschen brauchen Empathie.
  \item Dinge explizit zu sagen, macht es wahrscheinlicher, dass mein Gegenüber sie hört.
  \item Niemand kann Gedanken lesen.
  \item Menschen sind selbst dafür verantwortlich, ihre Bedürfnisse erfüllt zu bekommen.
  \item Andere Menschen sind für meine Gefühle nicht verantwortlich.
  \item Menschen haben immer einen guten Grund für das, was sie tun.
  \item Alle Menschen sind gewillt, zum Wohle ihrer Mitmenschen beizutragen.
  \item Konflikte sind im Miteinander wichtig und unvermeidbar.
  \item Einen Scheiß muss ich.
\end{itemize}

\section{Gefühle}
\label{gefuehle}
\index{Gefühle}
\index{Emotionen}

Das ursprüngliche Vokabular stammt von Marshall Rosenberg aus \cite[S.~216]{gfk-rosenberg} bzw.~im englischsprachigen Original \cite[S.~210]{nvc-rosenberg}. Das erweiterte Vokabular und die Kriterien hab ich aus \cite[S.~56~f]{gfk-dummies} übernommen.


\subsection{Echte Gefühle}

Woran man echte Gefühle erkennt:

\begin{enumerate}
 \item Ein echtes Gefühl kann jeder Mensch auf der Welt empfinden~-- vom Kindergartenkind bis zum alten Menschen.
 \item Echte Gefühle sind körperlich spürbar.
 \item Echte Gefühle enthalten keine Schuldzuweisung. In ihnen gibt es keine Täter\_innen und keine Opfer. Sie können sich aber durchaus auf einen Menschen richten.
\end{enumerate}

\subsubsection{Primärgefühle und Sekundärgefühle}

Der Psychologe Joseph LeDoux~\cite{emotional-brain} hat in Experimenten herausgefunden, dass Reize zwei Prozesse im Zentralnervensystem anstoßen:

\paragraph{\glqq quick and dirty\grqq:} ein schneller, ungenauer Prozess, der Reize in \glqq gefährlich\grqq{} und \glqq ungefährlich\grqq{} unterteilt, und der direkt von vom Thalamus zur Amygdala führt
\paragraph{kognitives Verarbeiten:} ein langsamerer Prozess, der im Thalamus beginnt und über den präfrontalen Cortex sowie auf einer Nebenroute über den Hippocampus läuft

Emotionen, die über der ersten Prozess entstehen, werden auch \emph{Primärgefühle} oder \emph{Basisemotionen} genannt. Emotionen aus dem zweiten Prozess werden \emph{Sekundärgefühle} genannt.


\paragraph{Primärgefühle/Basisemotionen}
\index{Primärgefühle}
\index{Basisemotionen}

Primärgefühle sind die erste Reaktion des Körpers auf ein Ereignis und meist sehr stark. Sie sind ursprüngliche und instinktive Überlebensreaktionen.

Diese Emotionen dauern etwas 90 Sekunden, wenn wir sie nicht \glqq erneuern\grqq.

Laut Robert Plutchik\cite{plutchik-emotions} sind dies diese acht Basisemotionen:

\begin{itemize}
  \item Akzeptanz, Vertrauen
  \item Ekel, Abscheu
  \item Freude, Ekstase
  \item Furcht, Panik
  \item Neugierde, Erwartung
  \item Traurigkeit, Kummer
  \item Zorn, Wut
  \item Überraschung, Erstaunen
\end{itemize}

Paul Ekman\cite{ekman-emotions} hat sieben universelle Basisemotionen empirisch nachgewiesen:

\begin{itemize}
  \item Ekel
  \item Freude
  \item Furcht
  \item Traurigkeit
  \item Verachtung
  \item Wut
  \item Überraschung
\end{itemize}

\paragraph{Sekundärgefühle}
\index{Sekundärgefühle}

Sekundärgefühle sind eine Mischung aus Primärgefühlen und einer bestimmten Art zu denken. Sie entstehen daher etwas weniger unmittelbar als die Primärgefühle.

Wenn wir die Bedürfnisse erkennen, die hinter Sekundärgefühlen stehen, können wir oft die dahinter stehenden Primärgefühle sehen.

Theoretisch können alle Primärgefühle auch sekundär sein. Einige Sekundärgefühle sind jedoch typischer als andere:

\begin{itemize}
 \item Aggression
 \item Angst
 \item Hoffnungslosigkeit
 \item Reizbarkeit
 \item Wut
 \item depressive Verstimmung
 \item innere Leere
\end{itemize}

Wenn wir über Gefühle kommunizieren, ist die Unterscheidung zwischen primären und sekundären Gefühlen in der Praxis nicht besonders relevant.

\subsubsection{Angenehme Gefühle, wenn Bedürfnisse erfüllt sind}
\index{Bedürfnisse}
\label{gefuehle-liste}
\label{angenehme-gefuehle}

\begin{multicols}{2}
  \begin{itemize}
    \item amüsiert
    \item angeregt
    \item aufgedreht
    \item aufgeregt
    \item ausgeglichen
    \item befreit
    \item begeistert
    \item behaglich
    \item belebt
    \item berauscht
    \item beruhigt
    \item berührt
    \item beschwingt
    \item bewegt
    \item dankbar
    \item eifrig
    \item ekstatisch
    \item ekstatisch
    \item energetisiert
    \item engagiert
    \item enthusiastisch
    \item entlastet
    \item entschlossen
    \item entspannt
    \item entzückt
    \item erfreut
    \item erfrischt
    \item erfrischt
    \item erfüllt
    \item ergriffen
    \item erleichtert
    \item erstaunt
    \item erwartungsvoll
    \item fasziniert
    \item frei
    \item friedlich
    \item froh
    \item fröhlich
    \item gebannt
    \item geborgen
    \item gefesselt
    \item gelassen
    \item gerührt
    \item gesammelt
    \item gespannt
    \item gesund
    \item glücklich
    \item großartig
    \item gut gelaunt
    \item heiter
    \item hellwach
    \item hoffnungsvoll
    \item inspiriert
    \item klar
    \item kraftvoll
    \item kribbelig
    \item lebendig
    \item leicht
    \item liebevoll
    \item locker
    \item lustig
    \item motiviert
    \item munter
    \item mutig
    \item neugierig
    \item optimistisch
    \item ruhig
    \item sanft
    \item satt
    \item schwungvoll
    \item selbstsicher
    \item selig
    \item sicher
    \item sorglos
    \item still
    \item stolz
    \item unbekümmert
    \item unbeschwert
    \item vergnügt
    \item verliebt
    \item vertrauensvoll
    \item voller Liebe
    \item wach
    \item weit
    \item wissbegierig
    \item wissbegierig
    \item zufrieden
    \item zugeneigt
    \item zuversichtlich
    \item zärtlich
    \item überglücklich
    \item überrascht
    \item überwältigt
  \end{itemize}
\end{multicols}

\subsubsection{Unangenehme Gefühle, wenn Bedürfnisse nicht (genug) erfüllt sind}
\index{Bedürfnisse}
\label{unangenehme-gefuehle}

\begin{multicols}{2}
  \begin{itemize}
    \item alarmiert
    \item angeekelt
    \item angespannt
    \item apathisch
    \item aufgeregt
    \item ausgelaugt
    \item bedrückt
    \item beklommen
    \item besorgt
    \item bestürzt
    \item betroffen
    \item beunruhigt
    \item bitter
    \item blockiert
    \item dumpf
    \item durcheinander
    \item durstig
    \item eifersüchtig
    \item einsam
    \item elend
    \item empört
    \item entrüstet
    \item enttäuscht
    \item ernüchtert
    \item erschlagen
    \item erschrocken
    \item erschöpft
    \item erschüttert
    \item erstarrt
    \item frustriert
    \item furchtsam
    \item gehemmt
    \item geladen
    \item gelangweilt
    \item genervt
    \item gestresst
    \item hart
    \item hasserfüllt
    \item hilflos
    \item hungrig
    \item irritiert
    \item kalt
    \item klein
    \item kraftlos
    \item kribbelig
    \item lasch
    \item leer
    \item lethargisch
    \item lustlos
    \item matt
    \item miserabel
    \item mutlos
    \item müde
    \item nervös
    \item niedergeschlagen
    \item ohnmächtig
    \item panisch
    \item perplex
    \item pessimistisch
    \item ratlos
    \item resigniert
    \item ruhelos
    \item sauer
    \item scheu
    \item schlapp
    \item schuldig
    \item schwer
    \item schwermütig
    \item schüchtern
    \item sorgenvoll
    \item teilnahmslos
    \item todtraurig
    \item tot
    \item traurig
    \item träge
    \item unbehaglich
    \item ungeduldig
    \item unglücklich
    \item unruhig
    \item unsicher
    \item unter Druck
    \item unwohl
    \item unzufrieden
    \item verbittert
    \item verspannt
    \item verstört
    \item verwirrt
    \item verzweifelt
    \item voller Angst
    \item voller Scham
    \item voller Sorgen
    \item widerwillig
    \item wütend
    \item zappelig
    \item zittrig
    \item zornig
    \item ängstlich
    \item ärgerlich
    \item überwältigt
  \end{itemize}
\end{multicols}


\subsection{Gedanken (\glqq Pseudogefühle, Interpretationen\grqq)}
\label{pseudogefuehle}
\index{Gedanken}
\index{Pseudogefühle}
\index{Interpretationen}

\emph{Gedanken} sind Begriffe, die als Gefühlsäußerung angekündigt werden, aber statt dessen Vorwürfe, Schuldzuweisungen, Analysen oder Interpretationen enthalten. Diese werden in der GfK auch \emph{Pseudogefühle} oder \emph{Interpretationen} genannt.

\begin{multicols}{2}
  \begin{itemize}
    \item abgelehnt
    \item abgeschnitten
    \item akzeptiert
    \item allein gelassen
    \item an den Pranger gestellt
    \item an die Wand gestellt
    \item angegriffen
    \item attackiert
    \item ausgebeutet
    \item ausgenutzt
    \item ausgeschlossen
    \item ausgestoßen
    \item beachtet
    \item bedroht
    \item beleidigt
    \item belogen
    \item belästigt
    \item benutzt
    \item beschuldigt
    \item beschämt
    \item beschützt
    \item bestraft
    \item bestätigt
    \item betrogen
    \item bevormundet
    \item blamiert
    \item deplatziert
    \item diskriminiert
    \item dominiert
    \item entmutigt
    \item enttäuscht
    \item erdrückt
    \item erniedrigt
    \item ernst genommen
    \item festgenagelt
    \item gedrängt
    \item geehrt
    \item geliebt
    \item gemaßregelt
    \item gemobbt
    \item gequält
    \item geschmeichelt
    \item gesehen
    \item getäuscht
    \item gewürdigt
    \item gezwungen
    \item gut beraten
    \item herabgesetzt
    \item hereingelegt
    \item hintergangen
    \item ignoriert
    \item im Mittelpunkt
    \item in die Ecke gedrängt
    \item in die Enge getrieben
    \item isoliert
    \item klein gemacht
    \item lächerlich gemacht
    \item manipuliert
    \item minderwertig
    \item missachtet
    \item missbrauchst
    \item missverstanden
    \item nicht anerkannt
    \item nicht ehrlich behandelt
    \item nicht einbezogen
    \item nicht ernst genommen
    \item nicht geliebt
    \item nicht gesehen
    \item nicht respektiert
    \item nicht unterstützt
    \item nicht verstanden
    \item nicht wertgeschätzt
    \item provoziert
    \item reingelegt
    \item sabotiert
    \item schikaniert
    \item schlecht behandelt
    \item schön
    \item sympathisch
    \item totgequatscht
    \item unerwünscht
    \item ungehört
    \item ungeliebt
    \item ungerecht behandelt
    \item uninteressant
    \item unter Druck gesetzt
    \item unterbezahlt
    \item unterdrückt
    \item unverstanden
    \item unwichtig
    \item verarscht
    \item verfolgt
    \item vernachlässigt
    \item über den Tisch gezogen
    \item überfordert
    \item übergangen
    \item überlistet
  \end{itemize}
\end{multicols}

\section{Bedürfnisse}
\label{beduerfnisse}
\index{Bedürfnisse}

\subsection{Was sind universelle Bedürfnisse?}

Bedürfnisse sind das, was wir erfüllt brauchen, damit es uns gut geht.

Ein universelles Bedürfnis ist eins, das jeder Mensch kennt~-- auch wenn sich Menschen darin unterscheiden, welche Bedürfnisse sie wie stark erfüllt brauchen.

Echte Bedürfnisse sind nicht an eine konkrete Person gebunden. Es gibt aber durchaus Bedürfnisse, die wir nur mit anderen Menschen zusammen erfüllen können, zum Beispiel unser Bedürfnis nach Gemeinschaft.

Ein Bedürfnis ist nicht an eine konkrete Handlung gebunden. Für jedes Bedürfnis gibt es viele verschiedene Strategien, um sie zu erfüllen~-- und wenn euch nur eine einzige Strategie dafür einfällt, dann habt ihr das Bedürfnis noch nicht genug verstanden.

\subsection{Liste von Bedürfnissen}
\label{beduerfnisse-liste}

Das ursprüngliche Vokabular stammt von Marshall Rosenberg aus \cite[S.~216~f]{gfk-rosenberg} bzw.~im englischsprachigen Original \cite[S.~210]{nvc-rosenberg}. Das erweiterte Vokabular kommt aus~\cite[S.~75~f]{gfk-dummies}.


\subsubsection{Autonomie}

\begin{multicols}{2}
  \begin{itemize}
    \item Freiheit
    \item Pläne für die Erfüllung der eigenen Träume, Ziele und Werte entwickeln
    \item Selbstbestimmung
    \item die eigenen Träume, Ziele und Werte wählen
  \end{itemize}
\end{multicols}


\subsubsection{Körperliche Bedürfnisse}

\begin{multicols}{2}
  \begin{itemize}
    \item Bewegung
    \item Distanz
    \item Gesundheit
    \item Heilung
    \item Kraft
    \item Körperkontakt
    \item Lebenserhaltung
    \item Luft
    \item Nahrung, Essen
    \item Schlaf
    \item Schutz (körperlich)
    \item Sexualität
    \item Sicherheit (körperlich)
    \item Unterkunft
    \item Wasser, Trinken
    \item Wärme
  \end{itemize}
\end{multicols}


\subsubsection{Stimmigkeit mit sich selbst}

\begin{multicols}{2}
  \begin{itemize}
    \item Authentizität
    \item Balance (von Geben und Nehmen)
    \item Eindeutigkeit
    \item Einklang
    \item Identität
    \item Individualität
    \item Integrität (im Einklang mit den eigenen Werten sein)
    \item Selbstwert
    \item Übereinstimmung mit den eigenen Werten
  \end{itemize}
\end{multicols}


\subsubsection{Einfühlung}

\begin{multicols}{2}
  \begin{itemize}
    \item Empathie (bekommen)
    \item Gerechtigkeit
    \item Gleichbehandlung
    \item verstanden/gesehen werden
  \end{itemize}
\end{multicols}


\subsubsection{Interaktion mit anderen Menschen}

\begin{multicols}{2}
  \begin{itemize}
    \item Achtsamkeit
    \item Akzeptanz
    \item Anerkennung
    \item Aufmerksamkeit
    \item Austausch
    \item Ehrlichkeit/Aufrichtigkeit
    \item Freundschaft
    \item Frieden (mit anderen Menschen)
    \item Geborgenheit
    \item Gemeinschaft
    \item Intimität
    \item Kontakt
    \item Liebe (erfahren)
    \item Nähe
    \item Offenheit
    \item Respekt
    \item Rücksichtnahme
    \item Schutz
    \item Sexualität
    \item Sicherheit
    \item Toleranz
    \item Unterstützung
    \item Vertrauen
    \item Wertschätzung
    \item Zugehörigkeit
    \item Zusammenarbeit
    \item Zärtlichkeit
  \end{itemize}
\end{multicols}


\subsubsection{Entspannung}

\begin{multicols}{2}
  \begin{itemize}
    \item Ausruhen
    \item Erholung
    \item Leichtigkeit
    \item Ruhe
    \item Spaß
    \item Spiel
  \end{itemize}
\end{multicols}


\subsubsection{Geistige Bedürfnisse}

\begin{multicols}{2}
  \begin{itemize}
    \item (innerer) Friede
    \item Abwechslung
    \item Ausgewogenheit
    \item Freude
    \item Glück
    \item Harmonie
    \item Humor
    \item Inspiration
    \item Klarheit
    \item Schönheit
    \item Spiel, Spielen
    \item \glqq Ordnung\grqq{} (im Sinne von Struktur, Klarheit)
    \item Ästhetik
  \end{itemize}
\end{multicols}


\subsubsection{Entwicklung}

\begin{multicols}{2}
  \begin{itemize}
    \item Anerkennung
    \item Bedeutung
    \item Beitragen
    \item Beteiligung
    \item Bildung
    \item Effektivität
    \item Engagement
    \item Erfolg (im Sinne von \glqq Gelingen\grqq)
    \item Feedback
    \item Feiern (von Gelungenem)
    \item Kompetenz
    \item Kreativität
    \item Lernen
    \item Rückmeldung
    \item Sinn
    \item Trauern (über Verluste oder wegen eines Scheiterns)
    \item Wachstum
    \item Wirksamkeit
  \end{itemize}
\end{multicols}

\section{Bitten}


\subsection{Kriterien für gute Bitten}

Diese Kriterien könnt ihr im Detail in \cite[S. 85f]{gfk-dummies} nachlesen

\paragraph{Konkret:} Das Verhalten sollte realistisch und überprüfbar sein.

\paragraph{Machbar:} für die andere Person

\paragraph{Positiv formuliert:} Sagt, was ihr braucht, anstatt, was ihr nicht haben wollt.

\paragraph{Im Hier und Jetzt erfüllbar:} Das schließt auch Vereinbarungen mit Wirkung auf die Zukunft ein.

\paragraph{Freiwillig:} Was passiert, wenn die andere Person Nein sagt?


\subsection{Arten von Bitten}

\paragraph{Handlungsbitte:} Könntest du bitte …?

\paragraph{Bitte um aufrichtige Rückmeldung:} Wie geht es dir damit? Was siehst du das?

\paragraph{Bitte um Empathie:} Ich würde gerne verstehen, was du verstanden hast.


\subsection{An wen kann ich eine Bitte richten?}

\begin{itemize}
  \item an mein Gegenüber
  \item an mich selbst
  \item an eine dritte Person
\end{itemize}



\chapter{Führung}
\section{Die Rolle der Führung}
\label{fuehrung-rolle}
\index{Führung: Rolle}


\subsection{Aufgaben der Führung nach Neuberger}

Laut Neuberger\cite{neuberger-fuehren} sind die Aufgaben der Führung,

\begin{itemize}
  \item andere Menschen
  \item zielgerichtet
  \item in einer formalen Organisation
  \item unter konkreten Umweltbedingungen dazu bewegen,
  \item Aufgaben zu übernehmen und erfolgreich auszuführen,
  \item wobei humane Ansprüche gewahrt werden.
\end{itemize}


\subsection{Neuberger, aber modernisiert}

Auf das moderne Arbeiten übertragen, wäre die Aufgabe der Führung,

\begin{itemize}
  \item eine Umgebung zu schaffen,
  \item die es einem Team oder einer Organisation möglich und leicht macht,
  \item für die Mission des Teams oder der Organisation zu arbeiten,
  \item wobei die Menschen nachhaltig körperlich und seelisch gesund zu bleiben
  \item und ihr Potenzial nutzen können.
\end{itemize}


\subsection{Aufgaben der Führung nach Malik}

Dies sind laut Fredmund Malik \cite{malik-fuehrung} die Aufgaben der Führung:

\begin{itemize}
  \item für Ziele sorgen
  \item organisieren
  \item entscheiden
  \item kontrollieren
  \item Menschen entwickeln und fördern
\end{itemize}

Auf moderne Führung übertragen, wäre es die Aufgabe der Führung, dafür zu sorgen, dass diese Dinge \emph{stattfinden} (also dass beispielsweise das Team Entscheidungen fällen und diese nachhalten kann), und nicht zwangsläufig, dass die Führung das auch selbst entscheidet.


\subsection{Was ergibt sich daraus?}
\index{Beziehungen}

Laut dem Podcast \emph{Manager Tools Basics} \cite{manager-tools-basics} ist eine der wichtigsten Verantwortung der Führung, \fett{gute Beziehungen} zu den geführten Personen \fett{aufzubauen und zu pflegen}.

Laut Amy Edmondson \cite{the-fearless-organisation} ist es die Hauptaufgabe der Führung, im Team bzw.~in der Organisation \fett{psychologische Sicherheit zu schaffen}.
\index{psychologische Sicherheit}

\section{Grundsätze der Führung nach Malik}
\label{fuehrung-grundsätze}
\index{Führung: Grundsätze}


Fredmund Malik \cite{malik-fuehrung} hat in seinen Büchern die Rolle und die Grundsätze von Führung beschrieben.

\begin{itemize}
  \item Ergebnisorientierung
  \item Beitrag zum Ganzen
  \item Konzentration auf weniges
  \item Stärken nutzen
  \item gegenseitiges Vertrauen
  \item positiv denken
\end{itemize}

\section{Konfliktmanagement aus Perspektive der Führung}
\label{konflikte-fuehrung}
\index{Führung}

Teil eurer Verantwortung (siehe S.~\pageref{fuehrung-aufgaben-neuberger}) in der Führung sind unter anderem diese Dinge:

\begin{itemize}
  \item eine Umgebung schaffen, in der Menschen produktiv sein können, und in der sie auch seelisch gesund bleiben
  \item gute Beziehungen zu euren Teammitgliedern aufbauen und pflegen
  \item psychologische Sicherheit schaffen
\end{itemize}

Konkret folgen daraus für euch diese Verantwortlichkeiten:

\begin{itemize}
  \item Konflikte zwischen euch und anderen Personen ansprechen und konstruktiv lösen, um eure Beziehung zu pflegen
  \item eurem Team ein Vorbild dafür sein, wie ihr mit Konflikten umgeht
  \item eure Teammitglieder dabei unterstützen, ihre Konflikte zu lösen
  \item einfordern, dass Teammitglieder die Konflikte lösen, die ihre Arbeit behindern
  \item es nicht akzeptieren, wenn Leute ihre Konflikte nicht ansprechen, nicht lösen oder nicht konstruktiv lösen
\end{itemize}

\section{Führen lernen}
\label{fuehren-lernen}
\label{fuehrung-lernen}
\index{Führung lernen}

Meiner Ansicht nach ist Führen zu lernen so ähnlich wie singen zu lernen.

Dafür sind diese Dinge notwendig:

\begin{itemize}
  \item viel \fett{üben} (und dabei aus Fehlern lernen)
  \item sehr viel \fett{Reflexion} \index{Reflexion}
  \item \fett{Außenwahrnehmung} bekommen in der Form von Feedback \index{Außenwahrnehmung} \index{Feedback}
  \item an \fett{Trainings} und \fett{Workshops} teilnehmen (oder anderweitig Unterricht nehmen)
  \item \fett{Bücher} oder anderen Quellen von Wissen konsumieren
  \item von \fett{guten Beispielen} lernen
\end{itemize}

Damit ihr andere Menschen gut führen könnt, ist es außerdem notwendig, dass ihr euch selbst gut kennt und versteht, wie ihr tickt und was euch antreibt. Dies könnt ihr durch diese Dinge (oder eine Kombination daraus) erreichen:

\begin{itemize}
  \item Gewaltfreie Kommunikation lernen \index{Gewaltfreie Kommunikation}
  \item eine Psychotherapie machen (Tiefenpsychologie oder Psychoanalyse; keine kognitive Verhaltenstherapie) \index{Therapie} \index{Psychotherapie}
\end{itemize}

Hilfreich zum kontinuierlichen Lernen ist außerdem eine Supervision, Intervision oder kollegiale Fallberatung.


\backmatter

\bibliography{../shared/bibliography/literatur}

\chapter{Lizenz}

\section*{Unter welchen Bedingungen könnt ihr dieses Handout benutzen?}
Dieses Handout ist unter einer \emph{Creative-Commons}-Lizenz lizensiert. Dies ist die \emph{Namensnennung-Share Alike 4.0 international (CC BY-SA 4.0)}\footnote{Die ausführliche Version dieser Lizenz findet ihr unter \url{https://creativecommons.org/licenses/by-sa/4.0/deed.de}.}. Das bedeutet, dass ihr dieses Handout unter diesen Bedingungen für euch kostenlos verbreiten, bearbeiten und nutzen könnt (auch kommerziell):

\begin{description}
  \item[Namensnennung.] Ihr müsst den Namen des Autors (Oliver Klee) nennen. Wenn ihr außerdem auch noch die Quelle\footnote{\url{https://github.com/oliverklee/workshop-handouts}} nennt, wäre das nett. Und wenn ihr mir zusätzlich eine Freude machen möchtet, sagt mir per E-Mail Bescheid.
  \item[Weitergabe unter gleichen Bedingungen.] Wenn ihr diesen Inhalt bearbeitet oder in anderer Weise umgestaltet, verändert oder als Grundlage für einen anderen Inhalt verwendet, dann dürft ihr den neu entstandenen Inhalt nur unter Verwendung identischer Lizenzbedingungen weitergeben.
  \item[Lizenz nennen.] Wenn ihr den Reader weiter verbreitet, müsst ihr dabei auch die Lizenzbedingungen nennen oder beifügen.
\end{description}


\printindex

\end{document}
