\section{Entschuldigungen}
\label{entschuldigungen}
\index{Entschuldigungen}

Die meisten Inhalte in diesem Abschnitt kommen aus dem Buch \emph{The 5 Apology Languages} \cite{apology-languages}, ergänzt durch Inhalt aus dem Podcast \emph{Familie verstehen} \cite{familie-verstehen-podcast} von Kathy Weber.


\subsection{Funktion von Entschuldigungen}

\begin{itemize}
  \item die gemeinsame Beziehung reparieren
  \item damit die Person sich besser fühlt, die sich entschuldigen möchte, indem sie sich dadurch eigene Bedürfnisse erfüllt, zum Beispiel:
    \begin{itemize}
      \item Verbindung
      \item Freundschaft
      \item Integrität (nach den eigenen Werten handeln)
      \item Zusammenarbeit
      \item Harmonie
    \end{itemize}
\end{itemize}


\subsection{Warum der Begriff \glqq Entschuldigung\grqq\ problematisch ist}

\begin{itemize}
  \item Es geht nicht darum, die schuldige Person zu finden und zu bestrafen oder zu beschämen, sondern darum, die Beziehung zu reparieren.
  \item Das Konzept von Schuld trennt Menschen, anstatt sie zu verbinden.
  \item Niemand ist gerne schuldig. Das macht es schwieriger als nötig, um Entschuldigung zu bitten.
  \item Niemand kann sich selbst entschuldigen. Ihr könnt nur die andere Person um Entschuldigung oder Verzeihung bitten, und diese kann dann die Entschuldigung annehmen (oder auch nicht).
\end{itemize}


\subsection{Elemente einer guten Entschuldigung}

Eine gut funktionierende Entschuldigung enthält einige dieser Elemente (wenn auch nicht notwendigerweise immer alle):

\paragraph{Zeigen, dass ihr wisst, was tatsächlich passiert ist.} Ansonsten kann eure Entschuldigung nicht glaubhaft sein. Oft wird aus der Situation auch schon deutlich, dass ihr das wisst~-- etwa, wenn ihr gerade jemanden auf den Fuß getreten seid.

\paragraph{Bedauern ausdrücken:} \glqq Es tut mir Leid.\grqq\ oder \glqq Ich bedauere total, dass \ldots\grqq\ Dadurch zeigt ihr, dass euch die andere Person und ihr Leid nicht gleichgültig sind.

\paragraph{Empathie ausdrücken:} \glqq Ich kann mit gut vorstellen, dass du gerade total sauer bist.\grqq\ oder \glqq Das war wahrscheinlich total unangenehm für dich.\grqq

\paragraph{Verantwortung für das eigene Handeln übernehmen:} \glqq Es war mein Fehler.\grqq\ oder \glqq Ich sehe, dass dadurch, dass ich \ldots\ getan habe, \ldots\ geschehen ist.\grqq

\paragraph{Wiedergutmachung:} \glqq Wie kann ich den Schaden reparieren?\grqq\ oder \glqq Was würde dir jetzt dazu helfen?\grqq

\paragraph{Die Wiederholungsgefahr reduzieren:} Bei Dingen, die theoretisch noch einmal geschehen könnten, könnt ihr erklären, was ihr tun werdet, damit das nicht noch einmal vorkommt.

\paragraph{Um Verzeihung bitten:} \glqq Ich bitte um Entschuldigung.\grqq\ oder \glqq Ist das so für dich okay?\grqq\ oder \glqq Mir ist es wichtig, dass es zwischen uns wieder okay ist. Ist jetzt noch etwas zwischen uns doof?\grqq


\subsection{Nicht-Entschuldigungen (Nonpologys)}
\index{Nicht-Entschuldigungen}
\index{Nonpologys}

Dies sind Aussagen, die wie Entschuldigungen klingen, aber tatsächlich keine Verantwortung für das Verhalten erkennen lassen.

Einige Merkmale und Muster:

\begin{itemize}
  \item Rechtfertigung und Erklärung des eigenen Verhaltens
  \item dem Gegenüber die Verantwortung geben (etwa für ein angebliches Missverständnis oder für dessen Gefühle)
  \item auf die Absicht fokussieren statt auf die Wirkung
  \item das eigene Verhalten oder die Auswirkung kleinreden
  \item auf andere Themen und Aspekte ablenken
  \item sich selbst als Opfer darstellen (beispielsweise der Berichterstattung)
  \item Gaslighting
\end{itemize}
