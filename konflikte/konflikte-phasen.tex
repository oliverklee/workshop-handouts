\section{Eskalationsstufen von Konflikten}
\label{eskalationsstufen}
\index{Eskalationsstufen}
\index{Phasen eines Konflikts}
\index{Konflikt: Phasen}

Friedrich Glas \cite{glasl-konfliktmanegement} teilt Konflikte in mehrere Eskalationsstufen oder Phasen ein:

\vspace{1em}

\begin{tabular}{ l l }
 \multirow{3}{5em}{win-win}   & 1.~Verhärtung \\
                              & 2.~Polarisierung \& Debatte \\
                              & 3.~Taten statt Worte! \\ \hline
 \multirow{3}{5em}{win-lose}  & 4.~Sorge um Image und Koalition \\
                              & 5.~Gesichtsverlust \\
                              & 6.~Drohstrategien \\ \hline
 \multirow{3}{5em}{lose-lose} & 7.~begrenzte Vernichtung \\
                              & 8.~Zersplitterung \\
                              & 9.~gemeinsam in den Abgrund \\
\end{tabular}


\subsection{1. Hauptphase (win-win)}

\subsubsection{Stufe 1~-- Verhärtung}

Konflikte beginnen mit Spannungen, z.\,B.~gelegentliches Aufeinanderprallen von Meinungen. Es ist alltäglich und wird nicht als Beginn eines Konflikts wahrgenommen. Wenn daraus doch ein Konflikt entsteht, werden die Meinungen fundamentaler. Der Konflikt könnte tiefere Ursachen haben.

\subsubsection{Stufe 2~-- Debatte, Polemik}

Ab hier überlegen sich die Konfliktparteien Strategien, um die anderen von ihren Argumenten zu überzeugen. Meinungsverschiedenheiten führen zu einem Streit. Man will die anderen unter Druck setzen. Schwarz-Weiß-Denken entsteht.

\subsubsection{Stufe 3~-- Taten statt Worte}

Die Konfliktparteien erhöhen den Druck auf die jeweils anderen, um sich oder die eigene Meinung durchzusetzen. Gespräche werden z.~,B.~abgebrochen. Es findet keine verbale Kommunikation mehr statt, und der Konflikt verschärft sich schneller. Das Mitgefühl für die \glqq anderen\grqq{} geht verloren.


\subsection{2.~Hauptphase (win-lose)}

\subsubsection{Stufe 4~-- Koalitionen, Images}

Der Konflikt verschärft sich dadurch, dass man Sympathisant\_innen für seine Sache sucht. Da man sich im Recht glaubt, kann man die Gegenseite denunzieren. Es geht nicht mehr um die Sache, sondern darum, den Konflikt zu gewinnen, damit die Gegenseite verliert.

\subsubsection{Stufe 5~-- Gesichtsverlust}

Die Gegenseite soll in ihrer Identität vernichtet werden, zum Beispiel durch Unterstellungen. Hier ist der Vertrauensverlust vollständig. Gesichtsverlust bedeutet in diesem Sinne Verlust der moralischen Glaubwürdigkeit.

\subsubsection{Stufe 6~-- Drohstrategien}

Mit Drohungen versuchen die Konfliktparteien, die Situation absolut zu kontrollieren. Sie sollen die eigene Macht veranschaulichen. Man droht z.\,B.~mit einer Forderung (\glqq 10 Millionen Euro\grqq), die durch eine Sanktion (\glqq Sonst sprenge ich Ihr Hauptgebäude in die Luft!\grqq) verschärft und durch das Sanktionspotenzial (Sprengstoff zeigen) untermauert wird. Hier entscheiden die Proportionen über die Glaubwürdigkeit der Drohung.


\subsection{3.~Hauptphase (lose-lose)}

\subsubsection{Stufe 7~-- Begrenzte Vernichtung(sschläge)}

Hier soll der Gegenseite mit allen Tricks empfindlich geschadet werden. Die Gegenseite wird nicht mehr als Mensch wahrgenommen. Ab hier wird ein begrenzter eigener Schaden schon als Gewinn angesehen, sollte der der Gegenseite größer sein.

\subsubsection{Stufe 8~-- Zersplitterung}

Das Unterstützungssystem der Gegenseite soll mit Vernichtungsaktionen zerstört werden.

\subsubsection{Stufe 9~-- Gemeinsam in den Abgrund}

Ab hier kalkuliert man die eigene Vernichtung mit ein, um die Gegenseite zu besiegen.
