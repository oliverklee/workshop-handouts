\section{Difficult Conversations}

Die Inhalt aus diesem Kapitel stammen aus dem Buch~\cite{difficult-conversations}.


\subsection{Stärken dieser Methode}

Dieser Ansatz fokussiert sich vor allem darauf, sich selbst, die andere Person und das Problem gut zu verstehen. Der Teil zu Lösungen ist hingegen eher kurz. Dadurch ist dieser Ansatz vor allem bei Problemen hilfreich, bei denen die Beziehung beschädigt ist und wo es weniger ums Lösen eines sachlichen Problems geht.


\subsection{Die Schritte}

\begin{enumerate}
  \item Vorbereitung: die drei Gespräche für sich selbst durchgehen
  \item Ziel des Gespräch überlegen: entscheiden, ob ich das Problem tatsächlich ansprechen möchte.
  \item die andere Person ansprechen und \fett{mit der dritten Geschichte beginnen}
  \item die beiden Geschichten aus der Sicht beider Personen austauschen (inklusive der Gefühle)
  \item die Probleme gemeinsam lösen
\end{enumerate}


\subsection{Die drei Gespräche}

Jedes Konfliktgespräch besteht eigentlich aus drei unterschiedlichen Gespräche:

\begin{enumerate}
  \item das \glqq Was ist passiert?\grqq-Gespräch
  \item das Gefühle-Gespräch
  \item das Gespräch zur Identität
\end{enumerate}

\subsubsection{Das \glqq Was ist passiert?\grqq-Gespräch}

Hier geht es (in den allermeisten Fällen) überhaupt nicht darum, wer \glqq Recht hat\grqq.

Die unterschiedlichen Geschichten:

\begin{itemize}
  \item Die Geschichte aus Sicht von Person 1.
  \item Die Geschichte aus Sicht von Person 2.
  \item Die Geschichte aus Sicht einer neutralen dritten Person, die nicht in die Köpfe der beiden beteiligten Personen hineinschauen kann (\glqq die dritte Geschichte\grqq).
\end{itemize}

Fragen zu der jeweils eigenen Geschichte:

\begin{itemize}
  \item Was habe ich selbst getan?
  \item Was habe ich wahrgenommen (ohne Interpretation)?
  \item Was waren meine Absichten?
  \item Welche Auswirkungen hatte das Ganze auf mich?
  \item Wie habe ich selbst zu dem Problem beigetragen?
\end{itemize}


\subsubsection{Das Gefühle-Gespräch}

Fragen dazu:

\begin{itemize}
  \item Wie habe ich mich dabei gefühlt?
  \item Wie fühle ich mich jetzt?
\end{itemize}

\subsubsection{Das Gespräch zur Identität}

Fragen dazu:

\begin{itemize}
  \item Als was für ein Mensch sehe ich mich? Und was für ein Mensch will ich sein?
  \item Wie hat das, was passiert ist, meine Identität und mein Selbstbild gefährdet?
\end{itemize}


\subsection{Strategien zur Problemlösung}

\begin{itemize}
  \item Haben wir alle Informationen? Was fehlt?
  \item Welche Ideen für eine Lösung haben wir?
  \item Wie können wir eine Lösung testen, um zu schauen, ob sie etwas taugt?
  \item Was fehlt mir persönlich noch?
  \item Was bräuchte es, damit ich einer Lösung zustimmen kann und will?
  \item Was bräuchte es, damit du einer Lösung zustimmen kannst und willst?
  \item Was wäre dein Rat?
  \item Was wären klare Standards, um eine Lösungsidee zu bewerten?
  \item Falls wir keine Lösungsidee finden, mit der wir beide einverstanden sind: Was wären die Alternativen?
\end{itemize}

