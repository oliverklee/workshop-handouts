\section{Strategien zu Konfliktlösungen}
\label{konfliktloesungs-strategien}
\index{Konfliktlösungen}
\index{Lösungsstrategien}

Gerhard Schwarz~\cite{schwarz-konfliktmanagement} stellt folgende Strategien vor, mit Konflikten umzugehen (und sie optimalerweise zu lösen). Die Strategien sind grob geordnet von \glqq wenig bis sehr hilfreich\grqq.

\subsection{Flucht}
\index{Flucht}

Man geht sich aus dem Weg oder spricht den Konflikt nicht an. Das ist das ursprünglich-instinktive Verhalten von Menschen bei Konflikten.

\subsubsection{Vorteile}
\begin{itemize}
  \item schmerzlos
  \item kurzfristig gelöst
  \item energiesparend
\end{itemize}

\subsubsection{Nachteile}
\begin{itemize}
  \item Konflikt verschärft sich
  \item unbefriedigend
\end{itemize}


\subsection{Vernichtung}
\index{Vernichtung}

\subsubsection{Vorteile}
\begin{itemize}
  \item dauerhaft gelöst
\end{itemize}

\subsubsection{Nachteile}
\begin{itemize}
  \item nicht korrigierbar
  \item destruktiv
  \item kein Lernen, keine Weiterentwicklung
\end{itemize}


\subsection{Unterordnung, Unterwerfung}
\index{Unterordnung}
\index{Unterwerfung}

\subsubsection{Vorteile}
\begin{itemize}
  \item umkehrbar
\end{itemize}

\subsubsection{Nachteile}
\begin{itemize}
  \item die stärkere Person gewinnt und nicht die, die im Recht ist
\end{itemize}


\subsection{Delegation}
\index{Delegation}

\subsubsection{Vorteile}
\begin{itemize}
  \item endgültige Lösung
  \item für die Konfliktparteien direkt wenig Arbeit
\end{itemize}

\subsubsection{Nachteile}
\begin{itemize}
  \item wenig Identifikation mit dem Ergebnis
  \item Stille-Post-Effekt
  \item Perspektiven und Bedürfnisse können untergehen
  \item umständlich, langwierig
\end{itemize}


\subsection{Kompromiss}
\index{Kompromiss}

\subsubsection{Vorteile}
\begin{itemize}
  \item Alle können ihr Gesicht wahren.
  \item zumindest teilweise Zufriedenheit: Alle sind gleichermaßen zufrieden.
  \item geteilte Verantwortung
\end{itemize}

\subsubsection{Nachteile}
\begin{itemize}
  \item Teilverluste: Alle sind gleichermaßen unzufrieden.
  \item wenig Blick auf die tatsächlichen Bedürfnisse und Interessen
\end{itemize}


\subsection{Konsens, Konsent}
\index{Konsens}
\index{Konsent}

Die Website \emph{The Decider}\footnote{\url{https://thedecider.app/}} erklärt den Unterschied zwischen Konsens und Konsent (und einigen anderen Entscheidungsfindungsmodellen). Sie führt euch auch durch den Prozess, ein passendes Entscheidungsfindungsmodell für eine Situation zu finden.

\subsubsection{Vorteile}
\begin{itemize}
  \item hohe Akzeptanz für die Lösung
  \item dauerhafte Lösung
  \item Berücksichtigung der Bedürfnisse und Interessen
\end{itemize}

\subsubsection{Nachteile}
\begin{itemize}
  \item tendenziell langwierig und aufwändig
  \item erfordert mehr emotionale und kommunikative Kompetenz bei den Konfliktparteien
\end{itemize}
