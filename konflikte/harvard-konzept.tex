\section{Das Harvard-Konzept}
\label{harvard-konzept}
\index{Harvard-Konzept}

Das \emph{Harvard-Konzept}~\cite{harvard-konzept} ist eine Methode für Verhandlungen und zur Konfliktlösung. Es ist ein Teil des \emph{Harvard Negotiation Project} der Harvard Law School.


\subsection{Stärken dieser Methode}

Ziel der Methode ist eine konstruktive und friedliche Einigung mit einem Win-Win-Ergebnis. Es kam unter anderem beim Camp-David-Abkommen im Nahen Osten zum Einsatz.

Diese Methode ist besonders stark bei Konflikten, bei denen es um die Lösung eines sachlichen Problems geht. Bei Konflikten, bei denen es vor allem um die Beziehung und um verletzte Gefühle geht, ist diese Methode hingegen nicht so gut geeignet.


\subsection{Grundsätzlicher Ansatz}

Das Ziel vom Harvard-Konzept ist, dass wir davon wegkommen, um Positionen zu feilschen, und stattdessen echt Win-Win-Lösungen zu finden versuchen.


\subsection{Prinzipien}


\subsubsection{Menschen und Probleme getrennt voneinander behandeln}

\begin{itemize}
  \item Sachprobleme als Sachprobleme behandeln
  \item wertschätzend miteinander umgehen
  \item Probleme auf der Beziehungsebene explizit ansprechen und behandeln
\end{itemize}


\subsubsection{Auf Interessen konzentrieren statt auf Positionen}

\paragraph{Positionen} sind sehr konkret, zum Beispiel: \glqq Ich will diese Tafel Schokolade haben.\grqq Positionen sind den \emph{Strategien} aus der Gewaltfreien Kommunikation ähnlich.

\paragraph{Interessen} sind das, weswegen ich eine Position vertrete, zum Beispiel für die Schokolade: \glqq Ich habe Hunger.\grqq\ oder \glqq Ich habe Lust auf etwas zu naschen.\grqq Interessen sind den \emph{Bedürfnissen} aus der Gewaltfreien Kommunikation ähnlich, aber etwas weniger universell.

Wenn wir auf die Interessen schauen, ist es viel einfacher, eine Win-Win-Lösung zu finden, die zu den Interessen aller beteiligten Parteien passt.


\subsubsection{Win-Win-Lösungsideen entwickeln}

Lösungen funktionieren dann besonders gut, wenn sie die Interessen aller Parteien erfüllen.


\subsubsection{Neutrale Beurteilungskriterien vereinbaren}

Nachdem ihr die gegenseitigen Interessen herausgefunden (und aufgeschrieben) habt, könnt ihr schauen, welche weiteren Beurteilungskriterien ihr noch wichtig findet.

Dies könnten zum Beispiel sein:

\begin{itemize}
  \item Gleichbehandlung
  \item Kosten
  \item zeitlicher Rahmen
\end{itemize}


\subsection{Ja, aber \ldots}


\subsubsection{Die \glqq Beste Alternative\grqq\ überlegen}

Überlegt euch vorher, was eure \glqq Beste Alternative\grqq\ (der \glqq Plan B\grqq) wäre für den Fall, dass ihr miteinander zu keinem Ergebnis kommt.


\subsubsection{Wenn die anderen nicht mitspielen: Verhandlungs-Judo}

\begin{itemize}
  \item Warum ist dir das wichtig?
  \item Was würde passieren, wenn wir das genau so machen, wie du es vorschlägst?
  \item Was würdest du stattdessen vorschlagen?
  \item Bitte korrigiere mich, falls ich falsch liege.
\end{itemize}
