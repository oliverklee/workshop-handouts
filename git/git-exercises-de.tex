\documentclass[a4paper,12pt]{scrartcl}

%----------------------------------------------------------------------------------
% Pakete und Parameter
%----------------------------------------------------------------------------------

% Pound-Zeichen statt $ in Code-Kommentaren fixen
\usepackage[T1]{fontenc}

% Input-Encoding für UTF-8
\usepackage[utf8]{inputenc}

% Mehr von der Seitenbreite nutzen
\usepackage{a4wide}

% Grafikpaket
\usepackage{color}
\usepackage[pdftex]{graphicx}

% Absätze werden nicht eingezogen, sondern vertikal abgesetzt
\setlength{\parindent}{0mm}
\addtolength{\parskip}{0.4em}

% Palatino und Helvetica statt Computer Modern als Standard-Fonts
\usepackage{palatino}

% lesbare Verweise
\usepackage[pdftex,plainpages=false,pdfpagelabels]{hyperref}

% nette URLs
\usepackage{url}

% für Boxen etc.
\usepackage{framed}
\definecolor{shadecolor}{rgb}{0.8,0.8,0.8}

% Anführungszeichen sprachabhängig machen
\usepackage[babel]{csquotes}


%----------------------------------------------------------------------------------
% Seitenlayout
%----------------------------------------------------------------------------------

% zweispaltiges Layout möglich machen
\usepackage{multicol}

% keine Seitenzahlen
\pagenumbering{gobble}
\pagestyle{empty}

% Seitenlayout
\topmargin0mm
\footskip0mm

% Descriptions ohne Einzug
\renewenvironment{description}[1][0pt]
{\list{}{
    \labelwidth=0pt \leftmargin=#1
     \let\makelabel\descriptionlabel
  }
}
{\endlist}

\raggedbottom
\raggedright

\newcommand{\fett}[1]{\textsf{\textbf{#1}}}

\newcommand{\answerspace}{\vspace{2.8em}}

% Deutsch als Dokumentsprache
\usepackage{babel}

\selectlanguage{ngerman}

\hyphenation{
}


\begin{document}
\raggedbottom

\section{Übungen zu Git}

\subsection{Git-Installation}
\begin{enumerate}
  \item Installiert euch Git (siehe die README auf GitHub).
\end{enumerate}

\subsection{Bash/Linux-Kommandozeile}
\begin{enumerate}
  \item Lasst euch anzeigen, in welchem Verzeichnis ihr gerade seid.
  \item Lasst euch anzeigen, welche Verzeichnisse und Dateien es im aktuellen Verzeichnis gibt.
  \item Wechselt in ein Unterverzeichnis.
  \item Lasst euch wieder anzeigen, in welchem Verzeichnis ihr gerade seid.
  \item Wechselt wieder ein Verzeichnis nach oben.
  \item Wechselt mit \texttt{cd -} wieder in das Unterverzeichnis und wieder zurück.
  \item Wechselt in das Verzeichnis, in dem ihr die Übungen für heute machen möchtet, und in dem ihr Dateien anlegen möchtet.
  \item Legt eine leere Datei an (\texttt{touch}).
  \item Schaut, dass die Datei jetzt tatsächlich existiert.
  \item Benennt die Datei um.
  \item Schaut, dass die Datei jetzt anders heißt.
  \item Löscht die Datei.
  \item Schaut, dass die Datei jetzt tatsächlich nicht mehr existiert.
\end{enumerate}

\subsection{Git-Konfiguration}
\begin{enumerate}
   \item Richtet euch Tab-Autovervollständigung für Git ein. Ihr könnt es testen, indem ihr \texttt{git check} tippt und dann Tab drückt. Wenn aus \texttt{check} ein \texttt{checkout} wird, tut's die Autovervollständigung.
  \item Konfiguriert Git anhand der Anleitung im README auf GitHub. Prüft die Einstellungen danach mit \texttt{git config $--$list}. Ersetzt dabei in den ersten beiden Zeilen die Platzhalter mit eurem echten vollen Namen und eurer Mailadresse.
\end{enumerate}


\subsection{Lokales Arbeiten}

\begin{enumerate}
  \item Lasst euch die Git-Hilfe anzeigen.
  \item Lasst euch die Git-Hilfe für \texttt{git init} anzeigen. Falls die Hilfe auf der Kommandozeile angezeigt wird, kommt ihr mit der \texttt{q}-Taste wieder raus.
  \item Legt das Verzeichnis an, mit dem ihr lokal üben möchtet, und wechselt in das Verzeichnis.
  \item Legt ein leeres Git-Repository an.
  \item Lasst euch den Status des Repositories anzeigen.
  \item Legt eine kleine Textdatei \texttt{hans.txt} an.
  \item Lasst euch den Status anzeigen.
  \item Fügt die Datei zur Staging-Area hinzu.
  \item Lasst euch den Status anzeigen.
  \item Lasst euch die Unterschiede zum letzten Commit anzeigen.
  \item Lasst euch die Unterschiede der Staging-Area zum letzten Commit anzeigen.
  \item Committet die Änderungen. Gebt dabei die Commit-Message direkt mit an.
  \item Lasst euch den Status anzeigen.
  \item Lasst euch das Log anzeigen.
  \item Lasst euch den letzten Commit anzeigen.
  \item Lasst euch die Liste der im letzen Commit geänderten Dateien anzeigen.
  \item Und jetzt noch einmal dasselbe mit der Commit-ID.
  \item Bearbeitet die existierende Datei \texttt{hans.txt} und legt außerdem eine zweite Datei \texttt{wurst.txt} neu an.
  \item Lasst euch die Unterschiede zum letzten Commit anzeigen. (Mit \texttt{q} kommt ihr da wieder raus.)
  \item Fügt die Datei zur Staging-Area hinzu.
  \item Lasst euch den Status anzeigen.
  \item Lasst euch die Unterschiede zum letzten Commit anzeigen.
  \item Lasst euch die Unterschiede der Staging-Area zum letzten Commit anzeigen.
  \item Commited die Änderungen, benutzt aber diesmal für die Commit-Message einen separaten Editor, indem ihr sie nicht direkt schon auf der Kommandozeile angebt.
  \item Lasst euch Status und Log anzeigen.
  \item Bearbeitet \texttt{hans.txt}. Commit die Datei, ohne vorher ein explizites \texttt{add} zu machen.
  \item Lasst euch das Log einzeilig anzeigen.
\end{enumerate}

\subsection{Schadensbegrenzung}
\begin{enumerate}
  \item Bearbeitet die \texttt{hans.txt} und fügt diese zum letzten Commit hinzu.
  \item Lasst euch das Log anzeigen.
  \item Bearbeitet \texttt{hans.txt} und setzt die Datei dann wieder die Version aus dem letzten Commit zurück.
  \item Setzt \texttt{hans.txt} auf die Version auf dem vorletzten Commit zurück (ohne zu commiten).
  \item Macht den kompletten letzten Commit rückgängig.
  \item Ändert eine Datei. Bringt die Änderungen mit dem Stash in Sicherheit.
  \item Holt die Änderungen wieder aus dem Stash.
  \item Setzt die Datei wieder auf den Ursprungszustand zurück.
\end{enumerate}

\subsection{Branches}
\begin{enumerate}
  \item Lasst euch anzeigen, welche Branches es gibt und auf welchem Branch ihr gerade seid.
  \item Legt einen Branch \texttt{pinguin} an.
  \item Wechselt auf den Branch.
  \item Legt in dem Branch eine Datei \texttt{watscheln.txt} an und committet sie.
  \item Schaut euch das Log an.
  \item Wechselt wieder auf den \texttt{master}-Branch und schaut euch wieder das Log an.
  \item Lasst euch den Unterschied zum \texttt{pinguin}-Branch anzeigen.
  \item Wechselt mit \texttt{git checkout -} zum \texttt{pinguin} und wieder zurück zum \texttt{master}.
  \item Merget den \texttt{pinguin}-Branch in den master und schaut euch das Log an.
  \item Löscht den \texttt{pinguin}-Branch.
  \item Legt mit einem Kommando einen Branch \texttt{nacktmull} an und wechselt direkt in den Branch.
  \item Legt eine Datei \texttt{frittiertes-mars.txt} an und committed sie.
  \item Wechselt wieder in den \texttt{master}.
  \item Löscht den {nacktmull}-Branch, ohne ihn zu mergen.
\end{enumerate}

\subsection{Rebase und Konflikte}
\begin{enumerate}
  \item Legt einen Branch \texttt{tee} an, wechselt in den Branch, legt dort eine Datei \texttt{earl grey.txt} (mit Leerzeichen!) mit ein paar Zeilen Text an und committed sie.
  \item Wechselt in den \texttt{master}, legt dort eine Datei {brot.txt} an und committed sie.
  \item Wechselt auf den \texttt{tee}-Branch und rebased von \texttt{master}.
  \item Wechselt in den \texttt{master}, bearbeitet \texttt{brot.txt} und committed die Änderungen.
  \item Wechselt nach \texttt{tee}, bearbeitet in \texttt{brot.txt} dieselben Zeilen und committed die Änderungen.
  \item Rebased von \texttt{master} und lasst euch den Status der Konflikte anzeigen.
  \item Brecht das Rebase ab.
  \item Rebased noch einmal. Löst diesmal die Konflikte im Editor, added die Datei und führt den Rebase dann zu Ende.
\end{enumerate}


\subsection{RSA-Keys und Accounts}
\begin{enumerate}
  \item Erzeugt euch ein RSA-Schlüsselpaar (falls noch nicht geschehen).
  \item Loggt euch bei GitHub ein und ladet euren öffentlichen RSA-Schlüssel hoch.
  \item Teilt Oli euren GitHub-Usernamen mit und bittet ihn, euch zum Übungs-Repository hinzuzufügen.
\end{enumerate}

\subsection{Verteiltes Arbeiten}
\begin{enumerate}
  \item Klont euch das Übungsrepository von GitHub. Achtet dabei darauf dass ihr es per SSH klont und nicht per HTTPS.
  \item Wechselt in das Verzeichnis des geklonten Repositories.
  \item Lasst euch die Remotes anzeigen.
  \item Und jetzt mit den URLs.
  \item Lasst euch die lokalen Branches anzeigen.
  \item Lasst euch die Remote-Branches anzeigen.
  \item Check den \texttt{gurkensalat}-Remote-Branch aus und wechselt direkt auf den Branch.
  \item Wartet, bis Oli etwas geändert und gepusht hat.
  \item Wechselt wieder auf den \texttt{master} und löscht den lokalen \texttt{gurkensalat}-Branch wieder.
  \item Pullt euch die Änderungen.
  \item Legt einen Branch an, legt dort einen Datei an und committed sie.
  \item Wartet, bis Oli etwas geändert und gepusht hat.
  \item Wechselt zum \texttt{master}, pullt, wechselt zu eurem Branch rebased, wechselt wieder zu \texttt{master}, merged euren Branch, pusht, und löscht euren Branch.
\end{enumerate}

\subsection{Noch mehr Schadensbegrenzung}
\begin{enumerate}
  \item Erzeugt einen Branch nicht von \texttt{master}, sondern vom vorletzten Commit von \texttt{master}. Löscht den Branch dann wieder.
  \item Schaut euch das Reflog an.
  \item Stellt den gelöschten Branch \texttt{nacktmull} wieder her.
  \item Wechselt auf den Branch.
  \item Vertauscht die letzten beiden Commits per interaktivem Rebase.
  \item Löscht den vorletzten Commit aus dem Branch (per interaktivem Rebase).
  \item Wechselt wieder auf den \texttt{master} und löscht den \texttt{nacktmull}-Branch.
\end{enumerate}


\end{document}
