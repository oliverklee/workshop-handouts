\section{GfK-Levels}
\label{gfk-levels}
\index{GfK-Levels}

Gewaltfreie Kommunikation könnt ihr auf verschiedenen Levels leben.

Voraussetzung dafür, dass ihr ein Level anfangt, ist, dass ihr das nächstkleinere Level grundlegend beherrscht und lebt.

\paragraph{Level 1: Selbstentwicklung:} Hier geht es darum, euch selbst besser zu verstehen und euch besser um euch selbst zu kümmern. Ziel dieses Level ist, eure eigenen Bedürfnisse zu erfüllen.

\paragraph{Level 2: In Beziehung zum Du:} Auf diesem Level gebt ihr einem Menschen Empathie und kommuniziert mit dem anderen Menschen darüber, was in euch lebendig ist. Ziel von diesem Level ist, die Bedürfnisse von euch selbst und eurem Gegenüber unter einen Hut zu bekommen. Das wäre beispielsweise in der Familie, in der Partnerschaft, oder in der Nachbarschaft.

\paragraph{Level 3: Innerhalb eines Teams oder einer Gruppe:} Hier geht es darum, die Bedürfnisse aller zu abzudecken, dass möglichst wenig Bedürfnisse verletzt werden. Das könnte auch die Gemeinde sein oder der Verein.

\paragraph{Level 4: Teamübergreifend, gruppenübergreifend:} Hier geht es darum, als Führungskraft oder mit Führungskräften zu arbeiten.

\paragraph{Level 4: Strukturelle Arbeit in Systemen:} Hier geht es um Organisationsentwicklung, politische Arbeit, Arbeit mit Vorständen oder in einem Parlament.
