\section{Bedürfnisse}
\label{beduerfnisse}
\index{Bedürfnisse}

\subsection{Was sind universelle Bedürfnisse?}

Bedürfnisse sind das, was wir erfüllt brauchen, damit es uns gut geht.

Ein universelles Bedürfnis ist eins, das jeder Mensch kennt~-- auch wenn sich Menschen darin unterscheiden, welche Bedürfnisse sie wie stark erfüllt brauchen.

Echte Bedürfnisse sind nicht an eine konkrete Person gebunden. Es gibt aber durchaus Bedürfnisse, die wir nur mit anderen Menschen zusammen erfüllen können, zum Beispiel unser Bedürfnis nach Gemeinschaft.

Ein Bedürfnis ist nicht an eine konkrete Handlung gebunden. Für jedes Bedürfnis gibt es viele verschiedene Strategien, um sie zu erfüllen~-- und wenn euch nur eine einzige Strategie dafür einfällt, dann habt ihr das Bedürfnis noch nicht genug verstanden.

\subsection{Liste von Bedürfnissen}
\label{beduerfnisse-liste}

Das ursprüngliche Vokabular stammt von Marshall Rosenberg aus \cite[S.~216~f]{gfk-rosenberg} bzw.~im englischsprachigen Original \cite[S.~210]{nvc-rosenberg}. Das erweiterte Vokabular kommt \cite[S.~75~f]{gfk-dummies}.


\subsubsection{Autonomie}

\begin{multicols}{2}
  \begin{itemize}
    \item Freiheit
    \item Selbstbestimmung
  \end{itemize}
\end{multicols}


\subsubsection{Körperliche Bedürfnisse}

\begin{multicols}{2}
  \begin{itemize}
    \item Luft
    \item Wasser
    \item Bewegung
    \item Nahrung
    \item Schlaf
    \item Distanz
    \item Unterkunft
    \item Wärme
    \item Gesundheit
    \item Heilung
    \item Kraft
    \item Lebenserhaltung
  \end{itemize}
\end{multicols}


\subsubsection{Integrität, Stimmigkeit mit sich selbst}

\begin{multicols}{2}
  \begin{itemize}
    \item Authentizität
    \item Einklang
    \item Eindeutigkeit
    \item Übereinstimmung mit den eigenen Werten
    \item Identität
    \item Individualität
  \end{itemize}
\end{multicols}


\subsubsection{Einfühlung}

\begin{multicols}{2}
  \begin{itemize}
    \item Empathie
    \item verstanden/gesehen werden
    \item Gleichbehandlung
    \item Gerechtigkeit
  \end{itemize}
\end{multicols}


\subsubsection{Verbindung}

\begin{multicols}{2}
  \begin{itemize}
    \item Wertschätzung
    \item Nähe
    \item Zugehörigkeit
    \item Liebe
    \item Intimität/Sexualität
    \item Unterstützung
    \item Ehrlichkeit/Aufrichtigkeit
    \item Gemeinschaft
    \item Geborgenheit
    \item Respekt
    \item Kontakt
    \item Akzeptanz
    \item Austausch
    \item Offenheit
    \item Vertrauen
    \item Anerkennung
    \item Freundschaft
    \item Achtsamkeit
    \item Aufmerksamkeit
    \item Toleranz
    \item Zusammenarbeit
  \end{itemize}
\end{multicols}


\subsubsection{Entspannung}

\begin{multicols}{2}
  \begin{itemize}
    \item Erholung
    \item Ausruhen
    \item Spiel
    \item Spaß
    \item Leichtigkeit
    \item Ruhe
  \end{itemize}
\end{multicols}


\subsubsection{Geistige Bedürfnisse}

\begin{multicols}{2}
  \begin{itemize}
    \item Harmonie
    \item Inspiration
    \item \glqq Ordnung\grqq
    \item (innerer) Friede
    \item Freude
    \item Humor
    \item Abwechslungsreichtum
    \item Ausgewogenheit
    \item Glück
    \item Ästhetik
  \end{itemize}
\end{multicols}


\subsubsection{Entwicklung}

\begin{multicols}{2}
  \begin{itemize}
    \item Beitragen
    \item Wachstum
    \item Anerkennung
    \item Feedback
    \item Rückmeldung
    \item Erfolg (im Sinne von \glqq Gelingen\grqq)
    \item Kreativität
    \item Sinn
    \item Bedeutung
    \item Effektivität
    \item Kompetenz
    \item Lernen
    \item Feiern
    \item Trauern
    \item Bildung
    \item Engagement
  \end{itemize}
\end{multicols}
