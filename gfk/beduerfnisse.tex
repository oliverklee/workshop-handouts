\section{Bedürfnisse}
\label{beduerfnisse}
\index{Bedürfnisse}

\subsection{Was sind universelle Bedürfnisse?}

Bedürfnisse sind das, was wir erfüllt brauchen, damit es uns gut geht.

Ein universelles Bedürfnis ist eins, das jeder Mensch kennt~-- auch wenn sich Menschen darin unterscheiden, welche Bedürfnisse sie wie stark erfüllt brauchen.

Echte Bedürfnisse sind nicht an eine konkrete Person gebunden. Es gibt aber durchaus Bedürfnisse, die wir nur mit anderen Menschen zusammen erfüllen können, zum Beispiel unser Bedürfnis nach Gemeinschaft.

Ein Bedürfnis ist nicht an eine konkrete Handlung gebunden. Für jedes Bedürfnis gibt es viele verschiedene Strategien, um sie zu erfüllen~-- und wenn euch nur eine einzige Strategie dafür einfällt, dann habt ihr das Bedürfnis noch nicht genug verstanden.

\subsection{Liste von Bedürfnissen}
\label{beduerfnisse-liste}

Das ursprüngliche Vokabular stammt von Marshall Rosenberg aus \cite[S.~216~f]{gfk-rosenberg} bzw.~im englischsprachigen Original \cite[S.~210]{nvc-rosenberg}. Das erweiterte Vokabular kommt aus~\cite[S.~75~f]{gfk-dummies}.


\subsubsection{Autonomie}

\begin{multicols}{2}
  \begin{itemize}
    \item Freiheit
    \item Pläne für die Erfüllung der eigenen Träume, Ziele und Werte entwickeln
    \item Selbstbestimmung
    \item die eigenen Träume, Ziele und Werte wählen
  \end{itemize}
\end{multicols}


\subsubsection{Körperliche Bedürfnisse}

\begin{multicols}{2}
  \begin{itemize}
    \item Bewegung
    \item Distanz
    \item Gesundheit
    \item Heilung
    \item Kraft
    \item Körperkontakt
    \item Lebenserhaltung
    \item Luft
    \item Nahrung, Essen
    \item Schlaf
    \item Schutz (körperlich)
    \item Sexualität
    \item Sicherheit (körperlich)
    \item Unterkunft
    \item Wasser, Trinken
    \item Wärme
  \end{itemize}
\end{multicols}


\subsubsection{Stimmigkeit mit sich selbst}

\begin{multicols}{2}
  \begin{itemize}
    \item Authentizität
    \item Balance (von Geben und Nehmen)
    \item Eindeutigkeit
    \item Einklang
    \item Identität
    \item Individualität
    \item Integrität (im Einklang mit den eigenen Werten sein)
    \item Selbstwert
    \item Übereinstimmung mit den eigenen Werten
  \end{itemize}
\end{multicols}


\subsubsection{Einfühlung}

\begin{multicols}{2}
  \begin{itemize}
    \item Empathie (bekommen)
    \item Gerechtigkeit
    \item Gleichbehandlung
    \item verstanden/gesehen werden
  \end{itemize}
\end{multicols}


\subsubsection{Interaktion mit anderen Menschen}

\begin{multicols}{2}
  \begin{itemize}
    \item Achtsamkeit
    \item Akzeptanz
    \item Anerkennung
    \item Aufmerksamkeit
    \item Austausch
    \item Ehrlichkeit/Aufrichtigkeit
    \item Freundschaft
    \item Frieden (mit anderen Menschen)
    \item Geborgenheit
    \item Gemeinschaft
    \item Intimität
    \item Kontakt
    \item Liebe (erfahren)
    \item Nähe
    \item Offenheit
    \item Respekt
    \item Rücksichtnahme
    \item Schutz
    \item Sexualität
    \item Sicherheit
    \item Toleranz
    \item Unterstützung
    \item Vertrauen
    \item Wertschätzung
    \item Zugehörigkeit
    \item Zusammenarbeit
    \item Zärtlichkeit
  \end{itemize}
\end{multicols}


\subsubsection{Entspannung}

\begin{multicols}{2}
  \begin{itemize}
    \item Ausruhen
    \item Erholung
    \item Leichtigkeit
    \item Ruhe
    \item Spaß
    \item Spiel
  \end{itemize}
\end{multicols}


\subsubsection{Geistige Bedürfnisse}

\begin{multicols}{2}
  \begin{itemize}
    \item (innerer) Friede
    \item Abwechslung
    \item Ausgewogenheit
    \item Freude
    \item Glück
    \item Harmonie
    \item Humor
    \item Inspiration
    \item Klarheit
    \item Schönheit
    \item Spiel, Spielen
    \item \glqq Ordnung\grqq{} (im Sinne von Struktur, Klarheit)
    \item Ästhetik
  \end{itemize}
\end{multicols}


\subsubsection{Entwicklung}

\begin{multicols}{2}
  \begin{itemize}
    \item Anerkennung
    \item Bedeutung
    \item Beitragen
    \item Beteiligung
    \item Bildung
    \item Effektivität
    \item Engagement
    \item Erfolg (im Sinne von \glqq Gelingen\grqq)
    \item Feedback
    \item Feiern (von Gelungenem)
    \item Kompetenz
    \item Kreativität
    \item Lernen
    \item Rückmeldung
    \item Sinn
    \item Trauern (über Verluste oder wegen eines Scheiterns)
    \item Wachstum
    \item Wirksamkeit
  \end{itemize}
\end{multicols}
