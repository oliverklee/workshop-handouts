\section{Empathie}
\label{empathie}
\index{Empathie}

\subsection{Arten von Empathie}

\cite{shaw-empathie} unterscheidet drei Arten von Empathie:

\paragraph{Emotionale Empathie:} die Fähigkeit, das Gleiche zu empfinden wie andere Menschen

\paragraph{Kognitive Empathie:} die Fähigkeit, nicht nur Gefühle, sondern auch Gedanken und Absichten anderer Menschen zu verstehen und daraus korrekte Schlussfolgerungen zu ihrem Verhalten abzuleiten

\paragraph{Soziale Empathie:} die Fähigkeit, das Verhalten komplexer sozialer Systeme zu verstehen und vorherzusagen

In der Gewaltfreien Kommunikation benutzen wir den Begriff \emph{Empathie} im Sinne der \fett{kognitiven Empathie}.

\subsection{Empathie lernen}

Bewusste Empathie ist zu einem großen Teil gelernt (und lässt sich auch später im Leben noch durch Üben verbessern).


\subsection{Mitleid vs.~Empathie}
\label{mitgleid}
\index{Mitleid}

Bei der \emph{Empathie} liegt der Fokus darauf, was die andere Person fühlt und braucht. Das ist das, was wir in der GfK üblicherweise möchten.

\emph{Mitleid} hingegen bedeutet, dass wir mit der anderen Person leiden. Dabei verschiebt sich der Fokus von der anderen Person auf uns, was wir bei der Empathie in der GfK vermeiden möchten.


\subsection{Alexithymie}
\label{alexithymie}
\index{Alexithymie}
\index{Gefühlsblindheit}

Alexithymie, auch \emph{Gefühlsblindheit}, bezeichnet Einschränkungen bei der Fähigkeit, Emotionen wahrzunehmen, zu erkennen und zu beschreiben. Emotionen sind bei Betroffenen prinzipiell vorhanden, werden jedoch als rein körperliche Symptome interpretiert. Der Schweregrad kann von nur leichten Schwierigkeiten beim Erkennen bestimmter Emotionen bis hin zu vollkommener \glqq Gefühlsblindheit\grqq{} reichen.

Für die Allgemeinbevölkerung wird eine Prävalenz von etwa 10\,\% angenommen, wobei Männer etwas häufiger betroffen sind als Frauen. Besonders verbreitet ist Alexithymie bei Menschen auf dem Autismus-Spektrum, und zwar mit ca.~ 50\,\%.
