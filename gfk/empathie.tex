\section{Empathie}
\label{empathie}
\index{Empathie}

\subsection{Arten von Empathie}

\cite{shaw-empathie} und \cite[S.~100~f]{gfk-dummies} unterscheiden drei Arten von Empathie:


\subsubsection{Affektive Empathie}

Dies die Fähigkeit, das Gleiche zu empfinden wie andere Menschen.

Innerhalb der affektiven Empathie ist noch ein weitere Unterschied für die GfK relevant:

\paragraph{Emotionale Mimikry} (Spiegeln) hilft uns, eine starke Verbindung zu dem zu haben, was unser Gegenüber fühlt. Wir sind dabei ganz bei der anderen Person.

\paragraph{Unbewusste emotionale Ansteckung} verschiebt den Fokus von der anderen Person zu uns und schränkt uns in unseren Handlungsmöglichkeiten ein: Wenn es uns mit diesen Emotionen selbst schlecht geht, können wir für die andere Person nicht mehr gut da sein.


\subsubsection{Kognitive Empathie}

Dies ist die Fähigkeit, nicht nur Gefühle, sondern auch Gedanken und Absichten anderer Menschen zu verstehen~-- und daraus korrekte Schlussfolgerungen zu ihrem Verhalten abzuleiten. Dies entspricht im Wesentlichen der \emph{Theory of Mind}.\cite{theory-of-mind-schrepfer}


\subsubsection{Soziale Empathie}

Dies ist die Fähigkeit, das Verhalten komplexer sozialer Systeme zu verstehen und vorherzusagen. Dazu gehört auch, ungeschriebene soziale Regeln wahrzunehmen.


\subsection{Empathie in der GfK}

In der Gewaltfreien Kommunikation benutzen wir den Begriff \emph{Empathie} im Sinne einer Kombination aus \fett{emotionaler Mimikry} (aus der affektiven Empathie) und \fett{kognitiver Empathie}.


\subsection{Empathie lernen}

Bewusste Empathie wird zu einem großen Teil in der Kindheit und der Pubertät gelernt. Insbesondere die kognitive und die soziale Empathie lässt sich auch später im Leben noch durch Üben verbessern.

Besonders hilfreich beim Lernen von Empathie ist es, selbst Empathie zu erhalten. Romane zu lesen ist Studien zufolge auch hilfreich.


\subsection{Mitleid vs.~Empathie}
\label{mitleid}
\index{Mitleid}

Bei der \emph{Empathie} liegt der Fokus darauf, was die andere Person fühlt und braucht. Das ist das, was wir in der GfK üblicherweise möchten.

\emph{Mitleid} hingegen bedeutet, dass wir mit der anderen Person leiden. Dabei verschiebt sich der Fokus von der anderen Person auf uns, was wir bei der Empathie in der GfK vermeiden möchten. Die entspricht der \emph{emotionalen Ansteckung}.


\subsection{Verstehen, Nachvollziehen, Zustimmen}
\index{verstehen}
\index{nachvollziehen}
\index{zustimmen}

\emph{Verstehen},\emph{Nachvollziehen} und \emph{Zustimmen} sind nicht dasselbe. Das ist wichtig, damit ihr Empathie geben könnt.

\begin{enumerate}
  \item \fett{Verstehen} bedeutet, dass bei euch keine Fragen mehr offen sind~-- und dass die andere Person euch sagen würde, dass ihr sie richtig verstanden habt.
  \item \fett{Nachvollziehen} bedeutet, dass ihr versteht, wie die andere Person zu ihren Gefühlen gekommen ist oder warum sie mit ihren Handlungen versucht hat, sich ein Bedürfnis zu erfüllen. Es kann auch bedeuten, dass ihr das Gefühl oder das Bedürfnis auch kennt, das die andere Person hat(te). Dies ist nicht dasselbe wie \emph{Zustimmen}.
  \item \fett{Zustimmen} ist, dass ihr die Meinung oder Einschätzung der anderen Person teilt.
\end{enumerate}

Für Empathie benötigt ihr \fett{die ersten beiden Schritte}, aber nicht notwendigerweise den dritten.


\subsection{Empathie geben}
\label{empathie-geben}
\index{Empathie geben}

\emph{Empathie geben} ist eine bewusste Handlung in der GfK, die darauf aufbaut, dass ihr euch mit affektiver und kognitiver Empathie in euer Gegenüber einfühlt.

Dies sind einige Dinge, die ihr tun könnt, um Empathie zu geben:

\begin{itemize}
  \item gut zuhören
  \item nicken
  \item \glqq therapeutisches Grunzen\grqq
  \item paraphrasieren (das Gehörte mit eigenen Worten wiederholen) \index{paraphrasieren}
  \item generell \glqq empathisch spekulieren\grqq
  \item geschilderte neutrale Beobachtungen herausdestillieren und wiedergeben
  \item Gefühle vermuten:
    \begin{itemize}
      \item \glqq Warst du gerade so richtig wütend?\grqq
      \item \glqq Du warst da mega genervt, oder?\grqq
      \item \glqq Das klingt, als seist du total frustriert gewesen. Höre ich das richtig?\grqq
    \end{itemize}
  \item Bedürfnisse vermuten:
    \begin{itemize}
      \item \glqq Dir fehlte da Zuverlässigkeit, oder?\grqq
      \item \glqq Brauchtest du da mehr Nähe und Intimität mir deiner Partnerin?\grqq
      \item \glqq Wolltest du mehr Unterstützung von ihm?\grqq
    \end{itemize}
\end{itemize}



\subsection{Alexithymie}
\label{alexithymie}
\index{Alexithymie}
\index{Gefühlsblindheit}

Alexithymie, auch \emph{Gefühlsblindheit}, bezeichnet Einschränkungen bei der Fähigkeit, Emotionen wahrzunehmen, zu erkennen und zu beschreiben. Emotionen sind bei Betroffenen prinzipiell vorhanden, werden jedoch als rein körperliche Symptome interpretiert. Der Schweregrad kann von nur leichten Schwierigkeiten beim Erkennen bestimmter Emotionen bis hin zu vollkommener \glqq Gefühlsblindheit\grqq{} reichen.

Für die Allgemeinbevölkerung wird eine Prävalenz von etwa 10\,\% angenommen, wobei Männer etwas häufiger betroffen sind als Frauen. Besonders verbreitet ist Alexithymie bei Menschen auf dem Autismus-Spektrum, und zwar mit ca.~ 50\,\%.
