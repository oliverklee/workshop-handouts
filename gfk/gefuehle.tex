\section{Gefühle}
\label{gefuehle}
\index{Gefühle}
\index{Emotionen}

Das ursprüngliche Vokabular stammt von Marshall Rosenberg aus \cite[S.~216]{gfk-rosenberg} bzw.~im englischsprachigen Original \cite[S.~210]{nvc-rosenberg}. Das erweiterte Vokabular und die Kriterien hab ich aus \cite[S.~56~f]{gfk-dummies} übernommen.


\subsection{Echte Gefühle}

Woran man echte Gefühle erkennt:

\begin{enumerate}
 \item Ein echtes Gefühl kann jeder Mensch auf der Welt empfinden~-- vom Kindergartenkind bis zum alten Menschen.
 \item Echte Gefühle sind körperlich spürbar.
 \item Echte Gefühle enthalten keine Schuldzuweisung. In ihnen gibt es keine Täter\_innen und keine Opfer. Sie können sich aber durchaus auf einen Menschen richten.
\end{enumerate}

\subsubsection{Primärgefühle und Sekundärgefühle}

\paragraph{Primärgefühle/Basisemotionen}
\index{Primärgefühle}
\index{Basisemotionen}

Primärgefühle sind die erste Reaktion des Körpers auf ein Ereignis und meist sehr stark. Sie sind ursprüngliche und instinktive Überlebensreaktionen.

Diese Emotionen dauern etwas 90 Sekunden, wenn wir sie nicht \glqq erneuern\grqq.

Laut Robert Plutchik\cite{plutchik-emotions} sind dies diese acht Basisemotionen:

\begin{itemize}
  \item Akzeptanz, Vertrauen
  \item Ekel, Abscheu
  \item Freude, Ekstase
  \item Furcht, Panik
  \item Neugierde, Erwartung
  \item Traurigkeit, Kummer
  \item Zorn, Wut
  \item Überraschung, Erstaunen
\end{itemize}

Paul Ekman\cite{ekman-emotions} hat sieben universelle Basisemotionen empirisch nachgewiesen:

\begin{itemize}
  \item Ekel
  \item Freude
  \item Furcht
  \item Traurigkeit
  \item Verachtung
  \item Wut
  \item Überraschung
\end{itemize}

\paragraph{Sekundärgefühle}
\index{Sekundärgefühle}

Sekundärgefühle sind eine Mischung aus Primärgefühlen und einer bestimmten Art zu denken. Sie entstehen daher etwas weniger unmittelbar als die Primärgefühle.

Wenn wir die Bedürfnisse erkennen, die hinter Sekundärgefühlen stehen, können wir oft die dahinter stehenden Primärgefühle sehen.

Theoretisch können alle Primärgefühle auch sekundär sein. Einige Sekundärgefühle sind jedoch typischer als andere:

\begin{itemize}
 \item Aggression
 \item Angst
 \item Hoffnungslosigkeit
 \item Reizbarkeit
 \item Wut
 \item depressive Verstimmung
 \item innere Leere
\end{itemize}

Wenn wir über Gefühle kommunizieren, ist die Unterscheidung zwischen primären und sekundären Gefühlen in der Praxis nicht besonders relevant.

\subsubsection{Angenehme Gefühle, wenn Bedürfnisse erfüllt sind}
\index{Bedürfnisse}
\label{gefuehle-liste}
\label{angenehme-gefuehle}

\begin{multicols}{2}
  \begin{itemize}
    \item amüsiert
    \item angeregt
    \item aufgedreht
    \item aufgeregt
    \item ausgeglichen
    \item befreit
    \item begeistert
    \item behaglich
    \item belebt
    \item berauscht
    \item beruhigt
    \item berührt
    \item beschwingt
    \item bewegt
    \item dankbar
    \item eifrig
    \item ekstatisch
    \item ekstatisch
    \item energetisiert
    \item engagiert
    \item enthusiastisch
    \item entlastet
    \item entschlossen
    \item entspannt
    \item entzückt
    \item erfreut
    \item erfrischt
    \item erfrischt
    \item erfüllt
    \item ergriffen
    \item erleichtert
    \item erstaunt
    \item erwartungsvoll
    \item fasziniert
    \item frei
    \item friedlich
    \item froh
    \item fröhlich
    \item gebannt
    \item geborgen
    \item gefesselt
    \item gelassen
    \item gerührt
    \item gesammelt
    \item gespannt
    \item gesund
    \item glücklich
    \item großartig
    \item gut gelaunt
    \item heiter
    \item hellwach
    \item hoffnungsvoll
    \item inspiriert
    \item klar
    \item kraftvoll
    \item kribbelig
    \item lebendig
    \item leicht
    \item liebevoll
    \item locker
    \item lustig
    \item motiviert
    \item munter
    \item mutig
    \item neugierig
    \item optimistisch
    \item ruhig
    \item sanft
    \item satt
    \item schwungvoll
    \item selbstsicher
    \item selig
    \item sicher
    \item sorglos
    \item still
    \item stolz
    \item unbekümmert
    \item unbeschwert
    \item vergnügt
    \item verliebt
    \item vertrauensvoll
    \item voller Liebe
    \item wach
    \item weit
    \item wissbegierig
    \item wissbegierig
    \item zufrieden
    \item zugeneigt
    \item zuversichtlich
    \item zärtlich
    \item überglücklich
    \item überrascht
    \item überwältigt
  \end{itemize}
\end{multicols}

\subsubsection{Unangenehme Gefühle, wenn Bedürfnisse nicht (genug) erfüllt sind}
\index{Bedürfnisse}
\label{unangenehme-gefuehle}

\begin{multicols}{2}
  \begin{itemize}
    \item alarmiert
    \item angeekelt
    \item angespannt
    \item apathisch
    \item aufgeregt
    \item ausgelaugt
    \item bedrückt
    \item beklommen
    \item besorgt
    \item bestürzt
    \item betroffen
    \item beunruhigt
    \item bitter
    \item blockiert
    \item dumpf
    \item durcheinander
    \item durstig
    \item eifersüchtig
    \item einsam
    \item elend
    \item empört
    \item entrüstet
    \item enttäuscht
    \item ernüchtert
    \item erschlagen
    \item erschrocken
    \item erschöpft
    \item erschüttert
    \item erstarrt
    \item frustriert
    \item furchtsam
    \item gehemmt
    \item geladen
    \item gelangweilt
    \item genervt
    \item hart
    \item hasserfüllt
    \item hilflos
    \item hungrig
    \item irritiert
    \item kalt
    \item klein
    \item kraftlos
    \item kribbelig
    \item lasch
    \item leer
    \item lethargisch
    \item lustlos
    \item matt
    \item miserabel
    \item mutlos
    \item müde
    \item nervös
    \item niedergeschlagen
    \item ohnmächtig
    \item panisch
    \item perplex
    \item pessimistisch
    \item ratlos
    \item resigniert
    \item ruhelos
    \item sauer
    \item scheu
    \item schlapp
    \item schuldig
    \item schwer
    \item schwermütig
    \item schüchtern
    \item sorgenvoll
    \item teilnahmslos
    \item todtraurig
    \item tot
    \item traurig
    \item träge
    \item unbehaglich
    \item ungeduldig
    \item unglücklich
    \item unruhig
    \item unsicher
    \item unter Druck
    \item unwohl
    \item unzufrieden
    \item verbittert
    \item verspannt
    \item verstört
    \item verwirrt
    \item verzweifelt
    \item voller Angst
    \item voller Scham
    \item voller Sorgen
    \item widerwillig
    \item wütend
    \item zappelig
    \item zittrig
    \item zornig
    \item ängstlich
    \item ärgerlich
    \item überwältigt
  \end{itemize}
\end{multicols}


\subsection{Gedanken (\glqq Pseudogefühle, Interpretationen\grqq)}
\label{pseudogefuehle}
\index{Gedanken}
\index{Pseudogefühle}
\index{Interpretationen}

\emph{Gedanken} sind Begriffe, die als Gefühlsäußerung angekündigt werden, aber statt dessen Vorwürfe, Schuldzuweisungen, Analysen oder Interpretationen enthalten. Diese werden in der GfK auch \emph{Pseudogefühle} oder \emph{Interpretationen} genannt.

\begin{multicols}{2}
  \begin{itemize}
    \item abgelehnt
    \item abgeschnitten
    \item akzeptiert
    \item allein gelassen
    \item an den Pranger gestellt
    \item an die Wand gestellt
    \item angegriffen
    \item attackiert
    \item ausgebeutet
    \item ausgenutzt
    \item ausgeschlossen
    \item ausgestoßen
    \item beachtet
    \item bedroht
    \item beleidigt
    \item belogen
    \item belästigt
    \item benutzt
    \item beschuldigt
    \item beschämt
    \item beschützt
    \item bestraft
    \item bestätigt
    \item betrogen
    \item bevormundet
    \item blamiert
    \item deplatziert
    \item diskriminiert
    \item dominiert
    \item entmutigt
    \item enttäuscht
    \item erdrückt
    \item erniedrigt
    \item ernst genommen
    \item festgenagelt
    \item gedrängt
    \item geehrt
    \item geliebt
    \item gemaßregelt
    \item gemobbt
    \item gequält
    \item geschmeichelt
    \item gesehen
    \item getäuscht
    \item gewürdigt
    \item gezwungen
    \item gut beraten
    \item herabgesetzt
    \item hereingelegt
    \item hintergangen
    \item ignoriert
    \item im Mittelpunkt
    \item in die Ecke gedrängt
    \item in die Enge getrieben
    \item isoliert
    \item klein gemacht
    \item lächerlich gemacht
    \item manipuliert
    \item minderwertig
    \item missachtet
    \item missbrauchst
    \item missverstanden
    \item nicht anerkannt
    \item nicht ehrlich behandelt
    \item nicht einbezogen
    \item nicht ernst genommen
    \item nicht geliebt
    \item nicht gesehen
    \item nicht respektiert
    \item nicht unterstützt
    \item nicht verstanden
    \item nicht wertgeschätzt
    \item provoziert
    \item reingelegt
    \item sabotiert
    \item schikaniert
    \item schlecht behandelt
    \item schön
    \item sympathisch
    \item totgequatscht
    \item unerwünscht
    \item ungehört
    \item ungeliebt
    \item ungerecht behandelt
    \item uninteressant
    \item unter Druck gesetzt
    \item unterbezahlt
    \item unterdrückt
    \item unverstanden
    \item unwichtig
    \item verarscht
    \item verfolgt
    \item vernachlässigt
    \item über den Tisch gezogen
    \item überfordert
    \item übergangen
    \item überlistet
  \end{itemize}
\end{multicols}
