\section{Gefühle}

Das ursprüngliche Vokabular stammt von Marshall Rosenberg aus \cite[S. 216]{gfk-rosenberg} bzw. im englischsprachigen Original \cite[S. 210]{nvc-rosenberg}. Das erweiterte Vokabular und die Kriterien kommen aus \cite[S. 56f]{gfk-dummies}.


\section{Primärgefühle ("`Echte Gefühle"')}

Woran man echte Gefühle erkennt:

\begin{enumerate}
 \item Ein echtes Gefühl kann jeder Mensch auf der Welt empfinden~-- vom Baby bis zum alten Menschen.
 \item Echte Gefühle sind körperlich spürbar.
 \item Echte Gefühle enthalten keine Schuldzuweisung. In ihnen gibt es keine Täter\_innen und keine Opfer.
\end{enumerate}

\subsection{Angenehme Gefühle, wenn Bedürfnisse erfüllt sind}

\begin{multicols}{2}
  \begin{itemize}
    \item angeregt
    \item aufgedreht
    \item aufgeregt
    \item ausgeglichen
    \item befreit
    \item begeistert
    \item behaglich
    \item belebt
    \item berauscht
    \item beruhigt
    \item berührt
    \item beschwingt
    \item bewegt
    \item dankbar
    \item eifrig
    \item ekstatisch
    \item energetisiert
    \item engagiert
    \item enthusiastisch
    \item entlastet
    \item entschlossen
    \item entspannt
    \item entzückt
    \item erfreut
    \item erfrischt
    \item erfüllt
    \item ergriffen
    \item erleichtert
    \item erstaunt
    \item erwartungsvoll
    \item fasziniert
    \item frei
    \item friedlich
    \item froh
    \item fröhlich
    \item gebannt
    \item geborgen
    \item gefesselt
    \item gelassen
    \item gerührt
    \item gesammelt
    \item gespannt
    \item gesund
    \item glücklich
    \item gut gelaunt
    \item heiter
    \item hellwach
    \item hoffnungsvoll
    \item inspiriert
    \item klar
    \item kraftvoll
    \item lebendig
    \item leicht
    \item locker
    \item lustig
    \item motiviert
    \item munter
    \item mutig
    \item neugierig
    \item optimistisch
    \item ruhig
    \item sanft
    \item satt
    \item schwungvoll
    \item selbstsicher
    \item selig
    \item sicher
    \item sorglos
    \item still
    \item stolz
    \item überglücklich
    \item überrascht
    \item überwältigt
    \item unbeschwert
    \item vergnügt
    \item verliebt
    \item vertrauensvoll
    \item wach
    \item weit
    \item wissbegierig
    \item zärtlich
    \item zufrieden
    \item zugeneigt
    \item zuversichtlich
  \end{itemize}
\end{multicols}

\subsection{Unangenehme Gefühle, wenn Bedürfnisse nicht (genug) erfüllt sind}
\begin{multicols}{2}
  \begin{itemize}
    \item alarmiert
    \item angespannt
    \item ängstlich
    \item apathisch
    \item ärgerlich
    \item aufgeregt
    \item ausgelaugt
    \item bedrückt
    \item besorgt
    \item bestürzt
    \item beunruhigt
    \item bitter
    \item blockiert
    \item deprimiert
    \item durcheinander
    \item eifersüchtig
    \item einsam
    \item elend
    \item empört
    \item enttäuscht
    \item ernüchtert
    \item erschlagen
    \item erschöpft
    \item erschrocken
    \item erschüttert
    \item erstarrt
    \item frustriert
    \item furchtsam
    \item gehemmt
    \item geladen
    \item gelähmt
    \item gelangweilt
    \item genervt
    \item hart
    \item hasserfüllt
    \item hilflos
    \item in Panik
    \item irritiert
    \item kalt
    \item kraftlos
    \item leer
    \item lethargisch
    \item matt
    \item miserabel
    \item müde
    \item mutlos
    \item nervös
    \item niedergeschlagen
    \item ohnmächtig
    \item panisch
    \item perplex
    \item ratlos
    \item resigniert
    \item ruhelos
    \item sauer
    \item scheu
    \item schlapp
    \item schüchtern
    \item schwer
    \item schwermütig
    \item sorgenvoll
    \item teilnahmslos
    \item tot
    \item träge
    \item traurig
    \item überwältigt
    \item unbehaglich
    \item ungeduldig
    \item unglücklich
    \item unruhig
    \item unsicher
    \item unter Druck
    \item unwohl
    \item unzufrieden
    \item verbittert
    \item verspannt
    \item verwirrt
    \item verzweifelt
    \item widerwillig
    \item wütend
    \item zappelig
    \item zornig
  \end{itemize}
\end{multicols}


\section{Gedanken (\glqq Pseudogefühle\grqq)}

\emph{Gedanken} sind Begriffe, die als Gefühlsäußerung angekündigt werden, aber statt dessen Vorwürfe, Schuldzuweisungen, Analysen oder Interpretationen enthalten.

\begin{multicols}{2}
  \begin{itemize}
    \item abgelehnt
    \item abgeschnitten
    \item akzeptiert
    \item allein gelassen
    \item an den Pranger gestellt
    \item an die Wand gestellt
    \item angegriffen
    \item attackiert
    \item ausgebeutet
    \item ausgenutzt
    \item ausgeschlossen
    \item ausgestoßen
    \item beachtet
    \item bedroht
    \item belästigt
    \item beleidigt
    \item belogen
    \item benutzt
    \item beschuldigt
    \item beschützt
    \item bestätigt
    \item bestraft
    \item betrogen
    \item bevormundet
    \item deplatziert
    \item diskriminiert
    \item dominiert
    \item entmutigt
    \item enttäuscht
    \item erdrückt
    \item erniedrigt
    \item ernst genommen
    \item festgenagelt
    \item gedrängt
    \item geehrt
    \item geliebt
    \item geliebt
    \item gemaßregelt
    \item gemobbt
    \item gequält
    \item geschmeichelt
    \item gesehen
    \item getäuscht
    \item gewürdigt
    \item gezwungen
    \item gut beraten
    \item herabgesetzt
    \item hereingelegt
    \item hintergangen
    \item ignoriert
    \item im Mittelpunkt
    \item in die Ecke gedrängt
    \item in die Enge getrieben
    \item isoliert
    \item klein gemacht
    \item lächerlich gemacht
    \item manipuliert
    \item minderwertig
    \item missachtet
    \item missbrauchst
    \item missverstanden
    \item nicht anerkannt
    \item nicht ehrlich behandelt
    \item nicht einbezogen
    \item nicht ernst genommen
    \item nicht geliebt
    \item ungerecht behandelt
    \item nicht gesehen
    \item nicht respektiert
    \item nicht unterstützt
    \item nicht verstanden
    \item nicht wertgeschätzt
    \item provoziert
    \item reingelegt
    \item sabotiert
    \item schikaniert
    \item schlecht behandelt
    \item schön
    \item sympathisch
    \item totgequatscht
    \item über den Tisch gezogen
    \item überfordert
    \item übergangen
    \item überlistet
    \item unerwünscht
    \item ungehört
    \item ungeliebt
    \item unter Druck gesetzt
    \item unterbezahlt
    \item unterdrückt
    \item unverstanden
    \item unwichtig
    \item verarscht
    \item verfolgt
  \end{itemize}
\end{multicols}
