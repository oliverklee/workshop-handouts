\section{Straßen-Giraffisch}
\label{strassengiraffisch}
\index{Straßen-Giraffisch}

Generell könnt ihr auch die Haltung und die Methoden der GfK mit Menschen benutzen, die (noch) kein GfK-Training haben. Ihr macht dann nur den größeren Teil der Arbeit, und ihr braucht etwas mehr Flexibilität bei den Begrifflichkeiten (\glqq Straßen-Giraffisch\grqq), weil die explizite Kommunikation mit den Begriffen \glqq Bedürfnis\grqq{} und \glqq Gefühl\grqq{} eher GfK-spezifisch ist.

Das funktioniert \emph{nicht}, wenn die andere Person gar nicht mit euch reden möchte oder überhaupt nicht dafür offen ist, Probleme konstruktiv zu lösen.


\subsection{Metakommunikation vorab}
\index{Metakommunikation}

Es kann hilfreich sein, wenn ihr ganz am Anfang des Gesprächs etwas Metakommunikation (siehe S.~\pageref{metakommunikation}) betreibt und euch die Zustimmung des Gegenübers einholt:

\begin{quote}
  Es ist mir wichtig, dass ich wirklich verstehe, was genau passiert ist, was dir wichtig ist und was du genau brauchst. Daher würde ich gerne genau nachfragen, damit ich sicher sein kann, dich richtig zu verstehen. Wäre das für dich okay, wenn wir das zusammen versuchen?
\end{quote}


\subsection{Beobachtungen}

\begin{itemize}
  \item Mir ist es wichtig, zuerst zu verstehen, was genau passiert ist.
  \item Was genau hast du gesehen?
  \item Was davon hast du beobachtet, und was hast du dir eher vermutet?
\end{itemize}


\subsection{Gefühle}

Hier helfen Umschreibungen der Gefühle.

\begin{itemize}
  \item Du klingst, als hättest du da echt einen Hals. Höre ich das richtig?
  \item Das nervt total, oder?
  \item Ich bin davon frustriert/verwirrt.
  \item Ich war echt erschrocken, als ich das hörte.
\end{itemize}


\subsection{Bedürfnisse}

\begin{itemize}
  \item Wofür ist dir das wichtig?
  \item Was fehlt dir dabei?
  \item Was bräuchtest du an der Stelle, damit es für dich besser funktioniert?
  \item Mir ist dabei \ldots{} wirklich wichtig.
  \item Ich brauche dabei mehr \ldots, damit es für mich funktioniert.
\end{itemize}

Das Wort \emph{Warum} würde ich dafür nicht empfehlen, weil es oft benutzt wird, um jemanden anzugreifen, und euer Gegenüber dann leicht dichtmacht.


\subsection{Bitten}

\begin{itemize}
  \item Was könnte ich möglicherweise tun, was dir für mehr \emph{(Bedürfnis)} hilfreich wäre?
  \item Was könnten wir zusammen tun?
  \item Ich glaube, das würde für mich nicht gut funktionieren, weil \emph{(verletztes Bedürfnis)}. Aber wäre vielleicht \emph{(andere Strategie)} für dich hilfreich?
  \item Wäre es für dich okay, wenn wir zusammen kurz brainstormen, wie wir für dich für mehr \emph{(Bedürfnis)} sorgen könnten?
  \item Wärst du bereit, \ldots? Es ist völlig okay, wenn du Nein sagst~-- Fragen kostet ja nichts.
  \item Wäre es für dich okay, \ldots?
\end{itemize}


\subsection{Straßen-Giraffisch für einige Bedürfnisse}
\index{Bedürfnisse}

Diese Formulierungen habe ich aus einem GfK-Adventskalender von Verena Ohn. \cite{verena-ohn}


\subsubsection{Authentizität}

\begin{itemize}
  \item Ich selbst sein können.
  \item Reden, wie mir der Schnabel gewachsen ist.
  \item Meine Gefühle zeigen und ausdrücken können, wie sie gerade sind.
\end{itemize}


\subsubsection{Autonomie}

\begin{itemize}
  \item Ich will es alleine schaffen und selbst entscheiden.
  \item Wenn ich über mich selbst bestimmen kann.
  \item Dass ich nicht immer jemanden (um Erlaubnis) fragen muss.
\end{itemize}


\subsubsection{Beteiligung}

\begin{itemize}
  \item Mitreden können, wenn es um wichtige Sachen geht.
  \item Etwas beitragen können, das zählt.
  \item Mitmachen, dabei sein, eingezogen werden.
\end{itemize}


\subsubsection{Disziplin}

\begin{itemize}
  \item Dranbleiben, auch wenn es anstrengend wird.
  \item Etwas üben, bis man es richtig gut kann.
  \item Dass sich alle an die Regeln halten, die ausgemacht wurden.
\end{itemize}


\subsubsection{Empathie}

\begin{itemize}
  \item Jemanden haben, der wirklich versteht, wie ich mich fühle.
  \item Nicht allein sein mit seinen Gefühlen, sondern jemanden haben, der mitfühlt.
  \item Dass jemand versteht, ohne dass man alles erklären muss.
\end{itemize}


\subsubsection{Fürsorge}

\begin{itemize}
  \item Sich um andere kümmern und für sie da sein, wenn sie jemanden brauchen.
  \item Dafür sorgen, dass es anderen gut geht.
  \item Für die Gesundheit und das seelische Wohl anderer Menschen sorgen.
\end{itemize}


\subsubsection{Gerechtigkeit}

\begin{itemize}
  \item Dass die Regeln für alle gelten und dass sich alle an die gleichen Absprachen halten müssen.
  \item Dass jeder und jeder das bekommt, was sie/er braucht.
  \item Dass jede und jeder mal bestimmen und entscheiden darf.
\end{itemize}


\subsubsection{Harmonie}

\begin{itemize}
  \item Dass sich niemand streitet und dass es ruhig und entspannt ist.
  \item Dass alle nett zueinander sind; dass sich alle lieb haben.
  \item Dass alle friedlich miteinander umgehen.
\end{itemize}


\subsubsection{Integrität}

\begin{itemize}
  \item Mir selbst treu bleiben.
  \item Für das einstehen, was mir wichtig ist, auch wenn es schwer ist.
  \item Das tun und zu dem stehen, was ich gesagt habe, und mich daran halten.
\end{itemize}


\subsubsection{Kreativität}

\begin{itemize}
  \item Mir eigene Ideen ausdenken und so malen, bauen oder basteln, wie ich es möchte.
  \item Nicht darüber nachdenken, wie und was ich mache, sondern einfach mal drauf los.
  \item Meinen Ideen freien Lauf lassen und auch Neues ausprobieren.
\end{itemize}


\subsubsection{Lebensfreude}

\begin{itemize}
  \item Spaß haben und lachen können, auch mal Quatsch machen und albern sein.
  \item Frei sein und das Leben genießen.
  \item Sachen tun, bei denen ich mich lebendig fühle.
\end{itemize}


\subsubsection{Mitbestimmung}

\begin{itemize}
  \item Mitreden können und Einfluss darauf haben, wenn es um Dinge geht, die mich betreffen.
  \item Gefragt werden, wie ich es gerne hätte und mitbestimmen dürfen.
  \item Dass ich gefragt werde, bevor jemand etwas für mich entscheidet.
\end{itemize}


\subsubsection{Nähe}

\begin{itemize}
  \item Kuscheln, in den Arm nehmen oder einfach nah sein.
  \item Dass ich dich sehen, hören, spüren und bei dir sein kann.
  \item Mit dir zusammen sein und dass du bei mir bleibst.
\end{itemize}


\subsubsection{Orientierung}

\begin{itemize}
  \item Dass jemand mir sagt, was ich tun soll, wenn ich unsicher bin.
  \item Einen Überblick haben, damit ich mich sicher fühle.
  \item Einen Plan haben, damit ich nicht durcheinander komme.
\end{itemize}


\subsubsection{Privatsphäre}

\begin{itemize}
  \item Dass ich auch mal allein sein kann und andere das respektieren.
  \item Einen Ort haben, an den ich mich zurückziehen und für mich sein kann.
  \item Dass nicht immer jemand fragt, was ich gerade mache.
\end{itemize}


\subsubsection{Rücksichtnahme}

\begin{itemize}
  \item Sehen, was ich gut kann, und stolz auf mich sein können.
  \item Dass ich merke, dass ich wichtig bin.
  \item Tief im Inneren wissen, dass ich genau richtig und gut bin, so wie ich bin.
\end{itemize}


\subsubsection{Selbstwert}

\begin{itemize}
  \item Sehen, was ich gut kann, und stolz auf mich sein können.
  \item Dass ich merke, dass ich wichtig bin.
  \item Tief im Inneren wissen, dass ich genau richtig und gut bin, so wie ich bin.
\end{itemize}


\subsubsection{Tatkraft}

\begin{itemize}
  \item Etwas motiviert anpacken, wenn ich Lust drauf habe.
  \item Wenn ich sofort loslegen will, weil ich gerade eine Idee habe.
  \item Die Energie, um etwas Neues zu probieren.
\end{itemize}


\subsubsection{Unterstützung}

\begin{itemize}
  \item Wenn mir jemand hilft, wenn ich etwas nicht allein schaffe.
  \item Dass jemand da ist, wenn ich nicht weiter weiß.
  \item Dass jemand mir Mut macht, wenn ich unsicher bin.
\end{itemize}


\subsubsection{Verständnis}

\begin{itemize}
  \item Jemand nimmt sich Zeit, um herauszufinden, was mit mir los ist.
  \item Dass keiner denkt, ich bin doof, nur weil ich etwas anders mache.
  \item Wenn jemand versucht, meine Sicht zu verstehen.
\end{itemize}


\subsubsection{Wachstum}

\begin{itemize}
  \item Wenn ich immer besser werde.
  \item Dass ich ausprobieren kann und herausfinde, was ich gut kann.
  \item Wenn ich Neues lerne.
\end{itemize}


\subsubsection{Verantwortung}

\begin{itemize}
  \item Mich selbst und ganz allein um etwas kümmern und es alleine regeln dürfen.
  \item Dass ich zeigen kann, dass ich für etwas sorgen kann.
  \item Dass ich für etwas ganz alleine zuständig bin.
\end{itemize}


\subsubsection{Wertschätzung}

\begin{itemize}
  \item Dass ich mir selbst sagen kann: ‚Das habe ich gut gemacht!‘
  \item Dass ich zufrieden mit mir selbst bin, egal was andere sagen.
  \item Mich freuen und stolz sein, wenn ich etwas geschafft habe.
\end{itemize}


\subsubsection{Zugehörigkeit}

\begin{itemize}
  \item die Sehnsucht, Teil der Gruppe zu sein
  \item der Wunsch dazu zu gehören und mitzumachen
  \item Freunde, die mit mir spielen wollen und gerne mit mir zusammen sind.
\end{itemize}
