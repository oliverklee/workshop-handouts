\section{Tagesrückblick}
\label{tagesrueckblick}
\index{Tagesrückblick}
\index{Rückblick}

GfK zu lernen ist wie eine Fremdsprache zu lernen: In der Hektik des Moments habt ihr die Vokabeln vielleicht nicht parat. Aber ihr könnt die Wörter später in Ruhe nachschauen, so dass ihr sie in der nächsten Situation besser parat habt (und die Sprache ein klein wenig besser sprecht).

Nehmt euch dafür am Abend ein paar Minuten Zeit, um \fett{eine Situation des Tages} besser zu verstehen, die euch bewegt hat.

\begin{itemize}
  \item Geht die Liste der \fett{Gefühle} auf Seite~\pageref{gefuehle-liste} durch: Welche Gefühle habt ihr gefühlt?
  \item Geht die Liste der \fett{Bedürfnisse} auf Seite~\pageref{beduerfnisse-liste} durch: Welche Bedürfnisse waren bei euch in der Situation gut erfüllt? Welche waren nicht gut erfüllt?
  \item Geht die Liste der \fett{Gedanken/Pseudogefühle} auf Seite~\pageref{pseudogefuehle} durch: Welche dieser Gedanken hattet ihr? Welche Gefühle und Bedürfnisse stecken eigentlich dahinter?
\end{itemize}

So lernt ihr Stück für Stück, eure Gefühle und Bedürfnisse zu erkennen und zu benennen, und ihr erweitert euer aktives Vokabular dafür.
