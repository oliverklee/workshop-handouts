\section{Was ist Gewaltfreie Kommunikation?}

Vielen Dank an den GfK-Trainer Jochen Hiester\footnote{\url{https://www.gewaltfrei-koblenz.de/}} aus Koblenz, der mich mit seinem tiefen Wissen zu GfK mit diesem Abschnitt unterstützt hat und geduldig meine vielen Fragen beantwortet hat.


\subsection{Bezeichnung \glqq Gewaltfreie Kommunikation\grqq}

Marshall Rosenberg hat selbst nie gewählt, dass sein Kommunikationsmodell Gewaltfreie Kommunikation heißen soll. Er lehrte eine rudimentäre Form der GFK ab den späten 70er Jahren in Seminaren in den USA, ab den 80er Jahren auch in anderen Ländern. Und er hatte jahrelang keinen Namen für sein Modell. In den ersten Jahren lehrte er sein Modell überwiegend vor Menschen, die auch politisch engagiert waren~-- z.\,B.~wegen Diskriminierung von Afroamerikaner\_innen, Gleichberechtigung von Frauen, Friedensbewegung, Umweltbewegung. In diesem Kreisen waren die Schriften von Mahatma Gandhi den meisten bekannt~-- und damit auch wie Gandhi den Begriff \glqq Gewaltfreiheit\grqq{} aufgefasst hatte. Und da Teilnehmende seiner Seminare immer wieder anmerkten, dass das Modell von Rosenberg beschreibt, wie man die Haltung der Gewaltfreiheit im Sinne von Gandhi erreichen kann, nannten die Teilnehmenden zunehmend sein Modell Gewaltfreie Kommunikation.

Rosenberg selbst hat eher den Begriff \glqq eine Sprache des Lebens\grqq{} für dein Modell benutzt.


\subsection{GfK als Haltung}
\label{gfk-haltung}
\index{Haltung}

GfK ist sowohl eine Haltung als auch eine Sammlung von Kommunikationstechniken, die helfen können, diese Haltung zu leben.

Die Haltung der GfK zu leben bedeutet, die Grundannahmen auf Seite~\pageref{gfk-annahmen} in die persönlichen Überzeugungen aufzunehmen und danach zu handeln. Dabei ist die Änderung der Haltung ein lebenslanger Lernprozess: Mal sind Menschen in dieser Haltung, mal nicht. Lernen heißt dabei auch, immer öfter in der Haltung zu sein und zu bleiben, auch wenn eine Person gerade getriggert ist.


\subsection{Ziele der GfK und der Weg dahin}
\label{gfk-ziele}
\index{Ziele}

Ziel der GfK ist, das Miteinander friedlicher zu machen.

Mehr Verbindung zu dem, was in einem selbst und in anderen lebendig ist (also vor allem Gefühle und Bedürfnisse) ist Teil des Weges, aber nicht das Ziel selbst.


\subsection{Definition von Gewalt in der GfK}
\label{gewalt-definition}
\index{Gewalt}

In der GfK ist Gewalt so definiert:

\begin{quote}
  Die eigenen Bedürfnisse versuchen zu erfüllen, wobei man die Bedürfnisse von anderen Personen mutwillig, bewusst oder absichtlich verletzt, soweit das vermeidbar wäre.
\end{quote}

Rosenberg selbst hat das nicht auf Menschen in der Zukunft (Klima) oder Tiere angewandt. Es würde aber funktionieren~-- dann wäre es entsprechend Gewalt gegen zukünftige Generationen oder gegen Tiere.


\subsection{Eigenverantwortung}
\label{gfk-eigenverantwortung}
\index{Eigenverantwortung}

Teil des Haltung der GfK ist auch, immer mehr Verantwortung für das eigene Handeln, die eigenen Worte und auch das eigene Wohlergehen zu übernehmen.

Die Annahme dahinter ist, dass jede Person ist für die eigene Bedürfnisse und Gefühle selbst verantwortlich ist. Und jede Person ist für die eigenen Worte und Handlungen verantwortlich. Eine Grauzone ist, wie weit man dann für die Folgen dieser Worte und Handlungen verantwortlich ist.


\subsection{Konflikte und GfK}
\label{gfk-konflikte}
\index{Konflikte}

GfK reduziert die Anzahl der Konflikte (oder Spannungen/Reibungen) nicht. Was sich reduziert, ist die Schärfe, wie Konflikte ausgetragen werden, wenn wenigstens eine der an einem Konflikt beteiligten Personen ausreichend GFK anwendet (vor allem mit sich selbst). GfK hilft aber, diese konstruktiv und friedlich zu lösen.

Tendenziell werdet ihr sogar \emph{mehr} Konflikte ansprechen und klären, wenn ihr GfK lernt.

