\documentclass[a4paper,openany,twoside,titlepage,10pt,headsepline]{scrbook}

%----------------------------------------------------------------------------------
% Typografie und Fonts
%----------------------------------------------------------------------------------

% Vektorfonts statt Pixelfonts
\usepackage[T1]{fontenc}
\usepackage{lmodern}

% Input-Encoding für UTF-8
\usepackage[utf8]{inputenc}

% Anderer Sans-serif-Font
% https://www.tug.org/FontCatalogue/allfonts.html
% https://www.draketo.de/anderes/latex-fonts.html
\usepackage{librefranklin}
\renewcommand{\familydefault}{\sfdefault}

\newcommand{\fett}[1]{\textsf{\textbf{#1}}}


%----------------------------------------------------------------------------------
% Mehrspachigkeit
%----------------------------------------------------------------------------------

% Mehrsprachigkeit für Deutsch und Englisch erlauben
\usepackage[ngerman,english]{babel}

% Anführungszeichen sprachabhängig machen
\usepackage[babel]{csquotes}


%----------------------------------------------------------------------------------
% Größen, Abstände und Einzüge
%----------------------------------------------------------------------------------

% Mehr von der Seite nutzen
\usepackage[top=2cm, bottom=3cm, right=3cm, left=2cm]{geometry}

% Absätze werden nicht eingezogen, sondern vertikal abgesetzt
\setlength{\parindent}{0mm}
\addtolength{\parskip}{0.5em}

% Descriptions ohne Einzug
\renewenvironment{description}[1][0pt]
{\list{}{
    \labelwidth=0pt \leftmargin=#1
    \let\makelabel\descriptionlabel
}
}
{\endlist}

% Im Zweifel die Seite nicht komplett füllen, aber keine zusätzlichen vertikalen Abstände hinzufügen
\raggedbottom

% Hurenkinder und Schusterjungen verhindern
\clubpenalty10000
\widowpenalty10000
\displaywidowpenalty=10000

%----------------------------------------------------------------------------------
% Literaturverzeichnis, Links und Querverweise
%----------------------------------------------------------------------------------

% Bibliographieeinstellungen
\bibliographystyle{alphadin}

% klickbare Verweise
\usepackage[pdftex,plainpages=false,pdfpagelabels]{hyperref}

% nette URLs
\usepackage{url}


%----------------------------------------------------------------------------------
% Dokument- und Seitenstruktur
%----------------------------------------------------------------------------------

% zweispaltiges Layout möglich machen
\usepackage{multicol}

% Seiten-Kopfzeilen und -Fußzeilen
\usepackage{scrlayer-scrpage}

% Maximal drei Ebenen nummerieren
\setcounter{secnumdepth}{2}

% Maximale 2 Ebenen im Inhaltsverzeichnis
\setcounter{tocdepth}{1}

% Mit jeder Section eine neue Seite anfangen
% https://tex.stackexchange.com/questions/9497/start-new-page-with-each-section
\usepackage{etoolbox}
\preto{\section}{%
  \ifnum\value{section}=0 \else\clearpage\fi
}


%----------------------------------------------------------------------------------
% Grafiken, Farben und Boxen
%----------------------------------------------------------------------------------

% Farben
\usepackage{xcolor}

% Grafiken
\usepackage[pdftex]{graphicx}

% Boxen inklusive Schattierung
\usepackage{framed}
\definecolor{shadecolor}{rgb}{0.8,0.8,0.8}


\AtBeginDocument{\selectlanguage{ngerman}}

\title{Gewaltfreie Kommunikation\\für Teams und Führungskräfte}
\author{Oliver Klee\\\texttt{www.oliverklee.de}\\\texttt{seminare@oliverklee.de}}
\date{Version vom \today}

\begin{document}

\frontmatter

\maketitle

\tableofcontents


\mainmatter

\chapter{Seminar-Handwerkszeug}
\section{Regeln für den Workshop}
\label{gfk-workshopregeln}
\index{Workshopregeln}

\paragraph{Vegas-Regel:} Was wir hier persönlichen Dingen teilen, bleibt im Workshop. Wir erzählen Dinge nur anonymisiert nach außen.

\paragraph{Keine dummen Fragen:} Es gibt keine dummen Fragen. Für Fragen, die nicht gut in den Rahmen des aktuellen Themas passen, haben wir einen Themenkühlschrank.

\paragraph{Joker-Regel:} Wir alle versuchen, uns auf dem Workshop gut um uns selbst zu kümmern. Wenn wir etwas brauchen, sprechen wir es an oder sorgen selbst dafür.

\paragraph{Aufrichtigkeit:} Wir tun unser Bestes, uns ehrlich und aufrichtig miteinander umzugehen.

\paragraph{Konstruktiv sein:} Wir tun unser Bestes, konstruktiv miteinander umzugehen und uns gut zu behandeln.

\section{Paarinterview zum Kennenlernen}
\label{paarinterview}

\begin{itemize}
 \item Wo und wie wohne ich?
 \item Was mache ich in Beruf und Ehrenamt so? Und was habe ich bisher so gemacht?
 \item Was sind ein paar Dinge, die mir im Leben zurzeit Freude bereiten?
 \item Was brauche ich (von anderen Personen oder der Umgebung), damit die Zusammenarbeit mit mir gut funktioniert?
 \item Was sollten andere Menschen über mich wissen, wenn sie mit mir zusammenarbeiten?
 \item Was mache ich, um trotz der aktuellen Krisen psychisch halbwegs gesund zu bleiben?
 \item Was ist ein \emph{Guilty Pleasure}, dem ich ab und an fröne?
\end{itemize}

\section{Feedback: Tipps und Tricks}
\label{feedback-regeln}
\index{Feedback}
\index{Feedbackregeln}

\subsection{Was ist Feedback?}
Feedback ist für euch eine Gelegenheit, in kurzer Zeit viel über euch selbst zu lernen. Feedback ist ein Anstoß, damit ihr danach an euch arbeiten könnt (wenn ihr wollt).

Feedback heißt, dass euch jemandem einen persönlichen, subjektiven Eindruck in Bezug auf konkrete Punkte mitteilt. Da es sich um einen persönlichen Eindruck im Kopf eines einzelnen Menschen handelt, sagt Feedback nichts darüber aus, wie ihr tatsächlich wart. Es bleibt allein euch selbst überlassen, das Feedback, das ihr bekommt, für euch selbst zu einem großen Gesamtbild zusammenzusetzen.

Es kann übrigens durchaus vorkommen, dass ihr zur selben Sache von verschiedenen Personen völlig unterschiedliches (oder gar gegensätzliches) Feedback bekommt.

Es geht beim Feedback \emph{nicht} darum, euch mitzuteilen, ob ihr ein guter oder schlechter Mensch, ein guter Redner, eine schlechte Rhetorikerin oder so seid. Solche Aussagen haben für euch keinen Lerneffekt. Stattdessen schrecken sie euch ab, Neues auszuprobieren und dabei auch einmal so genannte Fehler zu machen.

Insbesondere ist Feedback keine Grundsatzdiskussion, ob das eine oder andere Verhalten generell gut oder schlecht ist. Solche Diskussionen führt ihr besser am Abend bei einem Bierchen.

\subsection{Feedback geben}
\begin{itemize}
  \item  "`ich"' statt "`man"' oder "`wir"'
  \item die \emph{eigene} Meinung sagen
  \item die andere Person direkt ansprechen: "`du/Sie"' statt "`er/sie"'
  \item eine konkrete, spezifische Beobachtung schildern
  \item nicht verallgemeinern
  \item nicht analysieren oder psychologisieren (nicht: "`du machst das nur, weil \ldots"')
  \item Feedback möglichst unmittelbar danach geben
  \item konstruktiv: nur Dinge ansprechen, die die andere Person auch ändern kann
\end{itemize}

\subsection{Feedback entgegennehmen}
\begin{itemize}
  \item vorher den Rahmen für das Feedback abstecken: Inhalt, Vortragstechnik, Schriftbild \ldots
  \item gut zuhören und ausreden lassen
  \item sich nicht rechtfertigen, verteidigen oder entschuldigen
  \item Missverständnisse klären, Hintergründe erläutern
  \item Feedback als Chance zur Weiterentwicklung sehen
\end{itemize}

\section{Das Blitzlicht}

\begin{itemize}
  \item wird nicht visualisiert
  \item jede Person spricht nur für sich selbst
  \item keine Diskussion (Ausnahme: wichtige Verständnisfragen)
  \item nicht unterbrechen
  \item wer anfängt, fängt an
  \item kurz~-- ein Blitzlicht ist kein Flutlicht
\end{itemize}


\chapter{Kommunikation}
\section{Die Geschichte mit dem Hammer}
\label{hammer-geschichte}
\index{Geschichte mit dem Hammer}
\index{Hammer}

Diese Kurzgeschichte stammt aus dem Buch \emph{Anleitung zum Unglücklichsein}~\cite{anleitung-zum-ungluecklichsein} von Paul Watzlawick.

\begin{quote}

Ein Mann will ein Bild aufhängen. Den Nagel hat er, nicht aber den Hammer. Der Nachbar hat einen. Also beschließt unser Mann, hinüberzugehen und ihn auszuborgen.

Doch da kommt ihm ein Zweifel: Was, wenn der Nachbar mir den Hammer nicht leihen will? Gestern schon grüßte er mich nur so flüchtig. Vielleicht war er in Eile.

Aber vielleicht war die Eile nur vorgeschützt, und er hat etwas gegen mich. Und was? Ich habe ihm nichts angetan; der bildet sich da etwas ein. Wenn jemand von mir ein Werkzeug borgen wollte, ich gäbe es ihm sofort. Und warum er nicht? Wie kann man einem Mitmenschen einen so einfachen Gefallen abschlagen? Leute wie dieser Kerl vergiften einem das Leben. Und dann bildet er sich noch ein, ich sei auf ihn angewiesen. Bloß weil er einen Hammer hat. Jetzt reicht's mir wirklich.

Und so stürmt er hinüber, läutet, der Nachbar öffnet, doch noch bevor er \glqq Guten Tag\grqq{} sagen kann, schreit ihn unser Mann an: \glqq Behalten Sie sich Ihren Hammer, Sie Rüpel!\grqq

\end{quote}

\section{Prinzipien der Kommunikation}
\label{kommunikationsprinzipien}
\index{Kommunikation}
\index{Missverständnisse}
\index{Sender-Empfänger-Modell}

\begin{itemize}
  \item Es gibt bei Kommunikation immer sendende und (mindestens) eine empfangende Partei (Sender-Empfänger-Modell nach Shannon und Weaver). Diese Rollen können in einer Interaktion öfter wechseln.
  \item Die Partei, die etwas von der anderen will, hat die Zuständigkeit dafür dafür (und das Interesse daran), dass die Kommunikation erfolgreich ist.
  \item Jede Partei hat nur auf ihre eigene Hälfte der Kommunikation direkten Einfluss.
  \item Missverständnisse passieren, und sie sind eher die Regel denn die Ausnahme~-- wir bemerken sie nur oft nicht.
\end{itemize}


\subsection{Metakommunikation}
\label{metakommunikation}
\index{Metakommunikation}

\emph{Metakommunikation} (\glqq Kommunikation über Kommunikation\grqq) bedeutet, die Kommunikation auf eine höhere Ebene zu verlagern und darüber zu reden, wie wir miteinander reden, wie wir miteinander umgehen und was uns beschäftigt.

\section{Direkte vs.~indirekte Kommunikation}
\label{direkte-kommunikation}
\index{direkte Kommunikation}
\index{indirekte Kommunikation}

Bei \fett{direkter Kommunikation} sagt die Person das, was sie kommunizieren möchte, explizit mit ihren Worten.

Bei \fett{indirekter Kommunikation} benutzt die Person stattdessen Mehrdeutigkeit, Anspielungen, den Tonfall, den Rhythmus der Sprache, Gestik oder Mimik, Handlungen, Ironie oder Sarkasmus.

Direkte Kommunikation zu nutzen, senkt das Risiko für Missverständnisse deutlich. Außerdem können wir damit mehr Verantwortung für unsere eigene Kommunikation übernehmen.

GfK setzt sehr stark auf direkte Kommunikation.

\subsection{Beispiele}

\renewcommand{\arraystretch}{2.0}
\begin{tabular}{|p{20em}|p{20em}|}
\hline

\fett{direkte Kommunikation} & \fett{indirekte Kommunikation} \\
\hline

Ich würde gerne mit meinem Freund einen Abend zu zweit verbringen. Wärst du bereit, morgen Abend von 20 bis 23 Uhr die WG zu verlassen? &
Hättest du Lust, morgen Abend ohne mich ins Kino zu gehen? \\
\hline

Mir ist kalt. Wäre es okay, wenn ich das Fenster zumache? &
Ziemlich kalt hier. \\
\hline

Ich bin gerade echt genervt. &
(knurrt) \\
\hline

Könntest du die Musik vielleicht etwas leiser machen? &
Tolle Musik! \\
\hline

Ich habe im Moment echt wenig Geld. Zurzeit kaufen wir ein Kilo Kaffeebohnen im Monat. Können wir uns dazu mal zusammensetzen, wie wir unsere Ausgaben für Kaffee senken können? &
Du trinkst zu viel Kaffee. \\
\hline

Könntest du bitte jeden zweiten Tag den Müll runterbringen? &
(stellt dem Mitbewohner den vollen Mülleimer vor die Zimmertür) \\
\hline

Könntest du bitte damit aufhören, mit dem Kuli zu klicken? &
(knallt die Kaffeetasse auf den Schreibtisch) \\
\hline

\end{tabular}
\renewcommand{\arraystretch}{1.0}

\section{\emph{Ask-Culture} vs.~\emph{Guess-Culture}}
\label{ask-guess-culture}
\index{Ask-Culture}
\index{Guess-Culture}

\emph{Ask-Culture} vs.~\emph{Guess-Culture}\cite{ask-guess-culture} ist ein Kontinuum zwischen zwei Extremen:


\subsection{Annahmen der \emph{Ask-Culture}}

\begin{itemize}
  \item Ein Ja ist ein Ja, und ein Nein ist ein Nein.
  \item Wer Menschen um etwas bittet, hat die Verantwortung, möglicherweise ein Nein als Antwort zu erhalten und damit zurechtzukommen.
  \item Es ist völlig okay, Menschen um etwas zu bitten. Sie müssen ja nicht Ja sagen.
  \item Wenn mich jemand um etwas bittet, ist es meine Verantwortung, Nein zu sagen, wenn ich der Bitte nicht gerne nachkommen möchte.
\end{itemize}


\subsection{Annahmen der \emph{Guess-Culture}}

\begin{itemize}
  \item Wir bitten Menschen nur dann um etwas, wenn wir es für sehr wahrscheinlich halten, dass die Person Ja sagt.
  \item Wir streben an, dass wir nicht mehr um etwas zu bitten brauchen, sondern es uns die andere Person von sich aus anbietet.
  \item Bitten um Dinge, zu denen wir nicht gerne Ja sagen, sind unhöflich.
  \item Auf eine Bitte Nein zu sagen, ist unhöflich.
  \item Konflikte sind etwas, was unbedingt zu vermeiden ist.
  \item Wir sind dafür verantwortlich, zu erahnen, was andere Menschen brauchen könnten.
\end{itemize}


\subsection{Begegnungen zwischen den beiden Kulturen}

Wenn zwei Menschen aufeinandertreffen, die auf diesem Kontinuum weiter auseinanderliegen, sind Missverständnisse und Konflikte wahrscheinlich. In so einem Fall kann es hilfreich sein, miteinander in die Metakommunikation (siehe S.~\pageref{metakommunikation}) über dieses Konzept zu gehen.


\subsection{GfK auf dem Kontinuum}

GfK tendiert sehr stark zur \emph{Ask-Culture}.

\section{The Triforce for Communication}
\label{triforce-communication}
\index{Triforce of Communication}

Das Konzept des \emph{Triforce for Communication}\cite{triforce-of-communication} sagt, dass es hilfreich ist, vor einem Gespräch abzuklären, was die andere Person von dir braucht, wenn sie dir etwas erzählen möchte.

\paragraph{\textnumero 1 Soziale Verbindung:} Ihr möchtet einfach nur teilen, was bei euch gerade so los ist, um die Verbindung zwischen euch zu pflegen.

\paragraph{\textnumero 2 Empathie oder emotionale Unterstützung:} Ihr braucht empathisches Zuhören und vielleicht Trost.
\index{Empathie}

\paragraph{\textnumero 3 Ratschläge:} Ihr braucht konkrete Ideen, Ratschläge und Problemlösungen.
\index{Ratschläge}


\chapter{Gewaltfreie Kommunikation}
\index{Gewaltfreie Kommunikation}
\section{Verwandte Konzepte}
\label{verwandte-konzepte}
\index{Harvard-Konzept}
\index{Ich-Botschaften}
\index{4 Seiten einer Nachricht}

Diese Konzepte haben Berührungspunkte mit der GfK:

\begin{itemize}
  \item direkte Kommunikation (siehe Seite \pageref{direkte-kommunikation})
  \item \emph{Ask-Culture} (siehe Seite \pageref{ask-guess-culture})
  \item radikale Ehrlichkeit \cite{radical-honesty}
  \item Ich-Botschaften und die 4 Seiten einer Nachricht nach Schulz von Thun \cite{miteinander-reden-1}
  \item das Harvard-Konzept \cite{harvard-konzept} für Verhandlung und Konfliktlösung
  \item psychologische Sicherheit
\end{itemize}

\section{Was ist Gewaltfreie Kommunikation?}

Vielen Dank an den GfK-Trainer Jochen Hiester\footnote{\url{https://www.gewaltfrei-koblenz.de/}} aus Koblenz, der mich mit seinem tiefen Wissen zu GfK mit diesem Abschnitt unterstützt hat und geduldig meine vielen Fragen beantwortet hat.


\subsection{Bezeichnung \glqq Gewaltfreie Kommunikation\grqq}

Marshall Rosenberg hat selbst nie gewählt, dass sein Kommunikationsmodell Gewaltfreie Kommunikation heißen soll. Er lehrte eine rudimentäre Form der GFK ab den späten 70er Jahren in Seminaren in den USA, ab den 80er Jahren auch in anderen Ländern. Und er hatte jahrelang keinen Namen für sein Modell. In den ersten Jahren lehrte er sein Modell überwiegend vor Menschen, die auch politisch engagiert waren~-- z.\,B.~wegen Diskriminierung von Afroamerikaner\_innen, Gleichberechtigung von Frauen, Friedensbewegung, Umweltbewegung. In diesem Kreisen waren die Schriften von Mahatma Gandhi den meisten bekannt~-- und damit auch wie Gandhi den Begriff \glqq Gewaltfreiheit\grqq{} aufgefasst hatte. Und da Teilnehmende seiner Seminare immer wieder anmerkten, dass das Modell von Rosenberg beschreibt, wie man die Haltung der Gewaltfreiheit im Sinne von Gandhi erreichen kann, nannten die Teilnehmenden zunehmend sein Modell Gewaltfreie Kommunikation.

Rosenberg selbst hat eher den Begriff \glqq eine Sprache des Lebens\grqq{} für dein Modell benutzt.


\subsection{GfK als Haltung}
\label{gfk-haltung}
\index{Haltung}

GfK ist sowohl eine Haltung als auch eine Sammlung von Kommunikationstechniken, die helfen können, diese Haltung zu leben.

Die Haltung der GfK zu leben bedeutet, die Grundannahmen auf Seite~\pageref{gfk-annahmen} in die persönlichen Überzeugungen aufzunehmen und danach zu handeln. Dabei ist die Änderung der Haltung ein lebenslanger Lernprozess: Mal sind Menschen in dieser Haltung, mal nicht. Lernen heißt dabei auch, immer öfter in der Haltung zu sein und zu bleiben, auch wenn eine Person gerade sehr aufgebracht ist.


\subsection{Ziele der GfK und der Weg dahin}
\label{gfk-ziele}
\index{Ziele}

Ziel der GfK ist, das Miteinander friedlicher zu machen.

Mehr Verbindung zu dem, was in einem selbst und in anderen lebendig ist (also vor allem Gefühle und Bedürfnisse) ist Teil des Weges, aber nicht das Ziel selbst.


\subsection{Definition von Gewalt in der GfK}
\label{gewalt-definition}
\index{Gewalt}

In der GfK ist Gewalt so definiert:

\begin{quote}
  Die eigenen Bedürfnisse versuchen zu erfüllen, wobei man die Bedürfnisse von anderen Personen mutwillig, bewusst oder absichtlich verletzt, soweit das vermeidbar wäre.
\end{quote}

Rosenberg selbst hat das nicht auf Menschen in der Zukunft (Klima) oder Tiere angewandt. Es würde aber funktionieren~-- dann wäre es entsprechend Gewalt gegen zukünftige Generationen oder gegen Tiere.


\subsection{Eigenverantwortung}
\label{gfk-eigenverantwortung}
\index{Eigenverantwortung}

Teil des Haltung der GfK ist auch, immer mehr Verantwortung für das eigene Handeln, die eigenen Worte und auch das eigene Wohlergehen zu übernehmen.

Die Annahme dahinter ist, dass jede Person ist für die eigene Bedürfnisse und Gefühle selbst verantwortlich ist. Und jede Person ist für die eigenen Worte und Handlungen verantwortlich. Eine Grauzone ist, wie weit man dann für die Folgen dieser Worte und Handlungen verantwortlich ist.


\subsection{Konflikte und GfK}
\label{gfk-konflikte}
\index{Konflikte}

GfK reduziert die Anzahl der Konflikte (oder Spannungen/Reibungen) nicht. Was sich reduziert, ist die Schärfe, wie Konflikte ausgetragen werden, wenn wenigstens eine der an einem Konflikt beteiligten Personen ausreichend GFK anwendet (vor allem mit sich selbst). GfK hilft aber, diese konstruktiv und friedlich zu lösen.

Tendenziell werdet ihr sogar \emph{mehr} Konflikte ansprechen und klären, wenn ihr GfK lernt.


\section{Annahmen in der gewaltfreien Kommunikation}
\label{gfk-annahmen}

\begin{itemize}
  \item Alle Menschen sind zu Empathie fähig, und alle Menschen brauchen Empathie.
  \item Dinge explizit zu sagen, macht es wahrscheinlicher, dass mein Gegenüber sie hört.
  \item Niemand kann Gedanken lesen.
  \item Menschen sind selbst dafür verantwortlich, ihre Bedürfnisse erfüllt zu bekommen.
  \item Andere Menschen sind für meine Gefühle nicht verantwortlich.
  \item Menschen haben immer einen guten Grund für das, was sie tun.
  \item Alle Menschen sind gewillt, zum Wohle ihrer Mitmenschen beizutragen.
  \item Konflikte sind im Miteinander wichtig und unvermeidbar.
  \item Einen Scheiß muss ich.
\end{itemize}

\section{Die drei Säulen der gewaltfreien Kommunikation}
\label{gfk-saeulen}

Diese Liste könnt ihr im Detail in \cite[S. 33f]{gfk-dummies} nachlesen.


\subsection{Einfühlsames Zuhören}

\begin{itemize}
 \item ganz beim Gegenüber sein
 \item mitschwingen
 \item die Aussagen logisch einordnen können
 \item nicht zwingend: einverstanden sein
\end{itemize}


\subsection{Selbstempathie}

\begin{itemize}
 \item das innere Gefühlschaos sortieren
 \item die eigenen (unerfüllten) Bedürfnisse herausfinden
 \item für dich selbst sorgen
\end{itemize}


\subsection{Achtsamer und ehrlicher Selbstausdruck}

\begin{itemize}
 \item die eigenen Bedürfnisse ausdrücken
 \item konkrete Beobachtungen äußern
 \item um konkrete Unterstützung bitten
\end{itemize}

\section{Schlüsselunterscheidungen der GfK}
\label{gfk-schluesselunterscheidungen}
\index{GfK-Schlüsselunterscheidungen}

Mehr dazu findet ihr in \cite[S.~35~f]{gfk-dummies}.

\subsection{Beobachtung vs.~Bewertung}
\index{Beobachtungen}
\index{Bewertungen}

\begin{description}
 \item[Beobachtungen] sind etwas, was eine Filmkamera aufnehmen könnte: Bild und Ton, aber nicht, was die Menschen sich dabei denken, was sie wollen, wie es ihnen geht, oder was für Menschen sie sind.
 \item[Bewertungen und Interpretationen] Ist das gut oder schlecht, was jemand tut? Macht das die Person zu einem schlechten Menschen?
\end{description}


\subsection{Gefühle vs.~Gedanken}
\index{Gefühle}
\index{Gedanken}

\begin{description}
 \item[Gefühle] sind kurze körperliche kurze Reaktionen nur in uns selbst. Mehr dazu auf Seite~\pageref{gefuehle}.
 \item[Gedanken] beschreiben, was jemand mit uns macht: Hier gibt es Opfer und Täter\_innen.
\end{description}


\subsection{Bedürfnisse vs.~Strategien}
\index{Bedürfnisse}
\index{Strategien}

\begin{description}
 \item[Bedürfnisse] müssen langfristig erfüllt sein, damit es uns gut geht. Bedürfnisse sind nicht an konkrete Handlungen oder Menschen gebunden. Mehr dazu auf Seite~\pageref{beduerfnisse}.
 \item[Strategien] sind konkrete Handlungen, mit denen wir versuchen, ein oder mehrere Bedürfnisse zu erfüllen.
\end{description}

\section{GfK-Levels}
\label{gfk-levels}
\index{GfK-Levels}

Gewaltfreie Kommunikation könnt ihr auf verschiedenen Levels leben.

Voraussetzung dafür, dass ihr ein Level anfangt, ist, dass ihr das nächstkleinere Level grundlegend beherrscht und lebt.

\paragraph{Level 1: Selbstentwicklung:} Hier geht es darum, euch selbst besser zu verstehen und euch besser um euch selbst zu kümmern. Ziel dieses Level ist, eure eigenen Bedürfnisse zu erfüllen.

\paragraph{Level 2: In Beziehung zum Du:} Auf diesem Level gebt ihr einem Menschen Empathie und kommuniziert mit dem anderen Menschen darüber, was in euch lebendig ist. Ziel von diesem Level ist, die Bedürfnisse von euch selbst und eurem Gegenüber unter einen Hut zu bekommen. Das wäre beispielsweise in der Familie, in der Partnerschaft, oder in der Nachbarschaft.

\paragraph{Level 3: Innerhalb eines Teams oder einer Gruppe:} Hier geht es darum, die Bedürfnisse aller zu abzudecken, dass möglichst wenig Bedürfnisse verletzt werden. Das könnte auch die Gemeinde sein oder der Verein.

\paragraph{Level 4: Teamübergreifend, gruppenübergreifend:} Hier geht es darum, als Führungskraft oder mit Führungskräften zu arbeiten.

\paragraph{Level 4: Strukturelle Arbeit in Systemen:} Hier geht es um Organisationsentwicklung, politische Arbeit, Arbeit mit Vorständen oder in einem Parlament.

\section{Empathie}
\label{empathie}
\index{Empathie}

\subsection{Arten von Empathie}

\cite{shaw-empathie} unterscheidet drei Arten von Empathie:

\paragraph{Emotionale Empathie:} die Fähigkeit, das Gleiche zu empfinden wie andere Menschen

\paragraph{Kognitive Empathie:} die Fähigkeit, nicht nur Gefühle, sondern auch Gedanken und Absichten anderer Menschen zu verstehen und daraus korrekte Schlussfolgerungen zu ihrem Verhalten abzuleiten

\paragraph{Soziale Empathie:} die Fähigkeit, das Verhalten komplexer sozialer Systeme zu verstehen und vorherzusagen

In der Gewaltfreien Kommunikation benutzen wir den Begriff \emph{Empathie} im Sinne der \fett{kognitiven Empathie}.

\subsection{Empathie lernen}

Bewusste Empathie ist zu einem großen Teil gelernt (und lässt sich auch später im Leben noch durch Üben verbessern).


\subsection{Mitleid vs.~Empathie}
\label{mitgleid}
\index{Mitleid}

Bei der \emph{Empathie} liegt der Fokus darauf, was die andere Person fühlt und braucht. Das ist das, was wir in der GfK üblicherweise möchten.

\emph{Mitleid} hingegen bedeutet, dass wir mit der anderen Person leiden. Dabei verschiebt sich der Fokus von der anderen Person auf uns, was wir bei der Empathie in der GfK vermeiden möchten.


\subsection{Alexithymie}
\label{alexithymie}
\index{Alexithymie}
\index{Gefühlsblindheit}

Alexithymie, auch \emph{Gefühlsblindheit}, bezeichnet Einschränkungen bei der Fähigkeit, Emotionen wahrzunehmen, zu erkennen und zu beschreiben. Emotionen sind bei Betroffenen prinzipiell vorhanden, werden jedoch als rein körperliche Symptome interpretiert. Der Schweregrad kann von nur leichten Schwierigkeiten beim Erkennen bestimmter Emotionen bis hin zu vollkommener \glqq Gefühlsblindheit\grqq{} reichen.

Für die Allgemeinbevölkerung wird eine Prävalenz von etwa 10\,\% angenommen, wobei Männer etwas häufiger betroffen sind als Frauen. Besonders verbreitet ist Alexithymie bei Menschen auf dem Autismus-Spektrum, und zwar mit ca.~ 50\,\%.

\section{Die vier Schritte der gewaltfreien Kommunikation}

Mehr dazu findet ihr in \cite[S. 213]{gfk-rosenberg} und \cite{gfk-dummies}.

\subsection{Beobachtung}

\begin{itemize}
 \item Was ist wertfrei betrachtet geschehen?
 \item Was wurde gesagt und getan?
 \item Was habt ihr konkret beobachtet?
\end{itemize}


\subsection{Gefühle}

\begin{itemize}
 \item Wie geht es euch mit dem, was ihr gehört oder beobachtet hat?
 \item Wie fühlt ihr euch dabei?
 \item Was macht das mit euch?
\end{itemize}


\subsection{Bedürfnisse}

\begin{itemize}
 \item Was genau ist euch wichtig?
 \item Worum geht es euch?
 \item Was soll sich für euch dabei erfüllen?
\end{itemize}


\subsection{Bitten}

\begin{itemize}
 \item Um welche konkrete Handlung möchtet ihr bitten?
 \item Welchen nächsten Schritt wünscht ihr?
 \item Wie könnte euch die andere Person konkret unterstützen?
\end{itemize}

\section{Gefühle}
\label{gefuehle}

Das ursprüngliche Vokabular stammt von Marshall Rosenberg aus \cite[S. 216]{gfk-rosenberg} bzw. im englischsprachigen Original \cite[S. 210]{nvc-rosenberg}. Das erweiterte Vokabular und die Kriterien kommen aus \cite[S. 56f]{gfk-dummies}.


\section{Primärgefühle ("`Echte Gefühle"')}

Woran man echte Gefühle erkennt:

\begin{enumerate}
 \item Ein echtes Gefühl kann jeder Mensch auf der Welt empfinden~-- vom Baby bis zum alten Menschen.
 \item Echte Gefühle sind körperlich spürbar.
 \item Echte Gefühle enthalten keine Schuldzuweisung. In ihnen gibt es keine Täter\_innen und keine Opfer.
\end{enumerate}

\subsection{Angenehme Gefühle, wenn Bedürfnisse erfüllt sind}

\begin{multicols}{2}
  \begin{itemize}
    \item angeregt
    \item aufgedreht
    \item aufgeregt
    \item ausgeglichen
    \item befreit
    \item begeistert
    \item behaglich
    \item belebt
    \item berauscht
    \item beruhigt
    \item berührt
    \item beschwingt
    \item bewegt
    \item dankbar
    \item eifrig
    \item ekstatisch
    \item energetisiert
    \item engagiert
    \item enthusiastisch
    \item entlastet
    \item entschlossen
    \item entspannt
    \item entzückt
    \item erfreut
    \item erfrischt
    \item erfüllt
    \item ergriffen
    \item erleichtert
    \item erstaunt
    \item erwartungsvoll
    \item fasziniert
    \item frei
    \item friedlich
    \item froh
    \item fröhlich
    \item gebannt
    \item geborgen
    \item gefesselt
    \item gelassen
    \item gerührt
    \item gesammelt
    \item gespannt
    \item gesund
    \item glücklich
    \item gut gelaunt
    \item heiter
    \item hellwach
    \item hoffnungsvoll
    \item inspiriert
    \item klar
    \item kraftvoll
    \item lebendig
    \item leicht
    \item locker
    \item lustig
    \item motiviert
    \item munter
    \item mutig
    \item neugierig
    \item optimistisch
    \item ruhig
    \item sanft
    \item satt
    \item schwungvoll
    \item selbstsicher
    \item selig
    \item sicher
    \item sorglos
    \item still
    \item stolz
    \item überglücklich
    \item überrascht
    \item überwältigt
    \item unbeschwert
    \item vergnügt
    \item verliebt
    \item vertrauensvoll
    \item wach
    \item weit
    \item wissbegierig
    \item zärtlich
    \item zufrieden
    \item zugeneigt
    \item zuversichtlich
  \end{itemize}
\end{multicols}

\subsection{Unangenehme Gefühle, wenn Bedürfnisse nicht (genug) erfüllt sind}
\begin{multicols}{2}
  \begin{itemize}
    \item alarmiert
    \item angespannt
    \item ängstlich
    \item apathisch
    \item ärgerlich
    \item aufgeregt
    \item ausgelaugt
    \item bedrückt
    \item besorgt
    \item bestürzt
    \item beunruhigt
    \item bitter
    \item blockiert
    \item deprimiert
    \item durcheinander
    \item eifersüchtig
    \item einsam
    \item elend
    \item empört
    \item enttäuscht
    \item ernüchtert
    \item erschlagen
    \item erschöpft
    \item erschrocken
    \item erschüttert
    \item erstarrt
    \item frustriert
    \item furchtsam
    \item gehemmt
    \item geladen
    \item gelähmt
    \item gelangweilt
    \item genervt
    \item hart
    \item hasserfüllt
    \item hilflos
    \item in Panik
    \item irritiert
    \item kalt
    \item kraftlos
    \item leer
    \item lethargisch
    \item matt
    \item miserabel
    \item müde
    \item mutlos
    \item nervös
    \item niedergeschlagen
    \item ohnmächtig
    \item panisch
    \item perplex
    \item ratlos
    \item resigniert
    \item ruhelos
    \item sauer
    \item scheu
    \item schlapp
    \item schüchtern
    \item schwer
    \item schwermütig
    \item sorgenvoll
    \item teilnahmslos
    \item tot
    \item träge
    \item traurig
    \item überwältigt
    \item unbehaglich
    \item ungeduldig
    \item unglücklich
    \item unruhig
    \item unsicher
    \item unter Druck
    \item unwohl
    \item unzufrieden
    \item verbittert
    \item verspannt
    \item verwirrt
    \item verzweifelt
    \item widerwillig
    \item wütend
    \item zappelig
    \item zornig
  \end{itemize}
\end{multicols}


\section{Gedanken (\glqq Pseudogefühle\grqq)}

\emph{Gedanken} sind Begriffe, die als Gefühlsäußerung angekündigt werden, aber statt dessen Vorwürfe, Schuldzuweisungen, Analysen oder Interpretationen enthalten.

\begin{multicols}{2}
  \begin{itemize}
    \item abgelehnt
    \item abgeschnitten
    \item akzeptiert
    \item allein gelassen
    \item an den Pranger gestellt
    \item an die Wand gestellt
    \item angegriffen
    \item attackiert
    \item ausgebeutet
    \item ausgenutzt
    \item ausgeschlossen
    \item ausgestoßen
    \item beachtet
    \item bedroht
    \item belästigt
    \item beleidigt
    \item belogen
    \item benutzt
    \item beschuldigt
    \item beschützt
    \item bestätigt
    \item bestraft
    \item betrogen
    \item bevormundet
    \item deplatziert
    \item diskriminiert
    \item dominiert
    \item entmutigt
    \item enttäuscht
    \item erdrückt
    \item erniedrigt
    \item ernst genommen
    \item festgenagelt
    \item gedrängt
    \item geehrt
    \item geliebt
    \item geliebt
    \item gemaßregelt
    \item gemobbt
    \item gequält
    \item geschmeichelt
    \item gesehen
    \item getäuscht
    \item gewürdigt
    \item gezwungen
    \item gut beraten
    \item herabgesetzt
    \item hereingelegt
    \item hintergangen
    \item ignoriert
    \item im Mittelpunkt
    \item in die Ecke gedrängt
    \item in die Enge getrieben
    \item isoliert
    \item klein gemacht
    \item lächerlich gemacht
    \item manipuliert
    \item minderwertig
    \item missachtet
    \item missbrauchst
    \item missverstanden
    \item nicht anerkannt
    \item nicht ehrlich behandelt
    \item nicht einbezogen
    \item nicht ernst genommen
    \item nicht geliebt
    \item ungerecht behandelt
    \item nicht gesehen
    \item nicht respektiert
    \item nicht unterstützt
    \item nicht verstanden
    \item nicht wertgeschätzt
    \item provoziert
    \item reingelegt
    \item sabotiert
    \item schikaniert
    \item schlecht behandelt
    \item schön
    \item sympathisch
    \item totgequatscht
    \item über den Tisch gezogen
    \item überfordert
    \item übergangen
    \item überlistet
    \item unerwünscht
    \item ungehört
    \item ungeliebt
    \item unter Druck gesetzt
    \item unterbezahlt
    \item unterdrückt
    \item unverstanden
    \item unwichtig
    \item verarscht
    \item verfolgt
  \end{itemize}
\end{multicols}

\section{Bedürfnisse}
\label{beduerfnisse}
\index{Bedürfnisse}

\subsection{Was sind universelle Bedürfnisse?}

Bedürfnisse sind das, was wir erfüllt brauchen, damit es uns gut geht.

Ein universelles Bedürfnis ist eins, das jeder Mensch kennt~-- auch wenn sich Menschen darin unterscheiden, welche Bedürfnisse sie wie stark erfüllt brauchen.

Echte Bedürfnisse sind nicht an eine konkrete Person gebunden. Es gibt aber durchaus Bedürfnisse, die wir nur mit anderen Menschen zusammen erfüllen können, zum Beispiel unser Bedürfnis nach Gemeinschaft.

Ein Bedürfnis ist nicht an eine konkrete Handlung gebunden. Für jedes Bedürfnis gibt es viele verschiedene Strategien, um sie zu erfüllen~-- und wenn euch nur eine einzige Strategie dafür einfällt, dann habt ihr das Bedürfnis noch nicht genug verstanden.

\subsection{Liste von Bedürfnissen}

Das ursprüngliche Vokabular stammt von Marshall Rosenberg aus \cite[S.~216~f]{gfk-rosenberg} bzw.~im englischsprachigen Original \cite[S.~210]{nvc-rosenberg}. Das erweiterte Vokabular kommt \cite[S.~75~f]{gfk-dummies}.


\subsubsection{Autonomie}

\begin{multicols}{2}
  \begin{itemize}
    \item Freiheit
    \item Selbstbestimmung
  \end{itemize}
\end{multicols}


\subsubsection{Körperliche Bedürfnisse}

\begin{multicols}{2}
  \begin{itemize}
    \item Luft
    \item Wasser
    \item Bewegung
    \item Nahrung
    \item Schlaf
    \item Distanz
    \item Unterkunft
    \item Wärme
    \item Gesundheit
    \item Heilung
    \item Kraft
    \item Lebenserhaltung
  \end{itemize}
\end{multicols}


\subsubsection{Integrität, Stimmigkeit mit sich selbst}

\begin{multicols}{2}
  \begin{itemize}
    \item Authentizität
    \item Einklang
    \item Eindeutigkeit
    \item Übereinstimmung mit den eigenen Werten
    \item Identität
    \item Individualität
  \end{itemize}
\end{multicols}


\subsubsection{Einfühlung}

\begin{multicols}{2}
  \begin{itemize}
    \item Empathie
    \item verstanden/gesehen werden
    \item Gleichbehandlung
    \item Gerechtigkeit
  \end{itemize}
\end{multicols}


\subsubsection{Verbindung}

\begin{multicols}{2}
  \begin{itemize}
    \item Wertschätzung
    \item Nähe
    \item Zugehörigkeit
    \item Liebe
    \item Intimität/Sexualität
    \item Unterstützung
    \item Ehrlichkeit/Aufrichtigkeit
    \item Gemeinschaft
    \item Geborgenheit
    \item Respekt
    \item Kontakt
    \item Akzeptanz
    \item Austausch
    \item Offenheit
    \item Vertrauen
    \item Anerkennung
    \item Freundschaft
    \item Achtsamkeit
    \item Aufmerksamkeit
    \item Toleranz
    \item Zusammenarbeit
  \end{itemize}
\end{multicols}


\subsubsection{Entspannung}

\begin{multicols}{2}
  \begin{itemize}
    \item Erholung
    \item Ausruhen
    \item Spiel
    \item Spaß
    \item Leichtigkeit
    \item Ruhe
  \end{itemize}
\end{multicols}


\subsubsection{Geistige Bedürfnisse}

\begin{multicols}{2}
  \begin{itemize}
    \item Harmonie
    \item Inspiration
    \item \glqq Ordnung\grqq
    \item (innerer) Friede
    \item Freude
    \item Humor
    \item Abwechslungsreichtum
    \item Ausgewogenheit
    \item Glück
    \item Ästhetik
  \end{itemize}
\end{multicols}


\subsubsection{Entwicklung}

\begin{multicols}{2}
  \begin{itemize}
    \item Beitragen
    \item Wachstum
    \item Anerkennung
    \item Feedback
    \item Rückmeldung
    \item Erfolg (im Sinne von \glqq Gelingen\grqq
    \item Kreativität
    \item Sinne
    \item Bedeutung
    \item Effektivität
    \item Kompetenz
    \item Lernen
    \item Feiern
    \item Trauern
    \item Bildung
    \item Engagement
  \end{itemize}
\end{multicols}

\section{Bitten}
\label{bitten}

\subsection{Kriterien für gute Bitten}

Diese Kriterien könnt ihr im Detail in \cite[S. 85f]{gfk-dummies} nachlesen

\paragraph{Konkret:} Das Verhalten sollte realistisch und überprüfbar sein.

\paragraph{Machbar:} für die andere Person

\paragraph{Positiv formuliert:} Sagt, was ihr braucht, anstatt, was ihr nicht haben wollt.

\paragraph{Im Hier und Jetzt erfüllbar:} Das schließt auch Vereinbarungen mit Wirkung auf die Zukunft ein.

\paragraph{Freiwillig:} Was passiert, wenn die andere Person Nein sagt?


\subsection{Arten von Bitten}

\paragraph{Handlungsbitte:} Könntest du bitte …?

\paragraph{Bitte um aufrichtige Rückmeldung:} Wie geht es dir damit? Was siehst du das?

\paragraph{Bitte um Empathie:} Ich würde gerne verstehen, was du verstanden hast.


\subsection{An wen kann ich eine Bitte richten?}

\begin{itemize}
  \item an mein Gegenüber
  \item an mich selbst
  \item an eine dritte Person
\end{itemize}


\section{Wertschätzung ausdrücken}
\label{wertschaetzung}
\index{Wertschätzung}

Diese habe ich aus dem Buch \emph{GfK für Dummies} \cite[S.~206]{gfk-dummies} und mit einigen Dingen aus dem Podcast \emph{Familie verstehen} \cite{familie-verstehen-podcast} ergänzt, plus Ergänzungen aus dem Podcast \emph{Manager Tools Basics} \cite{manager-tools-basics}.

\begin{enumerate}
  \item Was hat die andere Person \fett{gesagt oder getan}?
  \item Welche \fett{Gefühle} hat dies bei mir ausgelöst? (optional, aber sehr hilfreich)
  \item Welche \fett{Bedürfnisse} von mir oder vom Team hat das erfüllt?
  \item \fett{Danke} dafür! \emph{oder:} Das \fett{feiere} ich!
  \item Gerne \fett{öfter/wieder tun}! (Bitte, optional)
\end{enumerate}

\section{Straßen-Giraffisch}
\label{strassengiraffisch}
\index{Straßen-Giraffisch}

Generell könnt ihr auch die Haltung und die Methoden der GfK mit Menschen benutzen, die (noch) kein GfK-Training haben. Ihr macht dann nur den größeren Teil der Arbeit, und ihr braucht etwas mehr Flexibilität bei den Begrifflichkeiten (\glqq Straßen-Giraffisch\grqq), weil die explizite Kommunikation mit den Begriffen \glqq Bedürfnis\grqq{} und \glqq Gefühl\grqq{} eher GfK-spezifisch ist.

Das funktioniert \emph{nicht}, wenn die andere Person gar nicht mit euch reden möchte oder überhaupt nicht dafür offen ist, Probleme konstruktiv zu lösen.


\subsection{Metakommunikation vorab}
\index{Metakommunikation}

Es kann hilfreich sein, wenn ihr ganz am Anfang des Gesprächs etwas Metakommunikation (siehe S.~\pageref{metakommunikation}) betreibt und euch die Zustimmung des Gegenübers einholt:

\begin{quote}
  Es ist mir wichtig, dass ich wirklich verstehe, was genau passiert ist, was dir wichtig ist und was du genau brauchst. Daher würde ich gerne genau nachfragen, damit ich sicher sein kann, dich richtig zu verstehen. Wäre das für dich okay, wenn wir das zusammen versuchen?
\end{quote}


\subsection{Beobachtungen}

\begin{itemize}
  \item Mir ist es wichtig, zuerst zu verstehen, was genau passiert ist.
  \item Was genau hast du gesehen?
  \item Was davon hast du beobachtet, und was hast du dir eher vermutet?
\end{itemize}


\subsection{Gefühle}

Hier helfen Umschreibungen der Gefühle.

\begin{itemize}
  \item Du klingst, als hättest du da echt einen Hals. Höre ich das richtig?
  \item Das nervt total, oder?
  \item Ich bin davon frustriert/verwirrt.
  \item Ich war echt erschrocken, als ich das hörte.
\end{itemize}


\subsection{Bedürfnisse}

\begin{itemize}
  \item Wofür ist dir das wichtig?
  \item Was fehlt dir dabei?
  \item Was bräuchtest du an der Stelle, damit es für dich besser funktioniert?
  \item Mir ist dabei \ldots{} wirklich wichtig.
  \item Ich brauche dabei mehr \ldots, damit es für mich funktioniert.
\end{itemize}

Das Wort \emph{Warum} würde ich dafür nicht empfehlen, weil es oft benutzt wird, um jemanden anzugreifen, und euer Gegenüber dann leicht dichtmacht.


\subsection{Bitten}

\begin{itemize}
  \item Was könnte ich möglicherweise tun, was dir für mehr \emph{(Bedürfnis)} hilfreich wäre?
  \item Was könnten wir zusammen tun?
  \item Ich glaube, das würde für mich nicht gut funktionieren, weil \emph{(verletztes Bedürfnis)}. Aber wäre vielleicht \emph{(andere Strategie)} für dich hilfreich?
  \item Wäre es für dich okay, wenn wir zusammen kurz brainstormen, wie wir für dich für mehr \emph{(Bedürfnis)} sorgen könnten?
  \item Wärst du bereit, \ldots? Es ist völlig okay, wenn du Nein sagst~-- Fragen kostet ja nichts.
  \item Wäre es für dich okay, \ldots?
\end{itemize}


\subsection{Straßen-Giraffisch für einige Bedürfnisse}
\index{Bedürfnisse}

Diese Formulierungen habe ich aus einem GfK-Adventskalender von Verena Ohn. \cite{verena-ohn}


\subsubsection{Authentizität}

\begin{itemize}
  \item Ich selbst sein können.
  \item Reden, wie mir der Schnabel gewachsen ist.
  \item Meine Gefühle zeigen und ausdrücken können, wie sie gerade sind.
\end{itemize}


\subsubsection{Autonomie}

\begin{itemize}
  \item Ich will es alleine schaffen und selbst entscheiden.
  \item Wenn ich über mich selbst bestimmen kann.
  \item Dass ich nicht immer jemanden (um Erlaubnis) fragen muss.
\end{itemize}


\subsubsection{Beteiligung}

\begin{itemize}
  \item Mitreden können, wenn es um wichtige Sachen geht.
  \item Etwas beitragen können, das zählt.
  \item Mitmachen, dabei sein, eingezogen werden.
\end{itemize}


\subsubsection{Disziplin}

\begin{itemize}
  \item Dranbleiben, auch wenn es anstrengend wird.
  \item Etwas üben, bis man es richtig gut kann.
  \item Dass sich alle an die Regeln halten, die ausgemacht wurden.
\end{itemize}


\subsubsection{Empathie}

\begin{itemize}
  \item Jemanden haben, der wirklich versteht, wie ich mich fühle.
  \item Nicht allein sein mit seinen Gefühlen, sondern jemanden haben, der mitfühlt.
  \item Dass jemand versteht, ohne dass man alles erklären muss.
\end{itemize}


\subsubsection{Fürsorge}

\begin{itemize}
  \item Sich um andere kümmern und für sie da sein, wenn sie jemanden brauchen.
  \item Dafür sorgen, dass es anderen gut geht.
  \item Für die Gesundheit und das seelische Wohl anderer Menschen sorgen.
\end{itemize}


\subsubsection{Gerechtigkeit}

\begin{itemize}
  \item Dass die Regeln für alle gelten und dass sich alle an die gleichen Absprachen halten müssen.
  \item Dass jeder und jeder das bekommt, was sie/er braucht.
  \item Dass jede und jeder mal bestimmen und entscheiden darf.
\end{itemize}


\subsubsection{Harmonie}

\begin{itemize}
  \item Dass sich niemand streitet und dass es ruhig und entspannt ist.
  \item Dass alle nett zueinander sind; dass sich alle lieb haben.
  \item Dass alle friedlich miteinander umgehen.
\end{itemize}


\subsubsection{Integrität}

\begin{itemize}
  \item Mir selbst treu bleiben.
  \item Für das einstehen, was mir wichtig ist, auch wenn es schwer ist.
  \item Das tun und zu dem stehen, was ich gesagt habe, und mich daran halten.
\end{itemize}


\subsubsection{Kreativität}

\begin{itemize}
  \item Mir eigene Ideen ausdenken und so malen, bauen oder basteln, wie ich es möchte.
  \item Nicht darüber nachdenken, wie und was ich mache, sondern einfach mal drauf los.
  \item Meinen Ideen freien Lauf lassen und auch Neues ausprobieren.
\end{itemize}


\subsubsection{Lebensfreude}

\begin{itemize}
  \item Spaß haben und lachen können, auch mal Quatsch machen und albern sein.
  \item Frei sein und das Leben genießen.
  \item Sachen tun, bei denen ich mich lebendig fühle.
\end{itemize}


\subsubsection{Mitbestimmung}

\begin{itemize}
  \item Mitreden können und Einfluss darauf haben, wenn es um Dinge geht, die mich betreffen.
  \item Gefragt werden, wie ich es gerne hätte und mitbestimmen dürfen.
  \item Dass ich gefragt werde, bevor jemand etwas für mich entscheidet.
\end{itemize}


\subsubsection{Nähe}

\begin{itemize}
  \item Kuscheln, in den Arm nehmen oder einfach nah sein.
  \item Dass ich dich sehen, hören, spüren und bei dir sein kann.
  \item Mit dir zusammen sein und dass du bei mir bleibst.
\end{itemize}


\subsubsection{Orientierung}

\begin{itemize}
  \item Dass jemand mir sagt, was ich tun soll, wenn ich unsicher bin.
  \item Einen Überblick haben, damit ich mich sicher fühle.
  \item Einen Plan haben, damit ich nicht durcheinander komme.
\end{itemize}


\subsubsection{Privatsphäre}

\begin{itemize}
  \item Dass ich auch mal allein sein kann und andere das respektieren.
  \item Einen Ort haben, an den ich mich zurückziehen und für mich sein kann.
  \item Dass nicht immer jemand fragt, was ich gerade mache.
\end{itemize}


\subsubsection{Rücksichtnahme}

\begin{itemize}
  \item Sehen, was ich gut kann, und stolz auf mich sein können.
  \item Dass ich merke, dass ich wichtig bin.
  \item Tief im Inneren wissen, dass ich genau richtig und gut bin, so wie ich bin.
\end{itemize}


\subsubsection{Selbstwert}

\begin{itemize}
  \item Sehen, was ich gut kann, und stolz auf mich sein können.
  \item Dass ich merke, dass ich wichtig bin.
  \item Tief im Inneren wissen, dass ich genau richtig und gut bin, so wie ich bin.
\end{itemize}


\subsubsection{Tatkraft}

\begin{itemize}
  \item Etwas motiviert anpacken, wenn ich Lust drauf habe.
  \item Wenn ich sofort loslegen will, weil ich gerade eine Idee habe.
  \item Die Energie, um etwas Neues zu probieren.
\end{itemize}


\subsubsection{Unterstützung}

\begin{itemize}
  \item Wenn mir jemand hilft, wenn ich etwas nicht allein schaffe.
  \item Dass jemand da ist, wenn ich nicht weiter weiß.
  \item Dass jemand mir Mut macht, wenn ich unsicher bin.
\end{itemize}


\subsubsection{Verständnis}

\begin{itemize}
  \item Jemand nimmt sich Zeit, um herauszufinden, was mit mir los ist.
  \item Dass keiner denkt, ich bin doof, nur weil ich etwas anders mache.
  \item Wenn jemand versucht, meine Sicht zu verstehen.
\end{itemize}


\subsubsection{Wachstum}

\begin{itemize}
  \item Wenn ich immer besser werde.
  \item Dass ich ausprobieren kann und herausfinde, was ich gut kann.
  \item Wenn ich Neues lerne.
\end{itemize}


\subsubsection{Verantwortung}

\begin{itemize}
  \item Mich selbst und ganz allein um etwas kümmern und es alleine regeln dürfen.
  \item Dass ich zeigen kann, dass ich für etwas sorgen kann.
  \item Dass ich für etwas ganz alleine zuständig bin.
\end{itemize}


\subsubsection{Wertschätzung}

\begin{itemize}
  \item Dass ich mir selbst sagen kann: ‚Das habe ich gut gemacht!‘
  \item Dass ich zufrieden mit mir selbst bin, egal was andere sagen.
  \item Mich freuen und stolz sein, wenn ich etwas geschafft habe.
\end{itemize}


\subsubsection{Zugehörigkeit}

\begin{itemize}
  \item die Sehnsucht, Teil der Gruppe zu sein
  \item der Wunsch dazu zu gehören und mitzumachen
  \item Freunde, die mit mir spielen wollen und gerne mit mir zusammen sind.
\end{itemize}

\section{GfK im Beruf}
\label{gfk-im-beruf}
\index{Beruf}
\index{Arbeitswelt}

Bei GfK geht es (generell) nicht darum, dass ihr immer euer Innerstes nach außen kehrt\footnote{siehe dazu auch die etwas unappetitliche Szene mit dem fehlgeschlagenen Teleporter-Test im Film \emph{Die Fliege}}, sondern darum, was hilfreich ist, euch mehr in Verbindung miteinander zu bringen und besser für euch zu sorgen.

GfK ist in der Arbeitswelt sehr hilfreich, wenn ihr schaut, wie es euch in dieser Rolle und diesem Kontext dient.

Dabei können diese Fragen für euch hilfreich sein:

\begin{itemize}
  \item Was hilft mir, besser zu verstehen, was die andere Person gerade bewegt und was ihr wichtig ist?
  \item Was ist für die andere Person hilfreich zu wissen, warum ich gerade etwas brauche?
  \item Wie können wir eine Lösung finden, die für alle Beteiligten gut funktioniert? Wer braucht dabei was, und wie bringen wir das zusammen?
  \item Welche kreativen Strategien können wir dafür brainstormen?
  \item Welche meiner Gefühle sind für die andere Person hilfreich zu wissen, damit ihr klarer wird, wie wichtig mir etwas ist?
  \item Was sollte die andere Person über meine Bedürfnisse wissen, um sinnvoll und kreativ zu einer Lösung beitragen zu können?
\end{itemize}

Und überhaupt tut \fett{Empathie} (siehe S.~\pageref{empathie}) total gut~-- insbesondere \fett{Selbstempathie} (wenn ihr gerade Empathie braucht, aber sie euch gerade niemand anderes geben kann oder will) und \fett{Empathie anderen Menschen gegenüber}. Meiner Erfahrung nach dürsten viele Menschen im Arbeitsleben nach Empathie, und es tut ihnen wirklich gut, wenn wir uns ihnen gegenüber empathisch verhalten.

\section{Wolfsbegriffe in Giraffisch übersetzen}
\label{wolfsbegriffe-ersetzen}
\index{Wolfsbegriffe}
\index{Wolfssprache}
\index{Giraffisch}


\subsection{Aber}
\index{aber}

Wir möchten die Begriffe von allen Beteiligten sehen.

\subsubsection{Ersatz: Und, gleichzeitig}

\paragraph{Wolfssprache:} Du hast das Feature nicht fertiggestellt, aber wir haben es für kommenden Montag versprochen.

\paragraph{Giraffisch:} Du hast das Feature nicht fertiggestellt, gleichzeitig haben wir es für kommenden Montag versprochen. Wie kriegen wir das gelöst?

\paragraph{Wolfssprache:} Du hast dir die ganze Tafel Schokolade genommen, aber ich möchte auch welche.

\paragraph{Giraffisch:} Du hast dir die ganze Tafel Schokolade genommen, und ich möchte auch welche. Sollen wir halbe-halbe machen?


\subsection{Man}
\index{man}

Wir möchten über unsere gegenseitigen, persönlichen Bedürfnisse sprechen.

\subsubsection{Ersatz: Ich, oder über Bedürfnisse sprechen}

\paragraph{Wolfssprache:} Man fängt erst mit dem Essen an, wenn alle am Tisch sind.

\paragraph{Giraffisch:} Mir ist Gemeinschaft wichtig, und dafür ist es für mich hilfreich, wenn wir gemeinsam mit dem Essen anfangen. Wärst du bereit, mit dem Anfangen zu warten, bis alle am Tisch sitzen?

\paragraph{Giraffisch:} \emph{Wir stellen fest, dass das eine Benimmregel ist, die von keinerr der beteiligten Personen tatsächlich ein Bedürfnis erfüllt. Deswegen fangen wir mit Essen an, wann wir möchten.}

\paragraph{Wolfssprache:} Man kann dich akustisch nicht gut verstehen.

\paragraph{Giraffisch:} Ich habe dich gerade akustisch nicht gut verstanden.


\subsection{Muss}
\index{muss}

Wir möchten über unsere gegenseitigen, persönlichen Bedürfnisse sprechen, und uns bewusst für oder gegen Dinge entscheiden, um uns damit gut zu tun.

\subsubsection{Ersatz: Sich explizit für oder gegen Dinge entscheiden.}

\paragraph{Wolfssprache:} Ich muss jetzt zur Arbeit.

\paragraph{Giraffisch:} Mir ist es wichtig, dass ich meine Miete und mein Essen bezahlen kann. Deswegen gehe ich jetzt zur Arbeit.

\paragraph{Giraffisch:} Dieser Job tut mir nicht gut. Ich möchte mir daher einen neuen Job suchen. Bis ich einen neuen Job habe, gehe ich zur bisherigen Arbeit, damit ich bis dahin finanziell für mich sorgen kann.


\subsection{Entschuldigung}
\index{entschuldigen}

Wir möchten verbindend kommunizieren. Darüber zu sprechen, wer Schuld hat, trennt uns eher.

\subsubsection{Ersatz: bedauern}

\paragraph{Wolfssprache:} Ich möchte mich für die Verspätung entschuldigen.

\paragraph{Giraffisch:} Ich bedaure, dass ich 15~Minuten zu spät war und ihr deswegen auch mich gewartet habt, weil mir Zuverlässigkeit und Effizienz wichtig ist.

\section{Die 4 Ohren der GfK}

\paragraph{Wolf (nach außen gerichtet):} Ich bin okay. Die andere Person ist nicht okay.

\paragraph{Wolf (nach innen gerichtet):} Ich bin nicht okay. Die andere Person ist okay.

\paragraph{Giraffe (nach innen gerichtet):} Was fühle ich? Was brauche ich?

\paragraph{Giraffe (nach außen gerichtet):} Was könnte die andere Person fühlen? Was könnte sie brauchen?

\section{Wie GfK bei Konflikten helfen kann}
\label{gfk-und-konflikte}
\index{Konflikte}

Gewaltfreie Kommunikation kann euch bei Konflikte an diesen Punkten helfen:

\paragraph{Selbstempathie:} Wenn es euch schlecht geht, könnt ihr euch selbst Empathie geben, damit ihr danach in der Lage seid, mit der anderen Person konstruktiv zu reden.

\paragraph{Empathie bekommen:} Eine andere Person kann euch Empathie geben, wenn es euch schlecht geht.

\paragraph{Selbstklärung:} GfK kann euch dabei helfen, euch darüber klarer zu werden, was ihr fühlt und was euch gerade fehlt.

\paragraph{Selbstfürsorge:} Wenn ihr wisst, welche Bedürfnisse bei euch untererfüllt sind, könnt ihr möglicherweise schon Wege finden, diese Bedürfnisse für euch auch ohne die andere Person zu erfüllen.

\paragraph{Empathie geben:} Ihr könnt der anderen Person Empathie geben, damit sie besser in der Lage ist, auch eure Bedürfnisse wahrnehmen zu können.

\paragraph{Gemeinsam Lösungen finden:} Sobald ihr herausgefunden habt, was die verletzten oder untererfüllten Bedürfnisse aller Beteiligten sind (was durchaus länger dauern kann), ist es vergleichsweise einfach, gemeinsam Strategien für diese Bedürfnisse zu finden.

\section{Nach dem Workshop weiterlernen}
\label{gfk-weiterlenen}

Dieser Workshop bietet euch einen Einstieg in die GfK. Wenn ihr danach weiter lernen möchtet, könnt ihr euch kontinuierlich weiterbilden und üben.

Zusätzlich könnt ihr beim internationalen GfK-Dachverband \emph{Center for Nonviolent Communication (CNVC)} eine Zertifizierung als GfK-Trainer\_in anstreben.

Außerdem bauen alle meine Workshops zur Führungskräfteentwicklung und Teamentwicklung auf GfK aus. Mehr findet ihr unter \url{https://www.oliverklee.de/workshops/}.


\subsection{Weitere Themen, die in diesem Workshop keinen Platz gefunden haben}

\begin{itemize}
  \item schützende Gewalt
  \item Konflikte lösen
  \item GfK in der Kindererziehung
  \item Mediation
  \item mit Ärger umgehen
  \item das innere Team
  \item Umgang mit Scham und Schuld
  \item Nein sagen, Grenzen setzen
  \item Arbeit mit Glaubenssätzen
  \item das innere Kind heilen
  \item systemisches Konsensieren
  \item empathisch unterbrechen
\end{itemize}

Viele dieser Themen könnt ihr in einer GfK-Grundausbildung oder in Einzelseminaren lernen.


\subsection{Kontinuierlich weiter lernen}

\begin{itemize}
  \item Seminare zu einzelnen Themen
  \item GfK-Übungsgruppe (vor Ort oder online)
  \item Podcast \glqq Familie verstehen\grqq\cite{familie-verstehen-podcast} von Kathy Weber (hauptsächlich über Kindererziehung, aber auch generell für GfK interessant, wenn ihr (noch) keine Kinder habt)
  \item GfK-Tage (eintägige Konferenzen, die in vielen Städten jährlich stattfinden)
  \item Bücher zu GfK lesen
\end{itemize}

\subsubsection{Literaturempfehlungen}

Das Buch \glqq Gewaltfreie Kommunikation\grqq{} von Marshall Rosenberg \cite{gfk-rosenberg} bzw.\ im englischen Original \cite{nvc-rosenberg} ist ein guter Einstieg und bietet eine gute Übersicht.

Gabriel Seils hat in \cite{gfk-gespraech} einige lange Gespräche mit Marshall Rosenberg geführt. Dieses Buch hilft, GfK als Haltung besser zu verstehen. Ich finde es außerdem sehr tröstlich zu lesen.

Rosenberg hat außerdem eine Reihe von kleinen Büchlein verfasst, zum Beispiel \cite{we-can-work-it-out} oder \cite{being-me-loving-you}. Diese Büchlein beleuchten die Anwendung von GfK in einzelnen Bereichen des Lebens und sind jeweils gut an einem Nachmittag als Snack lesbar.

\subsection{GfK-Zertifizierung}

So sieht der Weg zur Zertifizierung beim CNVC aus:

\begin{enumerate}
  \item an einen GfK-Einstiegsworkshop teilnehmen
  \item an einer 10- bis 20-tägigen GfK-Basisausbildung teilnehmen (auch \glqq Jahresausbildung\grqq\ oder \glqq Grundausbildung\grqq\ genannt)
  \item mit dem GfK-Zertifizierungsprozess\cite{gfk-trainer-werden} beginnen
  \item derweil weiter lernen (s.o.)
  \item an einem Workshop zum Thema \glqq GfK unterrichten\grqq{} teilnehmen (nicht verpflichtend, aber wirklich hilfreich)
  \item an einem allgemeinen Train-the-Trainer-Workshop teilnehmen (auch nicht verpflichtend, aber das hilft sehr, die didaktische Qualität eurer GfK-Workshops zu steigern)
\end{enumerate}

Wenn ihr an Workshops teilnehmt, achtet darauf, dass ihr dies bei Personen tut, die beim CNVC zertifiziert sind, da ihr 10 solche Tage nachweisen müsst, um zur Ausbildung als GfK-Trainer\_in zugelassen zu werden.

Auch dann, wenn ihr GfK nicht unterrichten möchtet, ist die Basisausbildung ein guter Weg, um nach einem Einstiegsworkshop wirklich tief in die GfK einzusteigen.

Ich persönlich habe sehr gute Erfahrungen mit der Basisausbildung von Lydia Kaiser\footnote{\url{https://kommunikation-bewegt.de/}} und Jochen Hiester\footnote{\url{https://www.gewaltfrei-koblenz.de/}} in Bonn gemacht und kann beide sehr empfehlen.



\chapter{Führung}
\section{Die Rolle der Führung}
\label{fuehrung-rolle}
\index{Führung: Rolle}


\subsection{Aufgaben der Führung nach Neuberger}

Laut Neuberger\cite{neuberger-fuehren} sind die Aufgaben der Führung,

\begin{itemize}
  \item andere Menschen
  \item zielgerichtet
  \item in einer formalen Organisation
  \item unter konkreten Umweltbedingungen dazu bewegen,
  \item Aufgaben zu übernehmen und erfolgreich auszuführen,
  \item wobei humane Ansprüche gewahrt werden.
\end{itemize}


\subsection{Neuberger, aber modernisiert}

Auf das moderne Arbeiten übertragen, wäre die Aufgabe der Führung,

\begin{itemize}
  \item eine Umgebung zu schaffen,
  \item die es einem Team oder einer Organisation möglich und leicht macht,
  \item für die Mission des Teams oder der Organisation zu arbeiten,
  \item wobei die Menschen nachhaltig körperlich und seelisch gesund zu bleiben
  \item und ihr Potenzial nutzen können.
\end{itemize}

(Dies ist meine eigene \glqq Übersetzung\grqq{} aus Perspektive der modernen Führung.)


\subsection{Aufgaben der Führung nach Malik}

Dies sind laut Fredmund Malik \cite{malik-fuehrung} die Aufgaben der Führung:

\begin{itemize}
  \item für Ziele sorgen
  \item organisieren
  \item entscheiden
  \item kontrollieren
  \item Menschen entwickeln und fördern
\end{itemize}

Auf moderne Führung übertragen, wäre es die Aufgabe der Führung, dafür zu sorgen, dass diese Dinge stattfinden (also dass beispielsweise das Team Entscheidungen fällen und nachhalten kann), und nicht zwangsläufig, dass die Führung das auch selbst entscheidet.

\subsection{Grundsätze der Führung nach Malik}
\label{fuehrung-grundsätze}
\index{Führung: Grundsätze}


Fredmund Malik \cite{malik-fuehrung} hat in seinen Büchern die Rolle und die Grundsätze von Führung beschrieben.

\begin{itemize}
  \item Ergebnisorientierung
  \item Beitrag zum Ganzen
  \item Konzentration auf weniges
  \item Stärken nutzen
  \item gegenseitiges Vertrauen
  \item positiv denken
\end{itemize}

\section{Führungsinstrumente (Führungswerkzeuge)}
\label{fuehrungsinstrumente}
\index{Führungswerkzeuge}
\index{Führungsinstrumente}


\subsection{Was ist ein Führungsinstrument?}

Ein Führungsinstrument (oder Führungswerkzeug) generell alles, was eine Führungskraft tun kann, um direkt oder indirekt zusammen mit den geführten Personen Ziele zu erreichen.


\subsubsection{Führungswerkzeuge nach Malik}

Diese Liste von Fredmund Malik \cite{malik-fuehrung} ist schon etwas angestaubt. Sie lässt sich allerdings gut in die heutige Zeit übertragen.

\begin{itemize}
  \item Besprechung
  \item Schriftstück
  \item Stellengestaltung und Einsatzsteuerung
  \item Persönliche Arbeitsmethodik
  \item Budget und Budgetierung
  \item Leistungsbeurteilung
  \item systematische Müllabfuhr
\end{itemize}


\subsection{Direkte Führungsinstrumente}

\begin{itemize}
  \item 1-zu-1-Gespräche (siehe Seite~\pageref{1-zu-1})
  \item Anweisung
  \item Besprechung
  \item delegieren
  \item Entscheidungen treffen
  \item Feedback einholen
  \item Feedback geben
  \item informieren
  \item Konflikte klären
  \item kontrollieren
  \item Kritik
  \item Lob
  \item um etwas bitten
  \item Wertschätzung ausdrücken
  \item Ziele vereinbaren
\end{itemize}


\subsection{Indirekte Führungsinstrumente}

\begin{itemize}
  \item Anreizsysteme schaffen
  \item dem Team Workshops und andere Fortbildungen anbieten
  \item den (physischen) Arbeitsplatz gestalten
  \item die Motivation verbessern
  \item die psychologische Sicherheit verbessern
  \item eine Mission definieren
  \item einen Spieleabend mit dem Team veranstalten
  \item ein Team zusammenstellen
  \item Gewaltfreie Kommunikation lernen und anwenden
  \item mit dem Team einen Escape-Room spielen
  \item mit dem Team lecker essen gehen
  \item Prozesse definieren
  \item Rollen und Verantwortlichkeiten definieren
  \item Supervision für das Team organisieren
\end{itemize}

\section{1-zu-1-Gespräche (\emph{One-on-Ones})}
\label{1-zu-1}
\index{1-zu-1-Gespräche}
\index{One-on-Ones}

\subsection{Ziele}

\begin{itemize}
 \item das Vertrauen und die Beziehung zwischen Teamlead und Teammitglied aufbauen und pflegen
 \item dem Teammitglied und Teamlead die Möglichkeit geben, sich gegenseitig regelmäßig Feedback zu geben
 \item ein Ort für Absprachen sein, damit die Ad-hoc-Absprachen und -Calls zwischendrin weniger werden
 \item Konflikte, Sorgen und Ideen zeitnah besprechen
\end{itemize}

\subsection{Struktur}

\begin{itemize}
 \item Die Gespräche sollten regelmäßig und zuverlässig stattfinden.
 \item Mit Vollzeitangestellten sollten die Gespräche wöchentlich stattfinden. In unserem ehrenamtlichen Sehr-Teilzeit-Team haben wir einen Rhythmus von 4 Wochen.
 \item Die Gespräche haben eine harte Begrenzung auf 30 Minuten, während die einzelnen Slots zeitlich flexibler sind.
 \item Generell sollte das Teammitglied ca.~90\,\% der Redeanteile im Gespräch haben und die Teamführung die restlichen 10\,\%.
\end{itemize}

Alle Themen hier sind nur Vorschläge. Wenn euch andere Themen wichtiger sind, sprecht über diese.

\subsubsection{Checkin}
Wie fühle ich mich gerade? Wie bin ich hier?

\subsubsection{Slot für das Teammitglied (ca.~10 Minuten)}

\paragraph{Allgemeines}
\begin{itemize}
 \item alles, worüber du sprechen möchtest
\end{itemize}

\paragraph{Strategie}
\begin{itemize}
 \item Was tun wir als Team/Organisation nicht, was wir tun sollten?
 \item Wenn wir eine Sache verbessern könnten, welche wäre das?
 \item Wenn du ich wärst, was würdest du verändern?
 \item Feedback an die Teamführung
\end{itemize}

\paragraph{Blick nach außen}
\begin{itemize}
 \item Was ist zur Zeit das größte Problem unseres Teams/unserer Organisation? Und warum?
 \item Was gefällt dir nicht an unseren Produkten und Dienstleistungen? Was könnten wir verbessern? Woran müssen wir noch arbeiten?
 \item Was ist die größte Chance, die wir gerade verpassen?
\end{itemize}

\paragraph{Team}
\begin{itemize}
 \item Mit wem würdest du gerne (mehr) zusammenarbeiten?
 \item Wer macht gerade einen richtig guten Job?
 \item Was macht dir gerade Spaß? Und warum?
 \item Und was macht dir keinen Spaß? Und warum?
 \item Gibt es etwas, vor dem du gerade Angst hast?
\end{itemize}


\subsubsection{Slot für die Teamführung (ca.~10 Minuten)}
\begin{itemize}
 \item alles, worüber du sprechen möchtest
 \item Feedback an das Teammitglied
\end{itemize}


\subsubsection{Persönliche Entwicklung (ca.~10 Minuten, wenn noch Zeit ist)}
\begin{itemize}
 \item Was hast du kürzlich (dazu-)gelernt?
 \item Was hast du kürzlich ausprobiert? Was ist dabei herausgekommen?
 \item Was würdest du gerne lernen?
 \item Was würdest du gerne mal ausprobieren?
 \item Welche Verantwortungsbereiche würdest du gerne annehmen oder abgeben?
 \item Wie kannst du deine Superkräfte am besten einsetzen?
\end{itemize}

\subsection{Quellen}

\begin{itemize}
 \item Podcast: Manager Tools Basics: One on Ones \cite[04.\,07.\,2005 bis 11.\,07.\,2005]{manager-tools-basics}
 \item Podcast: Female Leadership: Gamechanger für Gesprächsführung \cite[Folge 15]{female-leadership-gespraechsfuehrung}
 \item Buch: The Hard Thing About Hard Things: Building a Business When There Are No Easy Answers \cite{the-hard-thing-about-hard-things}
\end{itemize}

\section{Führen lernen}
\label{fuehren-lernen}
\label{fuehrung-lernen}
\index{Führung lernen}

Meiner Ansicht nach ist Führen zu lernen so ähnlich wie singen zu lernen.

Dafür sind diese Dinge notwendig:

\begin{itemize}
  \item viel \fett{üben} (und dabei aus Fehlern lernen)
  \item sehr viel \fett{Reflexion} \index{Reflexion}
  \item \fett{Außenwahrnehmung} bekommen in der Form von Feedback \index{Außenwahrnehmung} \index{Feedback}
  \item an \fett{Trainings} und \fett{Workshops} teilnehmen (oder anderweitig Unterricht nehmen)
  \item \fett{Bücher} oder anderen Quellen von Wissen konsumieren
  \item von \fett{guten Beispielen} lernen
\end{itemize}

Damit ihr andere Menschen gut führen könnt, ist es außerdem notwendig, dass ihr euch selbst gut kennt und versteht, wie ihr tickt und was euch antreibt. Dies könnt ihr durch diese Dinge (oder eine Kombination daraus) erreichen:

\begin{itemize}
  \item Gewaltfreie Kommunikation lernen \index{Gewaltfreie Kommunikation}
  \item eine Psychotherapie machen (Tiefenpsychologie oder Psychoanalyse; keine kognitive Verhaltenstherapie) \index{Therapie} \index{Psychotherapie}
\end{itemize}

Hilfreich zum kontinuierlichen Lernen ist außerdem eine Supervision, Intervision oder kollegiale Fallberatung.

\section{Lob und Tadel ersetzen}
\label{lob-tadel}
\index{Lob}
\index{Tadel}

\subsection{Warum wollen wir Lob und Tadel ersetzen?}

\begin{itemize}
  \item Wenn ihr lobt oder tadelt, dann verpasst ihr damit die Gelegenheit, der anderen Person ein Einblick in das zu geben, was in euch lebendig ist.
  \item Lob und Tadel kommt ihr von oben statt auf Augenhöhe. Das wirkt sich negativ auf eure Beziehung aus.
  \item Die übliche Reaktion auf Tadel ist, das getadelte Verhalten in Zukunft zu vermeiden, anstatt daran zu wachsen.
  \item Wenn ihr jemanden tadelt, ist nicht immer klar, ob der Fokus drauf liegt, dass ihr selbst etwas braucht, oder darauf, dass ihr der anderen Person die Gelegenheit zum Lernen geben möchtet.
\end{itemize}


\subsection{Lob ersetzen}

Gebt stattdessen echte, herzliche \fett{Wertschätzung}. Mehr dazu findet ihr auf Seite~\pageref{wertschaetzung}.


\subsection{Tadel ersetzen}

Wenn der Fokus darauf ist, dass ihr selbst eine \fett{Veränderung für euch} braucht, formuliert eine \fett{Bitte}. Mehr dazu auf Seite~\pageref{bitten}.

Wenn hingegen der Fokus darauf ist, dass ihr der anderen Person (mehr) \fett{Entwicklung ermöglichen} möchtet, dann bietet stattdessen \fett{Feedback} an. Auf Seite~\pageref{feedback-regeln} findet ihr Tipps dazu, wie ihr hilfreiches Feedback geben könnt.


\chapter{Übungsaufgaben}
\label{gfk-übungen}

\section{Übungen zur Haltung}


\subsection{Nur ungern!}

Schreibe 2~Dinge auf, die du sehr ungern tust, aber trotzdem tust. Warum hast du sie trotzdem getan? Was hast du dir damit erfüllt, was dir wichtig ist? Wie hast du dich dabei gefühlt?

Besprecht das danach in der Kleingruppe. Seht ihr ein Muster? Teilt danach im Plenum eure Erkenntnisse.


\subsection{Du musst!}

Schreibe eine Situation auf, in der du einen anderen Menschen dazu gebracht hast, etwas für dich zu tun, obwohl die andere Person es eigentlich nicht wollte. Was hatte das für Auswirkungen auf dich, auf die andere Person und auf eure Beziehung? Wie hast du dich dabei gefühlt?

Besprecht das danach in der Kleingruppe. Sehr ihr ein Muster? Teilt danach im Plenum eure Erkenntnisse.


\subsection{Aber gern!}

Schreibe 2~Dinge auf, die du sehr gern für einen anderen Menschen getan hast, und warum du sie so gern getan hast. Wie hast du dich dabei gefühlt?

Besprecht das danach in der Kleingruppe. Sehr ihr ein Muster? Teilt danach im Plenum eure Erkenntnisse.


\subsection{Gefordert!}

Schreibt 2~Situationen auf, in denen jemand etwas von dir gefordert hat. Wie hast du dich dabei gefühlt? Und wie hoch war danach deine Bereitschaft, dies für die andere Person zu tun (von $-5$ (großer Widerstand) bis $+5$ (freudige Bereitschaft))?

Besprecht das danach in der Kleingruppe. Welche Muster fallen euch auf? Teilt danach im Plenum eure Erkenntnisse.


\subsection{Kümmer dich um dich!}

Finde 2~Dinge/Tätigkeiten, mit denen du dir in der letzten Zeit so etwas richtig Gutes getan hast. Was war es? Und wie/warum hat es dein Leben besser gemacht? Wie hast du dich dabei/danach gefühlt?

Besprecht das dann in der Kleingruppe. Teilt davon ein Highlight im Plenum.


\subsection{Perspektivwechsel}

Finde eine Situation, wo du dich für \glqq das kleinere Übel\grqq{} entschieden hast. Finde heraus, wofür du dich dadurch entschieden hast (also wie du damit versucht hast, dein Leben wundervoller zu machen). Wie hat sich das damals angefühlt? Und wie fühlt es sich mit dieser veränderten Perspektive an? Würdest du dich mit dieser neuen Perspektive in derselben Situation wieder so entscheiden?

Finde außerdem eine Situation, in der du dich gegen etwas entschieden hast. Finde heraus, wofür du dich dadurch entschieden hast (also wie du damit versucht hast, dein Leben wundervoller zu machen). Wie hat sich das damals angefühlt? Und wie fühlt es sich mit dieser veränderten Perspektive an? Würdest du dich mit dieser neuen Perspektive in derselben Situation wieder so entscheiden?

Teilt dies in eurer Kleingruppe. Sucht euch dann aus jeder Kategorie 1 Highlight aus, das ihr dann im Plenum teilen möchtet.


\section{Übungen zu den 4~Schritten}


\subsection{Konfliktsituation analysieren}

Sucht euch eine Konfliktsituation in der Vergangenheit aus, die euch immer noch beschäftigt oder belastet. Schreibt dann diese Dinge für euch auf und teilt sie in eurer Kleingruppe miteinander:

\begin{enumerate}
  \item Beobachtung
  \item Gefühl(e)
  \item nicht erfüllte Bedürfnisse
  \item Bitte (an dich oder jemand anderen)
\end{enumerate}


\section{Übungen zu Bedürfnissen}


\subsection{Resonanz}

Sucht euch aus der Bedürfnisliste auf Seite~\pageref{beduerfnisse} ein Bedürfnis heraus, das mit euch besonders resoniert oder euch besonders anspricht. Überlegt, mit welchen 2~konkreten Strategien ihr euch dieses Bedürfnis in der Vergangenheit erfüllt habt.

Teilt das danach in eurer Kleingruppe.

Teilt danach eure Erkenntnisse im Plenum.


\subsection{Bedürfnis oder Strategie?}

Geht in eurer Kleingruppe diese Liste durch und entscheidet, ob dieser Begriff ein echtes Bedürfnis darstellt oder eine Strategie. Überlegt bei den Strategien, welche Bedürfnisse sich jemand mit dieser Strategie möglicherweise erfüllen könnte (oder es zumindest versuchen könnte).

Ihr könnt dabei diese Faustregeln benutzen:

\begin{itemize}
  \item Wenn es um einen konkreten Ort, einen konkreten Gegenstand oder eine konkrete Person geht, dann ist es eine Strategie, kein Bedürfnis.
  \item Wenn es (gesunde erwachsene) Menschen gibt, die das überhaupt nicht brauchen, dann ist es eine Strategie, kein Bedürfnis.
  \item Es gibt allerdings durchaus Bedürfnisse, die wir nur mit anderen Menschen erfüllen können, zum Beispiel das Bedürfnis nach Verbindung oder nach Gemeinschaft.
\end{itemize}

Schaut dabei bitte \emph{nicht} in die Liste mit Bedürfnissen in diesem Handout, sondern nutzt euer Verständnis davon, was Bedürfnisse von Strategien unterscheidet.

\begin{enumerate}
  \item Alkohol
  \item Autonomie
  \item Bücher lesen
  \item Dazugehören
  \item Effizienz
  \item Feiern
  \item Frühstück
  \item gehört werden
  \item Ibuprofen
  \item Kaffee
  \item gelobt werden
  \item Konflikte klären
  \item Kooperation
  \item massiert werden
  \item monogam leben
  \item Orientierung
  \item polyamor leben
  \item psychische Gesundheit
  \item Religion
  \item Schlaf
  \item Schokolade
  \item Sex
  \item Sinn
  \item Sport
  \item Vertrauen
  \item Videospiele spielen
  \item wertschätzender Umgang im eigenen Team
  \item Übersicht
  \item Unterstützung
  \item WLAN
\end{enumerate}


\subsection{Schandtat}

Tut euch in eurer Kleingruppe zusammen.

Eine Person, die in der Vergangenheit eine \glqq Schandtat\grqq{} getan hat (etwas, für das sie bestraft wurde oder worden wäre, wenn sie erwischt worden wäre oder eine andere Person dies gesehen hätte), nennt diese Schandtat.

\begin{enumerate}
  \item Die Gruppe bietet Bedürfnisse an, die die Person sich damit versucht haben könnte zu erfüllen.
  \item Die Gruppe bietet Bedürfnisse an, die für das \glqq Opfer\grqq{} oder die Umgebung da nicht gut erfüllt waren.
  \item Überlegt euch zusammen Strategien, wie das \glqq Opfer\grqq{} seine Bedürfnisse erfüllen könnte.
  \item Macht dann mit der nächsten Schandtat einer anderen Person weiter, bis ihr in der Gruppe keine Schandtaten mehr habt.
\end{enumerate}

Teilt danach ein, zwei besonders beeindruckende Schandtaten im Plenum.


\section{Übungen zu Gefühlen}


\subsection{Gefühl oder Gedanke/Pseudogefühl?}

Geht in eurer Kleingruppe diese Liste durch und entscheidet, ob dieser Begriff ein echtes Gefühl darstellt oder einen Gedanken/ein Pseudogefühl. Überlegt bei den Gedanken/Pseudogefühlen, welche Gefühle die Person dabei fühlen könnte, und welche Bedürfnisse bei der Person gerade unerfüllt sein könnten.

Ihr könnt dabei diese Faustregeln benutzen:

\begin{itemize}
  \item Wenn es ein direkter oder indirekter Vorwurf an eine andere Person ist, dann ist es ein Gedanke/Pseudogefühl.
  \item Wenn ihr sagen könnt: \glqq Du hast mich \ldots\grqq, dann ist es oft (nicht immer!) ein Gedanke/Pseudogefühl.
  \item Wenn ihr sagen könnt: \glqq Ich bin \ldots\grqq, dann ist es oft (nicht immer!) ein Gefühl.
\end{itemize}

Schaut dabei bitte \emph{nicht} in die Liste mit Gefühlen und Gedanken in diesem Handout, sondern nutzt euer Verständnis davon, was Gefühle von Gedanken und Pseudogefühlen unterscheidet.

\begin{enumerate}
  \item ärgerlich
  \item ausgeschlossen
  \item begeistert
  \item eifersüchtig
  \item entlastet
  \item erschrocken
  \item erwartungsvoll
  \item geehrt
  \item gefesselt
  \item geliebt
  \item gerührt
  \item gesehen
  \item ergriffen
  \item hasserfüllt
  \item hoffnungsvoll
  \item inspiriert
  \item minderwertig
  \item missverstanden
  \item neugierig
  \item ohnmächtig
  \item schwermütig
  \item überfordert
  \item überwältigt
  \item unterdrückt
  \item unter Druck
  \item unverstanden
  \item unwichtig
  \item unzufrieden
  \item verarscht
  \item zugeneigt
\end{enumerate}


\section{Übungen zu Bitten}


\subsection{Hilfreiche Bitten formulieren}

Entscheidet in eurer Kleingruppe, ob diese Sätze eindeutig darum bitten, dass die andere Person eine konkrete Handlung ausführt. Überlegt euch bei den Sätzen, die ihr für keine hilfreichen Bitten haltet, eine mögliche hilfreiche Bitte für das Bedürfnis, dass sich die Person damit möglicherweise erfüllen möchte.

Teilt eure Entscheidungen und Ideen danach im Plenum.


\begin{enumerate}
  \item Bitte nenne mir etwas an meiner Arbeit, das du an mir wertschätzt.
  \item Bitte respektiere meine Privatsphäre.
  \item Hör bitte mit dem Trinken auf.
  \item Ich hätte gerne, dass du mich verstehst.
  \item Ich hätte gerne, dass du öfter den Müll rausbringst.
  \item Ich möchte, dass du dich mir gegenüber respektvoll verhältst.
  \item Ich möchte dich gerne besser kennenlernen.
  \item Im neuen Jahr will ich mehr Sport machen.
  \item Könntest du bitte am Dienstag die Spülmaschine ausräumen?
  \item Würdest du mir bitte sagen, wie du mich gerade verstanden hast?
\end{enumerate}

\section{Tagesrückblick}
\label{tagesrueckblick}
\index{Tagesrückblick}
\index{Rückblick}

GfK zu lernen ist wie eine Fremdsprache zu lernen: In der Hektik des Moments habt ihr die Vokabeln vielleicht nicht parat. Aber ihr könnt die Wörter später in Ruhe nachschauen, so dass ihr sie in der nächsten Situation besser parat habt (und die Sprache ein klein wenig besser sprecht).

Nehmt euch dafür am Abend ein paar Minuten Zeit, um \fett{eine Situation des Tages} besser zu verstehen, die euch bewegt hat.

\begin{itemize}
  \item Geht die Liste der \fett{Gefühle} auf Seite~\pageref{gefuehle-liste} durch: Welche Gefühle habt ihr gefühlt?
  \item Geht die Liste der \fett{Bedürfnisse} auf Seite~\pageref{beduerfnisse-liste} durch: Welche Bedürfnisse waren bei euch in der Situation gut erfüllt? Welche waren nicht gut erfüllt?
  \item Geht die Liste der \fett{Gedanken/Pseudogefühle} auf Seite~\pageref{pseudogefuehle} durch: Welche dieser Gedanken hattet ihr? Welche Gefühle und Bedürfnisse stecken eigentlich dahinter?
\end{itemize}

So lernt ihr Stück für Stück, eure Gefühle und Bedürfnisse zu erkennen und zu benennen, und ihr erweitert euer aktives Vokabular dafür.

\section{GfK-Bingo}

Dieses Bingo-Blatt enthält sowohl Elemente der GfK als auch Dinge, die wir in der GfK zu vermeiden versuchen.

Wenn dir eines dieser Dinge auffällt, notiere dir zu der Situation ein Stichwort auf deinem Bingo-Blatt.

Wer vier Kästchen in einer Reihe ausgefüllt bekommt~-- waagerecht, senkrecht oder diagonal~--, ruft laut \fett{"`Bingo!"'}. Je mehr \emph{Bingo}s, desto besser. Das Spiel geht danach weiter.

\hspace{1em}

\renewcommand{\arraystretch}{1.27}
\noindent\begin{tabular}{|p{10.0em}|p{10.0em}|p{10.0em}|p{10.0em}|}
\hline

Bewertung statt Beobachtung &
echtes Bedürfnis ausdrücken &
echtes Gefühl ausdrücken &
Empathie geben \vspace{8em} \\
\hline

hilfreiche Bitte \vspace{8em} &
indirekte Kommunikation &
Interpretation statt Beobachtung &
man \\
\hline

muss  \vspace{8em} &
neutrale Beobachtung &
Pseudogefühl, Gedanke &
Selbstempathie \\
\hline

sollte \vspace{8em} &
Schuld(ige) suchen &
unkonkrete Bitte &
Verallgemeinerung \\
\hline
\end{tabular}
\renewcommand{\arraystretch}{1.0}


\backmatter

\bibliography{../shared/bibliography/literatur}

\chapter{Lizenz}
\label{lizenz}

\section*{Unter welchen Bedingungen könnt ihr dieses Handout benutzen?}
Dieses Handout ist unter einer \emph{Creative-Commons}-Lizenz lizensiert. Dies ist die \emph{Namensnennung-Share Alike 4.0 international (CC BY-SA 4.0)}\footnote{Die ausführliche Version dieser Lizenz findet ihr unter \url{https://creativecommons.org/licenses/by-sa/4.0/}.}. Das bedeutet, dass ihr dieses Handout unter diesen Bedingungen für euch kostenlos verbreiten, bearbeiten und nutzen könnt (auch kommerziell):

\begin{description}
  \item[Namensnennung.] Ihr müsst den Namen des Autors (Oliver Klee) nennen. Wenn ihr außerdem auch noch die Quelle\footnote{\url{https://github.com/oliverklee/workshop-handouts}} nennt, wäre das nett. Und wenn ihr mir zusätzlich eine Freude machen möchtet, sagt mir per E-Mail Bescheid.
  \item[Weitergabe unter gleichen Bedingungen.] Wenn ihr diesen Inhalt bearbeitet oder in anderer Weise umgestaltet, verändert oder als Grundlage für einen anderen Inhalt verwendet, dann dürft ihr den neu entstandenen Inhalt nur unter Verwendung identischer Lizenzbedingungen weitergeben.
  \item[Lizenz nennen.] Wenn ihr den Reader weiter verbreitet, müsst ihr dabei auch die Lizenzbedingungen nennen oder beifügen.
\end{description}


\printindex

\end{document}
