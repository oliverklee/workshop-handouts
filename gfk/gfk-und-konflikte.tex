\section{Wie GfK bei Konflikten helfen kann}
\label{gfk-und-konflikte}
\index{Konflikte}

Gewaltfreie Kommunikation kann euch bei Konflikte an diesen Punkten helfen:

\paragraph{Selbstempathie:} Wenn es euch schlecht geht, könnt ihr euch selbst Empathie geben, damit ihr danach in der Lage seid, mit der anderen Person konstruktiv zu reden.

\paragraph{Empathie bekommen:} Eine andere Person kann euch Empathie geben, wenn es euch schlecht geht.

\paragraph{Selbstklärung:} GfK kann euch dabei helfen, euch darüber klarer zu werden, was ihr fühlt und was euch gerade fehlt.

\paragraph{Selbstfürsorge:} Wenn ihr wisst, welche Bedürfnisse bei euch untererfüllt sind, könnt ihr möglicherweise schon Wege finden, diese Bedürfnisse für euch auch ohne die andere Person zu erfüllen.

\paragraph{Empathie geben:} Ihr könnt der anderen Person Empathie geben, damit sie besser in der Lage ist, auch eure Bedürfnisse wahrnehmen zu können.

\paragraph{Gemeinsam Lösungen finden:} Sobald ihr herausgefunden habt, was die verletzten oder untererfüllten Bedürfnisse aller Beteiligten sind (was durchaus länger dauern kann), ist es vergleichsweise einfach, gemeinsam Strategien für diese Bedürfnisse zu finden.
