\section{Schlüsselunterscheidungen der GfK}
\label{gfk-schluesselunterscheidungen}
\index{GfK-Schlüsselunterscheidungen}

Mehr dazu findet ihr in \cite[S. 35ff]{gfk-dummies}.

\subsection{Beobachtung vs.~Bewertung}

\begin{description}
 \item[Beobachtungen] sind etwas, was eine Filmkamera aufnehmen könnte: Bild und Ton, aber nicht, was die Menschen sich dabei denken, was sie wollen, wie es ihnen geht, oder was für Menschen sie sind.
 \item[Bewertungen und Interpretationen] Ist das gut oder schlecht, was jemand tut? Macht das die Person zu einem schlechten Menschen?
\end{description}


\subsection{Gefühle vs.~Gedanken}

\begin{description}
 \item[Gefühle] sind kurze körperliche kurze Reaktionen nur in uns selbst. Mehr dazu auf Seite~\pageref{gefuehle}.
 \item[Gedanken] beschreiben, was jemand mit uns macht: Hier gibt es Opfer und Täter\_innen.
\end{description}


\subsection{Bedürfnisse vs.~Strategien}

\begin{description}
 \item[Bedürfnisse] müssen langfristig erfüllt sein, damit es uns gut geht. Bedürfnisse sind nicht an konkrete Handlungen oder Menschen gebunden. Mehr dazu auf Seite~\pageref{beduerfnisse}.
 \item[Strategien] sind konkrete Handlungen, mit denen wir versuchen, ein oder mehrere Bedürfnisse zu erfüllen.
\end{description}
