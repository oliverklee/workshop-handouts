\section{Bitten in der GfK}
\label{bitten}
\index{Bitten}

\subsection{Kriterien für gute Bitten}

Diese Kriterien könnt ihr im Detail in \cite[S.~85~f]{gfk-dummies} nachlesen

\paragraph{Konkret:} Das Verhalten sollte realistisch und überprüfbar sein.

\paragraph{Machbar:} für die andere Person

\paragraph{Positiv formuliert:} Sagt, was ihr braucht, anstatt, was ihr nicht haben wollt.

\paragraph{Im Hier und Jetzt erfüllbar:} Das schließt auch Vereinbarungen mit Wirkung auf die Zukunft ein.

\paragraph{Freiwillig:} Was passiert, wenn die andere Person Nein sagt?


\subsection{Arten von Bitten}

\paragraph{Handlungsbitte:} Könntest du bitte \ldots?

\paragraph{Bitte um aufrichtige Rückmeldung:} Wie geht es dir damit? Was siehst du das?

\paragraph{Bitte um Empathie:} Ich würde gerne verstehen, was du verstanden hast.


\subsection{An wen kann ich eine Bitte richten?}

\begin{itemize}
  \item an mein Gegenüber
  \item an mich selbst
  \item an eine dritte Person
\end{itemize}

