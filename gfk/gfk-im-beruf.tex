\section{GfK im Beruf}
\label{gfk-im-beruf}
\index{Beruf}
\index{Arbeitswelt}

Bei GfK geht es (generell) nicht darum, dass ihr immer euer Innerstes nach außen kehrt\footnote{siehe dazu auch die etwas unappetitliche Szene mit dem fehlgeschlagenen Teleporter-Test im Film \emph{Die Fliege}}, sondern darum, was hilfreich ist, euch mehr in Verbindung miteinander zu bringen und besser für euch zu sorgen.

GfK ist in der Arbeitswelt sehr hilfreich, wenn ihr schaut, wie es euch in dieser Rolle und diesem Kontext dient.

Dabei können diese Fragen für euch hilfreich sein:

\begin{itemize}
  \item Was hilft mir, besser zu verstehen, was die andere Person gerade bewegt und was ihr wichtig ist?
  \item Was ist für die andere Person hilfreich zu wissen, warum ich gerade etwas brauche?
  \item Wie können wir eine Lösung finden, die für alle Beteiligten gut funktioniert? Wer braucht dabei was, und wie bringen wir das zusammen?
  \item Welche kreativen Strategien können wir dafür brainstormen?
  \item Welche meiner Gefühle sind für die andere Person hilfreich zu wissen, damit ihr klarer wird, wie wichtig mir etwas ist?
  \item Was sollte die andere Person über meine Bedürfnisse wissen, um sinnvoll und kreativ zu einer Lösung beitragen zu können?
\end{itemize}

Und überhaupt tut \fett{Empathie} (siehe S.~\pageref{empathie}) total gut~-- insbesondere \fett{Selbstempathie} (wenn ihr gerade Empathie braucht, aber sie euch gerade niemand anderes geben kann oder will) und \fett{Empathie anderen Menschen gegenüber}. Meiner Erfahrung nach dürsten viele Menschen im Arbeitsleben nach Empathie, und es tut ihnen wirklich gut, wenn wir uns ihnen gegenüber empathisch verhalten.
