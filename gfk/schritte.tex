\section{Die vier Schritte der gewaltfreien Kommunikation}
\label{gfk-schritte}
\index{Schritte der GfK}

Mehr dazu findet ihr in \cite[S. 213]{gfk-rosenberg} und \cite{gfk-dummies}.

\subsection{Beobachtung}

\begin{itemize}
 \item Was ist wertfrei betrachtet geschehen?
 \item Was wurde gesagt und getan?
 \item Was habt ihr konkret beobachtet?
\end{itemize}


\subsection{Gefühle}

\begin{itemize}
 \item Wie geht es euch mit dem, was ihr gehört oder beobachtet hat?
 \item Wie fühlt ihr euch dabei?
 \item Was macht das mit euch?
\end{itemize}


\subsection{Bedürfnisse}

\begin{itemize}
 \item Was genau ist euch wichtig?
 \item Worum geht es euch?
 \item Was soll sich für euch dabei erfüllen?
\end{itemize}


\subsection{Bitten}

\begin{itemize}
 \item Um welche konkrete Handlung möchtet ihr bitten?
 \item Welchen nächsten Schritt wünscht ihr?
 \item Wie könnte euch die andere Person konkret unterstützen?
\end{itemize}
