\section{GfK-Bingo}

Dieses Bingo-Blatt enthält sowohl Elemente der GfK als auch Dinge, die wir in der GfK zu vermeiden versuchen.

Wenn dir eines dieser Dinge auffällt, notiere dir zu der Situation ein Stichwort auf deinem Bingo-Blatt.

Wer vier Kästchen in einer Reihe ausgefüllt bekommt~-- waagerecht, senkrecht oder diagonal~--, ruft laut \fett{"`Bingo!"'}. Je mehr \emph{Bingo}s, desto besser. Das Spiel geht danach weiter.

\hspace{1em}

\renewcommand{\arraystretch}{1.27}
\noindent\begin{tabular}{|p{10.0em}|p{10.0em}|p{10.0em}|p{10.0em}|}
\hline

Bewertung statt Beobachtung &
echtes Bedürfnis ausdrücken &
echtes Gefühl ausdrücken &
Empathie geben \vspace{8em} \\
\hline

hilfreiche Bitte \vspace{8em} &
indirekte Kommunikation &
Interpretation statt Beobachtung &
man \\
\hline

muss  \vspace{8em} &
neutrale Beobachtung &
Pseudogefühl, Gedanke &
Selbstempathie \\
\hline

sollte \vspace{8em} &
Schuld(ige) suchen &
unkonkrete Bitte &
Verallgemeinerung \\
\hline
\end{tabular}
\renewcommand{\arraystretch}{1.0}
