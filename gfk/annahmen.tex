\section{Grundannahmen in der GfK}
\label{gfk-annahmen}

\begin{itemize}
  \item Alle Menschen sind zu Empathie fähig, und alle Menschen brauchen Empathie.
  \item Dinge explizit zu sagen, macht es wahrscheinlicher, dass mein Gegenüber sie hört.
  \item Niemand kann Gedanken lesen.
  \item Menschen sind selbst dafür zuständig, ihre Bedürfnisse erfüllt zu bekommen.
  \item Menschen sind für ihre Taten und Worte verantwortlich.
  \item Andere Menschen sind nicht für meine Gefühle zuständig.
  \item Menschen haben immer einen guten Grund für das, was sie tun.
  \item Alles, was ein Mensch jemals tut, ist ein Versuch, Bedürfnisse zu erfüllen.
  \item Menschen tun freiwillig und gerne etwas, um anderen das Leben zu verschönern. (Vorübergehende Nicht-Kooperation erfolgt lediglich, weil andere Bedürfnisse dem gerade entgegenstehen. \glqq Ein Nein ist ein Ja zu etwas Anderem.\glqq)
  \item Konflikte sind im Miteinander wichtig und unvermeidbar.
  \item Einen Scheiß muss ich.
  \item Alle Menschen haben die gleichen Bedürfnisse. Bedürfnisse (so wie die GfK sie definiert) dienen alle dem Leben. (Daher gibt es auch keine negativen Bedürfnisse.)
  \item Jeder Mensch hat bemerkenswerte Fähigkeiten, die uns erfahrbar werden, wenn wir durch Einfühlung mit ihnen in Kontakt kommen.
\end{itemize}
