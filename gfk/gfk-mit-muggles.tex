\section{GfK mit Muggles}
\label{gfk-mit-muggles}
\index{Muggles}

Generell könnt ihr auch die Haltung und die Methoden der GfK mit Menschen benutzen, die (noch) kein GfK-Training haben. Ihr macht dann nur den größeren Teil der Arbeit, und ihr braucht etwas mehr Flexibilität bei den Begrifflichkeiten (\glqq Straßen-Giraffisch\grqq), weil die explizite Kommunikation mit den Begriffen \glqq Bedürfnis\grqq{} und \glqq Gefühl\grqq{} eher GfK-spezifisch ist.
\index{Straßen-Giraffisch}

Das funktioniert \emph{nicht}, wenn die andere Person gar nicht mit euch reden möchte oder überhaupt nicht dafür offen ist, Probleme konstruktiv zu lösen.

Den Begriff \emph{Muggle} habe ich aus dem Harry-Potter-Universum übernommen\footnote{\url{https://de.wikipedia.org/wiki/Begriffe_der_Harry-Potter-Romane\#Muggel}}. Inzwischen wird er auch im Geocaching als Begriff für Menschen benutzt, die nicht mitspielen.


\subsection{Metakommunikation vorab}
\index{Metakommunikation}

Es kann hilfreich sein, wenn ihr ganz am Anfang des Gesprächs etwas Metakommunikation (siehe S.~\pageref{metakommunikation}) betreibt und euch die Zustimmung des Gegenübers einholt:

\begin{quote}
  Es ist mir wichtig, dass ich wirklich verstehe, was genau passiert ist, was dir wichtig ist und was du genau brauchst. Daher würde ich gerne genau nachfragen, damit ich sicher sein kann, dich richtig zu verstehen. Wäre das für dich okay, wenn wir das zusammen versuchen?
\end{quote}


\subsection{Beobachtungen}

\begin{itemize}
  \item Mir ist es wichtig, zuerst zu verstehen, was genau passiert ist.
  \item Was genau hast du gesehen?
  \item Was davon hast du beobachtet, und was hast du dir eher vermutet?
\end{itemize}


\subsection{Gefühle}

Hier helfen Umschreibungen der Gefühle.

\begin{itemize}
  \item Du klingst, als hättest du da echt einen Hals. Höre ich das richtig?
  \item Das nervt total, oder?
  \item Ich bin davon frustriert/verwirrt.
  \item Ich war echt erschrocken, als ich das hörte.
\end{itemize}


\subsection{Bedürfnisse}

\begin{itemize}
  \item Wofür ist dir das wichtig?
  \item Was fehlt dir dabei?
  \item Was bräuchtest du an der Stelle, damit es für dich besser funktioniert?
  \item Mir ist dabei \ldots{} wirklich wichtig.
  \item Ich brauche dabei mehr \ldots, damit es für mich funktioniert.
\end{itemize}

Das Wort \emph{Warum} würde ich dafür nicht empfehlen, weil es oft benutzt wird, um jemanden anzugreifen, und euer Gegenüber dann leicht dichtmacht.


\subsection{Bitten}

\begin{itemize}
  \item Was könnte ich möglicherweise tun, was dir für mehr \emph{(Bedürfnis)} hilfreich wäre?
  \item Was könnten wir zusammen tun?
  \item Ich glaube, das würde für mich nicht gut funktionieren, weil \emph{(verletztes Bedürfnis)}. Aber wäre vielleicht \emph{(andere Strategie)} für dich hilfreich?
  \item Wäre es für dich okay, wenn wir zusammen kurz brainstormen, wie wir für dich für mehr \emph{(Bedürfnis)} sorgen könnten?
  \item Wärst du bereit, \ldots? Es ist völlig okay, wenn du Nein sagst~-- Fragen kostet ja nichts.
  \item Wäre es für dich okay, \ldots?
\end{itemize}

