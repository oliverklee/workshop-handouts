\section{Nach dem Workshop weiterlernen}
\label{gfk-weiterlenen}

Dieser Workshop bietet euch einen Einstieg in die GfK. Wenn ihr danach weiter lernen möchtet, könnt ihr euch kontinuierlich weiterbilden und üben.

Zusätzlich könnt ihr beim internationalen GfK-Dachverband \emph{Center for Nonviolent Communication (CNVC)} eine Zertifizierung als GfK-Trainer\_in anstreben.

Außerdem bauen alle meine Workshops zur Führungskräfteentwicklung und Teamentwicklung auf GfK aus. Mehr findet ihr unter \url{https://www.oliverklee.de/workshops/}.


\subsection{Weitere Themen, die in diesem Workshop keinen Platz gefunden haben}

\begin{itemize}
  \item schützende Gewalt
  \item Konflikte lösen
  \item GfK in der Kindererziehung
  \item Mediation
  \item mit Ärger umgehen
  \item das innere Team
  \item Umgang mit Scham und Schuld
  \item Nein sagen, Grenzen setzen
  \item Arbeit mit Glaubenssätzen
  \item das innere Kind heilen
  \item systemisches Konsensieren
  \item empathisch unterbrechen
\end{itemize}

Viele dieser Themen könnt ihr in einer GfK-Grundausbildung oder in Einzelseminaren lernen.


\subsection{Kontinuierlich weiter lernen}

\begin{itemize}
  \item Seminare zu einzelnen Themen
  \item GfK-Übungsgruppe (vor Ort oder online)
  \item Podcast \glqq Familie verstehen\grqq\cite{familie-verstehen-podcast} von Kathy Weber (hauptsächlich über Kindererziehung, aber auch generell für GfK interessant, wenn ihr (noch) keine Kinder habt)
  \item GfK-Tage (eintägige Konferenzen, die in vielen Städten jährlich stattfinden)
  \item Bücher zu GfK lesen
\end{itemize}

\subsubsection{Literaturempfehlungen}

Das Buch \glqq Gewaltfreie Kommunikation\grqq{} von Marshall Rosenberg \cite{gfk-rosenberg} bzw.\ im englischen Original \cite{nvc-rosenberg} ist ein guter Einstieg und bietet eine gute Übersicht.

Gabriel Seils hat in \cite{gfk-gespraech} einige lange Gespräche mit Marshall Rosenberg geführt. Dieses Buch hilft, GfK als Haltung besser zu verstehen. Ich finde es außerdem sehr tröstlich zu lesen.

Rosenberg hat außerdem eine Reihe von kleinen Büchlein verfasst, darunter \cite{we-can-work-it-out} oder \cite{being-me-loving-you}. Diese Büchlein beleuchten die Anwendung von GfK in einzelnen Bereichen des Lebens und sind jeweils gut an einem Nachmittag als Snack lesbar.

\subsection{GfK-Zertifizierung}

So sieht der Weg zur Zertifizierung beim CNVC aus:

\begin{enumerate}
  \item an einen GfK-Einstiegsworkshop teilnehmen
  \item an einer 10- bis 20-tägigen GfK-Basisausbildung teilnehmen (auch \glqq Jahresausbildung\grqq\ oder \glqq Grundausbildung\grqq\ genannt)
  \item mit dem GfK-Zertifizierungsprozess\cite{gfk-trainer-werden} beginnen
  \item derweil weiter lernen (s.o.)
  \item an einem Workshop zum Thema \glqq GfK unterrichten\grqq{} teilnehmen (nicht verpflichtend, aber wirklich hilfreich)
  \item an einem allgemeinen Train-the-Trainer-Workshop teilnehmen (auch nicht verpflichtend, aber das hilft sehr, die didaktische Qualität eurer GfK-Workshops zu steigern)
\end{enumerate}

Wenn ihr an Workshops teilnehmt, achtet darauf, dass ihr dies bei Personen tut, die beim CNVC zertifiziert sind, da ihr 10 solche Tage nachweisen müsst, um zur Ausbildung als GfK-Trainer\_in zugelassen zu werden.

Auch dann, wenn ihr GfK nicht unterrichten möchtet, ist die Basisausbildung ein guter Weg, um nach einem Einstiegsworkshop wirklich tief in die GfK einzusteigen.

Ich persönlich habe sehr gute Erfahrungen mit der Basisausbildung von Lydia Kaiser\footnote{\url{https://kommunikation-bewegt.de/}} und Jochen Hiester\footnote{\url{https://www.gewaltfrei-koblenz.de/}} in Bonn gemacht und kann beide sehr empfehlen.

