\label{gfk-übungen}

\section{Übungen zur Haltung}


\subsection{Nur ungern!}

Schreibe 2~Dinge auf, die du sehr ungern tust, aber trotzdem tust. Warum hast du sie trotzdem getan? Was hast du dir damit erfüllt, was dir wichtig ist? Wie hast du dich dabei gefühlt?

Besprecht das danach in der Kleingruppe. Seht ihr ein Muster? Teilt danach im Plenum eure Erkenntnisse.


\subsection{Du musst!}

Schreibe eine Situation auf, in der du einen anderen Menschen dazu gebracht hast, etwas für dich zu tun, obwohl die andere Person es eigentlich nicht wollte. Was hatte das für Auswirkungen auf dich, auf die andere Person und auf eure Beziehung? Wie hast du dich dabei gefühlt?

Besprecht das danach in der Kleingruppe. Sehr ihr ein Muster? Teilt danach im Plenum eure Erkenntnisse.


\subsection{Aber gern!}

Schreibe 2~Dinge auf, die du sehr gern für einen anderen Menschen getan hast, und warum du sie so gern getan hast. Wie hast du dich dabei gefühlt?

Besprecht das danach in der Kleingruppe. Sehr ihr ein Muster? Teilt danach im Plenum eure Erkenntnisse.


\subsection{Gefordert!}

Schreibt 2~Situationen auf, in denen jemand etwas von dir gefordert hat. Wie hast du dich dabei gefühlt? Und wie hoch war danach deine Bereitschaft, dies für die andere Person zu tun (von $-5$ (großer Widerstand) bis $+5$ (freudige Bereitschaft))?

Besprecht das danach in der Kleingruppe. Welche Muster fallen euch auf? Teilt danach im Plenum eure Erkenntnisse.


\subsection{Kümmer dich um dich!}

Finde 2~Dinge/Tätigkeiten, mit denen du dir in der letzten Zeit so etwas richtig Gutes getan hast. Was war es? Und wie/warum hat es dein Leben besser gemacht? Wie hast du dich dabei/danach gefühlt?

Besprecht das dann in der Kleingruppe. Teilt davon ein Highlight im Plenum.


\subsection{Perspektivwechsel}

Finde eine Situation, wo du dich für \glqq das kleinere Übel\grqq{} entschieden hast. Finde heraus, wofür du dich dadurch entschieden hast (also wie du damit versucht hast, dein Leben wundervoller zu machen). Wie hat sich das damals angefühlt? Und wie fühlt es sich mit dieser veränderten Perspektive an? Würdest du dich mit dieser neuen Perspektive in derselben Situation wieder so entscheiden?

Finde außerdem eine Situation, in der du dich gegen etwas entschieden hast. Finde heraus, wofür du dich dadurch entschieden hast (also wie du damit versucht hast, dein Leben wundervoller zu machen). Wie hat sich das damals angefühlt? Und wie fühlt es sich mit dieser veränderten Perspektive an? Würdest du dich mit dieser neuen Perspektive in derselben Situation wieder so entscheiden?

Teilt dies in eurer Kleingruppe. Sucht euch dann aus jeder Kategorie 1 Highlight aus, das ihr dann im Plenum teilen möchtet.


\section{Übungen zu den 4~Schritten}


\subsection{Konfliktsituation analysieren}

Sucht euch eine Konfliktsituation in der Vergangenheit aus, die euch immer noch beschäftigt oder belastet. Schreibt dann diese Dinge für euch auf und teilt sie in eurer Kleingruppe miteinander:

\begin{enumerate}
  \item Beobachtung
  \item Gefühl(e)
  \item nicht erfüllte Bedürfnisse
  \item Bitte (an dich oder jemand anderen)
\end{enumerate}


\section{Übungen zu Bedürfnissen}


\subsection{Resonanz}

Sucht euch aus der Bedürfnisliste auf Seite~\pageref{beduerfnisse} ein Bedürfnis heraus, das mit euch besonders resoniert oder euch besonders anspricht. Überlegt, mit welchen 2~konkreten Strategien ihr euch dieses Bedürfnis in der Vergangenheit erfüllt habt.

Teilt das danach in eurer Kleingruppe.

Teilt danach eure Erkenntnisse im Plenum.


\subsection{Bedürfnis oder Strategie?}

Geht in eurer Kleingruppe diese Liste durch und entscheidet, ob dieser Begriff ein echtes Bedürfnis darstellt oder eine Strategie. Überlegt bei den Strategien, welche Bedürfnisse sich jemand mit dieser Strategie möglicherweise erfüllen könnte (oder es zumindest versuchen könnte).

Ihr könnt dabei diese Faustregeln benutzen:

\begin{itemize}
  \item Wenn es um einen konkreten Ort, einen konkreten Gegenstand oder eine konkrete Person geht, dann ist es eine Strategie, kein Bedürfnis.
  \item Wenn es (gesunde erwachsene) Menschen gibt, die das überhaupt nicht brauchen, dann ist es eine Strategie, kein Bedürfnis.
  \item Es gibt allerdings durchaus Bedürfnisse, die wir nur mit anderen Menschen erfüllen können, zum Beispiel das Bedürfnis nach Verbindung oder nach Gemeinschaft.
\end{itemize}

Schaut dabei bitte \emph{nicht} in die Liste mit Bedürfnissen in diesem Handout, sondern nutzt euer Verständnis davon, was Bedürfnisse von Strategien unterscheidet.

\begin{enumerate}
  \item Alkohol
  \item Autonomie
  \item Bücher lesen
  \item Dazugehören
  \item Effizienz
  \item Feiern
  \item Frühstück
  \item gehört werden
  \item Ibuprofen
  \item Kaffee
  \item gelobt werden
  \item Konflikte klären
  \item Kooperation
  \item massiert werden
  \item monogam leben
  \item Orientierung
  \item polyamor leben
  \item psychische Gesundheit
  \item Religion
  \item Schlaf
  \item Schokolade
  \item Sex
  \item Sinn
  \item Sport
  \item Vertrauen
  \item Videospiele spielen
  \item wertschätzender Umgang im eigenen Team
  \item Übersicht
  \item Unterstützung
  \item WLAN
\end{enumerate}


\subsection{Schandtat}

Tut euch in eurer Kleingruppe zusammen.

Eine Person, die in der Vergangenheit eine \glqq Schandtat\grqq{} getan hat (etwas, für das sie bestraft wurde oder worden wäre, wenn sie erwischt worden wäre oder eine andere Person dies gesehen hätte), nennt diese Schandtat.

\begin{enumerate}
  \item Die Gruppe bietet Bedürfnisse an, die die Person sich damit versucht haben könnte zu erfüllen.
  \item Die Gruppe bietet Bedürfnisse an, die für das \glqq Opfer\grqq{} oder die Umgebung da nicht gut erfüllt waren.
  \item Überlegt euch zusammen Strategien, wie das \glqq Opfer\grqq{} seine Bedürfnisse erfüllen könnte.
  \item Macht dann mit der nächsten Schandtat einer anderen Person weiter, bis ihr in der Gruppe keine Schandtaten mehr habt.
\end{enumerate}

Teilt danach ein, zwei besonders beeindruckende Schandtaten im Plenum.


\section{Übungen zu Gefühlen}


\subsection{Gefühl oder Gedanke/Pseudogefühl?}

Geht in eurer Kleingruppe diese Liste durch und entscheidet, ob dieser Begriff ein echtes Gefühl darstellt oder einen Gedanken/ein Pseudogefühl. Überlegt bei den Gedanken/Pseudogefühlen, welche Gefühle die Person dabei fühlen könnte, und welche Bedürfnisse bei der Person gerade unerfüllt sein könnten.

Ihr könnt dabei diese Faustregeln benutzen:

\begin{itemize}
  \item Wenn es ein direkter oder indirekter Vorwurf an eine andere Person ist, dann ist es ein Gedanke/Pseudogefühl.
  \item Wenn ihr sagen könnt: \glqq Du hast mich \ldots\grqq, dann ist es oft (nicht immer!) ein Gedanke/Pseudogefühl.
  \item Wenn ihr sagen könnt: \glqq Ich bin \ldots\grqq, dann ist es oft (nicht immer!) ein Gefühl.
\end{itemize}

Schaut dabei bitte \emph{nicht} in die Liste mit Gefühlen und Gedanken in diesem Handout, sondern nutzt euer Verständnis davon, was Gefühle von Gedanken und Pseudogefühlen unterscheidet.

\begin{enumerate}
  \item ärgerlich
  \item ausgeschlossen
  \item begeistert
  \item eifersüchtig
  \item entlastet
  \item erschrocken
  \item erwartungsvoll
  \item geehrt
  \item gefesselt
  \item geliebt
  \item gerührt
  \item gesehen
  \item ergriffen
  \item hasserfüllt
  \item hoffnungsvoll
  \item inspiriert
  \item minderwertig
  \item missverstanden
  \item neugierig
  \item ohnmächtig
  \item schwermütig
  \item überfordert
  \item überwältigt
  \item unterdrückt
  \item unter Druck
  \item unverstanden
  \item unwichtig
  \item unzufrieden
  \item verarscht
  \item zugeneigt
\end{enumerate}


\section{Übungen zu Bitten}


\subsection{Hilfreiche Bitten formulieren}

Entscheidet in eurer Kleingruppe, ob diese Sätze eindeutig darum bitten, dass die andere Person eine konkrete Handlung ausführt. Überlegt euch bei den Sätzen, die ihr für keine hilfreichen Bitten haltet, eine mögliche hilfreiche Bitte für das Bedürfnis, dass sich die Person damit möglicherweise erfüllen möchte.

Teilt eure Entscheidungen und Ideen danach im Plenum.


\begin{enumerate}
  \item Bitte nenne mir etwas an meiner Arbeit, das du an mir wertschätzt.
  \item Bitte respektiere meine Privatsphäre.
  \item Hör bitte mit dem Trinken auf.
  \item Ich hätte gerne, dass du mich verstehst.
  \item Ich hätte gerne, dass du öfter den Müll rausbringst.
  \item Ich möchte, dass du dich mir gegenüber respektvoll verhältst.
  \item Ich möchte dich gerne besser kennenlernen.
  \item Im neuen Jahr will ich mehr Sport machen.
  \item Könntest du bitte am Dienstag die Spülmaschine ausräumen?
  \item Würdest du mir bitte sagen, wie du mich gerade verstanden hast?
\end{enumerate}
