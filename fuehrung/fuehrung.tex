\documentclass[a4paper,openany,twoside,titlepage,10pt,headsepline]{scrbook}

%----------------------------------------------------------------------------------
% Typografie und Fonts
%----------------------------------------------------------------------------------

% Vektorfonts statt Pixelfonts
\usepackage[T1]{fontenc}
\usepackage{lmodern}

% Input-Encoding für UTF-8
\usepackage[utf8]{inputenc}

% Anderer Sans-serif-Font
% https://www.tug.org/FontCatalogue/allfonts.html
% https://www.draketo.de/anderes/latex-fonts.html
\usepackage{librefranklin}
\renewcommand{\familydefault}{\sfdefault}

\newcommand{\fett}[1]{\textsf{\textbf{#1}}}


%----------------------------------------------------------------------------------
% Mehrspachigkeit
%----------------------------------------------------------------------------------

% Mehrsprachigkeit für Deutsch und Englisch erlauben
\usepackage[ngerman,english]{babel}

% Anführungszeichen sprachabhängig machen
\usepackage[babel]{csquotes}


%----------------------------------------------------------------------------------
% Größen, Abstände und Einzüge
%----------------------------------------------------------------------------------

% Mehr von der Seite nutzen
\usepackage[top=2cm, bottom=3cm, right=3cm, left=2cm]{geometry}

% Absätze werden nicht eingezogen, sondern vertikal abgesetzt
\setlength{\parindent}{0mm}
\addtolength{\parskip}{0.5em}

% Descriptions ohne Einzug
\renewenvironment{description}[1][0pt]
{\list{}{
    \labelwidth=0pt \leftmargin=#1
    \let\makelabel\descriptionlabel
}
}
{\endlist}

% Im Zweifel die Seite nicht komplett füllen, aber keine zusätzlichen vertikalen Abstände hinzufügen
\raggedbottom

% Hurenkinder und Schusterjungen verhindern
\clubpenalty10000
\widowpenalty10000
\displaywidowpenalty=10000

%----------------------------------------------------------------------------------
% Literaturverzeichnis, Links und Querverweise
%----------------------------------------------------------------------------------

% Bibliographieeinstellungen
\bibliographystyle{alphadin}

% klickbare Verweise
\usepackage[pdftex,plainpages=false,pdfpagelabels]{hyperref}

% nette URLs
\usepackage{url}


%----------------------------------------------------------------------------------
% Dokument- und Seitenstruktur
%----------------------------------------------------------------------------------

% zweispaltiges Layout möglich machen
\usepackage{multicol}

% Seiten-Kopfzeilen und -Fußzeilen
\usepackage{scrlayer-scrpage}

% Maximal drei Ebenen nummerieren
\setcounter{secnumdepth}{2}

% Maximale 2 Ebenen im Inhaltsverzeichnis
\setcounter{tocdepth}{1}

% Mit jeder Section eine neue Seite anfangen
% https://tex.stackexchange.com/questions/9497/start-new-page-with-each-section
\usepackage{etoolbox}
\preto{\section}{%
  \ifnum\value{section}=0 \else\clearpage\fi
}


%----------------------------------------------------------------------------------
% Grafiken, Farben und Boxen
%----------------------------------------------------------------------------------

% Farben
\usepackage{xcolor}

% Grafiken
\usepackage[pdftex]{graphicx}

% Boxen inklusive Schattierung
\usepackage{framed}
\definecolor{shadecolor}{rgb}{0.8,0.8,0.8}


\AtBeginDocument{\selectlanguage{ngerman}}

\title{Laterale Führung und Führungdsstile}
\author{Oliver Klee\\\texttt{www.oliverklee.de}\\\texttt{seminare@oliverklee.de}}
\date{Version vom \today}

\begin{document}

\frontmatter

\maketitle

\tableofcontents


\mainmatter

\chapter{Seminar-Handwerkszeug}
\section{Regeln für den Workshop}
\label{gfk-workshopregeln}
\index{Workshopregeln}

\paragraph{Vegas-Regel:} Was wir hier persönlichen Dingen teilen, bleibt im Workshop. Wir erzählen Dinge nur anonymisiert nach außen.

\paragraph{Keine dummen Fragen:} Es gibt keine dummen Fragen. Für Fragen, die nicht gut in den Rahmen des aktuellen Themas passen, haben wir einen Themenkühlschrank.

\paragraph{Joker-Regel:} Wir alle versuchen, uns auf dem Workshop gut um uns selbst zu kümmern. Wenn wir etwas brauchen, sprechen wir es an oder sorgen selbst dafür.

\paragraph{Aufrichtigkeit:} Wir tun unser Bestes, uns ehrlich und aufrichtig miteinander umzugehen.

\paragraph{Konstruktiv sein:} Wir tun unser Bestes, konstruktiv miteinander umzugehen und uns gut zu behandeln.

\section{Paarinterview zum Kennenlernen}
\label{paarinterview}

\begin{itemize}
 \item Wo und wie wohne ich?
 \item Was mache ich in Beruf und Ehrenamt so? Und was habe ich bisher so gemacht?
 \item Was sind ein paar Dinge, die mir im Leben zurzeit Freude bereiten?
 \item Was brauche ich (von anderen Personen oder der Umgebung), damit die Zusammenarbeit mit mir gut funktioniert?
 \item Was sollten andere Menschen über mich wissen, wenn sie mit mir zusammenarbeiten?
 \item Was mache ich, um trotz der aktuellen Krisen psychisch halbwegs gesund zu bleiben?
 \item Was ist ein \emph{Guilty Pleasure}, dem ich ab und an fröne?
\end{itemize}

\section{Das Blitzlicht}

\begin{itemize}
  \item wird nicht visualisiert
  \item jede Person spricht nur für sich selbst
  \item keine Diskussion (Ausnahme: wichtige Verständnisfragen)
  \item nicht unterbrechen
  \item wer anfängt, fängt an
  \item kurz~-- ein Blitzlicht ist kein Flutlicht
\end{itemize}


\chapter{Kommunikation}
\section{Prinzipien der Kommunikation}
\label{kommunikationsprinzipien}
\index{Kommunikation}
\index{Missverständnisse}
\index{Sender-Empfänger-Modell}

\begin{itemize}
  \item Es gibt bei Kommunikation immer sendende und (mindestens) eine empfangende Partei (Sender-Empfänger-Modell nach Shannon und Weaver). Diese Rollen können in einer Interaktion öfter wechseln.
  \item Die Partei, die etwas von der anderen will, hat die Zuständigkeit dafür dafür (und das Interesse daran), dass die Kommunikation erfolgreich ist.
  \item Jede Partei hat nur auf ihre eigene Hälfte der Kommunikation direkten Einfluss.
  \item Missverständnisse passieren, und sie sind eher die Regel denn die Ausnahme~-- wir bemerken sie nur oft nicht.
\end{itemize}


\subsection{Metakommunikation}
\label{metakommunikation}
\index{Metakommunikation}

\emph{Metakommunikation} (\glqq Kommunikation über Kommunikation\grqq) bedeutet, die Kommunikation auf eine höhere Ebene zu verlagern und darüber zu reden, wie wir miteinander reden, wie wir miteinander umgehen und was uns beschäftigt.

\section{Direkte vs.~indirekte Kommunikation}
\label{direkte-kommunikation}
\index{direkte Kommunikation}
\index{indirekte Kommunikation}

Bei \fett{direkter Kommunikation} sagt die Person das, was sie kommunizieren möchte, explizit mit ihren Worten.

Bei \fett{indirekter Kommunikation} benutzt die Person stattdessen Mehrdeutigkeit, Anspielungen, den Tonfall, den Rhythmus der Sprache, Gestik oder Mimik, Handlungen, Ironie oder Sarkasmus.

Direkte Kommunikation zu nutzen, senkt das Risiko für Missverständnisse deutlich. Außerdem können wir damit mehr Verantwortung für unsere eigene Kommunikation übernehmen.

GfK setzt sehr stark auf direkte Kommunikation.

\subsection{Beispiele}

\renewcommand{\arraystretch}{2.0}
\begin{tabular}{|p{20em}|p{20em}|}
\hline

\fett{direkte Kommunikation} & \fett{indirekte Kommunikation} \\
\hline

Ich würde gerne mit meinem Freund einen Abend zu zweit verbringen. Wärst du bereit, morgen Abend von 20 bis 23 Uhr die WG zu verlassen? &
Hättest du Lust, morgen Abend ohne mich ins Kino zu gehen? \\
\hline

Mir ist kalt. Wäre es okay, wenn ich das Fenster zumache? &
Ziemlich kalt hier. \\
\hline

Ich bin gerade echt genervt. &
(knurrt) \\
\hline

Könntest du die Musik vielleicht etwas leiser machen? &
Tolle Musik! \\
\hline

Ich habe im Moment echt wenig Geld. Zurzeit kaufen wir ein Kilo Kaffeebohnen im Monat. Können wir uns dazu mal zusammensetzen, wie wir unsere Ausgaben für Kaffee senken können? &
Du trinkst zu viel Kaffee. \\
\hline

Könntest du bitte jeden zweiten Tag den Müll runterbringen? &
(stellt dem Mitbewohner den vollen Mülleimer vor die Zimmertür) \\
\hline

Könntest du bitte damit aufhören, mit dem Kuli zu klicken? &
(knallt die Kaffeetasse auf den Schreibtisch) \\
\hline

\end{tabular}
\renewcommand{\arraystretch}{1.0}

\section{Feedback: Tipps und Tricks}
\label{feedback-regeln}
\index{Feedback}
\index{Feedbackregeln}

\subsection{Was ist Feedback?}
Feedback ist für euch eine Gelegenheit, in kurzer Zeit viel über euch selbst zu lernen. Feedback ist ein Anstoß, damit ihr danach an euch arbeiten könnt (wenn ihr wollt).

Feedback heißt, dass euch jemandem einen persönlichen, subjektiven Eindruck in Bezug auf konkrete Punkte mitteilt. Da es sich um einen persönlichen Eindruck im Kopf eines einzelnen Menschen handelt, sagt Feedback nichts darüber aus, wie ihr tatsächlich wart. Es bleibt allein euch selbst überlassen, das Feedback, das ihr bekommt, für euch selbst zu einem großen Gesamtbild zusammenzusetzen.

Es kann übrigens durchaus vorkommen, dass ihr zur selben Sache von verschiedenen Personen völlig unterschiedliches (oder gar gegensätzliches) Feedback bekommt.

Es geht beim Feedback \emph{nicht} darum, euch mitzuteilen, ob ihr ein guter oder schlechter Mensch, ein guter Redner, eine schlechte Rhetorikerin oder so seid. Solche Aussagen haben für euch keinen Lerneffekt. Stattdessen schrecken sie euch ab, Neues auszuprobieren und dabei auch einmal so genannte Fehler zu machen.

Insbesondere ist Feedback keine Grundsatzdiskussion, ob das eine oder andere Verhalten generell gut oder schlecht ist. Solche Diskussionen führt ihr besser am Abend bei einem Bierchen.

\subsection{Feedback geben}
\begin{itemize}
  \item  "`ich"' statt "`man"' oder "`wir"'
  \item die \emph{eigene} Meinung sagen
  \item die andere Person direkt ansprechen: "`du/Sie"' statt "`er/sie"'
  \item eine konkrete, spezifische Beobachtung schildern
  \item nicht verallgemeinern
  \item nicht analysieren oder psychologisieren (nicht: "`du machst das nur, weil \ldots"')
  \item Feedback möglichst unmittelbar danach geben
  \item konstruktiv: nur Dinge ansprechen, die die andere Person auch ändern kann
\end{itemize}

\subsection{Feedback entgegennehmen}
\begin{itemize}
  \item vorher den Rahmen für das Feedback abstecken: Inhalt, Vortragstechnik, Schriftbild \ldots
  \item gut zuhören und ausreden lassen
  \item sich nicht rechtfertigen, verteidigen oder entschuldigen
  \item Missverständnisse klären, Hintergründe erläutern
  \item Feedback als Chance zur Weiterentwicklung sehen
\end{itemize}

\section{Feedback, Kritik, Bitten, Forderungen und Grenzen}
\label{feedback-vs-kritik}

\emph{Feedback}, \emph{Kritik}, \emph{Bitten}/\emph{Forderungen} und das \emph{Setzen von Grenzen} sind unterschiedliche Sachen, und es ist hilfreich, wenn ihr sie klar unterscheiden könnt, um dann bewusst auszuwählen und klar zu kommunizieren.


\subsection{Feedback}
\index{Feedback}

Das Ziel von Feedback ist, dass die andere Person daraus lernen und daran wachsen kann.

Bei Feedback geht es um die andere Person, nicht um euch.

Feedback ist ein Geschenk, und es ist okay, Geschenke nicht annehmen zu wollen.

Ein Geschenk kann man nicht aufzwingen~-- jemandem ohne Zustimmung Feedback zu geben, ist Gewalt.


\subsection{Bitten und Forderungen}
\index{Bitten}
\index{Forderungen}

Bei einer Bitte oder Forderung geht es darum, dass ihr etwas für euch von der anderen Person braucht oder möchtet.

Dort liegt der Fokus also bei euch, nicht bei der anderen Person.

Der wesentliche Unterschied zwischen Bitten und Forderungen ist, was bei einem Nein oder einer Ablehnung passiert, und wie weit ihr zu Verhandlungen bereit seid.

Mehr zu Kriterien für hilfreiche Bitten findet ihr ab Seite~\pageref{bitten}.


\subsection{Kritik}
\index{Kritik}

Wenn es tatsächlich darum geht, zusammen an einer Sache zu arbeiten oder zu klären, kann Kritik an der Sache (nicht an der Person!) hilfreich sein. In diesem Fall liegt der Fokus auf der Sache, nicht auf einer Person.

Ansonsten ist Kritik oft undeutlich kommuniziertes Feedback oder eine unklar kommunizierte Bitte. In solchen Fällen ist es hilfreicher, stattdessen klar Feedback oder eine Bitte zu äußern.


\subsection{Grenzen setzen}
\index{Grenzen}

Bei persönlichen Grenzen geht es darum, wozu ihr bereit seid~-- also was ihr zu tun bereit seid und was ihr mit euch zu machen bereit seid.

Das Ziel von euren persönlichen Grenzen ist, euch zu schützen. Persönliche Grenzen sind daher auch nicht verhandelbar.

Die Verantwortung, eure Grenzen zu wahren und zu verteidigen, liegt letztendlich bei euch selbst. Notfalls entfernt ihr euch aus der Situation, ruft die Polizei oder Ähnliches.

Wenn ihr Grenzen kommuniziert, ist dies daher keine Bitte oder Forderung an die andere Person, sondern vor allem eine Information.

\section{Das Johari-Fenster}
\label{johari-fenster}
\index{Johari-Fenster}

Das \emph{Johari-Fenster}~\cite{johari-window, gruppendnamik-einfuehrung} (benannt nach den Erfindern Joseph Luft und Harry Ingham) ist ein Modell zum Abgleich von Selbst- und Fremdwahrnehmung.

Es wird als \glqq Fenster\grqq\ bezeichnet, weil die Tabelle mit den vier Feldern einem alten Fenster mit Fensterkreuz ähnelt. (Die Älteren unter uns können sich vielleicht noch erinnern.)

\subsection{Die Bereiche des Johari-Fensters}

\renewcommand{\arraystretch}{2.0}
\begin{tabular}{p{10em}|p{10em}|p{7em}|}
& mir bekannt \cellcolor{lightgray}  & mir unbekannt \cellcolor{lightgray}
\\ \hline

anderen bekannt \cellcolor{lightgray} & öffentliche Person & blinder Fleck
\\ \hline

anderen unbekannt \cellcolor{lightgray} & mein Geheimnis & Unbekanntes
\\ \hline
\end{tabular}
\renewcommand{\arraystretch}{1.0}


\subsection{Bereiche des Johari-Fensters verkleinern}

Durch \fett{Feedback} könnt ihr den blinden Fleck verkleinern.
\index{Feedback}

Durch \fett{Selbstkundgabe} könnt ihr euer Geheimnis verkleinern.
\index{Selbstkundgabe}

Durch \fett{Therapie} könnt ihr das Unbekannte verkleinern.
\index{Therapie}


\chapter{Führung}
\section{Die Rolle der Führung}
\label{fuehrung-rolle}
\index{Führung: Rolle}


\subsection{Aufgaben der Führung nach Neuberger}

Laut Neuberger\cite{neuberger-fuehren} sind die Aufgaben der Führung,

\begin{itemize}
  \item andere Menschen
  \item zielgerichtet
  \item in einer formalen Organisation
  \item unter konkreten Umweltbedingungen dazu bewegen,
  \item Aufgaben zu übernehmen und erfolgreich auszuführen,
  \item wobei humane Ansprüche gewahrt werden.
\end{itemize}


\subsection{Neuberger, aber modernisiert}

Auf das moderne Arbeiten übertragen, wäre die Aufgabe der Führung,

\begin{itemize}
  \item eine Umgebung zu schaffen,
  \item die es einem Team oder einer Organisation möglich und leicht macht,
  \item für die Mission des Teams oder der Organisation zu arbeiten,
  \item wobei die Menschen nachhaltig körperlich und seelisch gesund zu bleiben
  \item und ihr Potenzial nutzen können.
\end{itemize}

(Dies ist meine eigene \glqq Übersetzung\grqq{} aus Perspektive der modernen Führung.)


\subsection{Aufgaben der Führung nach Malik}

Dies sind laut Fredmund Malik \cite{malik-fuehrung} die Aufgaben der Führung:

\begin{itemize}
  \item für Ziele sorgen
  \item organisieren
  \item entscheiden
  \item kontrollieren
  \item Menschen entwickeln und fördern
\end{itemize}

Auf moderne Führung übertragen, wäre es die Aufgabe der Führung, dafür zu sorgen, dass diese Dinge stattfinden (also dass beispielsweise das Team Entscheidungen fällen und nachhalten kann), und nicht zwangsläufig, dass die Führung das auch selbst entscheidet.

\section{Führungsstile}
\label{fuehrungsstile}
\index{Führungsstile}


\subsection{Tradierte Führungsstile}
\index{Tradierte Führungsstile}

Diese Führungsstile nach Max Weber~\cite{weber-wirtschaft-gesellschaft} sind nicht mehr aktuell und eher aus historischer Perspektive interessant.

\paragraph{Autokratisch:} uneingeschränkte Macht
\paragraph{Patriarchalisch:} Machtfülle/Vaterrolle
\paragraph{Charismatisch:} Leitfigur/Vorbild
\paragraph{Bürokratisch:} durch Strukturen; die Personen sind austauschbar


\subsection{Klassische Führungsstile}
\index{Klassische Führungsstile}

Diese Führungsstile nach Kurt Lewin~\cite{lewin-fuehrungsstile} sind immer noch aktuell.

\paragraph{Autoritär:} Die Aktivität liegt ausschließlich bei der Führungsperson. Die \glqq Untergebenen\grqq\ haben Weisungen zu akzeptieren und auszuführen. Dieser Stil entspricht dem tradierten autokratischen Stil.
\index{Autoritärer Führungsstil}

\paragraph{Kooperativ/demokratisch:} Die Führungsperson zieht die Mitarbeitenden in die Entscheidungen mit ein. Aus Kontrolle wird zunehmend Selbstkontrolle. Mitarbeitende können die Führung kritisieren
\index{Autoritärer Führungsstil}

\paragraph{Karitativ/partizipativ:} Diese Stil orientiert sich vorrangig an den Bedürfnissen der Mitarbeitenden. Der Mensch steht im Mittelpunkt; die Aufgaben sind dagegen nachrangig. Die Führungsperson hört viel zu, fördert und ermutigt.
\index{Karitativer Führungsstil}
\index{Partizipativ Führungsstil}

\paragraph{Laissez-faire:} Die Führungsperson überträgt/delegiert die Aufgaben, kümmert sich um die Arbeitsmittel und definiert klare Ziele.
\index{Laissez-faire}


\subsection{Situatives Führen}
\index{Situatives Führen}

\emph{Situatives Führen} nach Hersey~\cite{hersey-management} bezeichnet, dass eine Führungsperson je nach Situation unterschiedliche Führungsstile wählt, um erfolgreich zu führen.

\section{Managementtechniken}
\label{managementtechniken}
\index{Managementtechniken}
\index{Management-by-Techniken}

Dies sind einige (von vielen) Managementtechniken (oder \emph{Management-by}-Techniken), die ihr in Kombination mit verschiedenen Führungsstilen nutzen könnt.

\paragraph{Management by Delegation} bedeutet, Aufgaben und die Verantwortung dafür an die Mitarbeitenden zu delegieren.
\index{Management by Delegation}

\paragraph{Management by Direction and Control} bedeutet, dass Vorgesetzte die Entscheidungen treffen, Anweisungen geben und die Ausführung kontrollieren. Sie entspricht ungefähr dem autoritären Führungsstil.
\index{Management by Direction and Control}

\paragraph{Management by Exception} bedeutet, dass die Mitarbeitenden die Routineentscheidungen treffen und die Führungsperson nur in Ausnahmefällen selbst entscheidet.
\index{Management by Exception}

\paragraph{Management by Objectives} bedeutet, dass Teams und Mitarbeitende Ziele vorgegeben bekommen oder Zielvereinbarungen treffen, die sie dann selbstständig erfüllen.
\index{Management by Objectives}

\paragraph{Management by Systems} bedeutet, dass die Arbeit in einem sich selbst tragenden System erfolgt, bei dem die Führungsperson nur noch das System regelt.
\index{Management by Systems}

\section{Führungsinstrumente (Führungswerkzeuge)}
\label{fuehrungsinstrumente}
\index{Führungswerkzeuge}
\index{Führungsinstrumente}


\subsection{Was ist ein Führungsinstrument?}

Ein Führungsinstrument (oder Führungswerkzeug) generell alles, was eine Führungskraft tun kann, um direkt oder indirekt zusammen mit den geführten Personen Ziele zu erreichen.


\subsubsection{Führungswerkzeuge nach Malik}

Diese Liste von Fredmund Malik \cite{malik-fuehrung} ist schon etwas angestaubt. Sie lässt sich allerdings gut in die heutige Zeit übertragen.

\begin{itemize}
  \item Besprechung
  \item Schriftstück
  \item Stellengestaltung und Einsatzsteuerung
  \item Persönliche Arbeitsmethodik
  \item Budget und Budgetierung
  \item Leistungsbeurteilung
  \item systematische Müllabfuhr
\end{itemize}


\subsection{Direkte Führungsinstrumente}

\begin{itemize}
  \item 1-zu-1-Gespräche (siehe Seite~\pageref{1-zu-1})
  \item Anweisung
  \item Besprechung
  \item delegieren
  \item Entscheidungen treffen
  \item Feedback einholen
  \item Feedback geben
  \item informieren
  \item Konflikte klären
  \item kontrollieren
  \item Kritik
  \item Lob
  \item um etwas bitten
  \item Wertschätzung ausdrücken
  \item Ziele vereinbaren
\end{itemize}


\subsection{Indirekte Führungsinstrumente}

\begin{itemize}
  \item Anreizsysteme schaffen
  \item dem Team Workshops und andere Fortbildungen anbieten
  \item den (physischen) Arbeitsplatz gestalten
  \item die Motivation verbessern
  \item die psychologische Sicherheit verbessern
  \item eine Mission definieren
  \item einen Spieleabend mit dem Team veranstalten
  \item ein Team zusammenstellen
  \item Gewaltfreie Kommunikation lernen und anwenden
  \item mit dem Team einen Escape-Room spielen
  \item mit dem Team lecker essen gehen
  \item Prozesse definieren
  \item Rollen und Verantwortlichkeiten definieren
  \item Supervision für das Team organisieren
\end{itemize}

\section{Führen lernen}
\label{fuehren-lernen}
\label{fuehrung-lernen}
\index{Führung lernen}

Meiner Ansicht nach ist Führen zu lernen so ähnlich wie singen zu lernen.

Dafür sind diese Dinge notwendig:

\begin{itemize}
  \item viel \fett{üben} (und dabei aus Fehlern lernen)
  \item sehr viel \fett{Reflexion} \index{Reflexion}
  \item \fett{Außenwahrnehmung} bekommen in der Form von Feedback \index{Außenwahrnehmung} \index{Feedback}
  \item an \fett{Trainings} und \fett{Workshops} teilnehmen (oder anderweitig Unterricht nehmen)
  \item \fett{Bücher} oder anderen Quellen von Wissen konsumieren
  \item von \fett{guten Beispielen} lernen
\end{itemize}

Damit ihr andere Menschen gut führen könnt, ist es außerdem notwendig, dass ihr euch selbst gut kennt und versteht, wie ihr tickt und was euch antreibt. Dies könnt ihr durch diese Dinge (oder eine Kombination daraus) erreichen:

\begin{itemize}
  \item Gewaltfreie Kommunikation lernen \index{Gewaltfreie Kommunikation}
  \item eine Psychotherapie machen (Tiefenpsychologie oder Psychoanalyse; keine kognitive Verhaltenstherapie) \index{Therapie} \index{Psychotherapie}
\end{itemize}

Hilfreich zum kontinuierlichen Lernen ist außerdem eine Supervision, Intervision oder kollegiale Fallberatung.

\section{Laterale Führung}
\label{laterale-führung}
\index{laterale Führung}


\subsection{Definition}

Laterale Führung (\glqq zur Seite gerichtet führen\grqq) bedeutet \fett{Führen ohne direkte Weisungsbefugnis}.~\cite{kuhl-laterales-fuehren, fuehren-ohne-fuehrung, gtd-when-not-in-charge}

\subsection{Anlässe für laterales Führen}

\begin{itemize}
  \item Ihr arbeitet mit \fett{Ehrenamtlichen}.
  \item Ihr arbeitet mit \fett{Scrum} und seid Scrum-Master\_in oder Product-Owner\_in.
  \item Ihr seid Teil des Teams, und die Führungsperson hat bewusst \fett{Führungsaufgaben an euch abgegeben}.
  \item Ihr seid Teil eines \fett{selbstorganisierten Teams} ohne dedizierte Führungsperson.
  \item Ihr arbeitet \fett{bereichs- oder organisationsübergreifend} zu einem Projekt oder in Prozessketten zusammen.
  \item Ihr seid in einem \fett{Lehr- oder Mentoring-Verhältnis} zur anderen Person.
  \item Ihr seid eurem Team gegenüber zwar grundsätzlich weisungsbefugt, entscheidet euch aber bewusst, diese \fett{Macht nicht zu nutzen}.
  \item Worst-Case: Ihr habt eine Führungsperson, die aber ihre Führungsaufgabe aus Unwillen oder wegen mangelnder Kompetenz nicht ausfüllt, und der Laden muss trotzdem irgendwie laufen.
\end{itemize}
\index{Inkompetenz}
\index{Kompetenz}


\subsection{Laterale Führung vs.~\emph{Servant Leadership}}
\index{Servant Leadership}

Während das Konzept von lateraler Führung dadurch definiert ist, dass keine direkte Weisungsbefugnis besteht (das Konzept also eher \emph{deskriptiv} ist), beschreibt \emph{Servant Leadership} (\glqq dienende Führung\grqq) ~\cite{fuehren-durch-dienen} den Ansatz, dass die Führung die Geführten in ihrer Arbeit unterstützen und ermächtigen soll (wodurch das Konzept also eher \emph{normativ} ist).

Beide Konzepte ergänzen sich ganz ausgezeichnet: In einer Situation der lateralen Führung fahrt ihr mit \emph{Servant Leadership} sehr gut.


\subsection{Die drei Säulen der lateralen Führung}

\index{Verständigung}
\paragraph{Verständigung} darüber, wie die Zusammenarbeit funktioniert und wohin der Weg geht.

\index{Macht}
\paragraph{Macht} um Blockaden zu überwinden.

\index{Vertrauen}
\index{Beziehungen}
\paragraph{Vertrauen} erlauben den Menschen, bei der Zusammenarbeit in Vorleistung zu gehen.


\subsubsection{Zusammenspiel der drei Säulen}

Diese Prozesse laufen gleichzeitig ab und sind nicht immer eindeutig zu erkennen.

Sie können sich gegenseitig verstärken, und ihr könnt einen Mangel bei einer Säule durch eine der anderen Säulen teilweise ausgleichen: Zum Beispiel könnt ihr (insbesondere bei sehr lateraler Führung) wenig Macht durch gute Absprachen (Verständigung) und bewusst gepflegte Beziehungen (Vertrauen) ausgleichen.

Gleichzeitig können sich auch Prozesse gegenseitig behindern: Wenn ihr viel Macht einsetzt, kann das das Vertrauen verringern, das die andere Person in euch hat.

Wie diese drei Säulen bei euch aussehen, hängt zuerst von den Strukturen eurer Organisation und eures Teams ab. Ihr als Führungsperson habt darauf aufbauen direkten Einfluss darauf (und Verantwortung dafür).


\subsection{Verständigung}
\index{Verständigung}

Was ist Verständigung, und warum ist sie wichtig?

\begin{itemize}
  \item Erwartungen der Gruppe, in die man eingebunden ist
  \item Standards, Normen und Auffassungen
  \item gemeinsames Verständnis darüber, wie man arbeitet
  \item mobilisiert die Ansichten, Erfahrungen und Interessen vieler Personen
  \item reduzieren die Motivations- und Kontrollprobleme der Führung
\end{itemize}

Dies sind einige Instrumente, mit denen ihr zu mehr Verständigung beitragen könnt:

\begin{itemize}
  \item klare definierte und dokumentierte Rollen und Verantwortlichkeiten \index{Rollen}\index{Verantwortlichkeiten}
  \item Teamregeln oder -vereinbarungen \index{Teamregeln}\index{Regeln}\index{Vereinbarungen}
  \item klar dokumentierte Prozesse \index{Prozesse}
  \item ein Mission-Statement \index{Mission-Statement}
  \item eine gute Agenda für jedes Meeting (\glqq\emph{No agenda, no attenda.\grqq} \index{Agenda}
  \item klare, SMARTe Ziele
\end{itemize}

Wichtig für die Verständigung ist auch zu wissen, dass im Zweifel für die Einzelnen die \glqq lokalen Realitäten\grqq\ (also wie ihr im Team oder der Abteilung arbeitet) relevanter sind als das, was für die gesamte Organisation gilt.


\subsection{Macht}
\index{Macht}

Macht …

\begin{itemize}
  \item ist Teil jeder Beziehung
  \item ist Mechanismus, mit dem man bei anderen ein Verhalten erzeugt, das sie spontan nicht eingenommen hätten
  \item ist nur möglich, wenn irgendeine Beziehung zwischen den Beteiligten besteht
  \item basiert darauf, dass sie von den Beteiligten geteilt und (mehr oder minder) akzeptiert wird
\end{itemize}

Macht basiert im Wesentlichen darauf, dass jemand etwas Bestimmtes hat oder kontrolliert:

\begin{itemize}
  \item Hierarchie
  \item Fachwissen, Expertise
  \item Relaisstellen: Zugang zu Personen etc.
  \item Gatekeeper zu Informationen und Kommunikationskanälen
\end{itemize}


\subsection{Vertrauen}
\index{Vertrauen}

Kooperation ist riskant, weil wir damit ständig in Vorleistung gehen und von der anderen Person abhängig sind. Damit dies funktionieren kann, ist gegenseitiges Vertrauen zwingend notwendig.

Mehr dazu in einem separaten Kapitel.

\section{Die Team-Charta}
\label{team-charta}
\index{Charta}
\index{Team-Charta}


\subsection{Sinn der Team-Charta}

Eine Team-Charta (englisch Schreibweise: \emph{team charter}) ist ein zentrales Dokument, in die Mission und die Strukturen eines Teams festgehalten sind.


\subsection{Inhalte einer Team-Charta}

\begin{itemize}
  \item Name des Teams
  \item Mission-Statement
  \item Regeln oder Vereinbarungen
  \item Rollen
  \item wie und was das Team was kommuniziert
  \item wie in dem Team Entscheidungen gefällt werden
\end{itemize}

Optionale Punkte:

\begin{itemize}
  \item Kultur des Teams
  \item wichtige Punkte zur Organisation
  \item wo weiterführende Dokumentation zu finden ist
\end{itemize}

\section{Mission-Statement}
\label{mission-statement}
\index{Mission-Statement}

Eine komplette Organisation kann ein Mission-Statement haben.~\cite{corporate-purpose} Innerhalb der Organisation können einzelne Teams dann noch einmal als \glqq lokale Realität\grqq\ eigene Mission-Statements haben, die auf das Mission-Statement der Organisation einzahlen.

Ein Mission-Statement ist vor allem nach innen gerichtet.

\subsection{Sinn und Zweck eines Mission-Statements}

\begin{itemize}
  \item motivieren
  \item Entscheidungshilfe für Menschen sein, die überlegen, bei euch mitzuarbeiten
  \item Entscheidungshilfe sein, ob diese Umgebung mit euren Werten und Antreibern harmoniert und ihr vielleicht wechseln möchtet
  \item Entscheidungshilfe für Maßnahmen sein, wenn eine Person gegen die Mission arbeitet
  \item als Kompass bei alltäglichen Entscheidungen dienen: \glqq Zahlt diese Option auf unsere Mission ein?\grqq
  \item die Richtung bei der Planung der Aufgaben und Projekte vorgeben
  \item Grundlage für das sein, was ihr über über eure Organisation nach außen kommunizieren möchtet (z.\,B.~auf eurer Website)
\end{itemize}


\subsection{Elemente eines Mission-Statements}

Optimalerweise besteht ein Mission-Statement aus drei Abschnitten, sortiert von \emph{abstrakt} zu \emph{konkret}.

\paragraph{Motivation/Purpose (Daseinszweck):} \emph{Warum} gibt es uns?
\index{Motivation}
\index{Purpose}
\index{Daseinszweck}
\index{Zweck}

Dieser Punkt soll vor allem motivieren.

\paragraph{Vision:} \emph{Wie} arbeiten wir grundsätzlich auf unseren \emph{Purpose} hin?
\index{Vision}
\index{Kompass}

Dieser Punkt soll vor allem als Kompass bei täglichen Entscheidungen dienen.


\paragraph{Mission:} \emph{Was} tun wir konkret, um unsere Vision voranzutreiben?
\index{Mission}

Dieser Punkt soll vor allem bei der Planung der konkreten Aufgaben helfen.

\section{Rollen und Verantwortlichkeiten definieren}
\label{rollen}
\index{Rollen}
\index{Verantwortlichkeiten}

Eine Rolle definiert einen Bereich, für das eine Person verantwortlich ist, und wodurch sie zusätzliche Berechtigungen erhält. Eine Person kann dabei mehrere Rollen gleichzeitig ausfüllen, und mehrere Personen können sich auch eine Rolle teilen.

Ungeklärte Rollen sind eine häufige Ursache von Konflikten.

Klar definierte Rollen tragen bei lateraler Führung zur Verständigung bei.


\subsection{Rollen definieren}

Ein Teil dieser Anleitung kommt aus dem \fett{AKV-Prinzip} zur Analyse von Aufgaben, Kompetenzen und Verantwortlichkeiten.~\cite{kessler-projektmanagement}
\index{AKV-Prinzip}

Das hier sollte in einer guten Rollendefinition stehen:

\paragraph{Bezeichnung:} Wie benennen wir die Rolle in der Kommunikation?

\paragraph{Daseinszweck:} Warum brauchen/wollen wir diese Rolle überhaupt? Wie zahlt diese Rolle auf unsere Mission ein?
\index{Mission}

\paragraph{Dauer:} Ist dies eine dauerhafte oder vorübergehende Rolle? (Die Moderation oder Protokollführung eines Meetings sind beispielsweise oft temporäre Rollen.)

\paragraph{Aufgaben:} Was sind die 3--5 wichtigsten Tätigkeiten dieser Rolle? Und wofür sind Menschen dadurch Ansprechpersonen?
\index{Aufgaben}

\paragraph{Befugnisse/Kompetenzen:} Was dürfen diese Personen in dieser Rolle tun, was sie sonst nicht dürften?
\index{Befugnisse}
\index{Kompetenzen}

Beispiele: Verträge unterschreiben, Leute unterbrechen, Dinge (mit-)entscheiden \ldots

\paragraph{Verpflichtungen/Verantwortung:} Wozu verpflichtet diese Rolle? Woran messen wir, ob jemand diese Rolle erfolgreich ausfüllt?
\index{Verpflichtungen}
\index{Verantwortung}

\paragraph{Fähigkeiten/Kenntnisse:} Was muss eine Person fachlich können, wissen oder lernen, um diese Rolle gut erfüllen zu können?
\index{Fähigkeiten}
\index{Kenntnisse}

\section{1-zu-1-Gespräche (\emph{One-on-Ones})}
\label{1-zu-1}
\index{1-zu-1-Gespräche}
\index{One-on-Ones}

\subsection{Ziele}

\begin{itemize}
 \item das Vertrauen und die Beziehung zwischen Teamlead und Teammitglied aufbauen und pflegen
 \item dem Teammitglied und Teamlead die Möglichkeit geben, sich gegenseitig regelmäßig Feedback zu geben
 \item ein Ort für Absprachen sein, damit die Ad-hoc-Absprachen und -Calls zwischendrin weniger werden
 \item Konflikte, Sorgen und Ideen zeitnah besprechen
\end{itemize}

\subsection{Struktur}

\begin{itemize}
 \item Die Gespräche sollten regelmäßig und zuverlässig stattfinden.
 \item Mit Vollzeitangestellten sollten die Gespräche wöchentlich stattfinden. In unserem ehrenamtlichen Sehr-Teilzeit-Team haben wir einen Rhythmus von 4 Wochen.
 \item Die Gespräche haben eine harte Begrenzung auf 30 Minuten, während die einzelnen Slots zeitlich flexibler sind.
 \item Generell sollte das Teammitglied ca.~90\,\% der Redeanteile im Gespräch haben und die Teamführung die restlichen 10\,\%.
\end{itemize}

Alle Themen hier sind nur Vorschläge. Wenn euch andere Themen wichtiger sind, sprecht über diese.

\subsubsection{Checkin}
Wie fühle ich mich gerade? Wie bin ich hier?

\subsubsection{Slot für das Teammitglied (ca.~10 Minuten)}

\paragraph{Allgemeines}
\begin{itemize}
 \item alles, worüber du sprechen möchtest
\end{itemize}

\paragraph{Strategie}
\begin{itemize}
 \item Was tun wir als Team/Organisation nicht, was wir tun sollten?
 \item Wenn wir eine Sache verbessern könnten, welche wäre das?
 \item Wenn du ich wärst, was würdest du verändern?
 \item Feedback an die Teamführung
\end{itemize}

\paragraph{Blick nach außen}
\begin{itemize}
 \item Was ist zur Zeit das größte Problem unseres Teams/unserer Organisation? Und warum?
 \item Was gefällt dir nicht an unseren Produkten und Dienstleistungen? Was könnten wir verbessern? Woran müssen wir noch arbeiten?
 \item Was ist die größte Chance, die wir gerade verpassen?
\end{itemize}

\paragraph{Team}
\begin{itemize}
 \item Mit wem würdest du gerne (mehr) zusammenarbeiten?
 \item Wer macht gerade einen richtig guten Job?
 \item Was macht dir gerade Spaß? Und warum?
 \item Und was macht dir keinen Spaß? Und warum?
 \item Gibt es etwas, vor dem du gerade Angst hast?
\end{itemize}


\subsubsection{Slot für die Teamführung (ca.~10 Minuten)}
\begin{itemize}
 \item alles, worüber du sprechen möchtest
 \item Feedback an das Teammitglied
\end{itemize}


\subsubsection{Persönliche Entwicklung (ca.~10 Minuten, wenn noch Zeit ist)}
\begin{itemize}
 \item Was hast du kürzlich (dazu-)gelernt?
 \item Was hast du kürzlich ausprobiert? Was ist dabei herausgekommen?
 \item Was würdest du gerne lernen?
 \item Was würdest du gerne mal ausprobieren?
 \item Welche Verantwortungsbereiche würdest du gerne annehmen oder abgeben?
 \item Wie kannst du deine Superkräfte am besten einsetzen?
\end{itemize}

\subsection{Quellen}

\begin{itemize}
 \item Podcast: Manager Tools Basics: One on Ones \cite[04.\,07.\,2005 bis 11.\,07.\,2005]{manager-tools-basics}
 \item Podcast: Female Leadership: Gamechanger für Gesprächsführung \cite[Folge 15]{female-leadership-gespraechsfuehrung}
 \item Buch: The Hard Thing About Hard Things: Building a Business When There Are No Easy Answers \cite{the-hard-thing-about-hard-things}
\end{itemize}

\section{Teammitglieder (besser) kennenlernen}
\label{Kennenlernen}
\index{team-kennenlernen}

\subsection{Persönliches}

\begin{multicols}{2}
  \begin{itemize}
    \item vollständiger Vor- und Nachname
    \item Pronomen
    \item Geburtstag (ohne Jahr)
  \end{itemize}
\end{multicols}


\subsection{Wohnen und Verkehr}

\begin{multicols}{2}
  \begin{itemize}
    \item Wohnort
    \item Auto (und Modell)
    \item Hauptverkehrsmittel
    \item Länge des Arbeitswegs
  \end{itemize}
\end{multicols}


\subsection{Arbeit}

\begin{multicols}{2}
  \begin{itemize}
    \item seit wann im Team/der Organisation
    \item Rollen im Team/der Organisation
    \item Betriebssystem (zum Arbeiten)
    \item vorheriger Job
    \item weitere aktuelle Jobs
    \item liebste Aufgaben
    \item Lieblings-Kolleg\_in
  \end{itemize}
\end{multicols}


\subsection{Ausbildung}

\begin{multicols}{2}
  \begin{itemize}
    \item Studium/Abschluss
    \item Ausbildung
    \item weitere Fortbildungen/Zertifizierungen
  \end{itemize}
\end{multicols}


\subsection{Familie}

\begin{multicols}{2}
  \begin{itemize}
    \item aktuelle Partnerschaft(en)
    \item Anzahl Kinder
  \end{itemize}
\end{multicols}


\subsection{Freizeit}

\begin{multicols}{2}
  \begin{itemize}
    \item ehrenamtliches Engagement
    \item Hobbys
    \item Sport
    \item Vorlieben: Serien und Filme
    \item Vorlieben: Bücher, Literaturgenres
    \item Vorlieben: Computerspiele/Gesellschaftsspiele
    \item letzter Urlaub
  \end{itemize}
\end{multicols}


\section{Lob und Tadel ersetzen}
\label{lob-tadel}
\index{Lob}
\index{Tadel}

\subsection{Warum wollen wir Lob und Tadel ersetzen?}

\begin{itemize}
  \item Wenn ihr lobt oder tadelt, dann verpasst ihr damit die Gelegenheit, der anderen Person ein Einblick in das zu geben, was in euch lebendig ist.
  \item Lob und Tadel kommt ihr von oben statt auf Augenhöhe. Das wirkt sich negativ auf eure Beziehung aus.
  \item Die übliche Reaktion auf Tadel ist, das getadelte Verhalten in Zukunft zu vermeiden, anstatt daran zu wachsen.
  \item Wenn ihr jemanden tadelt, ist nicht immer klar, ob der Fokus drauf liegt, dass ihr selbst etwas braucht, oder darauf, dass ihr der anderen Person die Gelegenheit zum Lernen geben möchtet.
\end{itemize}


\subsection{Lob ersetzen}

Gebt stattdessen echte, herzliche \fett{Wertschätzung}. Mehr dazu findet ihr auf Seite~\pageref{wertschaetzung}.


\subsection{Tadel ersetzen}

Wenn der Fokus darauf ist, dass ihr selbst eine \fett{Veränderung für euch} braucht, formuliert eine \fett{Bitte}. Mehr dazu auf Seite~\pageref{bitten}.

Wenn hingegen der Fokus darauf ist, dass ihr der anderen Person (mehr) \fett{Entwicklung ermöglichen} möchtet, dann bietet stattdessen \fett{Feedback} an. Auf Seite~\pageref{feedback-regeln} findet ihr Tipps dazu, wie ihr hilfreiches Feedback geben könnt.

\section{Fallberatung für Führungskräfte}
\label{fallberatung}
\index{fallberatung}

Eine kontinuierliche Fallberatung ist sehr hilfreich, um als Führungskraft weiterzulernen und Unterstützung bei Problemen zu bekommen.


\subsection{Supervision vs.~Intervision vs.~kollegiale Fallberatung}

\paragraph{Supervision}~\cite{supervision} ist eine Form der beruflichen Beratung, die von einer Supervisorin oder einem Supervisor geleitet wird. Die Grenze zur Psychotherapie ist dabei schwer zu ziehen.
\index{Supervision}

\paragraph{Intervision}~\cite{intervision} ist ein aus den Niederlanden stammendes Modell, bei dem sich Kolleg\_innen gegenseitig beraten, wobei sie Unterstützung einer externen Moderation haben.
\index{Intervision}

\paragraph{Kollegiale Fallberatung}~\cite{kollegiale-fallberatung} ist ein aus Deutschland stammendes Modell, bei dem sich Kolleg\_innen gegenseitig beraten, wobei sie keine (externe) Moderation in Anspruch nehmen. Ein sehr verbreitetes Modell dafür ist das \emph{Heilbronner Modell}.
\index{Kollegiale Fallberatung}
\index{Heilbronner Modell}


\subsection{Kollegialen Fallberatung}


\subsubsection{Rollen}

Es sind mindestens diese Rollen notwendig:~\cite{kollegiale-fallberatung}

\begin{itemize}
  \item Moderation
  \item Fallgeber\_in
  \item Fallberater\_in
\end{itemize}


\subsubsection{Grober Ablauf}

\begin{enumerate}
  \item Rollen verteilen:
    \begin{itemize}
      \item Wer moderiert?
      \item Wer möchte beraten werden?
    \end{itemize}
  \item die problematische Situation schildern
  \item die beratenden Personen reagieren und beraten
\end{enumerate}


\subsubsection{Detaillierter Ablauf (Beispiel)}

Dies ist der Ablauf, den ich in meiner eigenen kollegialen Fallberatung nutze.


\begin{enumerate}
  \item Wer moderiert den Anfang?
  \item Checkin: Wie geht es mir? Wie fühle ich mich?
  \item Was seit dem letzten Treffen passiert ist
  \item Organisatorisches
  \item Themen auf Karten schreiben inklusive Wünsche dafür:
    \begin{itemize}
      \item Empathie
      \item Analyse
      \item Ratschläge
    \end{itemize}
  \item Themen kurz vorstellen
  \item Punkten, welche Themen wir gerne bearbeiten würden
  \item Thema für heute auswählen (das mit den meisten Punkten)
  \item Wer moderiert den Hauptteil?
  \item Und wer macht Notizen?
  \item Themen-Geber\_in stellt das Thema vor
  \item Thema bearbeiten
  \item Rückmeldung von Themen-Geber\_in:
    \begin{itemize}
      \item Wie geht es mir jetzt?
      \item Was habe ich gelernt?
      \item Welche Ratschläge nehme ich an?
      \item Was sind meine nächsten Schritte?
    \end{itemize}
  \item Runde: Was habe ich heute gelernt?
  \item Feedback zu Prozess, Ablauf, Moderation
  \item Checkout: Mit welchem Gefühl gehe ich raus?
\end{enumerate}


\chapter{Vertrauen}
\section{Vertrauen vs. psychologische Sicherheit}
\label{vertrauen-vs-ps}
\index{Vertrauen}
\index{psychologische Sicherheit}

Diese Tabelle gibt eine Übersicht darüber, wie sich das Konzept von Vertrauen~\cite{thin-book-of-trust, anatomy-of-trust} von dem von psychologische Sicherheit~\cite{the-fearless-organisation} unterscheidet.

\vspace{1em}

\renewcommand{\arraystretch}{2.0}
\begin{tabular}{|p{5em}|p{19em}|p{19em}|}
\hline
& \fett{Vertrauen} & \fett{psychologische Sicherheit}
\\ \hline

Definition
& \glqq Ich bin bereit, etwas mir Wichtiges zu riskieren, indem ich es in die Hände der anderen Person gebe.\grqq
& \glqq Unsere Arbeitsumgebung ist sicher genug dafür, dass wir dort zwischenmenschliche Risiken eingehen können.\grqq
\\ \hline

zu wem
& von einer einzelnen Person zu einer anderen Person oder Institution &
innerhalb eines Teams oder einer Gruppe
\\ \hline

zeitlicher Fokus
& ausgehend von der Summe der Erfahrungen in der Vergangenheit eine Prognose für die Zukunft
& ausgehend von der Vergangenheit für das Jetzt
\\ \hline

Haupt-Einfluss
& die andere Person/Institution
& die Führungskraft und das Team
\\ \hline

\end{tabular}
\renewcommand{\arraystretch}{1.0}

\section{Elemente von Vertrauen}
\label{vertrauen-element}
\index{Vertrauen: Elemente}


\subsection{Nach Brené Brown}

In ihrem Talk \emph{The Anatomy of Trust}~\cite{anatomy-of-trust} liefert Brené Brown mit dem Akronym \emph{BRAVING} eine schön knackige Merkhilfe, die (um das schöne Akronym zu bilden) inhaltlich allerdings etwas unscharf wird. Ihre Ergebnisse bauen auf einer Metaanalyse der Forschung zu Vertrauen auf.
\index{BRAVING}

Sie fasst die Elemente von Vertrauen wie folgt zusammen:

\fett{B}oundaries \\
\fett{R}eliability \\
\fett{A}ccountability \\
\fett{V}ault \\
\fett{I}ntegrity \\
\fett{N}on-judgement \\
\fett{G}enerosity


\subsubsection{Grenzen respektieren (Boundaries)}
\index{Grenzen respektieren}

\begin{itemize}
  \item die eigenen Grenzen kennen, kommunizieren und schützen
  \item die Grenzen anderer Menschen anerkennen und respektieren
\end{itemize}



\subsubsection{Zuverlässigkeit (Reliability)}
\index{Zuverlässigkeit}

\begin{itemize}
  \item das tun, was ihr versprochen/zugesagt habt
  \item immer und immer wieder
\end{itemize}


\subsubsection{Verantwortung übernehmen (Accountability)}
\index{Verantwortung übernehmen}

\begin{itemize}
  \item nach einem Fehler dafür geradestehen, um Entschuldigung bitten, und es wiedergutmachen
  \item anderen ermöglichen, für Fehler geradezustehen, um Entschuldigung zu bitten, und es wiedergutzumachen
\end{itemize}


\subsubsection{Vertraulichkeit wahren (Vault)}
\index{Vertraulichkeit}

\begin{itemize}
  \item \glqq Was ich mit dir teile, behandelst du vertraulich.\grqq
  \item \glqq Was du mit mir teilst, behandele ich vertraulich.\grqq
  \item \glqq Ich erlebe, dass du Sachen anderer Personen vertraulich behandelst.\grqq
\end{itemize}


\subsubsection{Integrität (Integrity)}
\index{Integrität}

\begin{itemize}
  \item die eigenen Werte und Prinzipien tatsächlich leben
  \item Worten Taten folgen lassen
  \item das Richtige tun (statt das Angenehme, Schnelle oder Einfache)
\end{itemize}


\subsubsection{Nicht-Verurteilen (Non-judgement)}
\index{Nicht-Verurteilen}

\begin{itemize}
  \item \glqq Ich kann dich um Hilfe bitten, ohne dass du mich dafür verurteilst.\grqq
  \item \glqq Du kannst mich um Hilfe bitten, ohne dass ich dich dafür verurteile.\grqq
\end{itemize}


\subsubsection{Großzügigkeit (Generosity)}
\index{Großzügigkeit}

\begin{itemize}
  \item \glqq Du gehst bei meinen Worten und Taten von guten Absichten aus und fragst nach.\grqq
\end{itemize}


\subsection{Nach Charles Feltman}

Charles Feltman geht mit seinem Büchlein \emph{The Thin Book of Trust}~\cite{thin-book-of-trust} eher in die Richtung Lebenshilfe-Buch oder Business-Buch und ist weniger wissenschaftlich fundiert als die anderen beiden Werke.

Er stellt die folgenden Elemente von Vertrauen vor:


\subsubsection{Aufrichtigkeit}
\index{Aufrichtigkeit}

\glqq Was ich sage, meine ich so, und ich handele entsprechend.\grqq


\subsubsection{Zuverlässigkeit}
\index{Zuverlässigkeit}

\glqq Du kannst darauf zählen, dass ich liefere, was ich versprochen habe.\grqq


\subsubsection{Kompetenz}
\index{Kompetenz}

\glqq Ich weiß, was ich kann. Und ich weiß auch, was ich nicht kann.\grqq


\subsubsection{Wohlwollen (Care)}
\index{Wohlwollen}

\glqq Ich berücksichtige unser beider Interessen, wenn ich Entscheidungen fälle oder handele.\grqq


\subsection{Nach Ariane Jäckel}

Ariane Jäckel stellt in ihrer Dissertation \emph{Gesundes Vertrauen in Organisationen: Eine Untersuchung der Vertrauensbeziehung zwischen Führungskraft und Mitarbeiter}~\cite{gesundes-vertrauen-in-organisationen} den aktuellen Stand der Forschung dar, geht sehr in die Tiefe und ist wissenschaftlich sehr sauber.

Sie hat die folgenden Dimensionen von Vertrauenswürdigkeit identifiziert:


\subsubsection{Kompetenz}
\index{Kompetenz}

Darunter fallen aus den anderen Konzepten diese Unterpunkte:

\begin{itemize}
  \item Zuverlässigkeit (Brené Brown, Charles Feltman)
  \item Kompetenz (Charles Feltman)
\end{itemize}


\subsubsection{Integrität}
\index{Integrität}

Darunter fallen aus den anderen Konzepten diese Unterpunkte:

\begin{itemize}
  \item Verantwortung übernehmen (Brené Brown)
  \item Vertraulichkeit wahren (Brené Brown)
  \item Integrität (Brené Brown)
  \item Aufrichtigkeit (Charles Feltman)
\end{itemize}


\subsubsection{Wohlwollen}
\index{Wohlwollen}

Darunter fallen aus den anderen Konzepten diese Unterpunkte:

\begin{itemize}
  \item Grenzen respektieren (Brené Brown)
  \item Nicht-Verurteilen (Brené Brown)
  \item Großzügigkeit (Brené Brown)
  \item Wohlwollen (Charles Feltman)
\end{itemize}


\chapter{Gewaltfreie Kommunikation}
\index{Gewaltfreie Kommunikation}
\section{Bedürfnisse}
\label{beduerfnisse}
\index{Bedürfnisse}

\subsection{Was sind universelle Bedürfnisse?}

Bedürfnisse sind das, was wir erfüllt brauchen, damit es uns gut geht.

Ein universelles Bedürfnis ist eins, das jeder Mensch kennt~-- auch wenn sich Menschen darin unterscheiden, welche Bedürfnisse sie wie stark erfüllt brauchen.

Echte Bedürfnisse sind nicht an eine konkrete Person gebunden. Es gibt aber durchaus Bedürfnisse, die wir nur mit anderen Menschen zusammen erfüllen können, zum Beispiel unser Bedürfnis nach Gemeinschaft.

Ein Bedürfnis ist nicht an eine konkrete Handlung gebunden. Für jedes Bedürfnis gibt es viele verschiedene Strategien, um sie zu erfüllen~-- und wenn euch nur eine einzige Strategie dafür einfällt, dann habt ihr das Bedürfnis noch nicht genug verstanden.

\subsection{Liste von Bedürfnissen}

Das ursprüngliche Vokabular stammt von Marshall Rosenberg aus \cite[S.~216~f]{gfk-rosenberg} bzw.~im englischsprachigen Original \cite[S.~210]{nvc-rosenberg}. Das erweiterte Vokabular kommt \cite[S.~75~f]{gfk-dummies}.


\subsubsection{Autonomie}

\begin{multicols}{2}
  \begin{itemize}
    \item Freiheit
    \item Selbstbestimmung
  \end{itemize}
\end{multicols}


\subsubsection{Körperliche Bedürfnisse}

\begin{multicols}{2}
  \begin{itemize}
    \item Luft
    \item Wasser
    \item Bewegung
    \item Nahrung
    \item Schlaf
    \item Distanz
    \item Unterkunft
    \item Wärme
    \item Gesundheit
    \item Heilung
    \item Kraft
    \item Lebenserhaltung
  \end{itemize}
\end{multicols}


\subsubsection{Integrität, Stimmigkeit mit sich selbst}

\begin{multicols}{2}
  \begin{itemize}
    \item Authentizität
    \item Einklang
    \item Eindeutigkeit
    \item Übereinstimmung mit den eigenen Werten
    \item Identität
    \item Individualität
  \end{itemize}
\end{multicols}


\subsubsection{Einfühlung}

\begin{multicols}{2}
  \begin{itemize}
    \item Empathie
    \item verstanden/gesehen werden
    \item Gleichbehandlung
    \item Gerechtigkeit
  \end{itemize}
\end{multicols}


\subsubsection{Verbindung}

\begin{multicols}{2}
  \begin{itemize}
    \item Wertschätzung
    \item Nähe
    \item Zugehörigkeit
    \item Liebe
    \item Intimität/Sexualität
    \item Unterstützung
    \item Ehrlichkeit/Aufrichtigkeit
    \item Gemeinschaft
    \item Geborgenheit
    \item Respekt
    \item Kontakt
    \item Akzeptanz
    \item Austausch
    \item Offenheit
    \item Vertrauen
    \item Anerkennung
    \item Freundschaft
    \item Achtsamkeit
    \item Aufmerksamkeit
    \item Toleranz
    \item Zusammenarbeit
  \end{itemize}
\end{multicols}


\subsubsection{Entspannung}

\begin{multicols}{2}
  \begin{itemize}
    \item Erholung
    \item Ausruhen
    \item Spiel
    \item Spaß
    \item Leichtigkeit
    \item Ruhe
  \end{itemize}
\end{multicols}


\subsubsection{Geistige Bedürfnisse}

\begin{multicols}{2}
  \begin{itemize}
    \item Harmonie
    \item Inspiration
    \item \glqq Ordnung\grqq
    \item (innerer) Friede
    \item Freude
    \item Humor
    \item Abwechslungsreichtum
    \item Ausgewogenheit
    \item Glück
    \item Ästhetik
  \end{itemize}
\end{multicols}


\subsubsection{Entwicklung}

\begin{multicols}{2}
  \begin{itemize}
    \item Beitragen
    \item Wachstum
    \item Anerkennung
    \item Feedback
    \item Rückmeldung
    \item Erfolg (im Sinne von \glqq Gelingen\grqq
    \item Kreativität
    \item Sinne
    \item Bedeutung
    \item Effektivität
    \item Kompetenz
    \item Lernen
    \item Feiern
    \item Trauern
    \item Bildung
    \item Engagement
  \end{itemize}
\end{multicols}

\section{Bitten}
\label{bitten}

\subsection{Kriterien für gute Bitten}

Diese Kriterien könnt ihr im Detail in \cite[S. 85f]{gfk-dummies} nachlesen

\paragraph{Konkret:} Das Verhalten sollte realistisch und überprüfbar sein.

\paragraph{Machbar:} für die andere Person

\paragraph{Positiv formuliert:} Sagt, was ihr braucht, anstatt, was ihr nicht haben wollt.

\paragraph{Im Hier und Jetzt erfüllbar:} Das schließt auch Vereinbarungen mit Wirkung auf die Zukunft ein.

\paragraph{Freiwillig:} Was passiert, wenn die andere Person Nein sagt?


\subsection{Arten von Bitten}

\paragraph{Handlungsbitte:} Könntest du bitte …?

\paragraph{Bitte um aufrichtige Rückmeldung:} Wie geht es dir damit? Was siehst du das?

\paragraph{Bitte um Empathie:} Ich würde gerne verstehen, was du verstanden hast.


\subsection{An wen kann ich eine Bitte richten?}

\begin{itemize}
  \item an mein Gegenüber
  \item an mich selbst
  \item an eine dritte Person
\end{itemize}


\section{Wertschätzung ausdrücken}
\label{wertschaetzung}
\index{Wertschätzung}

Diese habe ich aus dem Buch \emph{GfK für Dummies} \cite[S.~206]{gfk-dummies} und mit einigen Dingen aus dem Podcast \emph{Familie verstehen} \cite{familie-verstehen-podcast} ergänzt, plus Ergänzungen aus dem Podcast \emph{Manager Tools Basics} \cite{manager-tools-basics}.

\begin{enumerate}
  \item Was hat die andere Person \fett{gesagt oder getan}?
  \item Welche \fett{Gefühle} hat dies bei mir ausgelöst? (optional, aber sehr hilfreich)
  \item Welche \fett{Bedürfnisse} von mir oder vom Team hat das erfüllt?
  \item \fett{Danke} dafür! \emph{oder:} Das \fett{feiere} ich!
  \item Gerne \fett{öfter/wieder tun}! (Bitte, optional)
\end{enumerate}

\section{Nach dem Workshop weiterlernen}
\label{gfk-weiterlenen}

Dieser Workshop bietet euch einen Einstieg in die GfK. Wenn ihr danach weiter lernen möchtet, könnt ihr euch kontinuierlich weiterbilden und üben.

Zusätzlich könnt ihr beim internationalen GfK-Dachverband \emph{Center for Nonviolent Communication (CNVC)} eine Zertifizierung als GfK-Trainer\_in anstreben.

Außerdem bauen alle meine Workshops zur Führungskräfteentwicklung und Teamentwicklung auf GfK aus. Mehr findet ihr unter \url{https://www.oliverklee.de/workshops/}.


\subsection{Weitere Themen, die in diesem Workshop keinen Platz gefunden haben}

\begin{itemize}
  \item schützende Gewalt
  \item Konflikte lösen
  \item GfK in der Kindererziehung
  \item Mediation
  \item mit Ärger umgehen
  \item das innere Team
  \item Umgang mit Scham und Schuld
  \item Nein sagen, Grenzen setzen
  \item Arbeit mit Glaubenssätzen
  \item das innere Kind heilen
  \item systemisches Konsensieren
  \item empathisch unterbrechen
\end{itemize}

Viele dieser Themen könnt ihr in einer GfK-Grundausbildung oder in Einzelseminaren lernen.


\subsection{Kontinuierlich weiter lernen}

\begin{itemize}
  \item Seminare zu einzelnen Themen
  \item GfK-Übungsgruppe (vor Ort oder online)
  \item Podcast \glqq Familie verstehen\grqq\cite{familie-verstehen-podcast} von Kathy Weber (hauptsächlich über Kindererziehung, aber auch generell für GfK interessant, wenn ihr (noch) keine Kinder habt)
  \item GfK-Tage (eintägige Konferenzen, die in vielen Städten jährlich stattfinden)
  \item Bücher zu GfK lesen
\end{itemize}

\subsubsection{Literaturempfehlungen}

Das Buch \glqq Gewaltfreie Kommunikation\grqq{} von Marshall Rosenberg \cite{gfk-rosenberg} bzw.\ im englischen Original \cite{nvc-rosenberg} ist ein guter Einstieg und bietet eine gute Übersicht.

Gabriel Seils hat in \cite{gfk-gespraech} einige lange Gespräche mit Marshall Rosenberg geführt. Dieses Buch hilft, GfK als Haltung besser zu verstehen. Ich finde es außerdem sehr tröstlich zu lesen.

Rosenberg hat außerdem eine Reihe von kleinen Büchlein verfasst, zum Beispiel \cite{we-can-work-it-out} oder \cite{being-me-loving-you}. Diese Büchlein beleuchten die Anwendung von GfK in einzelnen Bereichen des Lebens und sind jeweils gut an einem Nachmittag als Snack lesbar.

\subsection{GfK-Zertifizierung}

So sieht der Weg zur Zertifizierung beim CNVC aus:

\begin{enumerate}
  \item an einen GfK-Einstiegsworkshop teilnehmen
  \item an einer 10- bis 20-tägigen GfK-Basisausbildung teilnehmen (auch \glqq Jahresausbildung\grqq\ oder \glqq Grundausbildung\grqq\ genannt)
  \item mit dem GfK-Zertifizierungsprozess\cite{gfk-trainer-werden} beginnen
  \item derweil weiter lernen (s.o.)
  \item an einem Workshop zum Thema \glqq GfK unterrichten\grqq{} teilnehmen (nicht verpflichtend, aber wirklich hilfreich)
  \item an einem allgemeinen Train-the-Trainer-Workshop teilnehmen (auch nicht verpflichtend, aber das hilft sehr, die didaktische Qualität eurer GfK-Workshops zu steigern)
\end{enumerate}

Wenn ihr an Workshops teilnehmt, achtet darauf, dass ihr dies bei Personen tut, die beim CNVC zertifiziert sind, da ihr 10 solche Tage nachweisen müsst, um zur Ausbildung als GfK-Trainer\_in zugelassen zu werden.

Auch dann, wenn ihr GfK nicht unterrichten möchtet, ist die Basisausbildung ein guter Weg, um nach einem Einstiegsworkshop wirklich tief in die GfK einzusteigen.

Ich persönlich habe sehr gute Erfahrungen mit der Basisausbildung von Lydia Kaiser\footnote{\url{https://kommunikation-bewegt.de/}} und Jochen Hiester\footnote{\url{https://www.gewaltfrei-koblenz.de/}} in Bonn gemacht und kann beide sehr empfehlen.



\backmatter

\bibliography{../shared/bibliography/literatur}

\chapter{Lizenz}
\label{lizenz}

\section*{Unter welchen Bedingungen könnt ihr dieses Handout benutzen?}
Dieses Handout ist unter einer \emph{Creative-Commons}-Lizenz lizensiert. Dies ist die \emph{Namensnennung-Share Alike 4.0 international (CC BY-SA 4.0)}\footnote{Die ausführliche Version dieser Lizenz findet ihr unter \url{https://creativecommons.org/licenses/by-sa/4.0/}.}. Das bedeutet, dass ihr dieses Handout unter diesen Bedingungen für euch kostenlos verbreiten, bearbeiten und nutzen könnt (auch kommerziell):

\begin{description}
  \item[Namensnennung.] Ihr müsst den Namen des Autors (Oliver Klee) nennen. Wenn ihr außerdem auch noch die Quelle\footnote{\url{https://github.com/oliverklee/workshop-handouts}} nennt, wäre das nett. Und wenn ihr mir zusätzlich eine Freude machen möchtet, sagt mir per E-Mail Bescheid.
  \item[Weitergabe unter gleichen Bedingungen.] Wenn ihr diesen Inhalt bearbeitet oder in anderer Weise umgestaltet, verändert oder als Grundlage für einen anderen Inhalt verwendet, dann dürft ihr den neu entstandenen Inhalt nur unter Verwendung identischer Lizenzbedingungen weitergeben.
  \item[Lizenz nennen.] Wenn ihr den Reader weiter verbreitet, müsst ihr dabei auch die Lizenzbedingungen nennen oder beifügen.
\end{description}


\printindex

\end{document}
