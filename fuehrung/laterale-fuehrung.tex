\section{Laterale Führung}
\label{laterale-führung}
\index{laterale Führung}


\subsection{Definition}

Laterale Führung (\glqq zur Seite gerichtet führen\grqq) bedeutet \fett{Führen ohne direkte Weisungsbefugnis}.~\cite{kuhl-laterales-fuehren, fuehren-ohne-fuehrung, gtd-when-not-in-charge}

\subsection{Anlässe für laterales Führen}

\begin{itemize}
  \item Ihr arbeitet mit \fett{Ehrenamtlichen}.
  \item Ihr arbeitet mit \fett{Scrum} und seid Scrum-Master\_in oder Product-Owner\_in.
  \item Ihr seid Teil des Teams, und die Führungsperson hat bewusst \fett{Führungsaufgaben an euch abgegeben}.
  \item Ihr seid Teil eines \fett{selbstorganisierten Teams} ohne dedizierte Führungsperson.
  \item Ihr arbeitet \fett{bereichs- oder organisationsübergreifend} zu einem Projekt oder in Prozessketten zusammen.
  \item Ihr seid in einem \fett{Lehr- oder Mentoring-Verhältnis} zur anderen Person.
  \item Ihr seid eurem Team gegenüber zwar grundsätzlich weisungsbefugt, entscheidet euch aber bewusst, diese \fett{Macht nicht zu nutzen}.
  \item Worst-Case: Ihr habt eine Führungsperson, die aber ihre Führungsaufgabe aus Unwillen oder wegen mangelnder Kompetenz nicht ausfüllt, und der Laden muss trotzdem irgendwie laufen.
\end{itemize}
\index{Inkompetenz}
\index{Kompetenz}


\subsection{Laterale Führung vs.~\emph{Servant Leadership}}
\index{Servant Leadership}

Während das Konzept von lateraler Führung dadurch definiert ist, dass keine direkte Weisungsbefugnis besteht (das Konzept also eher \emph{deskriptiv} ist), beschreibt \emph{Servant Leadership} (\glqq dienende Führung\grqq) ~\cite{fuehren-durch-dienen} den Ansatz, dass die Führung die Geführten in ihrer Arbeit unterstützen und ermächtigen soll (wodurch das Konzept also eher \emph{normativ} ist).

Beide Konzepte ergänzen sich ganz ausgezeichnet: In einer Situation der lateralen Führung fahrt ihr mit \emph{Servant Leadership} sehr gut.


\subsection{Die drei Säulen der lateralen Führung}

\index{Verständigung}
\paragraph{Verständigung} darüber, wie die Zusammenarbeit funktioniert und wohin der Weg geht.

\index{Macht}
\paragraph{Macht} um Blockaden zu überwinden.

\index{Vertrauen}
\index{Beziehungen}
\paragraph{Vertrauen} erlauben den Menschen, bei der Zusammenarbeit in Vorleistung zu gehen.


\subsubsection{Zusammenspiel der drei Säulen}

Diese Prozesse laufen gleichzeitig ab und sind nicht immer eindeutig zu erkennen.

Sie können sich gegenseitig verstärken, und ihr könnt einen Mangel bei einer Säule durch eine der anderen Säulen teilweise ausgleichen: Zum Beispiel könnt ihr (insbesondere bei sehr lateraler Führung) wenig Macht durch gute Absprachen (Verständigung) und bewusst gepflegte Beziehungen (Vertrauen) ausgleichen.

Gleichzeitig können sich auch Prozesse gegenseitig behindern: Wenn ihr viel Macht einsetzt, kann das das Vertrauen verringern, das die andere Person in euch hat.

Wie diese drei Säulen bei euch aussehen, hängt zuerst von den Strukturen eurer Organisation und eures Teams ab. Ihr als Führungsperson habt darauf aufbauen direkten Einfluss darauf (und Verantwortung dafür).


\subsection{Verständigung}
\index{Verständigung}

Was ist Verständigung, und warum ist sie wichtig?

\begin{itemize}
  \item Erwartungen der Gruppe, in die man eingebunden ist
  \item Standards, Normen und Auffassungen
  \item gemeinsames Verständnis darüber, wie man arbeitet
  \item mobilisiert die Ansichten, Erfahrungen und Interessen vieler Personen
  \item reduzieren die Motivations- und Kontrollprobleme der Führung
\end{itemize}

Dies sind einige Instrumente, mit denen ihr zu mehr Verständigung beitragen könnt:

\begin{itemize}
  \item klare definierte und dokumentierte Rollen und Verantwortlichkeiten (siehe dazu Seite~\pageref{rollen}) \index{Rollen}\index{Verantwortlichkeiten}
  \item Teamregeln oder -vereinbarungen \index{Teamregeln}\index{Regeln}\index{Vereinbarungen}
  \item klar dokumentierte Prozesse \index{Prozesse}
  \item ein Mission-Statement \index{Mission-Statement}
  \item eine gute Agenda für jedes Meeting (\glqq\emph{No agenda, no attenda.\grqq} \index{Agenda}
  \item klare, SMARTe Ziele
\end{itemize}

Wichtig für die Verständigung ist auch zu wissen, dass im Zweifel für die Einzelnen die \glqq lokalen Realitäten\grqq\ (also wie ihr im Team oder der Abteilung arbeitet) relevanter sind als das, was für die gesamte Organisation gilt.


\subsection{Macht}
\index{Macht}

Macht …

\begin{itemize}
  \item ist Teil jeder Beziehung
  \item ist Mechanismus, mit dem man bei anderen ein Verhalten erzeugt, das sie spontan nicht eingenommen hätten
  \item ist nur möglich, wenn irgendeine Beziehung zwischen den Beteiligten besteht
  \item basiert darauf, dass sie von den Beteiligten geteilt und (mehr oder minder) akzeptiert wird
\end{itemize}

Macht basiert im Wesentlichen darauf, dass jemand etwas Bestimmtes hat oder kontrolliert:

\begin{itemize}
  \item Hierarchie
  \item Fachwissen, Expertise
  \item Relaisstellen: Zugang zu Personen etc.
  \item Gatekeeper zu Informationen und Kommunikationskanälen
\end{itemize}


\subsection{Vertrauen}
\index{Vertrauen}

Kooperation ist riskant, weil wir damit ständig in Vorleistung gehen und von der anderen Person abhängig sind. Damit dies funktionieren kann, ist gegenseitiges Vertrauen zwingend notwendig.

Mehr dazu in einem separaten Kapitel.
