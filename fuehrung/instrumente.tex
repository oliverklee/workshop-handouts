\section{Führungsinstrumente (Führungswerkzeuge)}
\label{fuehrungsinstrumente}
\index{Führungswerkzeuge}
\index{Führungsinstrumente}


\subsection{Was ist ein Führungsinstrument?}

Ein Führungsinstrument (oder Führungswerkzeug) generell alles, was eine Führungskraft tun kann, um direkt oder indirekt zusammen mit den geführten Personen Ziele zu erreichen.


\subsubsection{Führungswerkzeuge nach Malik}

Diese Liste von Fredmund Malik \cite{malik-fuehrung} ist schon etwas angestaubt. Sie lässt sich allerdings gut in die heutige Zeit übertragen.

\begin{itemize}
  \item Besprechung
  \item Schriftstück
  \item Stellengestaltung und Einsatzsteuerung
  \item Persönliche Arbeitsmethodik
  \item Budget und Budgetierung
  \item Leistungsbeurteilung
  \item systematische Müllabfuhr
\end{itemize}


\subsection{Direkte Führungsinstrumente}

\begin{itemize}
  \item 1-zu-1-Gespräche (siehe Seite~\pageref{1-zu-1})
  \item Anweisung
  \item Besprechung
  \item delegieren
  \item Entscheidungen treffen
  \item Feedback einholen
  \item Feedback geben
  \item informieren
  \item Konflikte klären
  \item kontrollieren
  \item Kritik
  \item Lob
  \item um etwas bitten
  \item Wertschätzung ausdrücken
  \item Ziele vereinbaren
\end{itemize}


\subsection{Indirekte Führungsinstrumente}

\begin{itemize}
  \item Anreizsysteme schaffen
  \item dem Team Workshops und andere Fortbildungen anbieten
  \item den (physischen) Arbeitsplatz gestalten
  \item die Motivation verbessern
  \item die psychologische Sicherheit verbessern
  \item eine Mission definieren
  \item einen Spieleabend mit dem Team veranstalten
  \item ein Team zusammenstellen
  \item Gewaltfreie Kommunikation lernen und anwenden
  \item mit dem Team einen Escape-Room spielen
  \item mit dem Team lecker essen gehen
  \item Prozesse definieren
  \item Rollen und Verantwortlichkeiten definieren
  \item Supervision für das Team organisieren
\end{itemize}
