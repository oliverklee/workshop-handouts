\section{Managementtechniken}
\label{managementtechniken}
\index{Managementtechniken}
\index{Management-by-Techniken}

Dies sind einige (von vielen) Managementtechniken (oder \emph{Management-by}-Techniken), die ihr in Kombination mit verschiedenen Führungsstilen nutzen könnt.

\paragraph{Management by Delegation} bedeutet, Aufgaben und die Verantwortung dafür an die Mitarbeitenden zu delegieren.
\index{Management by Delegation}

\paragraph{Management by Direction and Control} bedeutet, dass Vorgesetzte die Entscheidungen treffen, Anweisungen geben und die Ausführung kontrollieren. Sie entspricht ungefähr dem autoritären Führungsstil.
\index{Management by Direction and Control}

\paragraph{Management by Exception} bedeutet, dass die Mitarbeitenden die Routineentscheidungen treffen und die Führungsperson nur in Ausnahmefällen selbst entscheidet.
\index{Management by Exception}

\paragraph{Management by Objectives} bedeutet, dass Teams und Mitarbeitende Ziele vorgegeben bekommen oder Zielvereinbarungen treffen, die sie dann selbstständig erfüllen.
\index{Management by Objectives}

\paragraph{Management by Systems} bedeutet, dass die Arbeit in einem sich selbst tragenden System erfolgt, bei dem die Führungsperson nur noch das System regelt.
\index{Management by Systems}
