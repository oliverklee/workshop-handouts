\section{Die Rolle der Führung}
\label{fuehrung-rolle}
\index{Führungsrolle}


\subsection{Führen lernen}
\index{Führen lernen}

Meiner Ansicht nach ist Führen zu lernen so ähnlich wie Singen zu lernen. Dafür sind diese Dinge notwendig:

\begin{itemize}
  \item viel \fett{üben} (und dabei aus Fehlern lernen)
  \item \fett{Reflexion}
  \item \fett{Außenwahrnehmung} bekommen in der Form von Feedback (oder beim Singen sich selbst aufzunehmen)
  \item an \fett{Trainings} und \fett{Workshops} teilnehmen (oder anderweitig Unterricht nehmen)
  \item \fett{Bücher} oder anderen Quellen von Wissen konsumieren
  \item von \fett{guten Beispielen} lernen
  \item \fett{Freude} daran haben!
\end{itemize}


\subsection{Aufgaben der Führung nach Neuberger}

Laut Neuberger\cite{neuberger-fuehren} sind die Aufgaben der Führung,

\begin{itemize}
  \item andere Menschen
  \item zielgerichtet
  \item in einer formalen Organisation
  \item unter konkreten Umweltbedingungen dazu bewegen,
  \item Aufgaben zu übernehmen und erfolgreich auszuführen,
  \item wobei humane Ansprüche gewahrt werden.
\end{itemize}


\subsection{Neuberger, aber modernisiert}

Auf das moderne Arbeiten übertragen, wäre die Aufgabe der Führung,

\begin{itemize}
  \item eine Umgebung zu schaffen,
  \item die es einem Team oder einer Organisation möglich und leicht macht,
  \item für die Mission des Teams oder der Organisation zu arbeiten,
  \item wobei die Menschen nachhaltig körperlich und seelisch gesund zu bleiben
  \item und ihr Potenzial nutzen können.
\end{itemize}

(Dies ist meine eigene \glqq Übersetzung\grqq{} aus Perspektive der modernen Führung.)


\subsection{Führung nach Malik}

Fredmund Malik \cite{malik-fuehrung} hat in seinen Büchern die Rolle und die Grundsätze von Führung beschrieben.


\subsubsection{Aufgaben der Führung nach Malik}

\begin{itemize}
  \item für Ziele sorgen
  \item organisieren
  \item entscheiden
  \item kontrollieren
  \item Menschen entwickeln und fördern
\end{itemize}

Auf moderne Führung übertragen, wäre es die Aufgabe der Führung, dafür zu sorgen, dass diese Dinge stattfinden (also dass beispielsweise das Team Entscheidungen fällen und nachhalten kann), und nicht zwangsläufig, dass die Führung das auch selbst entscheidet.


\subsubsection{Grundsätze der Führung nach Malik}

\begin{itemize}
  \item Ergebnisorientierung
  \item Beitrag zum Ganzen
  \item Konzentration auf weniges
  \item Stärken nutzen
  \item gegenseitiges Vertrauen
  \item positiv denken
\end{itemize}

\subsubsection{Führungswerkzeuge nach Malik}

Diese Liste ist schon etwas angestaubt. Sie lässt sich allerdings gut in die heutige Zeit übertragen.

\begin{itemize}
  \item Besprechung
  \item Schriftstück
  \item Stellengestaltung und Einsatzsteuerung
  \item Persönliche Arbeitsmethodik
  \item Budget und Budgetierung
  \item Leistungsbeurteilung
  \item systematische Müllabfuhr
\end{itemize}

\subsection{Führungsinstrumente (Führungswerkzeuge)}
\index{Führungswerkzeuge}
\index{Führungsinstrumente}

Dies ist eine unvollständige Auswahl.

\subsubsection{Direkte Führungsinstrumente}

\begin{itemize}
  \item Feedback geben
  \item Feedback einholen
  \item 1-zu-1-Gespräche (siehe Seite~\pageref{1-zu-1})
  \item Besprechung
  \item Anweisung
  \item Delegieren
  \item Wertschätzung ausdrücken
  \item Lob
  \item um etwas bitten
  \item Kritik
  \item Entscheidungen treffen
  \item Ziele vereinbaren
  \item Konflikte klären
  \item Informationen weitergeben
  \item kontrollieren
\end{itemize}


\subsubsection{Indirekte Führungsinstrumente}

\begin{itemize}
  \item ein Team zusammenstellen
  \item Rollen und Verantwortlichkeiten definieren
  \item die psychologische Sicherheit verbessern
  \item Prozesse definieren
  \item die Motivation verbessern
  \item eine Mission definieren
  \item einen Spieleabend mit dem Team veranstalten
  \item mit dem Team lecker essen gehen
  \item einen Escape-Room spielen
  \item Workshops besuchen
  \item Gewaltfreie Kommunikation lernen
  \item den (physischen) Arbeitsplatz gestalten
  \item Anreizsysteme schaffen
\end{itemize}
