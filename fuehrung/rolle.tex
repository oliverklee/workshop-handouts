\section{Die Rolle der Führung}
\label{fuehrung-rolle}
\index{Führung: Rolle}


\subsection{Aufgaben der Führung nach Neuberger}

Laut Neuberger\cite{neuberger-fuehren} sind die Aufgaben der Führung,

\begin{itemize}
  \item andere Menschen
  \item zielgerichtet
  \item in einer formalen Organisation
  \item unter konkreten Umweltbedingungen dazu bewegen,
  \item Aufgaben zu übernehmen und erfolgreich auszuführen,
  \item wobei humane Ansprüche gewahrt werden.
\end{itemize}


\subsection{Neuberger, aber modernisiert}

Auf das moderne Arbeiten übertragen, wäre die Aufgabe der Führung,

\begin{itemize}
  \item eine Umgebung zu schaffen,
  \item die es einem Team oder einer Organisation möglich und leicht macht,
  \item für die Mission des Teams oder der Organisation zu arbeiten,
  \item wobei die Menschen nachhaltig körperlich und seelisch gesund zu bleiben
  \item und ihr Potenzial nutzen können.
\end{itemize}


\subsection{Aufgaben der Führung nach Malik}

Dies sind laut Fredmund Malik \cite{malik-fuehrung} die Aufgaben der Führung:

\begin{itemize}
  \item für Ziele sorgen
  \item organisieren
  \item entscheiden
  \item kontrollieren
  \item Menschen entwickeln und fördern
\end{itemize}

Auf moderne Führung übertragen, wäre es die Aufgabe der Führung, dafür zu sorgen, dass diese Dinge \emph{stattfinden} (also dass beispielsweise das Team Entscheidungen fällen und diese nachhalten kann), und nicht zwangsläufig, dass die Führung das auch selbst entscheidet.


\subsection{Was ergibt sich daraus?}
\index{Beziehungen}

Laut dem Podcast \emph{Manager Tools Basics} \cite{manager-tools-basics} ist eine der wichtigsten Verantwortung der Führung, \fett{gute Beziehungen} zu den geführten Personen \fett{aufzubauen und zu pflegen}.

Laut Amy Edmondson \cite{the-fearless-organisation} ist es die Hauptaufgabe der Führung, im Team bzw.~in der Organisation \fett{psychologische Sicherheit zu schaffen}.
\index{psychologische Sicherheit}
