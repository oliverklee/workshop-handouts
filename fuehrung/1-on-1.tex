\section{1-zu-1-Gespräche (\emph{One-on-Ones})}
\label{1-zu-1}
\index{1-zu-1-Gespräche}
\index{One-on-Ones}

Das \emph{1-zu-1-Gespräch} ist ein Führungsinstrument (siehe S.~\pageref{fuehrungsinstrumente}).


\subsection{Ziele}

\begin{itemize}
 \item das Vertrauen und die Beziehung zwischen Teamlead und Teammitglied aufbauen und pflegen
 \item dem Teammitglied und Teamlead die Möglichkeit geben, sich gegenseitig regelmäßig Feedback zu geben
 \item ein Ort für Absprachen sein, damit die Ad-hoc-Absprachen und -Calls zwischendrin weniger werden
 \item Konflikte, Sorgen und Ideen zeitnah besprechen
\end{itemize}


\subsection{Organisation}

\begin{itemize}
 \item Die Gespräche sollten regelmäßig und zuverlässig stattfinden.
 \item Mit Vollzeitangestellten sollten die Gespräche wöchentlich stattfinden. In unserem ehrenamtlichen Sehr-Teilzeit-Team haben wir einen Rhythmus von 4 Wochen.
 \item Die Gespräche haben eine harte Begrenzung auf 30 Minuten, während die einzelnen Slots zeitlich flexibler sind.
 \item Generell sollte das Teammitglied ca.~90\,\% der Redeanteile im Gespräch haben und die Teamführung die restlichen 10\,\%.
\end{itemize}


\subsection{Struktur}

Alle Themen hier sind nur Vorschläge. Wenn euch andere Themen wichtiger sind, dann sprecht über diese.

\subsubsection{Checkin}
Wie fühle ich mich gerade? Wie bin ich hier?


\subsubsection{Slot für das Teammitglied (ca.~10 Minuten)}

\paragraph{Allgemeines}
\begin{itemize}
 \item alles, worüber du sprechen möchtest
\end{itemize}

\paragraph{Strategie}
\begin{itemize}
 \item Was tun wir als Team/Organisation nicht, was wir tun sollten?
 \item Wenn wir eine Sache verbessern könnten, welche wäre das?
 \item Wenn du ich wärst, was würdest du verändern?
\end{itemize}

\paragraph{Blick nach außen}
\begin{itemize}
 \item Was ist zur Zeit das größte Problem unseres Teams/unserer Organisation? Und warum?
 \item Was gefällt dir nicht an unseren Produkten und Dienstleistungen? Was könnten wir verbessern? Woran müssen wir noch arbeiten?
 \item Was ist die größte Chance, die wir gerade verpassen?
\end{itemize}

\paragraph{Team}
\begin{itemize}
 \item Mit wem würdest du gerne (mehr) zusammenarbeiten?
 \item Wer macht gerade einen richtig guten Job?
\end{itemize}

\paragraph{Persönliche Arbeit}
\begin{itemize}
 \item Was macht dir gerade Spaß? Und warum?
 \item Und was macht dir keinen Spaß? Was nervt? Und warum?
 \item Gibt es etwas, vor dem du gerade Angst hast?
 \item Was motiviert dich gerade?
 \item Was demotiviert dich gerade? Was bräuchtest du, um das zu ändern?
\end{itemize}

\paragraph{Feedback und Bitten an die Teamführung}
\begin{itemize}
 \item Was von dem, was ich tue, ist besonders hilfreich für dich?
 \item Wovon würdest du dir bei mir mehr wünschen?
 \item Was von dem, was ich tue, funktioniert für dich nicht gut?
 \item Was bräuchtest du von mir im Moment?
\end{itemize}


\subsubsection{Slot für die Teamführung (ca.~10 Minuten)}
\begin{itemize}
 \item alles, worüber du sprechen möchtest
 \item Feedback an das Teammitglied
\end{itemize}


\subsubsection{Persönliche Entwicklung (ca.~10 Minuten, wenn noch Zeit ist)}
\begin{itemize}
 \item Was hast du kürzlich (dazu-)gelernt?
 \item Was hast du kürzlich ausprobiert? Was ist dabei herausgekommen?
 \item Was würdest du gerne lernen?
 \item Was würdest du gerne mal ausprobieren?
 \item Welche Verantwortungsbereiche würdest du gerne annehmen oder abgeben?
 \item Wie kannst du deine Superkräfte am besten einsetzen?
\end{itemize}

\subsection{Quellen}

\begin{itemize}
 \item Podcast: Manager Tools Basics: One on Ones \cite[04.\,07.\,2005 bis 11.\,07.\,2005]{manager-tools-basics}
 \item Podcast: Female Leadership: Gamechanger für Gesprächsführung \cite[Folge 15]{female-leadership-gespraechsfuehrung}
 \item Buch: The Hard Thing About Hard Things: Building a Business When There Are No Easy Answers \cite{the-hard-thing-about-hard-things}
\end{itemize}
