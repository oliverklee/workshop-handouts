\section{Mission-Statement}
\label{mission-statement}
\index{Mission-Statement}

Eine komplette Organisation kann ein Mission-Statement haben.~\cite{corporate-purpose} Innerhalb der Organisation können einzelne Teams dann noch einmal als \glqq lokale Realität\grqq\ eigene Mission-Statements haben, die auf das Mission-Statement der Organisation einzahlen.

Ein Mission-Statement ist vor allem nach innen gerichtet.

\subsection{Sinn und Zweck eines Mission-Statements}

\begin{itemize}
  \item motivieren
  \item Entscheidungshilfe für Menschen sein, die überlegen, bei euch mitzuarbeiten
  \item Entscheidungshilfe sein, ob diese Umgebung mit euren Werten und Antreibern harmoniert und ihr vielleicht wechseln möchtet
  \item Entscheidungshilfe für Maßnahmen sein, wenn eine Person gegen die Mission arbeitet
  \item als Kompass bei alltäglichen Entscheidungen dienen: \glqq Zahlt diese Option auf unsere Mission ein?\grqq
  \item die Richtung bei der Planung der Aufgaben und Projekte vorgeben
  \item Grundlage für das sein, was ihr über über eure Organisation nach außen kommunizieren möchtet (z.\,B.~auf eurer Website)
\end{itemize}


\subsection{Elemente eines Mission-Statements}

Optimalerweise besteht ein Mission-Statement aus drei Abschnitten, sortiert von \emph{abstrakt} zu \emph{konkret}.

\paragraph{Motivation/Purpose (Daseinszweck):} \emph{Warum} gibt es uns?
\index{Motivation}
\index{Purpose}
\index{Daseinszweck}
\index{Zweck}

Dieser Punkt soll vor allem motivieren.

\paragraph{Vision:} \emph{Wie} arbeiten wir grundsätzlich auf unseren \emph{Purpose} hin?
\index{Vision}
\index{Kompass}

Dieser Punkt soll vor allem als Kompass bei täglichen Entscheidungen dienen.


\paragraph{Mission:} \emph{Was} tun wir konkret, um unsere Vision voranzutreiben?
\index{Mission}

Dieser Punkt soll vor allem bei der Planung der konkreten Aufgaben helfen.
