\section{Rollen und Verantwortlichkeiten definieren}
\label{rollen}
\index{Rollen}
\index{Verantwortlichkeiten}

Eine Rolle definiert einen Bereich, für das eine Person verantwortlich ist, und wodurch sie zusätzliche Berechtigungen erhält. Eine Person kann dabei mehrere Rollen gleichzeitig ausfüllen, und mehrere Personen können sich auch eine Rolle teilen.

Ungeklärte Rollen sind eine häufige Ursache von Konflikten.

Klar definierte Rollen tragen bei lateraler Führung zur Verständigung bei.


\subsection{Rollen definieren}

Ein Teil dieser Anleitung kommt aus dem \fett{AKV-Prinzip} zur Analyse von Aufgaben, Kompetenzen und Verantwortlichkeiten.~\cite{kessler-projektmanagement}
\index{AKV-Prinzip}

Das hier sollte in einer guten Rollendefinition stehen:

\paragraph{Bezeichnung:} Wie benennen wir die Rolle in der Kommunikation?

\paragraph{Daseinszweck:} Warum brauchen/wollen wir diese Rolle überhaupt?

\paragraph{Dauer:} Ist dies eine dauerhafte oder vorübergehende Rolle? (Die Moderation oder Protokollführung eines Meetings sind beispielsweise oft temporäre Rollen.)

\paragraph{Aufgaben:} Was sind die 3--5 wichtigsten Tätigkeiten dieser Rolle? Und wofür sind Menschen dadurch Ansprechpersonen?
\index{Aufgaben}

\paragraph{Befugnisse/Kompetenzen:} Was dürfen diese Personen in dieser Rolle tun, was sie sonst nicht dürften?
\index{Befugnisse}
\index{Kompetenzen}

Beispiele: Verträge unterschreiben, Leute unterbrechen, Dinge (mit-)entscheiden \ldots

\paragraph{Verpflichtungen/Verantwortung:} Wozu verpflichtet diese Rolle? Woran messen wir, ob jemand diese Rolle erfolgreich ausfüllt?
\index{Verpflichtungen}
\index{Verantwortung}

\paragraph{Fähigkeiten/Kenntnisse:} Was muss eine Person fachlich können, wissen oder lernen, um diese Rolle gut erfüllen zu können?
\index{Fähigkeiten}
\index{Kenntnisse}
