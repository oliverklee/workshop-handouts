\section{Konfliktmanagement aus Perspektive der Führung}
\label{konflikte-fuehrung}
\index{Führung}

Teil eurer Verantwortung (siehe S.~\pageref{fuehrung-aufgaben-neuberger}) in der Führung sind unter anderem diese Dinge:

\begin{itemize}
  \item eine Umgebung schaffen, in der Menschen produktiv sein können, und in der sie auch seelisch gesund bleiben
  \item gute Beziehungen zu euren Teammitgliedern aufbauen und pflegen
  \item psychologische Sicherheit schaffen
\end{itemize}

Konkret folgen daraus für euch diese Verantwortlichkeiten:

\begin{itemize}
  \item Konflikte zwischen euch und anderen Personen ansprechen und konstruktiv lösen, um eure Beziehung zu pflegen
  \item eurem Team ein Vorbild dafür sein, wie ihr mit Konflikten umgeht
  \item eure Teammitglieder dabei unterstützen, ihre Konflikte zu lösen
  \item einfordern, dass Teammitglieder die Konflikte lösen, die ihre Arbeit behindern
  \item es nicht akzeptieren, wenn Leute ihre Konflikte nicht ansprechen, nicht lösen oder nicht konstruktiv lösen
\end{itemize}
