\section{Führungsstile}
\label{fuehrungsstile}
\index{Führungsstile}


\subsection{Tradierte Führungsstile}
\index{Tradierte Führungsstile}

Diese Führungsstile nach Max Weber~\cite{weber-wirtschaft-gesellschaft} sind nicht mehr aktuell und eher aus historischer Perspektive interessant.

\paragraph{Autokratisch:} uneingeschränkte Macht
\paragraph{Patriarchalisch:} Machtfülle/Vaterrolle
\paragraph{Charismatisch:} Leitfigur/Vorbild
\paragraph{Bürokratisch:} durch Strukturen; die Personen sind austauschbar


\subsection{Klassische Führungsstile}
\index{Klassische Führungsstile}

Diese Führungsstile nach Kurt Lewin~\cite{lewin-fuehrungsstile} sind immer noch aktuell.

\paragraph{Autoritär:} Die Aktivität liegt ausschließlich bei der Führungsperson. Die \glqq Untergebenen\grqq\ haben Weisungen zu akzeptieren und auszuführen. Dieser Stil entspricht dem tradierten autokratischen Stil.
\index{Autoritärer Führungsstil}

\paragraph{Kooperativ/demokratisch:} Die Führungsperson zieht die Mitarbeitenden in die Entscheidungen mit ein. Aus Kontrolle wird zunehmend Selbstkontrolle. Mitarbeitende können die Führung kritisieren
\index{Autoritärer Führungsstil}

\paragraph{Karitativ/partizipativ:} Diese Stil orientiert sich vorrangig an den Bedürfnissen der Mitarbeitenden. Der Mensch steht im Mittelpunkt; die Aufgaben sind dagegen nachrangig. Die Führungsperson hört viel zu, fördert und ermutigt.
\index{Karitativer Führungsstil}
\index{Partizipativ Führungsstil}

\paragraph{Laissez-faire:} Die Führungsperson überträgt/delegiert die Aufgaben, kümmert sich um die Arbeitsmittel und definiert klare Ziele.
\index{Laissez-faire}


\subsection{Situatives Führen}
\index{Situatives Führen}

\emph{Situatives Führen} nach Hersey~\cite{hersey-management} bezeichnet, dass eine Führungsperson je nach Situation unterschiedliche Führungsstile wählt, um erfolgreich zu führen.
