\section{Lob und Tadel ersetzen}
\label{lob-tadel}
\index{Lob}
\index{Tadel}

\subsection{Warum wollen wir Lob und Tadel ersetzen?}

\begin{itemize}
  \item Wenn ihr lobt oder tadelt, dann verpasst ihr damit die Gelegenheit, der anderen Person ein Einblick in das zu geben, was in euch lebendig ist.
  \item Lob und Tadel kommt ihr von oben statt auf Augenhöhe. Das wirkt sich negativ auf eure Beziehung aus.
  \item Die übliche Reaktion auf Tadel ist, das getadelte Verhalten in Zukunft zu vermeiden, anstatt daran zu wachsen.
  \item Wenn ihr jemanden tadelt, ist nicht immer klar, ob der Fokus drauf liegt, dass ihr selbst etwas braucht, oder darauf, dass ihr der anderen Person die Gelegenheit zum Lernen geben möchtet.
\end{itemize}


\subsection{Lob ersetzen}

Gebt stattdessen echte, herzliche \fett{Wertschätzung}. Mehr dazu findet ihr auf Seite~\pageref{wertschaetzung}.


\subsection{Tadel ersetzen}

Wenn der Fokus darauf ist, dass ihr selbst eine \fett{Veränderung für euch} braucht, formuliert eine \fett{Bitte}. Mehr dazu auf Seite~\pageref{bitten}.

Wenn hingegen der Fokus darauf ist, dass ihr der anderen Person (mehr) \fett{Entwicklung ermöglichen} möchtet, dann bietet stattdessen \fett{Feedback} an. Auf Seite~\pageref{feedback-regeln} findet ihr Tipps dazu, wie ihr hilfreiches Feedback geben könnt.
