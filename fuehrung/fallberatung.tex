\section{Fallberatung für Führungskräfte}
\label{fallberatung}
\index{fallberatung}

Eine kontinuierliche Fallberatung ist sehr hilfreich, um als Führungskraft weiterzulernen und Unterstützung bei Problemen zu bekommen.


\subsection{Supervision vs.~Intervision vs.~kollegiale Fallberatung}

\paragraph{Supervision}~\cite{supervision} ist eine Form der beruflichen Beratung, die von einer Supervisorin oder einem Supervisor geleitet wird. Die Grenze zur Psychotherapie ist dabei schwer zu ziehen.
\index{Supervision}

\paragraph{Intervision}~\cite{intervision} ist ein aus den Niederlanden stammendes Modell, bei dem sich Kolleg\_innen gegenseitig beraten, wobei sie Unterstützung einer externen Moderation haben.
\index{Intervision}

\paragraph{Kollegiale Fallberatung}~\cite{kollegiale-fallberatung} ist ein aus Deutschland stammendes Modell, bei dem sich Kolleg\_innen gegenseitig beraten, wobei sie keine (externe) Moderation in Anspruch nehmen. Ein sehr verbreitetes Modell dafür ist das \emph{Heilbronner Modell}.
\index{Kollegiale Fallberatung}
\index{Heilbronner Modell}


\subsection{Kollegialen Fallberatung}


\subsubsection{Rollen}

Es sind mindestens diese Rollen notwendig:~\cite{kollegiale-fallberatung}

\begin{itemize}
  \item Moderation
  \item Fallgeber\_in
  \item Fallberater\_in
\end{itemize}


\subsubsection{Grober Ablauf}

\begin{enumerate}
  \item Rollen verteilen:
    \begin{itemize}
      \item Wer moderiert?
      \item Wer möchte beraten werden?
    \end{itemize}
  \item die problematische Situation schildern
  \item die beratenden Personen reagieren und beraten
\end{enumerate}


\subsubsection{Detaillierter Ablauf (Beispiel)}

Dies ist der Ablauf, den ich in meiner eigenen kollegialen Fallberatung nutze.


\begin{enumerate}
  \item Wer moderiert den Anfang?
  \item Checkin: Wie geht es mir? Wie fühle ich mich?
  \item Was seit dem letzten Treffen passiert ist
  \item Organisatorisches
  \item Themen auf Karten schreiben inklusive Wünsche dafür:
    \begin{itemize}
      \item Empathie
      \item Analyse
      \item Ratschläge
    \end{itemize}
  \item Themen kurz vorstellen
  \item Punkten, welche Themen wir gerne bearbeiten würden
  \item Thema für heute auswählen (das mit den meisten Punkten)
  \item Wer moderiert den Hauptteil?
  \item Und wer macht Notizen?
  \item Themen-Geber\_in stellt das Thema vor
  \item Thema bearbeiten
  \item Rückmeldung von Themen-Geber\_in:
    \begin{itemize}
      \item Wie geht es mir jetzt?
      \item Was habe ich gelernt?
      \item Welche Ratschläge nehme ich an?
      \item Was sind meine nächsten Schritte?
    \end{itemize}
  \item Runde: Was habe ich heute gelernt?
  \item Feedback zu Prozess, Ablauf, Moderation
  \item Checkout: Mit welchem Gefühl gehe ich raus?
\end{enumerate}
