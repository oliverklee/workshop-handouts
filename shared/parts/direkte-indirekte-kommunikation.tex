\section{Direkte vs.~indirekte Kommunikation}
\label{direkte-kommunikation}
\index{direkte Kommunikation}
\index{indirekte Kommunikation}

Bei \fett{direkter Kommunikation} sagt die Person das, was sie kommunizieren möchte, explizit mit ihren Worten.

Bei \fett{indirekter Kommunikation} benutzt die Person stattdessen Mehrdeutigkeit, Anspielungen, den Tonfall, den Rhythmus der Sprache, Ironie oder Sarkasmus.

Direkte Kommunikation zu nutzen, senkt das Risiko für Missverständnisse deutlich. Außerdem können wir damit mehr Verantwortung für unsere eigene Kommunikation übernehmen.

GfK setzt sehr stark auf direkte Kommunikation.

\subsection{Beispiele}

\renewcommand{\arraystretch}{2.0}
\begin{tabular}{|p{20em}|p{20em}|}
\hline

\fett{direkte Kommunikation} & \fett{indirekte Kommunikation} \\
\hline

Ich würde gerne mit meinem Freund einen Abend zu zweit verbringen. Wärst du bereit, morgen Abend von 20 bis 23 Uhr die WG zu verlassen? &
Hättest du Lust, morgen Abend ohne mich ins Kino zu gehen? \\
\hline

Mir ist kalt. Wäre es okay, wenn ich das Fenster zumache? &
Ziemlich kalt hier. \\
\hline

Ich bin gerade echt genervt. &
(knurrt) \\
\hline

Könntest du die Musik vielleicht etwas leiser machen? &
Tolle Musik! \\
\hline

Ich habe im Moment echt wenig Geld. Zurzeit kaufen wir ein Kilo Kaffeebohnen im Monat. Können wir uns dazu mal zusammensetzen, wie wir unsere Ausgaben für Kaffee senken können? &
Du trinkst zu viel Kaffee. \\
\hline

Könntest du bitte jeden zweiten Tag den Müll runterbringen? &
(stellt dem Mitbewohner den vollen Mülleimer vor die Zimmertür) \\
\hline

Könntest du bitte damit aufhören, mit dem Kuli zu klicken? &
(knallt die Kaffeetasse auf den Schreibtisch) \\
\hline

\end{tabular}
\renewcommand{\arraystretch}{1.0}
