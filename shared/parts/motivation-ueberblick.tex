\section{Definition und Herkunft}

Das Wort \emph{Motivation} ist auf das lateinische Verb \emph{movere} (bewegen, antreiben) zurückzuführen.~\cite{duden-herkunft}

Generell bezeichnet Motivation das, warum Menschen und Tiere generell etwas tun (oder mit etwas aufhören). In der Motivationspsychologie versteht man darunter, auf hinzuarbeiten, das man erreichen will. \cite{rheinberg-motivation}

\section{Arten von Motivation}

Grundsätzlich unterscheidet man zwischen \emph{intrinsischer} und \emph{extrinsischer} Motivation.~\cite{self-determiniation-theory, pelz-motivation} Je nach Quelle variieren Definitionen etwas.

Der Ansatz mit intrinsischer und extrinsischer ist schon etwas älter. Neuere Forschung setzt inzwischen auf das Konzept der Selbstbestimmungstheorie (\emph{self-determiniation theory}), die darauf aufbaut.
\index{Selbstbestimmungstheorie}


\subsection{Intrinsische Motivation}
\label{intrinsische-motivation}
\index{Intrinsische Motivation}

\emph{Intrinsische Motivation} bezeichnet das Handeln aus inneren Antrieben heraus. Intrinsische Motivation wirkt allgemein stärker und nachhaltiger als extrinsische Motivation.

Die Quellen der intrinsischen Motivation sind:

\paragraph{Die Aufgabe an sich:} Die Aufgabe selbst macht mir Spaß oder ist interessant.

\paragraph{Die Person:} Ich kann mit der Aufgabe meine persönlichen Werte und Ziele vorantreiben.
\index{persönliche Werte}

\paragraph{Selbstbestimmung und Kompetenz:} Ich lerne durch die Aufgabe dazu oder kann meine Fähigkeiten und Superkräfte nutzen.
\index{Selbstbestimmung}
\index{Kompetenz}

\paragraph{Mittel und Zweck stimmen überein:} Ich bin für den Zweck meiner Aufgabe intrinsisch motiviert. Zum Beispiel könnte ich einen Artikel lesen, weil ich ein Thema verstehen will.


\subsection{Extrinsische Motivation}
\label{extrinsische-motivation}
\index{Extrinsische Motivation}

Bei der \emph{extrinsischen Motivation} versuchen Menschen, durch ihre Handlungen Vorteile oder Belohnungen zu bekommen oder Nachteile (Bestrafung) zu vermeiden.

Die Quellen der intrinsischen Motivation sind:

\paragraph{Greifbare Anreize:} Geld, Punkte, Klausurzulassung \ldots

\paragraph{Das Umfeld:} Lob, Anerkennung, Wertschätzung, Status \ldots
\index{Lob}
\index{Wertschätzung}

\subsection{Zusammenspiel extrinsischer und intrinsischer Motivation}
\index{Korrumpierungs-Effekt}
\index{Overjustification-Effekt}

Extrinsische Motivation kann die (bestehende) intrinsische Motivation für eine Aufgabe zerstören, weil so das Gefühl der Selbstbestimmung zerstört wird. Dies ist vor allem dann der Fall, wenn die Tätigkeit tatsächlich interessant ist und ich die Belohnung/Bestrafung erwarte. Das wird als \emph{Korrumpierungs- oder Overjustification-Effekt} bezeichnet.~\cite{extrinsic-effects}
