\section{Rhetorisches Grundinstrumentarium}
\label{rhetorik-grundstrinumentarium}

\subsection{Körperhaltung}
\begin{itemize}
\item stabiler Stand: Beine schulterbreit geöffnet, Gewicht gleichmäßig verteilt (gibt Sicherheit)
\item aufrechte Haltung, Kopf gerade
\item dem Publikum zugewandt
\end{itemize}

\subsection{Blickkontakt}
\begin{itemize}
\item "`dynamisches Kreisen"'
\item freundliche Zielgesichter suchen
\item möglichst nie Boden oder Decke
\item kleine Gruppe: alle anschauen
\item große Gruppe: Leute im Raum verteilt anschauen
\end{itemize}

\subsection{Gestik}
\begin{itemize}
\item dient der Verstärkung des Vortrags
\item Hände möglichst frei im \emph{neutralen Bereich}
\item 3 Bereiche:
  \begin{description}
    \item [positiv:] oberhalb der Gürtellinie
    \item [neutral:] etwa auf Gürtellinie
    \item [negativ:] unterhalb der Gürtellinie
  \end{description}
\item kommt gleichzeitig oder kurz vor dem Wort, nie danach
\item soll vor allem \emph{angemessen} sein, kann sonst ablenken
\end{itemize}

\subsection{Mimik}
\begin{itemize}
\item gilt als "`Spiegel der Seele"'
\item kommt gleichzeitig oder kurz vor dem Wort, nie danach
\end{itemize}

\subsection{Stimme, Artikulation}
\begin{itemize}
\item Atmung: Bauchatmung
\item Stimmhöhe: im (angenehmen) Hauptbereich anfangen
\item Sprechtempo: eher langsamer als sonst, bewusste Pausen (Hektik und Ruhe übertragen sich auf das Publikum)
\item Lautstärke: eher lauter als sonst
\item sehr deutlich artikulieren, Endungen nicht verschlucken
\item keine Störlaute, möglichst wenig Füllwörter
\end{itemize}

\subsubsection{Literatur zum Grundinstrumentarium}
\cite{science-of-breath}

\subsubsection{Bücher zum Vorlesen-Üben}
\cite{anhalter, der-liebe-ziel, brenneisen, max-und-moritz, reimtopf, galgenlieder}


\subsection{Je nachdem und überhaupt: Vier Aspekte der Angemessenheit}

\unitlength 1mm
\begin{center}
\begin{picture}(120,120)
\thicklines

% Raute
%\put(60, 10) {\line( 2, 3){33}}
%\put(60, 10) {\line(-2, 3){33}}
%\put(60,109) {\line( 2,-3){33}}
%\put(60,109) {\line(-2,-3){33}}

% Pfeile
%\put(53,59.5) {\vector(-1, 0){23}}
%\put(67,59.5) {\vector( 1, 0){23}}
%\put(60,65  ) {\vector( 0, 1){40}}
%\put(60,54  ) {\vector( 0,-1){40}}

% Texte
\put(56,58.5){\fett{Rede}}
\put(54,112){\fett{Thema}}
\put(53,4){\fett{Situation}}
\put(8,58.5){\fett{RednerIn}}
\put(96,58.5){\fett{ZuhörerInnen}}

\thinlines
\end{picture}
\end{center}

Angemessenheit an Thema, eigenen Typ, ZuhörerInnen und Situation nicht vernachlässigen! Je nach persönlicher Zielsetzung kann man sich aber auch entscheiden, mal ein Element bewusst zu missachten (etwa völlig unangemessene Kleidung zu tragen).
