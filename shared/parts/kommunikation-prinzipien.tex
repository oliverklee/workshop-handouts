\section{Prinzipien der Kommunikation}
\label{kommunikationsprinzipien}
\index{Kommunikation}
\index{Missverständnisse}
\index{Sender-Empfänger-Modell}

\begin{itemize}
  \item Es gibt bei Kommunikation immer sendende und (mindestens) eine empfangende Partei (Sender-Empfänger-Modell nach Shannon und Weaver). Diese Rollen können in einer Interaktion öfter wechseln.
  \item Die sendende Partei ist hauptsächlich dafür zuständig, dass die Nachricht bei der anderen Person ankommt. Sprich: Falls etwas bei der anderen Person falsch oder gar nicht angekommen ist, ist es für euch nicht damit getan, dass ihr es der anderen Person \glqq ja gesagt habt\grqq.
  \item Jede Partei hat nur auf ihre eigene Hälfte der Kommunikation direkten Einfluss.
  \item Missverständnisse passieren, und sie sind eher die Regel denn die Ausnahme~-- wir bemerken sie nur oft nicht.
\end{itemize}


\subsection{Metakommunikation}
\label{metakommunikation}
\index{Metakommunikation}

\emph{Metakommunikation} (\glqq Kommunikation über Kommunikation\grqq) bedeutet, die Kommunikation auf eine höhere Ebene zu verlagern und darüber zu reden, wie wir miteinander reden, wie wir miteinander umgehen und was uns beschäftigt.
