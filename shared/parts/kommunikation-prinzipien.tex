\section{Prinzipien der Kommunikation}
\label{kommunikationsprinzipien}
\index{Kommunikation}

\begin{itemize}
  \item Es gibt bei Kommunikation immer sendende und (mindestens) eine empfangende Partei.
  \item Die sendende Partei hat die Verantwortung dafür (und das Interesse daran), dass die Kommunikation erfolgreich ist.
  \item Jede Partei hat nur auf ihre eigene Hälfte der Kommunikation direkten Einfluss.
  \item Missverständnisse passieren sind eher die Regel denn die Ausnahme~-- wir bemerken sie nur oft nicht.
\end{itemize}


\subsection{Metakommunikation}
\label{metakommunikation}
\index{Metakommunikation}

\emph{Metakommunikation} (\glqq Kommunikation über Kommunikation\grqq) bedeutet, die Kommunikation auf eine höhere Ebene zu verlagern und darüber zu reden, wie wir miteinander reden, wie wir miteinander umgehen und was uns beschäftigt.
