\section{Elemente von Vertrauen}
\label{vertrauen-element}
\index{Vertrauen: Elemente}


\subsection{Nach Brené Brown}

In ihrem Talk \emph{The Anatomy of Trust}~\cite{anatomy-of-trust} liefert Brené Brown mit dem Akronym \emph{BRAVING} eine schön knackige Merkhilfe, die (um das schöne Akronym zu bilden) inhaltlich allerdings etwas unscharf wird. Ihre Ergebnisse bauen auf einer Metaanalyse der Forschung zu Vertrauen auf.
\index{BRAVING}

Sie fasst die Elemente von Vertrauen wie folgt zusammen:

\fett{B}oundaries \\
\fett{R}eliability \\
\fett{A}ccountability \\
\fett{V}ault \\
\fett{I}ntegrity \\
\fett{N}on-judgement \\
\fett{G}enerosity


\subsubsection{Grenzen respektieren (Boundaries)}
\index{Grenzen respektieren}

\begin{itemize}
  \item die eigenen Grenzen kennen, kommunizieren und schützen
  \item die Grenzen anderer Menschen anerkennen und respektieren
\end{itemize}



\subsubsection{Zuverlässigkeit (Reliability)}
\index{Zuverlässigkeit}

\begin{itemize}
  \item das tun, was ihr versprochen/zugesagt habt
  \item immer und immer wieder
\end{itemize}


\subsubsection{Verantwortung übernehmen (Accountability)}
\index{Verantwortung übernehmen}

\begin{itemize}
  \item nach einem Fehler dafür geradestehen, um Entschuldigung bitten, und es wiedergutmachen
  \item anderen ermöglichen, für Fehler geradezustehen, um Entschuldigung zu bitten, und es wiedergutzumachen
\end{itemize}


\subsubsection{Vertraulichkeit wahren (Vault)}
\index{Vertraulichkeit}

\begin{itemize}
  \item \glqq Was ich mit dir teile, behandelst du vertraulich.\grqq
  \item \glqq Was du mit mir teilst, behandele ich vertraulich.\grqq
  \item \glqq Ich erlebe, dass du Sachen anderer Personen vertraulich behandelst.\grqq
\end{itemize}


\subsubsection{Integrität (Integrity)}
\index{Integrität}

\begin{itemize}
  \item die eigenen Werte und Prinzipien tatsächlich leben
  \item Worten Taten folgen lassen
  \item das Richtige tun (statt das Angenehme, Schnelle oder Einfache)
\end{itemize}


\subsubsection{Nicht-Verurteilen (Non-judgement)}
\index{Nicht-Verurteilen}

\begin{itemize}
  \item \glqq Ich kann dich um Hilfe bitten, ohne dass du mich dafür verurteilst.\grqq
  \item \glqq Du kannst mich um Hilfe bitten, ohne dass ich dich dafür verurteile.\grqq
\end{itemize}


\subsubsection{Großzügigkeit (Generosity)}
\index{Großzügigkeit}

\begin{itemize}
  \item \glqq Du gehst bei meinen Worten und Taten von guten Absichten aus und fragst nach.\grqq
\end{itemize}


\subsection{Nach Charles Feltman}

Charles Feltman geht mit seinem Büchlein \emph{The Thin Book of Trust}~\cite{thin-book-of-trust} eher in die Richtung Lebenshilfe-Buch oder Business-Buch und ist weniger wissenschaftlich fundiert als die anderen beiden Werke.

Er stellt die folgenden Elemente von Vertrauen vor:


\subsubsection{Aufrichtigkeit}
\index{Aufrichtigkeit}

\glqq Was ich sage, meine ich so, und ich handele entsprechend.\grqq


\subsubsection{Zuverlässigkeit}
\index{Zuverlässigkeit}

\glqq Du kannst darauf zählen, dass ich liefere, was ich versprochen habe.\grqq


\subsubsection{Kompetenz}
\index{Kompetenz}

\glqq Ich weiß, was ich kann. Und ich weiß auch, was ich nicht kann.\grqq


\subsubsection{Wohlwollen (Care)}
\index{Wohlwollen}

\glqq Ich berücksichtige unser beider Interessen, wenn ich Entscheidungen fälle oder handele.\grqq


\subsection{Nach Ariane Jäckel}

Ariane Jäckel stellt in ihrer Dissertation \emph{Gesundes Vertrauen in Organisationen: Eine Untersuchung der Vertrauensbeziehung zwischen Führungskraft und Mitarbeiter}~\cite{gesundes-vertrauen-in-organisationen} den aktuellen Stand der Forschung dar, geht sehr in die Tiefe und ist wissenschaftlich sehr sauber.

Sie hat die folgenden Dimensionen von Vertrauenswürdigkeit identifiziert:


\subsubsection{Kompetenz}
\index{Kompetenz}

Darunter fallen aus den anderen Konzepten diese Unterpunkte:

\begin{itemize}
  \item Zuverlässigkeit (Brené Brown, Charles Feltman)
  \item Kompetenz (Charles Feltman)
\end{itemize}


\subsubsection{Integrität}
\index{Integrität}

Darunter fallen aus den anderen Konzepten diese Unterpunkte:

\begin{itemize}
  \item Verantwortung übernehmen (Brené Brown)
  \item Vertraulichkeit wahren (Brené Brown)
  \item Integrität (Brené Brown)
  \item Aufrichtigkeit (Charles Feltman)
\end{itemize}


\subsubsection{Wohlwollen}
\index{Wohlwollen}

Darunter fallen aus den anderen Konzepten diese Unterpunkte:

\begin{itemize}
  \item Grenzen respektieren (Brené Brown)
  \item Nicht-Verurteilen (Brené Brown)
  \item Großzügigkeit (Brené Brown)
  \item Wohlwollen (Charles Feltman)
\end{itemize}
