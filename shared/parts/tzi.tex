\section{Themenzentrierte Interaktion (TZI)}
\label{tzi}
\index{Themenzentrierte Interaktion}
\index{TZI}

Die Themenzentrierte Interaktion (TZI)~\cite{tzi-cohn, tzi-reiser} von Ruth C.~Cohn ist ein Handlungskonzept zur Arbeit in Gruppen. Ziele sind soziales Lernen, die Förderung persönlicher Entwicklung und Fortschritte im Thema.


\subsection{Postulate der TZI}

\subsubsection{Sei deine eigene Chairperson!}

Jede Person spricht nur für sich selbst. Jede Person hat die Verantwortung, sich um sich selbst und ihre Bedürfnisse zu kümmern und sich dafür einzusetzen.

\subsubsection{Störungen haben Vorrang!}

Wenn etwas den Prozess, das Arbeiten oder das Lernen stört, kümmern wir uns so schnell wie möglich darum. Das schließt auch Konflikte ein.

\subsubsection{Verantworte dein Tun und Lassen~-- persönlich und gesellschaftlich!}

Deine Handlungen und Worte haben Konsequenzen. Setze dich damit auseinander und übernimm Verantwortung dafür.


\subsection{Vierfaktorenmodell}

\paragraph{Ich:} Ich selbst.

\paragraph{Wir:} Die Gruppe, in der wir gerade hier sind.

\paragraph{Es:} Die Aufgabe oder das Problem.

\paragraph{Umfeld:} Die Rahmenbedingungen und äußeren Umstände.
