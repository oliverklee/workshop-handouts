\section{\emph{Ask-Culture} vs.~\emph{Guess-Culture}}
\label{ask-guess-culture}
\index{Ask-Culture}
\index{Guess-Culture}

\emph{Ask-Culture} vs.~\emph{Guess-Culture}\cite{ask-guess-culture} ist ein Kontinuum zwischen zwei Extremen:


\subsection{Annahmen der \emph{Ask-Culture}}

\begin{itemize}
  \item Ein Ja ist ein Ja, und ein Nein ist ein Nein.
  \item Wer Menschen um etwas bittet, hat die Verantwortung, möglicherweise ein Nein als Antwort zu erhalten und damit zurechtzukommen.
  \item Es ist völlig okay, Menschen um etwas zu bitten. Sie müssen ja nicht Ja sagen.
  \item Wenn mich jemand um etwas bittet, ist es meine Verantwortung, Nein zu sagen, wenn ich der Bitte nicht gerne nachkommen möchte.
\end{itemize}


\subsection{Annahmen der \emph{Guess-Culture}}

\begin{itemize}
  \item Wir bitten Menschen nur dann um etwas, wenn wir es für sehr wahrscheinlich halten, dass die Person Ja sagt.
  \item Wir streben an, dass wir nicht mehr um etwas zu bitten brauchen, sondern es uns die andere Person von sich aus anbietet.
  \item Bitten um Dinge, zu denen wir nicht gerne Ja sagen, sind unhöflich.
  \item Auf eine Bitte Nein zu sagen, ist unhöflich.
  \item Konflikte sind etwas, was unbedingt zu vermeiden ist.
  \item Wir sind dafür verantwortlich, zu erahnen, was andere Menschen brauchen könnten.
\end{itemize}


\subsection{Begegnungen zwischen den beiden Kulturen}

Wenn Menschen aufeinandertreffen, die auf diesem Kontinuum weiter auseinanderliegen, sind Missverständnisse und Konflikte wahrscheinlich. In so einem Fall kann es hilfreich sein, miteinander in die Metakommunikation (siehe S.~\pageref{metakommunikation}) über dieses Konzept zu gehen.


\subsection{GfK auf dem Kontinuum}

GfK tendiert sehr stark zur \emph{Ask-Culture}.
