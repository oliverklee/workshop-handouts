\section{Design-Grunds"atze: La Crap}
\label{lacrap}

\index{LA CRAP}
\index{Design-Grundsätze}
\index{Grafikdesign|\see{Design-Grundsätze}}
\index{Lesbarkeit}
\index{Augenbewegungen}
\index{Contrast}
\index{Kontrast|\see{Contract}}
\index{Repetition}
\index{Konsistenz|\see{Repetition}}
\index{Aligmment}
\index{Ausrichtung|\see{Aligmment}}
\index{Proximity}
\index{Nähe|\see{Proximity}}

\begin{itemize}
\item \fett{L}esbarkeit sichern
\item \fett{A}ugenbewegung berücksichtigen

\smallskip
\item \fett{C}ontrast zwischen Designelementen (Weichei-Regel beachten)
\item \fett{R}epetition (Wiederholung von Designelementen, Konsistenz)
\item \fett{A}lignment (Ausrichtung an unsichtbaren Linien)
\item \fett{P}roximity (räumliche Nähe von inhaltlich Zusammengehörendem)
\end{itemize}

\subsection{Lesbarkeit}
Euer Text sollte vor allem lesbar sein~-- dass den Text jemand liest, ist in den meisten Fällen der Grund, ihn überhaupt zu schreiben. Wenn ihr den Text trotzdem so gestalten wollt, dass es schwer lesbar ist, braucht ihr dafür eine sehr gute Begründung.

\subsection{Augenbewegungen berücksichtigen}
Dies ist dann relevant, wenn ihr mit mehrspaltigem Satz oder mit Folien (oder Plakaten) arbeitet. Grundregel
\begin{itemize}
  \item von oben nach unten
  \item von links nach rechts
  \item das Auge folge einem \emph{Z} (oder einem umgekehrten \emph{S})
  \item Fokuspunkte: macht durch auffällige Objekte klar, wo das Auge zuerst hinwandern soll
\end{itemize}

\subsection{Contrast}
Kontrast kann bestehen in:
\begin{itemize}
  \item Helligkeit
  \item Farbe
  \item Schriftart
  \item Duktus
  \item Ausrichtung
  \item \ldots
\end{itemize}

\subsection{Repetition}
Inhaltlich ähnliche Objekte sollten auch optisch ähnlich sein. So sollten etwa die Aufzählungspunkte einer Ebene alle gleich aussehen in Form, Farbe, Größe, Ausrichtung.

\subsection{Alignment}
Richtet eure Objekte an unsichtbaren Kanten aus. Wenn ihr dies bewusst nicht tun wollt, dann macht es wirklich \emph{deutlich} (Weichei-Regel) und habt eine Begründung.

\subsection{Proximity}
Objekte, die inhaltlich zusammen gehören, sollen auch optisch nahe zusammen sein. Der Umkehrschluss gilt entsprechend.

\subsection{Quelle}
Das CRAP (ohne LA) kommt aus \cite{non-designer}.
