\chapter{Lehrmethoden}
\section{Frontalunterricht, Vortrag}
\begin{itemize}
 \item höchstens 20 Minuten am Stück
 \item vorher klarmachen, ob Mitschrieben gewünscht bzw. gefordert ist
\end{itemize}

\section{Lehrgespräch}
\begin{itemize}
 \item Gruppe durch geschicktes Fragen anregen, sich mit einem Thema auseinanderzusetzen
\end{itemize}

\section{Einzelarbeit}
\begin{itemize}
 \item Voraussetzung: Teilis haben schon genug Vorkenntnisse für die Aufgabe
 \item Aufgabenstellung muss schriftlich und eindeutig sein
 \item TutorIn geht durch die Reihen (für Fragen und um Anmerkungen zu Arbeitstechniken zu geben)
 \item Einzelarbeit kurz unterbrechen und erklären, wenn alle Probleme an der gleichen Stelle haben
\end{itemize}

\section{Partnerarbeit}
\begin{itemize}
 \item bietet sich an, Vorwissen aufzubauen, wenn Einzelarbeit für die Aufgabe zu schwierig ist
\end{itemize}

\section{Gruppenarbeit}
\begin{itemize}
 \item für Aufgaben, bei denen tatsächlich Diskussionsbedarf besteht und die von ``verschiedenen Köpfen'' profitieren
 \item Zeit für die Vorstellung der Ergebnisse einplanen
 \item prima Gelegenheit für die Teilis, Präsentation zu üben
\end{itemize}

\section{Planspiel, Simulation}
\begin{itemize}
 \item Beispiel: Routing in Netzwerken, bei dem die Leute die Datenpakete durch den Raum tragen
 \item macht den Stoff noch anschaulicher
\end{itemize}

\section{Weitere Methoden}

\twocolumn
\begin{itemize}
  \item auf Ecken verteilen
  \item Aufstellung
  \item Blitzlicht
  \item Brainstorming
  \item Christkindl-Effekt
  \item Clustern
  \item Diskussion (moderiert)
  \item Fahrplan visualisieren
  \item Fallbeispiel vorstellen
  \item Feedbackrunde
  \item Fishbowl
  \item Haribo-Analyse
  \item Hausaufgaben
  \item der heiße Stuhl
  \item Kartenabfrage
  \item Kennenlernspiele
  \item Kopfstand
  \item Kurzinput mit Plakat
  \item Loreley-Phase
  \item Mindmapping
  \item Open Space
  \item Partnerarbeit
  \item Pause
  \item Poster-Walk
  \item Pro-/Kontra-Diskussion
  \item Punkten
  \item Quiz
  \item Rollenspiel
  \item Schnitzeljagd, Ralley
  \item Spiele
  \item stille Diskussion
  \item Stimmungsbarometer
  \item Test schreiben
  \item Theater der Unterdrückten
  \item Übungen am PC
  \item Übungsaufgaben
  \item Umfrage
  \item Vernissage
  \item wilde Fragestunde
  \item World-Café
  \item Zukunftswerkstatt
\end{itemize}

\onecolumn
