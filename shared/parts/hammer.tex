\section{Die Geschichte mit dem Hammer}
\label{hammer-geschichte}
\index{Geschichte mit dem Hammer}
\index{Hammer}

Diese Kurzgeschichte stammt aus dem Buch \emph{Anleitung zum Unglücklichsein}~\cite{anleitung-zum-ungluecklichsein} von Paul Watzlawick.

\begin{quote}

Ein Mann will ein Bild aufhängen. Den Nagel hat er, nicht aber den Hammer. Der Nachbar hat einen. Also beschließt unser Mann, hinüberzugehen und ihn auszuborgen.

Doch da kommt ihm ein Zweifel: Was, wenn der Nachbar mir den Hammer nicht leihen will? Gestern schon grüßte er mich nur so flüchtig. Vielleicht war er in Eile.

Aber vielleicht war die Eile nur vorgeschützt, und er hat etwas gegen mich. Und was? Ich habe ihm nichts angetan; der bildet sich da etwas ein. Wenn jemand von mir ein Werkzeug borgen wollte, ich gäbe es ihm sofort. Und warum er nicht? Wie kann man einem Mitmenschen einen so einfachen Gefallen abschlagen? Leute wie dieser Kerl vergiften einem das Leben. Und dann bildet er sich noch ein, ich sei auf ihn angewiesen. Bloß weil er einen Hammer hat. Jetzt reicht's mir wirklich.

Und so stürmt er hinüber, läutet, der Nachbar öffnet, doch noch bevor er \glqq Guten Tag\grqq{} sagen kann, schreit ihn unser Mann an: \glqq Behalten Sie sich Ihren Hammer, Sie Rüpel!\grqq

\end{quote}
