\section{Paarinterview zum Kennenlernen}
\label{paarinterview}
\index{Paarinterview}

Nehmt euch für das Interview 10~Minuten Zeit pro Person. Wechselt selbstständig.

\subsection{Leitfragen}

\begin{itemize}
  \item Wo und wie wohnst du?
  \item Was machst du in Beruf und Ehrenamt so? Und was hast du vorher so Interessantes gemacht?
  \item Was sind ein paar Dinge, die dir im Leben zurzeit Freude bereiten?
  \item Was brauchst du (von anderen Personen oder der Umgebung), damit die Zusammenarbeit mit dir gut funktioniert?
  \item Was sollten andere Menschen über dich wissen, wenn sie mit dir zusammenarbeiten?
  \item Was machst du, um trotz der aktuellen Krisen psychisch halbwegs gesund zu bleiben?
  \item Was ist ein \emph{Guilty Pleasure}, dem du ab und an frönst?
\end{itemize}


\subsection{Bonusfragen}

Falls ihr euch schon gut kennt und noch etwas Zeit habt, könnt ihr euch mit diesen Zusatzfragen noch besser kennenlernen.

Diese Fragen kommen aus dem Spiel \emph{Gesprächsstoff XL} \cite{gespraechsstoff}.

\begin{itemize}
  \item Welche Fernsehsendung hast du nie verpasst, als du noch jünger warst?
  \item Wie meinst du, würden dich deine Freund\_innen beschreiben, wenn sie nur drei Wörter verwenden dürften?
  \item Kannst du ein Beispiel dafür nennen, wann du einmal zum richtigen Zeitpunkt am richtigen Ort warst?
  \item Was ist, von der Persönlichkeit her, der größte Unterschied zwischen dir und deinen Eltern?
  \item Was antwortest du einem Kind, das fragt, ob es einen Gott gibt?
  \item Was ist dein Lieblings-Knabberzeug (oder -Nascherei)?
  \item Erzähle von bedeutungslosem Wissen, das du hast.
  \item Kannst du etwas nennen, von dem du wünscht, früher damit angefangen zu haben?
  \item Welche Person hat dich zuletzt so richtig wütend gemacht?
  \item Wenn du wählen müsstest: Würdest du lieber einen anderen Menschen umbringen und straffrei davonkommen oder 25 Jahre für einen Mord im Gefängnis sitzen, den du nicht begangen hast?
  \item Bist du schon einmal im Kino eingeschlafen?
  \item Wenn du den Leben noch einmal leben könntest, was würdest du nicht wieder tun?
  \item Erzähle etwas über ein Ereignis der letzten Jahre, an das du dich für immer erinnern wirst.
\end{itemize}

Die Antworten auf diese Fragen sind \emph{nicht} Teil der Vorstellung vor der Gruppe.
