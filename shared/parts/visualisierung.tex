\section{Visualisierung: Plakate \& Co}

\subsection{Warum Visualisierung?}
\begin{itemize}
  \item Der Mensch ist ein Augentier!
  \item besserer Überblick über Erarbeitetes
  \item "`externer Speicher"' für Arbeitsgedächtnis/Kurzzeitgedächtnis
\end{itemize}

\subsection{Grundregeln}

\subsubsection{Erst planen, dann malen!}
Sonst kann es nämlich passieren, dass ihr irgendwann feststellt, dass der Platz nicht reicht, und eure ganze bisherige Arbeit für die Katz ist.

\subsubsection{Aufbau}
\begin{itemize}
  \item jedem Plakat eine Überschrift geben
  \item Überschrift in Container setzen
  \item Rahmen um das komplette Plakat benutzen
  \item von links nach rechts und von oben nach unten aufbauen (ansonsten klare Fokuspunkte setzen)
  \item viele Bilder und Symbole benutzen (siehe dazu \cite{bikablo}), wenig Text
  \item von vorne nach hinten aufbauen (also erst den Vordergrund malen, dann den Hintergrund)
  \item Schatten sind rechts unten, da das Licht von links oben kommt
\end{itemize}

\subsubsection{Sprache}
\begin{itemize}
  \item Halbsätze verwenden
  \item keine Abk.~verw.
\end{itemize}

\subsubsection{Schrift}
\begin{itemize}
  \item Druckschrift und normale Groß- und Kleinschreibung benutzen
  \item VERSALIEN vermeiden
  \item nicht visuell {\tiny nuscheln} oder {\LARGE brüllen}!
  \item kurze Ober-/Unterlängen benutzen
\end{itemize}

\subsubsection{Farben}
\begin{itemize}
  \item schwarz für Schrift (gilt nur für Plakate)
  \item zusätzlich noch 1 oder 2 Farben pro Plakat/Folie, in Ausnahmen auch mehr
\end{itemize}

\subsection{Literatur}
Zur Visualisierung auf Plakaten findet ihr in \cite{visualisieren-praesentieren-moderieren} sehr viel.

Speziell mit Folien (im Sinne von Keynote, OpenOffice.org Impress, PowerPoint) beschäftigt sich \emph{Presentation Zen} \cite{presentation-zen}.

Zur Visualisierung von Inhalten und Zusammenhängen ist \emph{The Back of the Napkin} \cite{napkin} sehr gut.

Für schnell zu zeichnende visuelle Vokabeln empfehle ich die drei BiKaBlo-Bücher. \cite{bikablo, bikablo-2, bikablo-emotions}
