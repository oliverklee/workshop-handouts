\section{Das Johari-Fenster}
\label{johari-fenster}
\index{Johari-Fenster}

Das \emph{Johari-Fenster}~\cite{johari-window, gruppendnamik-einfuehrung} (benannt nach den Erfindern Joseph Luft und Harry Ingham) ist ein Modell zum Abgleich von Selbst- und Fremdwahrnehmung.

Es wird als \glqq Fenster\grqq\ bezeichnet, weil die Tabelle mit den vier Feldern einem alten Fenster mit Fensterkreuz ähnelt. (Die Älteren unter uns können sich vielleicht noch erinnern.)

\subsection{Die Bereiche des Johari-Fensters}

\renewcommand{\arraystretch}{2.0}
\begin{tabular}{p{10em}|p{10em}|p{7em}|}
& mir bekannt \cellcolor{lightgray}  & mir unbekannt \cellcolor{lightgray}
\\ \hline

anderen bekannt \cellcolor{lightgray} & öffentliche Person & blinder Fleck
\\ \hline

anderen unbekannt \cellcolor{lightgray} & mein Geheimnis & Unbekanntes
\\ \hline
\end{tabular}
\renewcommand{\arraystretch}{1.0}


\subsection{Bereiche des Johari-Fensters verkleinern}

Durch \fett{Feedback} könnt ihr den blinden Fleck verkleinern.
\index{Feedback}

Durch \fett{Selbstkundgabe} könnt ihr euer Geheimnis verkleinern.
\index{Selbstkundgabe}

Durch \fett{Therapie} könnt ihr das Unbekannte verkleinern.
\index{Therapie}
