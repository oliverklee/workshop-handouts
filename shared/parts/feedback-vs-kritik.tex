\section{Feedback, Kritik, Bitten, Forderungen und Grenzen}
\label{feedback-vs-kritik}

\emph{Feedback}, \emph{Kritik}, \emph{Bitten}/\emph{Forderungen} und das \emph{Setzen von Grenzen} sind unterschiedliche Sachen, und es ist hilfreich, wenn ihr sie klar unterscheiden könnt, um dann bewusst auszuwählen und klar zu kommunizieren.


\subsection{Feedback}
\index{Feedback}

Das Ziel von Feedback ist, dass die andere Person daraus lernen und daran wachsen kann.

Bei Feedback geht es um die andere Person, nicht um euch.

Feedback ist ein Geschenk, und es ist okay, Geschenke nicht annehmen zu wollen.

Ein Geschenk kann man nicht aufzwingen~-- jemandem ohne Zustimmung Feedback zu geben, ist Gewalt.


\subsection{Bitten und Forderungen}
\index{Bitten}
\index{Forderungen}

Bei einer Bitte oder Forderung geht es darum, dass ihr etwas für euch von der anderen Person braucht oder möchtet.

Dort liegt der Fokus also bei euch, nicht bei der anderen Person.

Der wesentliche Unterschied zwischen Bitten und Forderungen ist, was bei einem Nein oder einer Ablehnung passiert, und wie weit ihr zu Verhandlungen bereit seid.


\subsection{Kritik}
\index{Kritik}

Wenn es tatsächlich darum geht, zusammen an einer Sache zu arbeiten oder zu klären, kann Kritik an der Sache (nicht an der Person!) hilfreich sein. In diesem Fall liegt der Fokus auf der Sache, nicht auf einer Person.

Ansonsten ist Kritik oft undeutlich kommuniziertes Feedback oder eine unklar kommunizierte Bitte. In solchen Fällen ist es hilfreicher, stattdessen klar Feedback oder eine Bitte zu äußern.


\subsection{Grenzen setzen}
\index{Grenzen}

Bei persönlichen Grenzen geht es darum, wozu ihr bereit seid~-- also was ihr zu tun bereit seid und was ihr mit euch zu machen bereit seid.

Das Ziel von euren persönlichen Grenzen ist, euch zu schützen. Persönliche Grenzen sind daher auch nicht verhandelbar.

Die Verantwortung, eure Grenzen zu wahren und zu verteidigen, liegt letztendlich bei euch selbst. Notfalls entfernt ihr euch aus der Situation, ruft die Polizei oder Ähnliches.

Wenn ihr Grenzen kommuniziert, ist dies daher keine Bitte oder Forderung an die andere Person, sondern vor allem eine Information.
