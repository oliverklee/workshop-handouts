\section{Feedback vs.~Kritik vs.~Bitten}
\label{feedback-vs-kritik}

\emph{Feedback}, \emph{Kritik} und \emph{Bitten} sind drei unterschiedliche Sachen, und es ist hilfreich, wenn ihr sie klar unterscheiden könnt, um dann bewusst auszuwählen und klar zu kommunizieren.


\subsection{Feedback}
\index{Feedback}

Das Ziel von Feedback ist, dass die andere Person daraus lernen und daran wachsen kann.

Bei Feedback geht es um die andere Person, nicht um euch.

Feedback ist ein Geschenk, und es ist okay, Geschenke nicht annehmen zu wollen.

Ein Geschenk kann man nicht aufzwingen~-- jemandem ohne Zustimmung Feedback zu geben, ist Gewalt.


\subsection{Bitten}
\index{Bitten}

Bei einer Bitte geht es darum, dass ihr etwas für euch von der anderen Person braucht oder möchtet.

Dort liegt der Fokus also bei euch, nicht bei der anderen Person.

Bitten sind keine Forderungen: Es ist also auch okay, eine Bitte abzulehnen.

Mehr zu Kriterien für hilfreiche Bitten findet ihr ab Seite~\pageref{bitten}.


\subsection{Kritik}
\index{Kritik}

Wenn es tatsächlich darum geht, zusammen an einer Sache zu arbeiten oder zu klären, kann Kritik an der Sache (nicht an der Person!) hilfreich sein. In diesem Fall liegt der Fokus auf der Sache, nicht auf einer Person.

Ansonsten ist Kritik oft undeutlich kommuniziertes Feedback oder eine unklar kommunizierte Bitte. In solchen Fällen ist es hilfreicher, stattdessen klar Feedback oder eine Bitte zu äußern.
