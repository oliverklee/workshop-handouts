%----------------------------------------------------------------------------------
% Typografie und Fonts
%----------------------------------------------------------------------------------

% Vektorfonts statt Pixelfonts
\usepackage[T1]{fontenc}
\usepackage{lmodern}

% Input-Encoding für UTF-8
\usepackage[utf8]{inputenc}

% Anderer Sans-serif-Font
% https://www.tug.org/FontCatalogue/allfonts.html
% https://www.draketo.de/anderes/latex-fonts.html
\usepackage{librefranklin}
\renewcommand{\familydefault}{\sfdefault}

\newcommand{\fett}[1]{\textsf{\textbf{#1}}}


%----------------------------------------------------------------------------------
% Mehrspachigkeit
%----------------------------------------------------------------------------------

% Mehrsprachigkeit für Deutsch und Englisch erlauben
\usepackage[ngerman,english]{babel}

% Anführungszeichen sprachabhängig machen
\usepackage[babel]{csquotes}


%----------------------------------------------------------------------------------
% Größen, Abstände und Einzüge
%----------------------------------------------------------------------------------

% Mehr von der Seite nutzen
\usepackage[top=2cm, bottom=3cm, right=3cm, left=2cm]{geometry}

% Absätze werden nicht eingezogen, sondern vertikal abgesetzt
\setlength{\parindent}{0mm}
\addtolength{\parskip}{0.5em}

% Descriptions ohne Einzug
\renewenvironment{description}[1][0pt]
{\list{}{
    \labelwidth=0pt \leftmargin=#1
    \let\makelabel\descriptionlabel
}
}
{\endlist}

% Im Zweifel die Seite nicht komplett füllen, aber keine zusätzlichen vertikalen Abstände hinzufügen
\raggedbottom

% Hurenkinder und Schusterjungen verhindern
\clubpenalty10000
\widowpenalty10000
\displaywidowpenalty=10000

%----------------------------------------------------------------------------------
% Literaturverzeichnis, Links und Querverweise
%----------------------------------------------------------------------------------

% Bibliographieeinstellungen
\bibliographystyle{alphadin}

% klickbare Verweise
\usepackage[pdftex,plainpages=false,pdfpagelabels]{hyperref}

% nette URLs
\usepackage{url}

% Stichwortverzeichnis
\usepackage{makeidx}
\makeindex

%----------------------------------------------------------------------------------
% Dokument- und Seitenstruktur
%----------------------------------------------------------------------------------

% zweispaltiges Layout möglich machen
\usepackage{multicol}

% Seiten-Kopfzeilen und -Fußzeilen
\usepackage{scrlayer-scrpage}

% Maximal drei Ebenen nummerieren
\setcounter{secnumdepth}{2}

% Maximale 2 Ebenen im Inhaltsverzeichnis
\setcounter{tocdepth}{1}

% Mit jeder Section eine neue Seite anfangen
% https://tex.stackexchange.com/questions/9497/start-new-page-with-each-section
\usepackage{etoolbox}
\preto{\section}{%
  \ifnum\value{section}=0 \else\clearpage\fi
}


%----------------------------------------------------------------------------------
% Grafiken, Farben und Boxen
%----------------------------------------------------------------------------------

% Farben
\usepackage{xcolor}

% Grafiken
\usepackage[pdftex]{graphicx}

% Boxen inklusive Schattierung
\usepackage{framed}
\definecolor{shadecolor}{rgb}{0.8,0.8,0.8}
