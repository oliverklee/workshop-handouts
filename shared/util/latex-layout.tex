%----------------------------------------------------------------------------------
% Typografie und Fonts
%----------------------------------------------------------------------------------

% Vektorfonts statt Pixelfonts
\usepackage[T1]{fontenc}
\usepackage{lmodern}

% Input-Encoding für UTF-8
\usepackage[utf8]{inputenc}

\newcommand{\fett}[1]{\textsf{\textbf{#1}}}

%----------------------------------------------------------------------------------
% Pakete und Parameter
%----------------------------------------------------------------------------------

% Mehrsprachigkeit für Deutsch und Englisch erlauben
\usepackage[ngerman,english]{babel}

% Mehr von der Seitenbreite nutzen
% \usepackage{a4wide}

% Grafikpaket
\usepackage{color}
\usepackage[pdftex]{graphicx}

% Absätze werden nicht eingezogen, sondern vertikal abgesetzt
\setlength{\parindent}{0mm}
\addtolength{\parskip}{0.4em}

% Bibliographieeinstellungen
\usepackage{natbib}
\bibliographystyle{alpha}

% lesbare Verweise
\usepackage[pdftex,plainpages=false,pdfpagelabels]{hyperref}

% nette URLs
\usepackage{url}

% für Boxen etc.
% \usepackage{framed}
\definecolor{shadecolor}{rgb}{0.8,0.8,0.8}

% Anführungszeichen sprachabhängig machen
% \usepackage[babel]{csquotes}


%----------------------------------------------------------------------------------
% Seitenlayout
%----------------------------------------------------------------------------------


% Seiten-Kopfzeilen und -Fußzeilen
\usepackage{scrlayer-scrpage}
\pagestyle{headings}

% Kopfzeile auf linker Seite: "1  Einführung"
\renewcommand{\chaptermark}[1]{%
\markboth{\thechapter\ \ \ \ #1}{}}

% Kopfzeile auf rechter Seite: "1.1  Basics"
\renewcommand{\sectionmark}[1]{%
\markright{\thesection\ \ \ \ #1}{}}

% Seitenlayout
\topmargin0mm
\footskip10mm % Abstand von unserem Rand zu Datum

\newlength{\fullwidth} % Seites des Textes plus Randnotizen
\setlength{\fullwidth}{\textwidth}
\addtolength{\fullwidth}{\marginparsep}
\addtolength{\fullwidth}{\marginparwidth}

% Maximal drei Ebenen nummerieren
\setcounter{secnumdepth}{3}
% Maximale Gliederungstiefe, die noch ins Inhaltsverzeichnis aufgenommen wird
\setcounter{tocdepth}{1}

% zweispaltiges Layout möglich machen
\usepackage{multicol}

% Descriptions ohne Einzug
\renewenvironment{description}[1][0pt]
{\list{}{
    \labelwidth=0pt \leftmargin=#1
     \let\makelabel\descriptionlabel
  }
}
{\endlist}

\raggedbottom
