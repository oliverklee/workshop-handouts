\chapter{Korrigieren von Übungszetteln}

\section{Allgemeines Tipps}
\begin{itemize}
  \item zuerst Überblick über alle Lösungen holen
  \item erst eine Aufgabe komplett, dann die nächste
  \item nach einer Aufgabe: Zettel durchmischen!
  \item nicht auf die Namen gucken
  \item leserlich schreiben
  \item nicht mit Rot
  \item bei Genervtsein: Pause!
  \item Zettel schnell zurückgeben
  \item sagen, wie die Aufgaben ausgefallen sind
\end{itemize}

\section{Bepunktung}
\begin{itemize}
  \item Bewertungskriterien ankündigen
  \item erst korrigieren, dann punkten
  \item vorher Punktekriterien überlegen
  \item Punkte für Teillösungen geben
  \item nur Pluspunkte geben
  \item Tutor-übergreifen einheitlich korrigieren
\end{itemize}

\section{Wertschätzender Umgang}
\begin{itemize}
  \item Respekt!
  \item Nicht: "`Unsinn"', "`Nein!"', "`Lächerlich!"'
  \item loben
  \item Feedback zur mathematischen Ausdrucksweise
  \item Feedback zur Beweisstruktur
\end{itemize}
