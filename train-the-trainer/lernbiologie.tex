\section{Biologie des Lernens}
\label{lernbiologie}

\subsection{Die verschiedenen Gedächtnisstufen}
\subsubsection{Filtern der von den Sinnesorganen aufgenommenen Informationen}
\begin{itemize}
  \item das limbische System bewertet/filtert, was ins Ultrakurzzeitgedächtnis gelangt
  \item dadurch beeinflussen Gefühle wesentlich die Filterung
  \item Ärger, Freude und andere Gefühle verändern so die Wahrnehmung
  \item Freude, Erfolgserlebnisse, erotische Anregung, Neugier, Spaß und Spiel sorgen für Hormonausschüttungen, die den Lerneffekt deutlich verbessern
\end{itemize}

\subsubsection{Sensorisches Gedächtnis/Register}
Früher wurde dieses Gedächtnis auch als \emph{Ultrakurzzeitgedächtnis} oder \emph{Immediatgedächtnis} bezeichnet.

\begin{itemize}
  \item speichert für bis zu 20~Sekunden
  \item speichert elektrisch
\end{itemize}

\subsubsection{Arbeitsgedächtnis}
Dieser Teil des Gedächntnisses ist ein System, das unter anderem das \emph{Kurzzeitgedächtnis} enthält.
\begin{itemize}
  \item speichert für ca.~20 Minuten
  \item speichert das, was aus dem Ultrakurzzeitgedächtnis eine Assoziation gefunden hat (über je mehr Kanäle, desto besser)
  \item kann $7\pm 2$ Informationseinheiten (\emph{Chunks}) speichern
  \item neue Informationen überschreiben alte (Interferenz)~-- \fett{deswegen sind kurze Pausen beim Erklären auch so wichtig!}
  \item speichert in Proteinen
  \item Proteinproduktion lässt im Alter nach
\end{itemize}

\subsubsection{Langzeitgedächtnis}
\begin{itemize}
  \item unbegrenzte Kapazität
  \item speichert die (Primär-)Informationen immer zusammen mit den Begleitumständen (Sekundärinformationen) der Speicherung
  \item Informationen lassen sich meist leichter abrufen, wenn die abgespeicherten Begleitumstände angenehm sind
  \item Wiederholung macht die Informationen leichter abrufbar und beugt dem Vergessen vor
  \item speichert in Synapsen
\end{itemize}

\subsection{Stress}
Stresshormone (Adrenalin, Noradrenalin)
\begin{itemize}
  \item blockieren chemisch die normale Funktion der für Assoziationen zuständigen Synapsen
  \item sorgen dafür dass der Mensch kämpfen oder flüchten kann, ohne Zeit mit Nachdenken zu verschwenden
\end{itemize}

\subsubsection{Stressfaktoren beim Lernen}
\begin{itemize}
  \item Angst
  \item fremde Inhalte, Aufmachung der Inhalte, Umstände, Umgebung
  \item abstrakte Darstellung des Wissens
\end{itemize}

Hintergrund: Das Gehirn ist darauf ausgerichtet, eine feindliche Umgebung möglichst zu meiden, anstatt sie zu erlernen.

\subsection{11 Gebote für gehirngerechtes Lehren und Lernen}
\begin{enumerate}
  \item Überblick vor Einzelinformationen!
  \item Transparenz der Lehr- und Lernziele!
  \item Interesse wecken!
  \item Wiederholen!
  \item Mehrere Sinne ansprechen!
  \item Auf die Gefühle achten!
  \item Rückmelden!
  \item Pausen einlegen!
  \item In der richtigen Reihenfolge lehren und lernen!
  \item Vernetzen!
  \item Beachten der individuellen Begabungen!
\end{enumerate}

Diese Informationen stammen aus dem Buch \cite{kopf}.
