\documentclass[a4paper,openany,twoside,titlepage,10pt,headsepline]{scrbook}

%----------------------------------------------------------------------------------
% Typografie und Fonts
%----------------------------------------------------------------------------------

% Vektorfonts statt Pixelfonts
\usepackage[T1]{fontenc}
\usepackage{lmodern}

% Input-Encoding für UTF-8
\usepackage[utf8]{inputenc}

% Anderer Sans-serif-Font
% https://www.tug.org/FontCatalogue/allfonts.html
% https://www.draketo.de/anderes/latex-fonts.html
\usepackage{librefranklin}
\renewcommand{\familydefault}{\sfdefault}

\newcommand{\fett}[1]{\textsf{\textbf{#1}}}


%----------------------------------------------------------------------------------
% Mehrspachigkeit
%----------------------------------------------------------------------------------

% Mehrsprachigkeit für Deutsch und Englisch erlauben
\usepackage[ngerman,english]{babel}

% Anführungszeichen sprachabhängig machen
\usepackage[babel]{csquotes}


%----------------------------------------------------------------------------------
% Größen, Abstände und Einzüge
%----------------------------------------------------------------------------------

% Mehr von der Seite nutzen
\usepackage[top=2cm, bottom=3cm, right=3cm, left=2cm]{geometry}

% Absätze werden nicht eingezogen, sondern vertikal abgesetzt
\setlength{\parindent}{0mm}
\addtolength{\parskip}{0.5em}

% Descriptions ohne Einzug
\renewenvironment{description}[1][0pt]
{\list{}{
    \labelwidth=0pt \leftmargin=#1
    \let\makelabel\descriptionlabel
}
}
{\endlist}

% Im Zweifel die Seite nicht komplett füllen, aber keine zusätzlichen vertikalen Abstände hinzufügen
\raggedbottom

% Hurenkinder und Schusterjungen verhindern
\clubpenalty10000
\widowpenalty10000
\displaywidowpenalty=10000

%----------------------------------------------------------------------------------
% Literaturverzeichnis, Links und Querverweise
%----------------------------------------------------------------------------------

% Bibliographieeinstellungen
\bibliographystyle{alphadin}

% klickbare Verweise
\usepackage[pdftex,plainpages=false,pdfpagelabels]{hyperref}

% nette URLs
\usepackage{url}


%----------------------------------------------------------------------------------
% Dokument- und Seitenstruktur
%----------------------------------------------------------------------------------

% zweispaltiges Layout möglich machen
\usepackage{multicol}

% Seiten-Kopfzeilen und -Fußzeilen
\usepackage{scrlayer-scrpage}

% Maximal drei Ebenen nummerieren
\setcounter{secnumdepth}{2}

% Maximale 2 Ebenen im Inhaltsverzeichnis
\setcounter{tocdepth}{1}

% Mit jeder Section eine neue Seite anfangen
% https://tex.stackexchange.com/questions/9497/start-new-page-with-each-section
\usepackage{etoolbox}
\preto{\section}{%
  \ifnum\value{section}=0 \else\clearpage\fi
}


%----------------------------------------------------------------------------------
% Grafiken, Farben und Boxen
%----------------------------------------------------------------------------------

% Farben
\usepackage{xcolor}

% Grafiken
\usepackage[pdftex]{graphicx}

% Boxen inklusive Schattierung
\usepackage{framed}
\definecolor{shadecolor}{rgb}{0.8,0.8,0.8}


\AtBeginDocument{\selectlanguage{ngerman}}

\title{Train the Tutor}
\author{Oliver Klee\\\texttt{www.oliverklee.de}\\\texttt{seminare@oliverklee.de}}
\date{Version vom \today}

\begin{document}

\frontmatter

\maketitle

\tableofcontents

\mainmatter

\chapter{Seminar-Handwerkszeug}
\section{Feedback: Tipps und Tricks}
\label{feedback-regeln}
\index{Feedback}
\index{Feedbackregeln}

\subsection{Was ist Feedback?}
Feedback ist für euch eine Gelegenheit, in kurzer Zeit viel über euch selbst zu lernen. Feedback ist ein Anstoß, damit ihr danach an euch arbeiten könnt (wenn ihr wollt).

Feedback heißt, dass euch jemandem einen persönlichen, subjektiven Eindruck in Bezug auf konkrete Punkte mitteilt. Da es sich um einen persönlichen Eindruck im Kopf eines einzelnen Menschen handelt, sagt Feedback nichts darüber aus, wie ihr tatsächlich wart. Es bleibt allein euch selbst überlassen, das Feedback, das ihr bekommt, für euch selbst zu einem großen Gesamtbild zusammenzusetzen.

Es kann übrigens durchaus vorkommen, dass ihr zur selben Sache von verschiedenen Personen völlig unterschiedliches (oder gar gegensätzliches) Feedback bekommt.

Es geht beim Feedback \emph{nicht} darum, euch mitzuteilen, ob ihr ein guter oder schlechter Mensch, ein guter Redner, eine schlechte Rhetorikerin oder so seid. Solche Aussagen haben für euch keinen Lerneffekt. Stattdessen schrecken sie euch ab, Neues auszuprobieren und dabei auch einmal so genannte Fehler zu machen.

Insbesondere ist Feedback keine Grundsatzdiskussion, ob das eine oder andere Verhalten generell gut oder schlecht ist. Solche Diskussionen führt ihr besser am Abend bei einem Bierchen.

\subsection{Feedback geben}
\begin{itemize}
  \item  "`ich"' statt "`man"' oder "`wir"'
  \item die \emph{eigene} Meinung sagen
  \item die andere Person direkt ansprechen: "`du/Sie"' statt "`er/sie"'
  \item eine konkrete, spezifische Beobachtung schildern
  \item nicht verallgemeinern
  \item nicht analysieren oder psychologisieren (nicht: "`du machst das nur, weil \ldots"')
  \item Feedback möglichst unmittelbar danach geben
  \item konstruktiv: nur Dinge ansprechen, die die andere Person auch ändern kann
\end{itemize}

\subsection{Feedback entgegennehmen}
\begin{itemize}
  \item vorher den Rahmen für das Feedback abstecken: Inhalt, Vortragstechnik, Schriftbild \ldots
  \item gut zuhören und ausreden lassen
  \item sich nicht rechtfertigen, verteidigen oder entschuldigen
  \item Missverständnisse klären, Hintergründe erläutern
  \item Feedback als Chance zur Weiterentwicklung sehen
\end{itemize}

\section{Das Blitzlicht}

\begin{itemize}
  \item wird nicht visualisiert
  \item jede Person spricht nur für sich selbst
  \item keine Diskussion (Ausnahme: wichtige Verständnisfragen)
  \item nicht unterbrechen
  \item wer anfängt, fängt an
  \item kurz~-- ein Blitzlicht ist kein Flutlicht
\end{itemize}

\section{Visualisierung: Plakate \& Co}

\subsection{Warum Visualisierung?}
\begin{itemize}
  \item Der Mensch ist ein Augentier!
  \item besserer Überblick über Erarbeitetes
  \item "`externer Speicher"' für Arbeitsgedächtnis/Kurzzeitgedächtnis
\end{itemize}

\subsection{Grundregeln}

\subsubsection{Erst planen, dann malen!}
Sonst kann es nämlich passieren, dass ihr irgendwann feststellt, dass der Platz nicht reicht, und eure ganze bisherige Arbeit für die Katz ist.

\subsubsection{Aufbau}
\begin{itemize}
  \item jedem Plakat eine Überschrift geben
  \item Überschrift in Container setzen
  \item Rahmen um das komplette Plakat benutzen
  \item von links nach rechts und von oben nach unten aufbauen (ansonsten klare Fokuspunkte setzen)
  \item viele Bilder und Symbole benutzen (siehe dazu \cite{bikablo}), wenig Text
  \item von vorne nach hinten aufbauen (also erst den Vordergrund malen, dann den Hintergrund)
  \item Schatten sind rechts unten, da das Licht von links oben kommt
\end{itemize}

\subsubsection{Sprache}
\begin{itemize}
  \item Halbsätze verwenden
  \item keine Abk.~verw.
\end{itemize}

\subsubsection{Schrift}
\begin{itemize}
  \item Druckschrift und normale Groß- und Kleinschreibung benutzen
  \item VERSALIEN vermeiden
  \item nicht visuell {\tiny nuscheln} oder {\LARGE brüllen}!
  \item kurze Ober-/Unterlängen benutzen
\end{itemize}

\subsubsection{Farben}
\begin{itemize}
  \item schwarz für Schrift (gilt nur für Plakate)
  \item zusätzlich noch 1 oder 2 Farben pro Plakat/Folie, in Ausnahmen auch mehr
\end{itemize}

\subsection{Literatur}
Zur Visualisierung auf Plakaten findet ihr in \cite{visualisieren-praesentieren-moderieren} sehr viel.

Speziell mit Folien (im Sinne von Keynote, OpenOffice.org Impress, PowerPoint) beschäftigt sich \emph{Presentation Zen} \cite{presentation-zen}.

Zur Visualisierung von Inhalten und Zusammenhängen ist \emph{The Back of the Napkin} \cite{napkin} sehr gut.

Für schnell zu zeichnende visuelle Vokabeln empfehle ich die drei BiKaBlo-Bücher. \cite{bikablo, bikablo-2, bikablo-emotions}

\section{Design-Grunds"atze: La Crap}
\label{lacrap}

\index{LA CRAP}
\index{Design-Grundsätze}
\index{Grafikdesign|\see{Design-Grundsätze}}
\index{Lesbarkeit}
\index{Augenbewegungen}
\index{Contrast}
\index{Kontrast|\see{Contract}}
\index{Repetition}
\index{Konsistenz|\see{Repetition}}
\index{Aligmment}
\index{Ausrichtung|\see{Aligmment}}
\index{Proximity}
\index{Nähe|\see{Proximity}}

\begin{itemize}
\item \fett{L}esbarkeit sichern
\item \fett{A}ugenbewegung berücksichtigen

\smallskip
\item \fett{C}ontrast zwischen Designelementen (Weichei-Regel beachten)
\item \fett{R}epetition (Wiederholung von Designelementen, Konsistenz)
\item \fett{A}lignment (Ausrichtung an unsichtbaren Linien)
\item \fett{P}roximity (räumliche Nähe von inhaltlich Zusammengehörendem)
\end{itemize}

\subsection{Lesbarkeit}
Euer Text sollte vor allem lesbar sein~-- dass den Text jemand liest, ist in den meisten Fällen der Grund, ihn überhaupt zu schreiben. Wenn ihr den Text trotzdem so gestalten wollt, dass es schwer lesbar ist, braucht ihr dafür eine sehr gute Begründung.

\subsection{Augenbewegungen berücksichtigen}
Dies ist dann relevant, wenn ihr mit mehrspaltigem Satz oder mit Folien (oder Plakaten) arbeitet. Grundregel
\begin{itemize}
  \item von oben nach unten
  \item von links nach rechts
  \item das Auge folge einem \emph{Z} (oder einem umgekehrten \emph{S})
  \item Fokuspunkte: macht durch auffällige Objekte klar, wo das Auge zuerst hinwandern soll
\end{itemize}

\subsection{Contrast}
Kontrast kann bestehen in:
\begin{itemize}
  \item Helligkeit
  \item Farbe
  \item Schriftart
  \item Duktus
  \item Ausrichtung
  \item \ldots
\end{itemize}

\subsection{Repetition}
Inhaltlich ähnliche Objekte sollten auch optisch ähnlich sein. So sollten etwa die Aufzählungspunkte einer Ebene alle gleich aussehen in Form, Farbe, Größe, Ausrichtung.

\subsection{Alignment}
Richtet eure Objekte an unsichtbaren Kanten aus. Wenn ihr dies bewusst nicht tun wollt, dann macht es wirklich \emph{deutlich} (Weichei-Regel) und habt eine Begründung.

\subsection{Proximity}
Objekte, die inhaltlich zusammen gehören, sollen auch optisch nahe zusammen sein. Der Umkehrschluss gilt entsprechend.

\subsection{Quelle}
Das CRAP (ohne LA) kommt aus \cite{non-designer}.

\section{Welches Medium wofür benutzen?}

\subsection{Längerer Vortrag}
\begin{itemize}
  \item Beamer
  \item Tafel
\end{itemize}

\subsection{Kurzinput, Fahrplan}
\begin{itemize}
  \item Plakate
\end{itemize}

\subsection{"`Schmierpapier"'}
\begin{itemize}
  \item Tafel
  \item Plakate
\end{itemize}


\chapter{Lehrmethoden}
\label{lehrmethoden}

\section{Frontalunterricht, Vortrag}
\begin{itemize}
 \item höchstens 20~Minuten am Stück
 \item vorher klarmachen, ob Mitschrieben gewünscht bzw.~gefordert ist
\end{itemize}

\section{Lehrgespräch}
\begin{itemize}
 \item Gruppe durch geschicktes Fragen anregen, sich mit einem Thema auseinanderzusetzen
\end{itemize}

\section{Einzelarbeit}
\begin{itemize}
 \item Voraussetzung: Teilis haben schon genug Vorkenntnisse für die Aufgabe
 \item Aufgabenstellung muss schriftlich und eindeutig sein
 \item ReferentIn geht durch die Reihen (für Fragen und um Anmerkungen zu Arbeitstechniken zu geben)
 \item Einzelarbeit kurz unterbrechen und erklären, wenn alle Probleme an der gleichen Stelle haben
\end{itemize}

\section{Partnerarbeit}
\begin{itemize}
 \item bietet sich an, Vorwissen aufzubauen, wenn Einzelarbeit für die Aufgabe zu schwierig ist
\end{itemize}

\section{Gruppenarbeit}
\begin{itemize}
 \item für Aufgaben, bei denen tatsächlich Diskussionsbedarf besteht und die von ``verschiedenen Köpfen'' profitieren
 \item Zeit für die Vorstellung der Ergebnisse einplanen
 \item prima Gelegenheit für die Teilis, Präsentation zu üben
\end{itemize}

\section{Planspiel, Simulation}
\begin{itemize}
 \item Beispiel: Routing in Netzwerken, bei dem die Leute die Datenpakete durch den Raum tragen
 \item macht den Stoff noch anschaulicher
\end{itemize}

\section{Weitere Methoden}

\twocolumn
\begin{itemize}
  \item auf Ecken verteilen
  \item Aufstellung
  \item Blitzlicht
  \item Brainstorming
  \item Christkindl-Effekt
  \item Clustern
  \item Diskussion (moderiert)
  \item Fahrplan visualisieren
  \item Fallbeispiel vorstellen
  \item Feedbackrunde
  \item Fishbowl
  \item Haribo-Analyse
  \item Hausaufgaben
  \item der heiße Stuhl
  \item Kartenabfrage
  \item Kennenlernspiele
  \item Kopfstand
  \item Kurzinput mit Plakat
  \item Loreley-Phase
  \item Mindmapping
  \item Open Space
  \item Partnerarbeit
  \item Pause
  \item Poster-Walk
  \item Pro-/Kontra-Diskussion
  \item Punkten
  \item Quiz
  \item Rollenspiel
  \item Schnitzeljagd, Ralley
  \item Spiele
  \item stille Diskussion
  \item Stimmungsbarometer
  \item Test schreiben
  \item Theater der Unterdrückten
  \item Übungen am PC
  \item Übungsaufgaben
  \item Umfrage
  \item Vernissage
  \item wilde Fragestunde
  \item World-Café
  \item Zukunftswerkstatt
\end{itemize}

\onecolumn


\chapter{Grafische Darstellungsformen}

Mit diesen Darstellungsformen könnt ihr Stoff grafisch organisieren und darstellen. Demonstriert sie, benutzt sie, bringt sie den Studis bei.

\begin{itemize}
 \item Mind-Map
 \item Flowchart
 \item Karteikarten/Kartenabfrage
\end{itemize}


\chapter{Rhetorik-Handwerkszeug}
\section{Was ist Rhetorik?}
\label{was-ist-rhetorik}

\paragraph*{Redeschulung ist Denkschulung:} Gute Reden verlangen strukturierte Gedanken und setzen die Fähigkeit zum sauberen Argumentieren voraus.
\paragraph*{Überzeugendes Reden setzt Überzeugung voraus:} Glaubwürdigkeit ist der entscheidende Maßstab für jegliche Rhetorik. Sie bedeutet, dass Form und Inhalt nicht zu trennen sind.
\paragraph*{Reden und Redlichkeit gehören zusammen:} Wer redet, übernimmt Verantwortung für die Wirkung seiner Rhetorik~-- und hat deswegen mit den rhetorischen Mitteln verantwortlich umzugehen!

\section{Aufgaben der RednerIn}
\subsubsection{Informieren}
Sachverhalte, Probleme, Situationen, Vorgänge, Abläufe, \ldots

\subsubsection{Überzeugen}
Vorgehensweisen, Entscheidungen, Lösungen, \ldots

\subsubsection{Unterhalten}
Lachen, Schmunzeln, Nachdenken, Erinnern, \ldots

\section{Rhetorisches Grundinstrumentarium}
\subsection{Körperhaltung}
\begin{itemize}
\item stabiler Stand: Beine schulterbreit geöffnet, Gewicht gleichmäßig verteit (gibt Sicherheit)
\item aufrechte Haltung, Kopf gerade
\item dem Publikum zugewandt
\end{itemize}

\subsection{Blickkontakt}
\begin{itemize}
\item "`dynamisches Kreisen"'
\item freundliche Zielgesichter suchen
\item möglichst nie Boden oder Decke
\item kleine Gruppe: alle anschauen
\item große Gruppe: Leute im Raum verteilt anschauen
\end{itemize}

\subsection{Gestik}
\begin{itemize}
\item dient der Verstärkung des Vortrags
\item Hände möglichst frei im \emph{neutralen Bereich}
\item 3 Bereiche:
  \begin{description}
    \item [positiv:] oberhalb der Gürtellinie
    \item [neutral:] etwa auf Gürtellinie
    \item [negativ:] unterhalb der Gürtellinie
  \end{description}
\item kommt gleichzeitig oder kurz vor dem Wort, nie danach
\item soll vor allem \emph{angemessen} sein, kann sonst ablenken
\end{itemize}

\subsection{Mimik}
\begin{itemize}
\item gilt als "`Spiegel der Seele"'
\item kommt gleichzeitig oder kurz vor dem Wort, nie danach
\end{itemize}

\subsection{Stimme, Artikulation}
\begin{itemize}
\item Atmung: Bauchatmung
\item Stimmhöhe: im (angenehmen) Hauptbereich anfangen
\item Sprechtempo: eher langsamer als sonst, bewusste Pausen (Hektik und Ruhe übertragen sich auf das Publikum)
\item Lautstärke: eher lauter als sonst
\item sehr deutlich artikulieren, Endungen nicht verschlucken
\item keine Störlaute, möglichst wenig Füllwörter
\end{itemize}

\subsubsection{Literatur zum Grundinstrumentarium}
\cite{science-of-breath}

\subsubsection{Bücher zum Vorlesen-Üben}
\cite{anhalter, der-liebe-ziel, brenneisen, max-und-moritz, reimtopf, galgenlieder}


\subsection{Je nachdem und überhaupt: Vier Aspekte der Angemessenheit}

\unitlength 1mm
\begin{center}
\begin{picture}(120,120)
\thicklines

% Raute
%\put(60, 10) {\line( 2, 3){33}}
%\put(60, 10) {\line(-2, 3){33}}
%\put(60,109) {\line( 2,-3){33}}
%\put(60,109) {\line(-2,-3){33}}

% Pfeile
%\put(53,59.5) {\vector(-1, 0){23}}
%\put(67,59.5) {\vector( 1, 0){23}}
%\put(60,65  ) {\vector( 0, 1){40}}
%\put(60,54  ) {\vector( 0,-1){40}}

% Texte
\put(56,58.5){\fett{Rede}}
\put(54,112){\fett{Thema}}
\put(53,4){\fett{Situation}}
\put(8,58.5){\fett{RednerIn}}
\put(96,58.5){\fett{ZuhörerInnen}}

\thinlines
\end{picture}
\end{center}

Angemessenheit an Thema, eigenen Typ, ZuhörerInnen und Situation nicht vernachlässigen! Je nach persönlicher Zielsetzung kann man sich aber auch entscheiden, mal ein Element bewusst zu missachten (etwa völlig unangemessene Kleidung zu tragen).

\section{Zungenbrecher}
\label{zungenbrecher}
\selectlanguage{english}

Elizabeth George

I can bet: A cat ban can't ban a bed pan.

Which witch's wit-wish switches which witches' widths with his witch's width?

Three sweet switched Swiss witches watch three washed Swiss witch Swatch watch switches.

Which sweet switched Swiss witch watches which washed Swiss witch Swatch watch switch?

\selectlanguage{ngerman}

Wenn dank bankrotter Konkurskotten Cloppenburger Rokokokötter Bangkoker Blanko-Bankkonten blank kloppen, kloppen dank bankrotter Konkurskotten Cloppenburger Rokokokötter Bangkoker Blanko-Bankkonten blank.

Brunos Brat-Brot-Breit-Brett-Brad-Britt-Brunch-Brauch


\chapter{Kommunikation}
\section{Das Vier-Seiten-Modell}

Dieses Modell ist auch bekannt als \fett{Vier-Ohren-Modell} oder \fett{Kommunikationsquadrat}. Es ist bekannst aus dem Buch \emph{Miteinander reden} \cite{miteinander-reden-1}.

\begin{description}
  \item[Sachseite:] Daten, Fakten, Sachverhalte
  \item[Selbstkundgabe (Selbstoffenbarung):] Was sagt er/sie über sich selbst? Das eigene Selbstverständnis, seine/ihre Motive, Werte, Emotionen \ldots
  \item[Beziehungsseite:] Was hält er/sie von mir? Wie stehen wir zueinander?
  \item[Appell:] Wozu will er/sie mich veranlassen?
\end{description}

\section{Aktives Zuhören}
\label{aktives-zuhoeren}

\begin{itemize}
  \item Aufmerksamkeit signalisieren
  \item Blickkontakt halten
  \item ganz zuwenden
  \item nicken
  \item \emph{genau} zuhören
  \item wiederholen, umschreiben
  \item nachfragen
  \item \glqq Zwischentöne\grqq{} zur Sprache bringen
\end{itemize}

\section{Der Kontrollierte Dialog}
\label{kontrollierter-dialog}
\index{kontrollierter Dialog}

\begin{itemize}
  \item A und B diskutieren.
  \item C beobachtet, achtet auf die Zeit und gibt hinterher Feedback.
\end{itemize}

\vspace*{1em}

\begin{tabular}{rcl}
 \fett{A} & & \fett{B} \\
\hline\\
 bringt ein Argument           & & \\
                               & & wiederholt mit eigenen Worten \\
 bestätigt/korrigiert          & & \\
                               & & bringt ein Argument \\
 wiederholt mit eigenen Worten & & \\
                               & & bestätigt/korrigiert \\
 bringt ein Argument           & & \\
                               & & \ldots \\
\end{tabular}

\section{Fragetypen}
\label{fragetypen}

\begin{description}
\item[Geschlossene Frage:] Ja-/Nein-Frage. Oder eine Entweder-oder-Frage.
\item[Offene Frage:] W-Frage, die zum Erzählen anregt: Warum \ldots? Wie \ldots?
\end{description}

Wenn man eine Frage an eine Gruppe richtet, sind Oder-Fragen ungünstig, weil dann alle Leute etwas sagen müssten:

\begin{quotation}
Habt ihr das alles verstanden oder soll ich es noch weiter erklären?
\end{quotation}

Günstiger sind dann einzelne Ja-Nein-Fragen, wo die Leute nicken oder den Kopf schütteln können:

\begin{quotation}
Ist das soweit klar?
\end{quotation}

\section{Fragen beantworten}
\label{fragen-beantworten}

\subsection{Fragemodus klären: Kösers goldene Regel}
\begin{itemize}
  \item Verständnisfragen bitte sofort.
  \item Weiterführende Fragen in der Diskussion am Ende.
  \item Eitle Selbstdarstellung bitte im nächsten Slot.
\end{itemize}

Vielen Dank an Sören Köser für diese Regel.


\subsection{Herausfinden, ob die Antwort geholfen hat}
\begin{itemize}
  \item War die Antwort für dich verständlich?
  \item Ist damit deine Frage beantwortet?
\end{itemize}

\subsection{Wenn dem Teili die Antwort nicht geholfen hat}
\begin{itemize}
  \item Habe ich dir unbekannte Begriffe benutzt?
  \item Was genau ist das Problem?
  \item weiteres Beispiel bringen
  \item Antwort anders formulieren
\end{itemize}

\subsection{Wenn ihr die Antwort selbst nicht wisst}
\begin{itemize}
  \item Frage an die Gruppe geben
  \item ehrlich zugeben, dass ihr es nicht wisst
  \item anbieten, die Sache nachzuschlagen
\end{itemize}

\subsection{Im Rahmen bleiben}
\begin{itemize}
  \item Abschweifungen deutlich machen
  \item bei Bedarf Einzelnen Sachen alleine erklären
  \item Literatur oder Websites empfehlen
\end{itemize}


\chapter{Lernen}
\section{Effektivität von Lernkanälen}
\subsection{Wahrnehmen}
Von wenig zu viel:
\begin{enumerate}
  \item riechen, schmecken oder fühlen
  \item hören
  \item sehen
\end{enumerate}

\subsection{Behalten}
Von wenig zu viel:
\begin{enumerate}
  \item lesen
  \item hören
  \item sehen
  \item hören und sehen
  \item sagen
  \item selber machen
  \item es anderen erklären, lehren
\end{enumerate}


\section{Lerneffekt bei den Inhalten steigern}

\subsubsection{Viel}
\begin{itemize}
\item mit Plakat/Folie
\item unkonventionelle Methoden (f"ur manches)
\item vorher "Uberblick geben
\item Spaßiges, Anspielungen
\item viele Bilder
\item klare visuelle Strukturen, Farben
\end{itemize}

\subsubsection{Wenig}
\begin{itemize}
\item reiner m"undlicher Vortrag
\item Sketch ohne Struktur
\item reine "`Zuschauerrolle"'
\end{itemize}

\subsubsection{Anmerkungen}
\begin{itemize}
\item Methode soll angemessen zu Thema sein
\item Visualisiertes sollte stimmen ("`Scheinsicherheit"')
\item auf einzelne Leute eingehen (bei Mathe)
\end{itemize}

\section{Biologie des Lernens}

\subsection{Ausbilden des persönlichen Lerntyps}
Drei Monate nach der Geburt haben sich die meisten Zellen im Gehirn gebildet und verknüpft. Danach verändert sich nur noch vergleichsweise sehr wenig. Somit prägt sich der Lerntyp eines jeden Menschen sehr früh in der Kindheit und verändert sich danach nicht mehr.

\subsection{Die verschiedenen Gedächtnisstufen}
\subsubsection{Filtern der von den Sinnesorganen aufgenommenen Informationen}
\begin{itemize}
  \item das limbische System bewertet/filtert, was ins Ultrakurzzeitgedächtnis gelangt
  \item dadurch beeinflussen Gefühle wesentlich die Filterung
  \item Ärger, Freude und andere Gefühle verändern so die Wahrnehmung
  \item Freude, Erfolgserlebnisse, erotische Anregung, Neugier, Spaß und Spiel sorgen für Hormonausschüttungen, die den Lerneffekt deutlich verbessern
\end{itemize}

\subsubsection{Sensorisches Gedächtnis/Register}
Früher wurde dieses Gedächtnis auch als \emph{Ultrakurzzeitgedächtnis} oder \emph{Immediatgedächtnis} bezeichnet.

\begin{itemize}
  \item speichert für bis zu 20~Sekunden
  \item speichert elektrisch
\end{itemize}

\subsubsection{Arbeitsgedächtnis}
Dieser Teil des Gedächntnisses ist ein System, das unter anderem das \emph{Kurzzeitgedächtnis} enthält.
\begin{itemize}
  \item speichert für ca.~20 Minuten
  \item speichert das, was aus dem Ultrakurzzeitgedächtnis eine Assoziation gefunden hat (über je mehr Kanäle, desto besser)
  \item kann $7\pm 2$ Informationseinheiten (\emph{Chunks}) speichern
  \item neue Informationen überschreiben alte (Interferenz)~-- \fett{deswegen sind kurze Pausen beim Erklären auch so wichtig!}
  \item speichert in Proteinen
  \item Proteinproduktion lässt im Alter nach
\end{itemize}

\subsubsection{Langzeitgedächtnis}
\begin{itemize}
  \item unbegrenzte Kapazität
  \item speichert die (Primär-)Informationen immer zusammen mit den Begleitumständen (Sekundärinformationen) der Speicherung
  \item Informationen lassen sich meist leichter abrufen, wenn die abgespeicherten Begleitumstände angenehm sind
  \item Wiederholung macht die Informationen leichter abrufbar und beugt dem Vergessen vor
  \item speichert in Synapsen
\end{itemize}

\subsection{Stress}
Stresshormone (Adrenalin, Noradrenalin)
\begin{itemize}
  \item blockieren chemisch die normale Funktion der für Assoziationen zuständigen Synapsen
  \item sorgen dafür dass der Mensch kämpfen oder flüchten kann, ohne Zeit mit Nachdenken zu verschwenden
\end{itemize}

\subsubsection{Stressfaktoren beim Lernen}
\begin{itemize}
  \item Angst
  \item fremde Inhalte, Aufmachung der Inhalte, Umstände, Umgebung
  \item abstrakte Darstellung des Wissens
\end{itemize}

Hintergrund: Das Gehirn ist darauf ausgerichtet, eine feindliche Umgebung möglichst zu meiden, anstatt sie zu erlernen.

\subsection{11 Gebote für gehirngerechtes Lehren und Lernen}
\begin{enumerate}
  \item Überblick vor Einzelinformationen!
  \item Transparenz der Lehr- und Lernziele!
  \item Interesse wecken!
  \item Wiederholen!
  \item Mehrere Sinne ansprechen!
  \item Auf die Gefühle achten!
  \item Rückmelden!
  \item Pausen einlegen!
  \item In der richtigen Reihenfolge lehren und lernen!
  \item Vernetzen!
  \item Beachten der individuellen Begabungen!
\end{enumerate}

Diese Informationen stammen aus dem Buch \cite{kopf}.
\section{11 Gebote für erfolgreiches Lehren und Lernen}
\label{lerngebote}

Diese Gebote stammen aus \cite{kopf}.

\begin{enumerate}
  \item Überblick vor Einzelinformationen!
  \item Transparenz der Lehr- und Lernziele!
  \item Interesse wecken!
  \item Wiederholen!
  \item Mehrere Sinne ansprechen!
  \item Auf die Gefühle achten!
  \item Rückmelden!
  \item Pausen einlegen!
  \item In der richtigen Reihenfolge lehren und lernen!
  \item Vernetzen!
  \item Individuelle Begabungen beachten!
\end{enumerate}

\section{Erklären leicht gemacht}

\subsection{Top-Down-Methode}
Diese Methode eignet sich sowohl für den gesamten Termin als auch für eine einzelne Einheit.

\begin{quote}
  \emph{Tell 'em what you gonna say,\\
  say it,\\
  tell 'em what you've just said.}
\end{quote} 

Mit anderen Worten:
\begin{enumerate}
  \item Fahrplan vorstellen
  \item machen
  \item zusammenfassen
\end{enumerate}

\subsection{Sandwich-Methode}
Diese Methode eignet sich, um komplizierte Sachverhalte zu erklären.

\begin{enumerate}
  \item einfaches Beispiel, dass den Sachverhalt veranschaulicht (möglichst ohne neue Begriffe)
  \item allgemeine Formulierung, Definition, Fachbegriffe
  \item komplexeres Beispiel mit Fachbegriffen
\end{enumerate}

Wichtig ist dabei, Fachbegriffe \emph{vor} der ersten Benutzung zu definieren.

\section{Fragen für die Übung zum Thema Lernen}
\begin{itemize}
 \item Wie bist du an das Lernen herangegangen?
 \item Wie und wo hast du dir Informationen besorgt?
 \item Wo und welche Art von Hilfe hast du dir geholt? Welche Rolle haben diese anderen Personen gespielt?
 \item Auf welche Probleme bist du gestoßen? Wie hast du dich dabei gefühlt?
 \item Welche Rolle haben Fehler gespielt? Haben sie dich trotzdem vorangebracht?
 \item Wann ist bei dir Frustration aufgekommen? Wie bist du damit umgegangen?
 \item Welche Phasen hat deine Lernstrategie?
\end{itemize}


\documentclass[a4paper,openany,twoside,titlepage,10pt,headsepline]{scrbook}

%----------------------------------------------------------------------------------
% Typografie und Fonts
%----------------------------------------------------------------------------------

% Vektorfonts statt Pixelfonts
\usepackage[T1]{fontenc}
\usepackage{lmodern}

% Input-Encoding für UTF-8
\usepackage[utf8]{inputenc}

% Anderer Sans-serif-Font
% https://www.tug.org/FontCatalogue/allfonts.html
% https://www.draketo.de/anderes/latex-fonts.html
\usepackage{librefranklin}
\renewcommand{\familydefault}{\sfdefault}

\newcommand{\fett}[1]{\textsf{\textbf{#1}}}


%----------------------------------------------------------------------------------
% Mehrspachigkeit
%----------------------------------------------------------------------------------

% Mehrsprachigkeit für Deutsch und Englisch erlauben
\usepackage[ngerman,english]{babel}

% Anführungszeichen sprachabhängig machen
\usepackage[babel]{csquotes}


%----------------------------------------------------------------------------------
% Größen, Abstände und Einzüge
%----------------------------------------------------------------------------------

% Mehr von der Seite nutzen
\usepackage[top=2cm, bottom=3cm, right=3cm, left=2cm]{geometry}

% Absätze werden nicht eingezogen, sondern vertikal abgesetzt
\setlength{\parindent}{0mm}
\addtolength{\parskip}{0.5em}

% Descriptions ohne Einzug
\renewenvironment{description}[1][0pt]
{\list{}{
    \labelwidth=0pt \leftmargin=#1
    \let\makelabel\descriptionlabel
}
}
{\endlist}

% Im Zweifel die Seite nicht komplett füllen, aber keine zusätzlichen vertikalen Abstände hinzufügen
\raggedbottom

% Hurenkinder und Schusterjungen verhindern
\clubpenalty10000
\widowpenalty10000
\displaywidowpenalty=10000

%----------------------------------------------------------------------------------
% Literaturverzeichnis, Links und Querverweise
%----------------------------------------------------------------------------------

% Bibliographieeinstellungen
\bibliographystyle{alphadin}

% klickbare Verweise
\usepackage[pdftex,plainpages=false,pdfpagelabels]{hyperref}

% nette URLs
\usepackage{url}


%----------------------------------------------------------------------------------
% Dokument- und Seitenstruktur
%----------------------------------------------------------------------------------

% zweispaltiges Layout möglich machen
\usepackage{multicol}

% Seiten-Kopfzeilen und -Fußzeilen
\usepackage{scrlayer-scrpage}

% Maximal drei Ebenen nummerieren
\setcounter{secnumdepth}{2}

% Maximale 2 Ebenen im Inhaltsverzeichnis
\setcounter{tocdepth}{1}

% Mit jeder Section eine neue Seite anfangen
% https://tex.stackexchange.com/questions/9497/start-new-page-with-each-section
\usepackage{etoolbox}
\preto{\section}{%
  \ifnum\value{section}=0 \else\clearpage\fi
}


%----------------------------------------------------------------------------------
% Grafiken, Farben und Boxen
%----------------------------------------------------------------------------------

% Farben
\usepackage{xcolor}

% Grafiken
\usepackage[pdftex]{graphicx}

% Boxen inklusive Schattierung
\usepackage{framed}
\definecolor{shadecolor}{rgb}{0.8,0.8,0.8}


\AtBeginDocument{\selectlanguage{ngerman}}

\title{Motivation in Teams}
\author{Oliver Klee\\\texttt{www.oliverklee.de}\\\texttt{seminare@oliverklee.de}}
\date{Version vom \today}

\begin{document}

\frontmatter

\maketitle

\tableofcontents


\mainmatter

\chapter{Seminar-Handwerkszeug}
\section{Regeln für den Workshop}
\label{gfk-workshopregeln}
\index{Workshopregeln}

\paragraph{Vegas-Regel:} Was wir hier persönlichen Dingen teilen, bleibt im Workshop. Wir erzählen Dinge nur anonymisiert nach außen.

\paragraph{Keine dummen Fragen:} Es gibt keine dummen Fragen. Für Fragen, die nicht gut in den Rahmen des aktuellen Themas passen, haben wir einen Themenkühlschrank.

\paragraph{Joker-Regel:} Wir alle versuchen, uns auf dem Workshop gut um uns selbst zu kümmern. Wenn wir etwas brauchen, sprechen wir es an oder sorgen selbst dafür.

\paragraph{Aufrichtigkeit:} Wir tun unser Bestes, uns ehrlich und aufrichtig miteinander umzugehen.

\paragraph{Konstruktiv sein:} Wir tun unser Bestes, konstruktiv miteinander umzugehen und uns gut zu behandeln.

\section{Paarinterview zum Kennenlernen}
\label{paarinterview}

\begin{itemize}
 \item Wo und wie wohne ich?
 \item Was mache ich in Beruf und Ehrenamt so? Und was habe ich bisher so gemacht?
 \item Was sind ein paar Dinge, die mir im Leben zurzeit Freude bereiten?
 \item Was brauche ich (von anderen Personen oder der Umgebung), damit die Zusammenarbeit mit mir gut funktioniert?
 \item Was sollten andere Menschen über mich wissen, wenn sie mit mir zusammenarbeiten?
 \item Was mache ich, um trotz der aktuellen Krisen psychisch halbwegs gesund zu bleiben?
 \item Was ist ein \emph{Guilty Pleasure}, dem ich ab und an fröne?
\end{itemize}

\section{Feedback: Tipps und Tricks}
\label{feedback-regeln}
\index{Feedback}
\index{Feedbackregeln}

\subsection{Was ist Feedback?}
Feedback ist für euch eine Gelegenheit, in kurzer Zeit viel über euch selbst zu lernen. Feedback ist ein Anstoß, damit ihr danach an euch arbeiten könnt (wenn ihr wollt).

Feedback heißt, dass euch jemandem einen persönlichen, subjektiven Eindruck in Bezug auf konkrete Punkte mitteilt. Da es sich um einen persönlichen Eindruck im Kopf eines einzelnen Menschen handelt, sagt Feedback nichts darüber aus, wie ihr tatsächlich wart. Es bleibt allein euch selbst überlassen, das Feedback, das ihr bekommt, für euch selbst zu einem großen Gesamtbild zusammenzusetzen.

Es kann übrigens durchaus vorkommen, dass ihr zur selben Sache von verschiedenen Personen völlig unterschiedliches (oder gar gegensätzliches) Feedback bekommt.

Es geht beim Feedback \emph{nicht} darum, euch mitzuteilen, ob ihr ein guter oder schlechter Mensch, ein guter Redner, eine schlechte Rhetorikerin oder so seid. Solche Aussagen haben für euch keinen Lerneffekt. Stattdessen schrecken sie euch ab, Neues auszuprobieren und dabei auch einmal so genannte Fehler zu machen.

Insbesondere ist Feedback keine Grundsatzdiskussion, ob das eine oder andere Verhalten generell gut oder schlecht ist. Solche Diskussionen führt ihr besser am Abend bei einem Bierchen.

\subsection{Feedback geben}
\begin{itemize}
  \item  "`ich"' statt "`man"' oder "`wir"'
  \item die \emph{eigene} Meinung sagen
  \item die andere Person direkt ansprechen: "`du/Sie"' statt "`er/sie"'
  \item eine konkrete, spezifische Beobachtung schildern
  \item nicht verallgemeinern
  \item nicht analysieren oder psychologisieren (nicht: "`du machst das nur, weil \ldots"')
  \item Feedback möglichst unmittelbar danach geben
  \item konstruktiv: nur Dinge ansprechen, die die andere Person auch ändern kann
\end{itemize}

\subsection{Feedback entgegennehmen}
\begin{itemize}
  \item vorher den Rahmen für das Feedback abstecken: Inhalt, Vortragstechnik, Schriftbild \ldots
  \item gut zuhören und ausreden lassen
  \item sich nicht rechtfertigen, verteidigen oder entschuldigen
  \item Missverständnisse klären, Hintergründe erläutern
  \item Feedback als Chance zur Weiterentwicklung sehen
\end{itemize}

\section{Das Blitzlicht}

\begin{itemize}
  \item wird nicht visualisiert
  \item jede Person spricht nur für sich selbst
  \item keine Diskussion (Ausnahme: wichtige Verständnisfragen)
  \item nicht unterbrechen
  \item wer anfängt, fängt an
  \item kurz~-- ein Blitzlicht ist kein Flutlicht
\end{itemize}


\chapter{Motivation}

\chapter{Gewaltfreie Kommunikation}
\index{Gewaltfreie Kommunikation}
\section{Bedürfnisse}
\label{beduerfnisse}
\index{Bedürfnisse}

\subsection{Was sind universelle Bedürfnisse?}

Bedürfnisse sind das, was wir erfüllt brauchen, damit es uns gut geht.

Ein universelles Bedürfnis ist eins, das jeder Mensch kennt~-- auch wenn sich Menschen darin unterscheiden, welche Bedürfnisse sie wie stark erfüllt brauchen.

Echte Bedürfnisse sind nicht an eine konkrete Person gebunden. Es gibt aber durchaus Bedürfnisse, die wir nur mit anderen Menschen zusammen erfüllen können, zum Beispiel unser Bedürfnis nach Gemeinschaft.

Ein Bedürfnis ist nicht an eine konkrete Handlung gebunden. Für jedes Bedürfnis gibt es viele verschiedene Strategien, um sie zu erfüllen~-- und wenn euch nur eine einzige Strategie dafür einfällt, dann habt ihr das Bedürfnis noch nicht genug verstanden.

\subsection{Liste von Bedürfnissen}

Das ursprüngliche Vokabular stammt von Marshall Rosenberg aus \cite[S.~216~f]{gfk-rosenberg} bzw.~im englischsprachigen Original \cite[S.~210]{nvc-rosenberg}. Das erweiterte Vokabular kommt \cite[S.~75~f]{gfk-dummies}.


\subsubsection{Autonomie}

\begin{multicols}{2}
  \begin{itemize}
    \item Freiheit
    \item Selbstbestimmung
  \end{itemize}
\end{multicols}


\subsubsection{Körperliche Bedürfnisse}

\begin{multicols}{2}
  \begin{itemize}
    \item Luft
    \item Wasser
    \item Bewegung
    \item Nahrung
    \item Schlaf
    \item Distanz
    \item Unterkunft
    \item Wärme
    \item Gesundheit
    \item Heilung
    \item Kraft
    \item Lebenserhaltung
  \end{itemize}
\end{multicols}


\subsubsection{Integrität, Stimmigkeit mit sich selbst}

\begin{multicols}{2}
  \begin{itemize}
    \item Authentizität
    \item Einklang
    \item Eindeutigkeit
    \item Übereinstimmung mit den eigenen Werten
    \item Identität
    \item Individualität
  \end{itemize}
\end{multicols}


\subsubsection{Einfühlung}

\begin{multicols}{2}
  \begin{itemize}
    \item Empathie
    \item verstanden/gesehen werden
    \item Gleichbehandlung
    \item Gerechtigkeit
  \end{itemize}
\end{multicols}


\subsubsection{Verbindung}

\begin{multicols}{2}
  \begin{itemize}
    \item Wertschätzung
    \item Nähe
    \item Zugehörigkeit
    \item Liebe
    \item Intimität/Sexualität
    \item Unterstützung
    \item Ehrlichkeit/Aufrichtigkeit
    \item Gemeinschaft
    \item Geborgenheit
    \item Respekt
    \item Kontakt
    \item Akzeptanz
    \item Austausch
    \item Offenheit
    \item Vertrauen
    \item Anerkennung
    \item Freundschaft
    \item Achtsamkeit
    \item Aufmerksamkeit
    \item Toleranz
    \item Zusammenarbeit
  \end{itemize}
\end{multicols}


\subsubsection{Entspannung}

\begin{multicols}{2}
  \begin{itemize}
    \item Erholung
    \item Ausruhen
    \item Spiel
    \item Spaß
    \item Leichtigkeit
    \item Ruhe
  \end{itemize}
\end{multicols}


\subsubsection{Geistige Bedürfnisse}

\begin{multicols}{2}
  \begin{itemize}
    \item Harmonie
    \item Inspiration
    \item \glqq Ordnung\grqq
    \item (innerer) Friede
    \item Freude
    \item Humor
    \item Abwechslungsreichtum
    \item Ausgewogenheit
    \item Glück
    \item Ästhetik
  \end{itemize}
\end{multicols}


\subsubsection{Entwicklung}

\begin{multicols}{2}
  \begin{itemize}
    \item Beitragen
    \item Wachstum
    \item Anerkennung
    \item Feedback
    \item Rückmeldung
    \item Erfolg (im Sinne von \glqq Gelingen\grqq
    \item Kreativität
    \item Sinne
    \item Bedeutung
    \item Effektivität
    \item Kompetenz
    \item Lernen
    \item Feiern
    \item Trauern
    \item Bildung
    \item Engagement
  \end{itemize}
\end{multicols}


\chapter{Führung}
\section{Die Rolle der Führung}
\label{fuehrung-rolle}
\index{Führung: Rolle}


\subsection{Aufgaben der Führung nach Neuberger}

Laut Neuberger\cite{neuberger-fuehren} sind die Aufgaben der Führung,

\begin{itemize}
  \item andere Menschen
  \item zielgerichtet
  \item in einer formalen Organisation
  \item unter konkreten Umweltbedingungen dazu bewegen,
  \item Aufgaben zu übernehmen und erfolgreich auszuführen,
  \item wobei humane Ansprüche gewahrt werden.
\end{itemize}


\subsection{Neuberger, aber modernisiert}

Auf das moderne Arbeiten übertragen, wäre die Aufgabe der Führung,

\begin{itemize}
  \item eine Umgebung zu schaffen,
  \item die es einem Team oder einer Organisation möglich und leicht macht,
  \item für die Mission des Teams oder der Organisation zu arbeiten,
  \item wobei die Menschen nachhaltig körperlich und seelisch gesund zu bleiben
  \item und ihr Potenzial nutzen können.
\end{itemize}

(Dies ist meine eigene \glqq Übersetzung\grqq{} aus Perspektive der modernen Führung.)


\subsection{Aufgaben der Führung nach Malik}

Dies sind laut Fredmund Malik \cite{malik-fuehrung} die Aufgaben der Führung:

\begin{itemize}
  \item für Ziele sorgen
  \item organisieren
  \item entscheiden
  \item kontrollieren
  \item Menschen entwickeln und fördern
\end{itemize}

Auf moderne Führung übertragen, wäre es die Aufgabe der Führung, dafür zu sorgen, dass diese Dinge stattfinden (also dass beispielsweise das Team Entscheidungen fällen und nachhalten kann), und nicht zwangsläufig, dass die Führung das auch selbst entscheidet.

\section{Führungsinstrumente (Führungswerkzeuge)}
\label{fuehrungsinstrumente}
\index{Führungswerkzeuge}
\index{Führungsinstrumente}


\subsection{Was ist ein Führungsinstrument?}

Ein Führungsinstrument (oder Führungswerkzeug) generell alles, was eine Führungskraft tun kann, um direkt oder indirekt zusammen mit den geführten Personen Ziele zu erreichen.


\subsubsection{Führungswerkzeuge nach Malik}

Diese Liste von Fredmund Malik \cite{malik-fuehrung} ist schon etwas angestaubt. Sie lässt sich allerdings gut in die heutige Zeit übertragen.

\begin{itemize}
  \item Besprechung
  \item Schriftstück
  \item Stellengestaltung und Einsatzsteuerung
  \item Persönliche Arbeitsmethodik
  \item Budget und Budgetierung
  \item Leistungsbeurteilung
  \item systematische Müllabfuhr
\end{itemize}


\subsection{Direkte Führungsinstrumente}

\begin{itemize}
  \item 1-zu-1-Gespräche (siehe Seite~\pageref{1-zu-1})
  \item Anweisung
  \item Besprechung
  \item delegieren
  \item Entscheidungen treffen
  \item Feedback einholen
  \item Feedback geben
  \item informieren
  \item Konflikte klären
  \item kontrollieren
  \item Kritik
  \item Lob
  \item um etwas bitten
  \item Wertschätzung ausdrücken
  \item Ziele vereinbaren
\end{itemize}


\subsection{Indirekte Führungsinstrumente}

\begin{itemize}
  \item Anreizsysteme schaffen
  \item dem Team Workshops und andere Fortbildungen anbieten
  \item den (physischen) Arbeitsplatz gestalten
  \item die Motivation verbessern
  \item die psychologische Sicherheit verbessern
  \item eine Mission definieren
  \item einen Spieleabend mit dem Team veranstalten
  \item ein Team zusammenstellen
  \item Gewaltfreie Kommunikation lernen und anwenden
  \item mit dem Team einen Escape-Room spielen
  \item mit dem Team lecker essen gehen
  \item Prozesse definieren
  \item Rollen und Verantwortlichkeiten definieren
  \item Supervision für das Team organisieren
\end{itemize}

\section{Führen lernen}
\label{fuehren-lernen}
\label{fuehrung-lernen}
\index{Führung lernen}

Meiner Ansicht nach ist Führen zu lernen so ähnlich wie singen zu lernen.

Dafür sind diese Dinge notwendig:

\begin{itemize}
  \item viel \fett{üben} (und dabei aus Fehlern lernen)
  \item sehr viel \fett{Reflexion} \index{Reflexion}
  \item \fett{Außenwahrnehmung} bekommen in der Form von Feedback \index{Außenwahrnehmung} \index{Feedback}
  \item an \fett{Trainings} und \fett{Workshops} teilnehmen (oder anderweitig Unterricht nehmen)
  \item \fett{Bücher} oder anderen Quellen von Wissen konsumieren
  \item von \fett{guten Beispielen} lernen
\end{itemize}

Damit ihr andere Menschen gut führen könnt, ist es außerdem notwendig, dass ihr euch selbst gut kennt und versteht, wie ihr tickt und was euch antreibt. Dies könnt ihr durch diese Dinge (oder eine Kombination daraus) erreichen:

\begin{itemize}
  \item Gewaltfreie Kommunikation lernen \index{Gewaltfreie Kommunikation}
  \item eine Psychotherapie machen (Tiefenpsychologie oder Psychoanalyse; keine kognitive Verhaltenstherapie) \index{Therapie} \index{Psychotherapie}
\end{itemize}

Hilfreich zum kontinuierlichen Lernen ist außerdem eine Supervision, Intervision oder kollegiale Fallberatung.


\backmatter

\bibliography{../shared/bibliography/literatur}

\chapter{Lizenz}
\label{lizenz}

\section*{Unter welchen Bedingungen könnt ihr dieses Handout benutzen?}
Dieses Handout ist unter einer \emph{Creative-Commons}-Lizenz lizensiert. Dies ist die \emph{Namensnennung-Share Alike 4.0 international (CC BY-SA 4.0)}\footnote{Die ausführliche Version dieser Lizenz findet ihr unter \url{https://creativecommons.org/licenses/by-sa/4.0/}.}. Das bedeutet, dass ihr dieses Handout unter diesen Bedingungen für euch kostenlos verbreiten, bearbeiten und nutzen könnt (auch kommerziell):

\begin{description}
  \item[Namensnennung.] Ihr müsst den Namen des Autors (Oliver Klee) nennen. Wenn ihr außerdem auch noch die Quelle\footnote{\url{https://github.com/oliverklee/workshop-handouts}} nennt, wäre das nett. Und wenn ihr mir zusätzlich eine Freude machen möchtet, sagt mir per E-Mail Bescheid.
  \item[Weitergabe unter gleichen Bedingungen.] Wenn ihr diesen Inhalt bearbeitet oder in anderer Weise umgestaltet, verändert oder als Grundlage für einen anderen Inhalt verwendet, dann dürft ihr den neu entstandenen Inhalt nur unter Verwendung identischer Lizenzbedingungen weitergeben.
  \item[Lizenz nennen.] Wenn ihr den Reader weiter verbreitet, müsst ihr dabei auch die Lizenzbedingungen nennen oder beifügen.
\end{description}


\printindex

\end{document}


\chapter{Tipps für die Zeitplanung}
\label{workshop-zeitplanung}

\section{Organisatorisches}
\begin{itemize}
  \item \emph{überhaupt} vorher einen Zeitplan machen
  \item Zeitplan mit Tabellenkalkulation machen
  \item nach 40--60~Minuten eine Pause machen
  \item tatsächliche Zeiten notieren
\end{itemize}

\section{Reihenfolge}
\begin{itemize}
  \item am Anfang Einstieg machen
  \item \emph{niemals} überziehen
  \item am Schluss 20\,\% Pufferzeit einplanen
  \item Vorkenntnisse für eine Einheit sicherstellen:
    \begin{itemize}
      \item Vorkenntnisse vorher abfragen
      \item Vorkenntnisse vorher behandeln
    \end{itemize}
\end{itemize}

\section{Methoden}
\begin{itemize}
  \item nicht zweimal die gleiche Methode hintereinander
  \item Vortrag soll < 20 Minuten sein, besser: < 10 Minuten
  \item Gedanken über Ziel der Einheit machen
  \item Passt die Methode zum Ziel?
\end{itemize}


\chapter{Korrigieren von Übungszetteln}

\section{Allgemeines Tipps}
\begin{itemize}
  \item zuerst Überblick über alle Lösungen holen
  \item erst eine Aufgabe komplett, dann die nächste
  \item nach einer Aufgabe: Zettel durchmischen!
  \item nicht auf die Namen gucken
  \item leserlich schreiben
  \item nicht mit Rot
  \item bei Genervtsein: Pause!
  \item Zettel schnell zurückgeben
  \item sagen, wie die Aufgaben ausgefallen sind
\end{itemize}

\section{Bepunktung}
\begin{itemize}
  \item Bewertungskriterien ankündigen
  \item erst korrigieren, dann punkten
  \item vorher Punktekriterien überlegen
  \item Punkte für Teillösungen geben
  \item nur Pluspunkte geben
  \item Tutor-übergreifen einheitlich korrigieren
\end{itemize}

\section{Wertschätzender Umgang}
\begin{itemize}
  \item Respekt!
  \item Nicht: "`Unsinn"', "`Nein!"', "`Lächerlich!"'
  \item loben
  \item Feedback zur mathematischen Ausdrucksweise
  \item Feedback zur Beweisstruktur
\end{itemize}


\backmatter

\chapter{Literaturtipps}

\section{Literatur speziell zu Tutorien}
\cite{tutorenschulung-kassel, fackel-oder-funzel, torch-or-firehose-eggert, torch-or-firehose, tutor-leitfaden}

\section{Literatur zu Workshops und Unterricht}
\cite{moderationstechniken, visualisieren-praesentieren-moderieren, workshop-buch, treibhaeuser-der-zukunft}

\section{Allgemeine Literatur zur Lehre in Mathematik, IT, Informatik}
\cite{teaching-engineering, red-pen, it-trainer}

\section{Literatur über Arbeitstechniken in Mathematik und Informatik}
\cite{obda, schule-des-denkens, solve-it, prove-it}

\section{Bücher, um Studis Themen der Mathematik und Informatik interessant und leicht verständlich zu vermitteln}
\cite{mathe-humor, ideen-der-informatik, mathe-westentasche, in-mathe-war-ich-immer-schlecht}

\section{Literatur rund ums Lernen}
\cite{denken-lernen-vergessen, lernen-zu-lernen}

\section{Literatur, die ich bei den Übungstutorien benutze}
\cite{algorithms}

\bibliography{../shared/bibliography/literatur}

\chapter{Lizenz}
\label{lizenz}

\section*{Unter welchen Bedingungen könnt ihr dieses Handout benutzen?}
Dieses Handout ist unter einer \emph{Creative-Commons}-Lizenz lizensiert. Dies ist die \emph{Namensnennung-Share Alike 4.0 international (CC BY-SA 4.0)}\footnote{Die ausführliche Version dieser Lizenz findet ihr unter \url{https://creativecommons.org/licenses/by-sa/4.0/}.}. Das bedeutet, dass ihr dieses Handout unter diesen Bedingungen für euch kostenlos verbreiten, bearbeiten und nutzen könnt (auch kommerziell):

\begin{description}
  \item[Namensnennung.] Ihr müsst den Namen des Autors (Oliver Klee) nennen. Wenn ihr außerdem auch noch die Quelle\footnote{\url{https://github.com/oliverklee/workshop-handouts}} nennt, wäre das nett. Und wenn ihr mir zusätzlich eine Freude machen möchtet, sagt mir per E-Mail Bescheid.
  \item[Weitergabe unter gleichen Bedingungen.] Wenn ihr diesen Inhalt bearbeitet oder in anderer Weise umgestaltet, verändert oder als Grundlage für einen anderen Inhalt verwendet, dann dürft ihr den neu entstandenen Inhalt nur unter Verwendung identischer Lizenzbedingungen weitergeben.
  \item[Lizenz nennen.] Wenn ihr den Reader weiter verbreitet, müsst ihr dabei auch die Lizenzbedingungen nennen oder beifügen.
\end{description}


\end{document}
