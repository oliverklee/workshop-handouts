\chapter{Motivation}
\label{motivation}

\section{Extrinsische Motivation}
\begin{itemize}
  \item Motivation durch äußere Anreize
  \item motiviert nicht so stark wie die intrinsische Motivation
\end{itemize}

Beispiele:

\begin{itemize}
  \item Belohnungen
  \item Bestrafungen
  \item Bepunktung
  \item Klausurzulassung
  \item Möhre an der Schnur vor dem Maul des Esels
\end{itemize}


\section{Intrinsische Motivation}
\begin{itemize}
  \item Motivation durch die Sache an sich
  \item motiviert stärker als die extrinsische Motivation
\end{itemize}

Beispiele:

\begin{itemize}
  \item Interesse am Thema selbst
  \item positive Gefühle dabei: Spaß, Erfolgserlebnisse, anderen Menschen helfen
  \item Befriedigung von Grundbedürfnissen: Spiel, persönliches Wachstum, soziale Kontakte, \ldots
\end{itemize}

\subsection{Interesse am Thema wecken}
\begin{itemize}
  \item schöne Anwendungsfälle aus der Praxis zeigen
  \item zeigen, warum das Thema für die \emph{momentane} Arbeit der Studis wichtig ist
\end{itemize}
