\documentclass[a4paper,openany,twoside,titlepage,10pt,headsepline]{scrbook}

%----------------------------------------------------------------------------------
% Typografie und Fonts
%----------------------------------------------------------------------------------

% Vektorfonts statt Pixelfonts
\usepackage[T1]{fontenc}
\usepackage{lmodern}

% Input-Encoding für UTF-8
\usepackage[utf8]{inputenc}

\newcommand{\fett}[1]{\textsf{\textbf{#1}}}

%----------------------------------------------------------------------------------
% Pakete und Parameter
%----------------------------------------------------------------------------------

% Mehrsprachigkeit für Deutsch und Englisch erlauben
\usepackage[ngerman,english]{babel}

% Mehr von der Seitenbreite nutzen
% \usepackage{a4wide}

% Grafikpaket
\usepackage{color}
\usepackage[pdftex]{graphicx}

% Absätze werden nicht eingezogen, sondern vertikal abgesetzt
\setlength{\parindent}{0mm}
\addtolength{\parskip}{0.4em}

% Bibliographieeinstellungen
\usepackage{natbib}
\bibliographystyle{alpha}

% lesbare Verweise
\usepackage[pdftex,plainpages=false,pdfpagelabels]{hyperref}

% nette URLs
\usepackage{url}

% für Boxen etc.
% \usepackage{framed}
\definecolor{shadecolor}{rgb}{0.8,0.8,0.8}

% Anführungszeichen sprachabhängig machen
% \usepackage[babel]{csquotes}


%----------------------------------------------------------------------------------
% Seitenlayout
%----------------------------------------------------------------------------------


% Seiten-Kopfzeilen und -Fußzeilen
\usepackage{scrlayer-scrpage}
\pagestyle{headings}

% Kopfzeile auf linker Seite: "1  Einführung"
\renewcommand{\chaptermark}[1]{%
\markboth{\thechapter\ \ \ \ #1}{}}

% Kopfzeile auf rechter Seite: "1.1  Basics"
\renewcommand{\sectionmark}[1]{%
\markright{\thesection\ \ \ \ #1}{}}

% Seitenlayout
\topmargin0mm
\footskip10mm % Abstand von unserem Rand zu Datum

\newlength{\fullwidth} % Seites des Textes plus Randnotizen
\setlength{\fullwidth}{\textwidth}
\addtolength{\fullwidth}{\marginparsep}
\addtolength{\fullwidth}{\marginparwidth}

% Maximal drei Ebenen nummerieren
\setcounter{secnumdepth}{3}
% Maximale Gliederungstiefe, die noch ins Inhaltsverzeichnis aufgenommen wird
\setcounter{tocdepth}{1}

% zweispaltiges Layout möglich machen
\usepackage{multicol}

% Descriptions ohne Einzug
\renewenvironment{description}[1][0pt]
{\list{}{
    \labelwidth=0pt \leftmargin=#1
     \let\makelabel\descriptionlabel
  }
}
{\endlist}

\raggedbottom


\AtBeginDocument{\selectlanguage{ngerman}}

\title{ Psychologische Sicherheit in Teams }
\author{Oliver Klee\\\texttt{www.oliverklee.de}\\\texttt{seminare@oliverklee.de}}
\date{Version vom \today}

\begin{document}

\frontmatter

\maketitle

\tableofcontents


\mainmatter

\chapter{Seminar-Handwerkszeug}
\section{Regeln für den Workshop}
\label{gfk-workshopregeln}
\index{Workshopregeln}

\paragraph{Vegas-Regel:} Was wir hier persönlichen Dingen teilen, bleibt im Workshop. Wir erzählen Dinge nur anonymisiert nach außen.

\paragraph{Keine dummen Fragen:} Es gibt keine dummen Fragen. Für Fragen, die nicht gut in den Rahmen des aktuellen Themas passen, haben wir einen Themenkühlschrank.

\paragraph{Für-sich-sorgen-Regeln:} Wir alle versuchen, uns auf dem Workshop gut um uns selbst zu kümmern. Wenn wir etwas brauchen, sprechen wir es an oder sorgen selbst dafür.

\paragraph{Freiwilligkeit:} Alle Übungen im Workshop sind freiwillig. Es ist okay, bei eine Übung komplett auszusetzen oder eine Runde zu passen.

\paragraph{Gut zueinander sein:} Wir tun unser Bestes, konstruktiv miteinander umzugehen und uns gut zu behandeln.

\section{Feedback: Tipps und Tricks}
\index{Feedback}

\subsection{Was ist Feedback?}
Feedback ist für euch eine Gelegenheit, in kurzer Zeit viel über euch selbst zu lernen. Feedback ist ein Anstoß, damit ihr danach an euch arbeiten könnt (wenn ihr wollt).

Feedback heißt, dass euch jemandem einen persönlichen, subjektiven Eindruck in Bezug auf konkrete Punkte mitteilt. Da es sich um einen persönlichen Eindruck im Kopf eines einzelnen Menschen handelt, sagt Feedback nichts darüber aus, wie ihr tatsächlich wart. Es bleibt allein euch selbst überlassen, das Feedback, das ihr bekommt, für euch selbst zu einem großen Gesamtbild zusammenzusetzen.

Es kann übrigens durchaus vorkommen, dass ihr zur selben Sache von verschiedenen Personen völlig unterschiedliches (oder gar gegensätzliches) Feedback bekommt.

Es geht beim Feedback \emph{nicht} darum, euch mitzuteilen, ob ihr ein guter oder schlechter Mensch, ein guter Redner, eine schlechte Rhetorikerin oder so seid. Solche Aussagen haben für euch keinen Lerneffekt. Stattdessen schrecken sie euch ab, Neues auszuprobieren und dabei auch einmal so genannte Fehler zu machen.

Insbesondere ist Feedback keine Grundsatzdiskussion, ob das eine oder andere Verhalten generell gut oder schlecht ist. Solche Diskussionen führt ihr besser am Abend bei einem Bierchen.

\subsection{Feedback geben}
\begin{itemize}
  \item  "`ich"' statt "`man"' oder "`wir"'
  \item die \emph{eigene} Meinung sagen
  \item die andere Person direkt ansprechen: "`du/Sie"' statt "`er/sie"'
  \item eine konkrete, spezifische Beobachtung schildern
  \item nicht verallgemeinern
  \item nicht analysieren oder psychologisieren (nicht: "`du machst das nur, weil \ldots"')
  \item Feedback möglichst unmittelbar danach geben
  \item konstruktiv: nur Dinge ansprechen, die die andere Person auch ändern kann
\end{itemize}

\subsection{Feedback entgegennehmen}
\begin{itemize}
  \item vorher den Rahmen für das Feedback abstecken: Inhalt, Vortragstechnik, Schriftbild \ldots
  \item gut zuhören und ausreden lassen
  \item sich nicht rechtfertigen, verteidigen oder entschuldigen
  \item Missverständnisse klären, Hintergründe erläutern
  \item Feedback als Chance zur Weiterentwicklung sehen
\end{itemize}

\section{Paarinterview zum Kennenlernen}
\label{paarinterview}
\index{Paarinterview}

Nehmt euch für das Interview 15~Minuten Zeit pro Person. Wechselt selbstständig.

\subsection{Leitfragen}

\begin{itemize}
 \item Wo und wie wohnst du?
 \item Was machst du in Beruf und Ehrenamt so? Und was hast du vorher so Interessantes gemacht?
 \item Was sind ein paar Dinge, die dir im Leben zurzeit Freude bereiten?
 \item Was brauchst du (von anderen Personen oder der Umgebung), damit die Zusammenarbeit mit dir gut funktioniert?
 \item Was sollten andere Menschen über dich wissen, wenn sie mit dir zusammenarbeiten?
 \item Was machst du, um trotz der aktuellen Krisen psychisch halbwegs gesund zu bleiben?
 \item Was ist ein \emph{Guilty Pleasure}, dem du ab und an frönst?
\end{itemize}


\subsection{Bonusfragen}

Falls ihr euch schon gut kennt und noch etwas Zeit habt, könnt ihr euch mit diesen Zusatzfragen noch besser kennenlernen.

Diese Fragen kommen aus dem Spiel \emph{Gesprächsstoff XL} \cite{gespraechsstoff}.

\begin{itemize}
  \item Welche Fernsehsendung hast du nie verpasst, als du noch jünger warst?
  \item Wie meinst du, würden dich deine Freund\_innen beschreiben, wenn sie nur drei Wörter verwenden dürften?
  \item Kannst du ein Beispiel dafür nennen, wann du einmal zum richtigen Zeitpunkt am richtigen Ort warst?
  \item Was ist, von der Persönlichkeit her, der größte Unterschied zwischen dir und deinen Eltern?
  \item Was antwortest du einem Kind, das fragt, ob es einen Gott gibt?
  \item Was ist dein Lieblings-Knabberzeug (oder -Nascherei)?
  \item Erzähle von bedeutungslosem Wissen, das du hast.
  \item Kannst du etwas nennen, von dem du wünscht, früher damit angefangen zu haben?
  \item Welche Person hat dich zuletzt so richtig wütend gemacht?
  \item Wenn du wählen müsstest: Würdest du lieber einen anderen Menschen umbringen und straffrei davonkommen oder 25 Jahre für einen Mord im Gefängnis sitzen, den du nicht begangen hast?
  \item Bist du schon einmal im Kino eingeschlafen?
  \item Wenn du den Leben noch einmal leben könntest, was würdest du nicht wieder tun?
  \item Erzähle etwas über ein Ereignis der letzten Jahre, an das du dich für immer erinnern wirst.
\end{itemize}

Die Antworten auf diese Fragen sind \emph{nicht} Teil der Vorstellung vor der Gruppe.


\chapter{Kommunikation}
\section{Prinzipien der Kommunikation}
\label{kommunikationsprinzipien}
\index{Kommunikation}
\index{Missverständnisse}
\index{Sender-Empfänger-Modell}

\begin{itemize}
  \item Es gibt bei Kommunikation immer sendende und (mindestens) eine empfangende Partei (Sender-Empfänger-Modell nach Shannon und Weaver). Diese Rollen können in einer Interaktion öfter wechseln.
  \item Die Partei, die etwas von der anderen will, hat die Zuständigkeit dafür dafür (und das Interesse daran), dass die Kommunikation erfolgreich ist.
  \item Jede Partei hat nur auf ihre eigene Hälfte der Kommunikation direkten Einfluss.
  \item Missverständnisse passieren, und sie sind eher die Regel denn die Ausnahme~-- wir bemerken sie nur oft nicht.
\end{itemize}


\subsection{Metakommunikation}
\label{metakommunikation}
\index{Metakommunikation}

\emph{Metakommunikation} (\glqq Kommunikation über Kommunikation\grqq) bedeutet, die Kommunikation auf eine höhere Ebene zu verlagern und darüber zu reden, wie wir miteinander reden, wie wir miteinander umgehen und was uns beschäftigt.

\section{Direkte vs.~indirekte Kommunikation}
\label{direkte-kommunikation}
\index{direkte Kommunikation}
\index{indirekte Kommunikation}

Bei \fett{direkter Kommunikation} sagt die Person das, was sie kommunizieren möchte, explizit mit ihren Worten.

Bei \fett{indirekter Kommunikation} benutzt die Person stattdessen Mehrdeutigkeit, Anspielungen, den Tonfall, den Rhythmus der Sprache, Ironie oder Sarkasmus.

Direkte Kommunikation zu nutzen, senkt das Risiko für Missverständnisse deutlich. Außerdem können wir damit mehr Verantwortung für unsere eigene Kommunikation übernehmen.

GfK setzt sehr stark auf direkte Kommunikation.

\subsection{Beispiele}

\renewcommand{\arraystretch}{2.0}
\begin{tabular}{|p{20em}|p{20em}|}
\hline

\fett{direkte Kommunikation} & \fett{indirekte Kommunikation} \\
\hline

Ich würde gerne mit meinem Freund einen Abend zu zweit verbringen. Wärst du bereit, morgen Abend von 20 bis 23 Uhr die WG zu verlassen? &
Hättest du Lust, morgen Abend ohne mich ins Kino zu gehen? \\
\hline

Mir ist kalt. Wäre es okay, wenn ich das Fenster zumache? &
Ziemlich kalt hier. \\
\hline

Ich bin gerade echt genervt. &
(knurrt) \\
\hline

Könntest du die Musik vielleicht etwas leiser machen? &
Tolle Musik! \\
\hline

Ich habe im Moment echt wenig Geld. Zurzeit kaufen wir ein Kilo Kaffeebohnen im Monat. Können wir uns dazu mal zusammensetzen, wie wir unsere Ausgaben für Kaffee senken können? &
Du trinkst zu viel Kaffee. \\
\hline

Könntest du bitte jeden zweiten Tag den Müll runterbringen? &
(stellt dem Mitbewohner den vollen Mülleimer vor die Zimmertür) \\
\hline

Könntest du bitte damit aufhören, mit dem Kuli zu klicken? &
(knallt die Kaffeetasse auf den Schreibtisch) \\
\hline

\end{tabular}
\renewcommand{\arraystretch}{1.0}

\section{Feedback vs.~Kritik vs.~Bitten}
\label{feedback-vs-kritik}

\emph{Feedback}, \emph{Kritik} und \emph{Bitten} sind drei unterschiedliche Sachen, und es ist hilfreich, wenn ihr sie klar unterscheiden könnt, um dann bewusst auszuwählen und klar zu kommunizieren.


\subsection{Feedback}
\index{Feedback}

Das Ziel von Feedback ist, dass die andere Person daraus lernen und daran wachsen kann.

Bei Feedback geht es um die andere Person, nicht um euch.

Feedback ist ein Geschenk, und es ist okay, Geschenke nicht annehmen zu wollen.

Ein Geschenk kann man nicht aufzwingen~-- jemandem ohne Zustimmung Feedback zu geben, ist Gewalt.


\subsection{Bitten}
\index{Bitten}

Bei einer Bitte geht es darum, dass ihr etwas für euch von der anderen Person braucht oder möchtet.

Dort liegt der Fokus also bei euch, nicht bei der anderen Person.

Bitten sind keine Forderungen: Es ist also auch okay, eine Bitte abzulehnen.

Mehr zu Kriterien für hilfreiche Bitten findet ihr ab Seite~\pageref{bitten}.


\subsection{Kritik}
\index{Kritik}

Wenn es tatsächlich darum geht, zusammen an einer Sache zu arbeiten oder zu klären, kann Kritik an der Sache (nicht an der Person!) hilfreich sein. In diesem Fall liegt der Fokus auf der Sache, nicht auf einer Person.

Ansonsten ist Kritik oft undeutlich kommuniziertes Feedback oder eine unklar kommunizierte Bitte. In solchen Fällen ist es hilfreicher, stattdessen klar Feedback oder eine Bitte zu äußern.


\chapter{Psychologische Sicherheit}
\section{Was ist psychologische Sicherheit?}
\label{ps-definition}
\index{psychologische Sicherheit}

Amy C.~Edmondson \cite{the-fearless-organisation} definiert psychologische Sicherheit so:

\selectlanguage{english}
\begin{quote}
  I have defined psychological safety as the belief that the work environment is safe for interpersonal risk taking. The concept refers to the experience of feeling able to speak up with relevant ideas, question, or concerns.
\end{quote}
\selectlanguage{ngerman}

Karolin Helbig und Minette Norman \cite{psychological-safety-playbook} fassen das noch einmal etwas prägnanter zusammen:

\selectlanguage{english}
\begin{quote}
  Psychological safety is a belief that one will not be punished or humiliated for speaking up with ideas, questions, concerns, or mistakes.
\end{quote}
\selectlanguage{ngerman}


\section{Psychologische Sicherheit und Leistung}
\label{ps-leistung}
\index{Leistung}

Die psychologische Sicherheit in einem Team hat einen extrem großen Einfluss auf die Leistung bzw.~Leistungsfähigkeit eines Teams \cite{high-performing-teams}.

Dabei sind psychologische Sicherheit und Leistungsstandards in einem Team unabhängig voneinander. Nur bei hoher psychologischer Sicherheit und hohen Leistungsstandards kann ein Team zu Bestleistungen aufblühen:~\cite{the-fearless-organisation}

\vspace{1em}

\renewcommand{\arraystretch}{2.0}
\begin{tabular}{|p{17em}|p{10em}|p{11em}|}
\hline
& \fett{Niedrige Standards} & \fett{Hohe Standards}
\\ \hline

\fett{Niedrige psychologische Sicherheit}
& Apathie-Zone
& Angst-Zone
\\ \hline

\fett{Hohe psychologische Sicherheit}
& Komfort-Zone
& Zone des Lernens und der Bestleistung
\\ \hline

\end{tabular}
\renewcommand{\arraystretch}{1.0}

\section{Die Entdeckung der psychologischen Sicherheit}
\label{ps-entdeckung}
\index{Entdeckung der psychologischen Sicherheit}

Die Forscherin Amy C.~Edmonson \cite{the-fearless-organisation} war an einem Forschungsprojekt beteiligt, in dem sie die Folgen von Teamarbeit auf Fehler in der Arbeit in Krankenhäusern untersuchte. Dabei erhob sie die Fehlerquote von Teams (Anzahl der gemeldeten Fehler pro 1000 Patient\_innen-Tage). Weiterhin befragte sie die Teams bezüglich gegenseitigem Respekt, der Qualität der Zusammenarbeit, deren Zufriedenheit und anderer Punkte.

Dabei kam heraus, dass die gemeldete Fehlerquote und die Qualität der Zusammenarbeit im Team \emph{umgekehrt} miteinander korreliert waren~-- also die Teams mit besserer Zusammenarbeit mehr Fehler gemeldet hatten.

Bei weiteren stillen Beobachtungen kam dann heraus, dass sich die Teams tatsächlich sehr darin unterschieden, ob sie offen über Fehler gesprochen haben.

Dafür wurde später dann der Begriff \emph{psychologische Sicherheit} geprägt.

Die psychologische Sicherheit war bei der Untersuchung auch zwischen unterschiedlichen Teams innerhalb desselben Krankenhauses deutlich unterschiedlich. Das hängt damit zusammen, dass diese sehr von den Führungskräften des einzelnen Teams abhängt.

\section{Einige bekannte Fälle, in denen mangelnde psychologische Sicherheit eine Rolle gespielt hat}
\label{ps-faelle}


\subsection{Der Abgasskandal bei Volkswagen und anderen Autoherstellern}
\index{Abgasskandal}
\index{Dieselgate}
\index{Volkswagen}

Beim Abgasskandal (\glqq Dieselgate\grqq) hatten mehrere Autohersteller illegale Abschalteinrichtung in der Motorsteuerung ihrer Diesel-Fahrzeuge verwendet, um die von den Vorgesetzten vorgegebenen strengeren Abgaswerte auf dem Prüfstand einzuhalten (aber nicht im realen Betrieb auf der Straße).

Dies hat (neben dem Vertrauensverlust) zum Rückruf von mehreren Millionen Fahrzeugen und Strafen in Millionenhöhe geführt.


\subsection{Nokia}
\index{Nokia}

Nach dem großen Erfolg mit einfachen Handys setze der finnische Handyhersteller Nokia sehr auf Feature-Phones. Durch eine Kultur der Angst in dem Unternehmen wagten es die Angestellten nicht, ihr Bedenken darüber zu äußern, dass Feature-Phones gegenüber Smartphones wenig Chancen haben würden.

Wegen des geringen wirtschaftlichen Erfolgs dieser Strategie folgten Entlassungswellen und Teil-Verkäufe der Firma. Nokia wäre daran beinahe zugrunde gegangen. Nach 2020 konnte Nokia durch ein Rebranding und eine radikale Neuausrichtung gerettet werden.


\subsection{Wells Fargo}
\index{Wells Fargo}

Beim US-amerikanische Finanzdienstleister Wells Fargo bekamen die Angestellten in den frühen 2000ern unrealistische Vorgaben, wie viele Produkte sie verkaufen mussten. Dabei übte die Bank sehr großen Druck auf das Personal aus.

Dies führte dazu, dass die Angestellten den Kund\_innen ohne deren Einverständnis Produkte verkauften oder sie belogen.

Dies führte dazu, dass Wells Fargo insgesamt mehr als 7~Milliarden Dollar Strafe zahlen musste.


\subsection{Columbia-Katastrophe}
\index{Columbia-Katastrophe}

2003 verglühte das Space-Shuttle \emph{Columbia} beim Wiedereintritt in die Erdatmosphäre. Dabei kamen alle Besatzungsmitglieder ums Leben. Ursache war ein technischer Defekt an der äußeren Schaumstoffisolierung.

Ein Ingenieur hatte vorher das Problem festgestellt und wollte es an das zuständige Verteidigungsministerium melden. Sein Vorgesetzter teilte ihm mit, dass \glqq die Ingenieure keine Anfragen an jemanden oberhalb ihrer Hierarchieebene schicken durften\grqq, und leitete die Anfrage nicht weiter.


\subsection{Zusammenstoß zweier Boeing 747}
\index{Boeing 747}
\index{Flugzeugzusammenstoß}

1977 stießen zwei Boeing 747 auf einer Landebahn zusammen. 583 Menschen an Bord starben.

Der Pilot hatte dabei das Flugzeug beschleunigt, obwohl die Freigabe der Flugsicherung noch nicht vorlag. Der Erst Offizier wies den Piloten darauf hin. Der Pilot unterbrach ihn und startete das Flugzeug.


\section{Psychologische Sicherheit und Leistung}
\label{ps-leistung}
\index{Leistung}

Die psychologische Sicherheit in einem Team hat einen extrem großen Einfluss auf die Leistung bzw.~Leistungsfähigkeit eines Teams \cite{high-performing-teams}.

Dabei sind psychologische Sicherheit und Leistungsstandards in einem Team unabhängig voneinander. Nur bei hoher psychologischer Sicherheit und hohen Leistungsstandards kann ein Team zu Bestleistungen aufblühen:~\cite{the-fearless-organisation}

\vspace{1em}

\renewcommand{\arraystretch}{2.0}
\begin{tabular}{|p{17em}|p{10em}|p{11em}|}
\hline
& \fett{Niedrige Standards} & \fett{Hohe Standards}
\\ \hline

\fett{Niedrige psychologische Sicherheit}
& Apathie-Zone
& Angst-Zone
\\ \hline

\fett{Hohe psychologische Sicherheit}
& Komfort-Zone
& Zone des Lernens und der Bestleistung
\\ \hline

\end{tabular}
\renewcommand{\arraystretch}{1.0}

\section{Auswirkungen von psychologischer Sicherheit}
\label{ps-auswirkungen}
\index{Auswirkungen psychologischer Sicherheit}
\index{folgen psychologischer Sicherheit}


Neben der Leistung des Teams (siehe Seite~\pageref{ps-leistung}) wirkt sich der Grad der psychologischen Sicherheit in einem Team auf diese Punkte aus:

\begin{itemize}
  \item ob das Team aus vermeidbaren Fehlern lernt \index{Fehler}\index{vermeidbare Fehler}
  \item ob das Team an Konflikten wächst oder Konflikte die Arbeit des Teams behindern \index{Konflikte}
  \item ob die Führung überhaupt von Problemen erfährt (das ist der Aspekt, bei dem die Führung am wenigsten mitbekommt, wenn es dem Team an psychologischer Sicherheit mangelt) \index{Probleme}
  \item ob die Führung rückgemeldet bekommt (sowohl überhaupt als auch frühzeitig), ob Ziele sinnvoll und realistisch sind \index{Ziele}
\end{itemize}

\section{Vertrauen vs. psychologische Sicherheit}
\label{vertrauen-vs-ps}
\index{Vertrauen}
\index{psychologische Sicherheit}

Diese Tabelle gibt eine Übersicht darüber, wie sich das Konzept von Vertrauen~\cite{thin-book-of-trust, anatomy-of-trust} von dem von psychologische Sicherheit~\cite{the-fearless-organisation} unterscheidet.

\vspace{1em}

\renewcommand{\arraystretch}{2.0}
\begin{tabular}{|p{5em}|p{19em}|p{19em}|}
\hline
& \fett{Vertrauen} & \fett{psychologische Sicherheit}
\\ \hline

Definition
& \glqq Ich bin bereit, etwas mir Wichtiges zu riskieren, indem ich es in die Hände der anderen Person gebe.\grqq
& \glqq Unsere Arbeitsumgebung ist sicher genug dafür, dass wir dort zwischenmenschliche Risiken eingehen können.\grqq
\\ \hline

zu wem
& von einer einzelnen Person zu einer anderen Person oder Institution &
innerhalb eines Teams oder einer Gruppe
\\ \hline

zeitlicher Fokus
& ausgehend von der Summe der Erfahrungen in der Vergangenheit eine Prognose für die Zukunft
& ausgehend von der Vergangenheit für das Jetzt
\\ \hline

Haupt-Einfluss
& die andere Person/Institution
& die Führungskraft und das Team
\\ \hline

\end{tabular}
\renewcommand{\arraystretch}{1.0}

\section{Elemente von Vertrauen}
\label{vertrauen-elemente}
\index{Vertrauen: Elemente}


\subsection{Nach Brené Brown}

In ihrem Talk \emph{The Anatomy of Trust}~\cite{anatomy-of-trust} liefert Brené Brown mit dem Akronym \emph{BRAVING} eine schön knackige Merkhilfe, die (um das schöne Akronym zu bilden) inhaltlich allerdings etwas unscharf wird. Ihre Ergebnisse bauen auf einer Metaanalyse der Forschung zu Vertrauen auf.
\index{BRAVING}

Sie fasst die Elemente von Vertrauen wie folgt zusammen:

\fett{B}oundaries \\
\fett{R}eliability \\
\fett{A}ccountability \\
\fett{V}ault \\
\fett{I}ntegrity \\
\fett{N}on-judgement \\
\fett{G}enerosity


\subsubsection{Grenzen respektieren (Boundaries)}
\index{Grenzen respektieren}

\begin{itemize}
  \item die eigenen Grenzen kennen, kommunizieren und schützen
  \item die Grenzen anderer Menschen anerkennen und respektieren
\end{itemize}



\subsubsection{Zuverlässigkeit (Reliability)}
\index{Zuverlässigkeit}

\begin{itemize}
  \item das tun, was ihr versprochen/zugesagt habt
  \item immer und immer wieder
\end{itemize}


\subsubsection{Verantwortung übernehmen (Accountability)}
\index{Verantwortung übernehmen}

\begin{itemize}
  \item nach einem Fehler dafür geradestehen, um Entschuldigung bitten, und es wiedergutmachen
  \item anderen ermöglichen, für Fehler geradezustehen, um Entschuldigung zu bitten, und es wiedergutzumachen
\end{itemize}


\subsubsection{Vertraulichkeit wahren (Vault)}
\index{Vertraulichkeit}

\begin{itemize}
  \item \glqq Was ich mit dir teile, behandelst du vertraulich.\grqq
  \item \glqq Was du mit mir teilst, behandele ich vertraulich.\grqq
  \item \glqq Ich erlebe, dass du Sachen anderer Personen vertraulich behandelst.\grqq
\end{itemize}


\subsubsection{Integrität (Integrity)}
\index{Integrität}

\begin{itemize}
  \item die eigenen Werte und Prinzipien tatsächlich leben
  \item Worten Taten folgen lassen
  \item das Richtige tun (statt das Angenehme, Schnelle oder Einfache)
\end{itemize}


\subsubsection{Nicht-Verurteilen (Non-judgement)}
\index{Nicht-Verurteilen}

\begin{itemize}
  \item \glqq Ich kann dich um Hilfe bitten, ohne dass du mich dafür verurteilst.\grqq
  \item \glqq Du kannst mich um Hilfe bitten, ohne dass ich dich dafür verurteile.\grqq
\end{itemize}


\subsubsection{Großzügigkeit (Generosity)}
\index{Großzügigkeit}

\begin{itemize}
  \item \glqq Du gehst bei meinen Worten und Taten von guten Absichten aus und fragst nach.\grqq
\end{itemize}


\subsection{Nach Charles Feltman}

Charles Feltman geht mit seinem Büchlein \emph{The Thin Book of Trust}~\cite{thin-book-of-trust} eher in die Richtung Lebenshilfe-Buch oder Business-Buch und ist weniger wissenschaftlich fundiert als die anderen beiden Werke.

Er stellt die folgenden Elemente von Vertrauen vor:


\subsubsection{Aufrichtigkeit}
\index{Aufrichtigkeit}

\glqq Was ich sage, meine ich so, und ich handele entsprechend.\grqq


\subsubsection{Zuverlässigkeit}
\index{Zuverlässigkeit}

\glqq Du kannst darauf zählen, dass ich liefere, was ich versprochen habe.\grqq


\subsubsection{Kompetenz}
\index{Kompetenz}

\glqq Ich weiß, was ich kann. Und ich weiß auch, was ich nicht kann.\grqq


\subsubsection{Wohlwollen (Care)}
\index{Wohlwollen}

\glqq Ich berücksichtige unser beider Interessen, wenn ich Entscheidungen fälle oder handele.\grqq


\subsection{Nach Ariane Jäckel}

Ariane Jäckel stellt in ihrer Dissertation \emph{Gesundes Vertrauen in Organisationen: Eine Untersuchung der Vertrauensbeziehung zwischen Führungskraft und Mitarbeiter}~\cite{gesundes-vertrauen-in-organisationen} den aktuellen Stand der Forschung dar, geht sehr in die Tiefe und ist wissenschaftlich sehr sauber.

Sie hat die folgenden Dimensionen von Vertrauenswürdigkeit identifiziert:


\subsubsection{Kompetenz}
\index{Kompetenz}

Darunter fallen aus den anderen Konzepten diese Unterpunkte:

\begin{itemize}
  \item Zuverlässigkeit (Brené Brown, Charles Feltman)
  \item Kompetenz (Charles Feltman)
\end{itemize}


\subsubsection{Integrität}
\index{Integrität}

Darunter fallen aus den anderen Konzepten diese Unterpunkte:

\begin{itemize}
  \item Verantwortung übernehmen (Brené Brown)
  \item Vertraulichkeit wahren (Brené Brown)
  \item Integrität (Brené Brown)
  \item Aufrichtigkeit (Charles Feltman)
\end{itemize}


\subsubsection{Wohlwollen}
\index{Wohlwollen}

Darunter fallen aus den anderen Konzepten diese Unterpunkte:

\begin{itemize}
  \item Grenzen respektieren (Brené Brown)
  \item Nicht-Verurteilen (Brené Brown)
  \item Großzügigkeit (Brené Brown)
  \item Wohlwollen (Charles Feltman)
\end{itemize}

\section{Was tun für mehr psychologische Sicherheit?}
\label{ws-was-tun}

Generell ist psychologische Sicherheit ein kontinuierlicher Prozess~-- ihr seid damit niemals \glqq fertig\grqq.

Dies ist eine ungeordnete Liste von Ideen aus \emph{The Fearless Organization} \cite{the-fearless-organisation}.

\begin{itemize}
  \item die Strategie der Organisation als Hypothese behandeln anstelle als Plan \index{Strategie} \index{Hypothesen}
  \item es zur Routine machen, explizit nach möglichen Problemen zu fragen \index{Probleme} \index{Routinen}
  \item fragen: \glqq War alles so sicher, wie du es gerne hättest?\grqq \index{Sicherheit}
  \item demokratisch/kooperativ führen anstelle von autoritär (zu Führungsstilen siehe Seite~\pageref{fuehrungsstile}) \index{Führungsstile}
  \item eine Kultur der radikalen Ehrlichkeit \cite{radical-honesty} schaffen \index{radikale Ehrlichkeit}
  \item eine Kultur des konstruktiven Feedbacks etablieren (Seite \pageref{feedback-regeln}) und zwischen Feedback, Bitten, Forderungen und Kritik sauber trennen \index{Feedback} (Seite \pageref{feedback-vs-kritik})
  \item Projekte und Aufgaben als Experimente betrachten und explizit die Erlaubnis geben, dass etwas fehlschlagen kann \index{Experimente}
  \item etablieren, dass es okay und gewünscht ist, ein Projekt abzubrechen, sobald klar ist, dass es nicht sinnvoll oder machbar ist \index{Projekte abbrechen}
  \item etablieren, dass niemand das Recht hat, eine kritische Meinung zu haben, ohne sie auch zu äußern
  \item etablieren, dass es okay ist, Fehler zu machen \index{Fehler}
  \item etablieren, dass es nicht okay ist, Fehler zu machen, ohne sie zu analysieren und daraus zu lernen
  \item Fehlschläge feiern \index{Fehlschläge}
  \item regelmäßige 1-zu-1-Gespräche etablieren (siehe Seite~\pageref{1-zu-1}) \index{1-zu-1-Gespräche}
  \item nach Input fragen und dann gut zuhören \index{Input} \index{zuhören}
  \item Fehler zugeben \index{Fehler}
  \item klare, direkte Kommunikation (siehe Seite~\pageref{direkte-kommunikation}) \index{direkte Kommunikation}
  \item Menschen ermutigen, Sicherheitsrisiken und Bedenken anzusprechen
  \item konstruktiv und mit Wertschätzung und Respekt antworten, wenn jemand etwas anspricht \index{Wertschätzung} \index{Respekt}
  \item klare Verstöße gegen Regeln mit Sanktionen belegen \index{Verstöße} \index{Sanktionen}
  \item etablieren, dass kontinuierliches Nachsteuern völlig okay ist und niemand etwas direkt von Anfang an perfekt hinbekommen muss \index{nachsteuern} \index{Perfektion}
  \item etablieren, dass Scheitern ein unvermeidliches Element von Lernen ist \index{scheitern}
  \item sagen: \glqq Vielleicht habe ich etwas übersehen. Ich brauche da eure Mithilfe.\grqq \index{übersehen}
  \item Leuten dafür danken, dass sie etwas ansprechen \index{danken}
  \item Scheitern-Partys feiern \index{Scheitern-Partys}
  \item Strukturen zum Feedback-Sammeln etablieren \index{Feedback sammeln}
\end{itemize}


\subsection{Kraftvolle Formulierungen}

\begin{itemize}
  \item \glqq Ich weiß es nicht.\grqq
  \item \glqq Ich brauche Hilfe.\grqq
  \item \glqq Ich habe einen Fehler gemacht.\grqq
  \item \glqq Es tut mir Leid.\grqq
  \item \glqq Wie kann ich helfen?\grqq
  \item \glqq Mit welchen Problemen hast du zur Zeit zu kämpfen?\grqq
  \item \glqq Was sind deine Bedenken?\grqq
\end{itemize}


\chapter{Führung}
\section{Die Rolle der Führung}
\label{fuehrung-rolle}
\index{Führung: Rolle}


\subsection{Aufgaben der Führung nach Neuberger}

Laut Neuberger\cite{neuberger-fuehren} sind die Aufgaben der Führung,

\begin{itemize}
  \item andere Menschen
  \item zielgerichtet
  \item in einer formalen Organisation
  \item unter konkreten Umweltbedingungen dazu bewegen,
  \item Aufgaben zu übernehmen und erfolgreich auszuführen,
  \item wobei humane Ansprüche gewahrt werden.
\end{itemize}


\subsection{Neuberger, aber modernisiert}

Auf das moderne Arbeiten übertragen, wäre die Aufgabe der Führung,

\begin{itemize}
  \item eine Umgebung zu schaffen,
  \item die es einem Team oder einer Organisation möglich und leicht macht,
  \item für die Mission des Teams oder der Organisation zu arbeiten,
  \item wobei die Menschen nachhaltig körperlich und seelisch gesund zu bleiben
  \item und ihr Potenzial nutzen können.
\end{itemize}


\subsection{Aufgaben der Führung nach Malik}

Dies sind laut Fredmund Malik \cite{malik-fuehrung} die Aufgaben der Führung:

\begin{itemize}
  \item für Ziele sorgen
  \item organisieren
  \item entscheiden
  \item kontrollieren
  \item Menschen entwickeln und fördern
\end{itemize}

Auf moderne Führung übertragen, wäre es die Aufgabe der Führung, dafür zu sorgen, dass diese Dinge \emph{stattfinden} (also dass beispielsweise das Team Entscheidungen fällen und diese nachhalten kann), und nicht zwangsläufig, dass die Führung das auch selbst entscheidet.


\subsection{Was ergibt sich daraus?}
\index{Beziehungen}

Laut dem Podcast \emph{Manager Tools Basics} \cite{manager-tools-basics} ist eine der wichtigsten Verantwortung der Führung, \fett{gute Beziehungen} zu den geführten Personen \fett{aufzubauen und zu pflegen}.

Laut Amy Edmondson \cite{the-fearless-organisation} ist es die Hauptaufgabe der Führung, im Team bzw.~in der Organisation \fett{psychologische Sicherheit zu schaffen}.
\index{psychologische Sicherheit}

\section{Grundsätze der Führung nach Malik}
\label{fuehrung-grundsätze}
\index{Führung: Grundsätze}


Fredmund Malik \cite{malik-fuehrung} hat in seinen Büchern die Rolle und die Grundsätze von Führung beschrieben.

\begin{itemize}
  \item Ergebnisorientierung
  \item Beitrag zum Ganzen
  \item Konzentration auf weniges
  \item Stärken nutzen
  \item gegenseitiges Vertrauen
  \item positiv denken
\end{itemize}

\section{Führungsstile}
\label{fuehrungsstile}
\index{Führungsstile}


\subsection{Tradierte Führungsstile}
\index{Tradierte Führungsstile}

Diese Führungsstile nach Max Weber~\cite{weber-wirtschaft-gesellschaft} sind nicht mehr aktuell und eher aus historischer Perspektive interessant.

\paragraph{Autokratisch:} uneingeschränkte Macht
\paragraph{Patriarchalisch:} Machtfülle/Vaterrolle
\paragraph{Charismatisch:} Leitfigur/Vorbild
\paragraph{Bürokratisch:} durch Strukturen; die Personen sind austauschbar


\subsection{Klassische Führungsstile}
\index{Klassische Führungsstile}

Diese Führungsstile nach Kurt Lewin~\cite{lewin-fuehrungsstile} sind immer noch aktuell.

\paragraph{Autoritär:} Die Aktivität liegt ausschließlich bei der Führungsperson. Die \glqq Untergebenen\grqq\ haben Weisungen zu akzeptieren und auszuführen. Dieser Stil entspricht dem tradierten autokratischen Stil.
\index{Autoritärer Führungsstil}

\paragraph{Kooperativ/demokratisch:} Die Führungsperson zieht die Mitarbeitenden in die Entscheidungen mit ein. Aus Kontrolle wird zunehmend Selbstkontrolle. Mitarbeitende können die Führung kritisieren
\index{Autoritärer Führungsstil}

\paragraph{Karitativ/partizipativ:} Diese Stil orientiert sich vorrangig an den Bedürfnissen der Mitarbeitenden. Der Mensch steht im Mittelpunkt; die Aufgaben sind dagegen nachrangig. Die Führungsperson hört viel zu, fördert und ermutigt.
\index{Karitativer Führungsstil}
\index{Partizipativ Führungsstil}

\paragraph{Laissez-faire:} Die Führungsperson überträgt/delegiert die Aufgaben, kümmert sich um die Arbeitsmittel und definiert klare Ziele.
\index{Laissez-faire}


\subsection{Situatives Führen}
\index{Situatives Führen}

\emph{Situatives Führen} nach Hersey~\cite{hersey-management} bezeichnet, dass eine Führungsperson je nach Situation unterschiedliche Führungsstile wählt, um erfolgreich zu führen.

\section{Führungsinstrumente (Führungswerkzeuge)}
\label{fuehrungsinstrumente}
\index{Führungswerkzeuge}
\index{Führungsinstrumente}


\subsection{Was ist ein Führungsinstrument?}

Ein Führungsinstrument (oder Führungswerkzeug) generell alles, was eine Führungskraft tun kann, um direkt oder indirekt zusammen mit den geführten Personen Ziele zu erreichen.


\subsubsection{Führungswerkzeuge nach Malik}

Diese Liste von Fredmund Malik \cite{malik-fuehrung} ist schon etwas angestaubt. Sie lässt sich allerdings gut in die heutige Zeit übertragen.

\begin{itemize}
  \item Besprechung
  \item Schriftstück
  \item Stellengestaltung und Einsatzsteuerung
  \item Persönliche Arbeitsmethodik
  \item Budget und Budgetierung
  \item Leistungsbeurteilung
  \item systematische Müllabfuhr
\end{itemize}


\subsection{Direkte Führungsinstrumente}

\begin{itemize}
  \item 1-zu-1-Gespräche (siehe Seite~\pageref{1-zu-1})
  \item Anweisung
  \item Besprechung
  \item delegieren
  \item Entscheidungen treffen
  \item Feedback einholen
  \item Feedback geben
  \item informieren
  \item Konflikte klären
  \item kontrollieren
  \item Kritik
  \item Lob
  \item um etwas bitten
  \item Wertschätzung ausdrücken
  \item Ziele vereinbaren
\end{itemize}


\subsection{Indirekte Führungsinstrumente}

\begin{itemize}
  \item Anreizsysteme schaffen
  \item dem Team Workshops und andere Fortbildungen anbieten
  \item den (physischen) Arbeitsplatz gestalten
  \item die Motivation verbessern
  \item die psychologische Sicherheit verbessern
  \item eine Mission definieren
  \item einen Spieleabend mit dem Team veranstalten
  \item ein Team zusammenstellen
  \item Gewaltfreie Kommunikation lernen und anwenden
  \item mit dem Team einen Escape-Room spielen
  \item mit dem Team lecker essen gehen
  \item Prozesse definieren
  \item Rollen und Verantwortlichkeiten definieren
  \item Supervision für das Team organisieren
\end{itemize}

\section{1-zu-1-Gespräche (\emph{One-on-Ones})}
\label{1-zu-1}
\index{1-zu-1-Gespräche}
\index{One-on-Ones}

Das \emph{1-zu-1-Gespräch} ist ein Führungsinstrument (siehe S.~\pageref{fuehrungsinstrumente}).


\subsection{Ziele}

\begin{itemize}
 \item das Vertrauen und die Beziehung zwischen Teamlead und Teammitglied aufbauen und pflegen
 \item dem Teammitglied und Teamlead die Möglichkeit geben, sich gegenseitig regelmäßig Feedback zu geben
 \item ein Ort für Absprachen sein, damit die Ad-hoc-Absprachen und -Calls zwischendrin weniger werden
 \item Konflikte, Sorgen und Ideen zeitnah besprechen
\end{itemize}


\subsection{Organisation}

\begin{itemize}
 \item Die Gespräche sollten regelmäßig und zuverlässig stattfinden.
 \item Mit Vollzeitangestellten sollten die Gespräche wöchentlich stattfinden. In unserem ehrenamtlichen Sehr-Teilzeit-Team haben wir einen Rhythmus von 4 Wochen.
 \item Die Gespräche haben eine harte Begrenzung auf 30 Minuten, während die einzelnen Slots zeitlich flexibler sind.
 \item Generell sollte das Teammitglied ca.~90\,\% der Redeanteile im Gespräch haben und die Teamführung die restlichen 10\,\%.
\end{itemize}


\subsection{Struktur}

Alle Themen hier sind nur Vorschläge. Wenn euch andere Themen wichtiger sind, dann sprecht über diese.

\subsubsection{Checkin}
Wie fühle ich mich gerade? Wie bin ich hier?


\subsubsection{Slot für das Teammitglied (ca.~10 Minuten)}

\paragraph{Allgemeines}
\begin{itemize}
 \item alles, worüber du sprechen möchtest
\end{itemize}

\paragraph{Strategie}
\begin{itemize}
 \item Was tun wir als Team/Organisation nicht, was wir tun sollten?
 \item Wenn wir eine Sache verbessern könnten, welche wäre das?
 \item Wenn du ich wärst, was würdest du verändern?
\end{itemize}

\paragraph{Blick nach außen}
\begin{itemize}
 \item Was ist zur Zeit das größte Problem unseres Teams/unserer Organisation? Und warum?
 \item Was gefällt dir nicht an unseren Produkten und Dienstleistungen? Was könnten wir verbessern? Woran müssen wir noch arbeiten?
 \item Was ist die größte Chance, die wir gerade verpassen?
\end{itemize}

\paragraph{Team}
\begin{itemize}
 \item Mit wem würdest du gerne (mehr) zusammenarbeiten?
 \item Wer macht gerade einen richtig guten Job?
\end{itemize}

\paragraph{Persönliche Arbeit}
\begin{itemize}
 \item Was macht dir gerade Spaß? Und warum?
 \item Und was macht dir keinen Spaß? Was nervt? Und warum?
 \item Gibt es etwas, vor dem du gerade Angst hast?
 \item Was motiviert dich gerade?
 \item Was demotiviert dich gerade? Was bräuchtest du, um das zu ändern?
\end{itemize}

\paragraph{Feedback und Bitten an die Teamführung}
\begin{itemize}
 \item Was von dem, was ich tue, ist besonders hilfreich für dich?
 \item Wovon würdest du dir bei mir mehr wünschen?
 \item Was von dem, was ich tue, funktioniert für dich nicht gut?
 \item Was bräuchtest du von mir im Moment?
\end{itemize}


\subsubsection{Slot für die Teamführung (ca.~10 Minuten)}
\begin{itemize}
 \item alles, worüber du sprechen möchtest
 \item Feedback an das Teammitglied
\end{itemize}


\subsubsection{Persönliche Entwicklung (ca.~10 Minuten, wenn noch Zeit ist)}
\begin{itemize}
 \item Was hast du kürzlich (dazu-)gelernt?
 \item Was hast du kürzlich ausprobiert? Was ist dabei herausgekommen?
 \item Was würdest du gerne lernen?
 \item Was würdest du gerne mal ausprobieren?
 \item Welche Verantwortungsbereiche würdest du gerne annehmen oder abgeben?
 \item Wie kannst du deine Superkräfte am besten einsetzen?
\end{itemize}

\subsection{Quellen}

\begin{itemize}
 \item Podcast: Manager Tools Basics: One on Ones \cite[04.\,07.\,2005 bis 11.\,07.\,2005]{manager-tools-basics}
 \item Podcast: Female Leadership: Gamechanger für Gesprächsführung \cite[Folge 15]{female-leadership-gespraechsfuehrung}
 \item Buch: The Hard Thing About Hard Things: Building a Business When There Are No Easy Answers \cite{the-hard-thing-about-hard-things}
\end{itemize}

\section{Wertschätzung ausdrücken}
\label{wertschaetzung}
\index{Wertschätzung}

Diese habe ich aus dem Buch \emph{GfK für Dummies} \cite[S.~206]{gfk-dummies} und mit einigen Dingen aus dem Podcast \emph{Familie verstehen} \cite{familie-verstehen-podcast} ergänzt, plus Ergänzungen aus dem Podcast \emph{Manager Tools Basics} \cite{manager-tools-basics}.

\begin{enumerate}
  \item Was hat die andere Person \fett{gesagt oder getan}?
  \item Welche \fett{Gefühle} hat dies bei mir ausgelöst? (optional, aber sehr hilfreich)
  \item Welche \fett{Bedürfnisse} von mir oder vom Team hat das erfüllt?
  \item \fett{Danke} dafür! \emph{oder:} Das \fett{feiere} ich!
  \item Gerne \fett{öfter/wieder tun}! (Bitte, optional)
\end{enumerate}

\section{Führen lernen}
\label{fuehren-lernen}
\label{fuehrung-lernen}
\index{Führung lernen}

Meiner Ansicht nach ist Führen zu lernen so ähnlich wie singen zu lernen.

Dafür sind diese Dinge notwendig:

\begin{itemize}
  \item viel \fett{üben} (und dabei aus Fehlern lernen)
  \item sehr viel \fett{Reflexion} \index{Reflexion}
  \item \fett{Außenwahrnehmung} bekommen in der Form von Feedback \index{Außenwahrnehmung} \index{Feedback}
  \item an \fett{Trainings} und \fett{Workshops} teilnehmen (oder anderweitig Unterricht nehmen)
  \item \fett{Bücher} oder anderen Quellen von Wissen konsumieren
  \item von \fett{guten Beispielen} lernen
\end{itemize}

Damit ihr andere Menschen gut führen könnt, ist es außerdem notwendig, dass ihr euch selbst gut kennt und versteht, wie ihr tickt und was euch antreibt. Dies könnt ihr durch diese Dinge (oder eine Kombination daraus) erreichen:

\begin{itemize}
  \item Gewaltfreie Kommunikation lernen \index{Gewaltfreie Kommunikation}
  \item eine Psychotherapie machen (Tiefenpsychologie oder Psychoanalyse; keine kognitive Verhaltenstherapie) \index{Therapie} \index{Psychotherapie}
\end{itemize}

Hilfreich zum kontinuierlichen Lernen ist außerdem eine Supervision, Intervision oder kollegiale Fallberatung.


\backmatter

\bibliography{../shared/bibliography/literatur}

\chapter{Lizenz}

\section*{Unter welchen Bedingungen könnt ihr dieses Handout benutzen?}
Dieses Handout ist unter einer \emph{Creative-Commons}-Lizenz lizensiert. Dies ist die \emph{Namensnennung-Share Alike 4.0 international (CC BY-SA 4.0)}\footnote{Die ausführliche Version dieser Lizenz findet ihr unter \url{https://creativecommons.org/licenses/by-sa/4.0/deed.de}.}. Das bedeutet, dass ihr dieses Handout unter diesen Bedingungen für euch kostenlos verbreiten, bearbeiten und nutzen könnt (auch kommerziell):

\begin{description}
  \item[Namensnennung.] Ihr müsst den Namen des Autors (Oliver Klee) nennen. Wenn ihr außerdem auch noch die Quelle\footnote{\url{https://github.com/oliverklee/workshop-handouts}} nennt, wäre das nett. Und wenn ihr mir zusätzlich eine Freude machen möchtet, sagt mir per E-Mail Bescheid.
  \item[Weitergabe unter gleichen Bedingungen.] Wenn ihr diesen Inhalt bearbeitet oder in anderer Weise umgestaltet, verändert oder als Grundlage für einen anderen Inhalt verwendet, dann dürft ihr den neu entstandenen Inhalt nur unter Verwendung identischer Lizenzbedingungen weitergeben.
  \item[Lizenz nennen.] Wenn ihr den Reader weiter verbreitet, müsst ihr dabei auch die Lizenzbedingungen nennen oder beifügen.
\end{description}


\printindex

\end{document}
