\section{Was ist psychologische Sicherheit?}
\index{psychologische Sicherheit}

\subsection{Definition}
\label{ps-definition}

Amy C.~Edmondson \cite{the-fearless-organisation} definiert psychologische Sicherheit so:

\selectlanguage{english}
\begin{quote}
  I have defined psychological safety as the belief that the work environment is safe for interpersonal risk taking. The concept refers to the experience of feeling able to speak up with relevant ideas, question, or concerns.
\end{quote}
\selectlanguage{ngerman}

Karolin Helbig und Minette Norman \cite{psychological-safety-playbook} fassen das noch einmal etwas prägnanter zusammen:

\selectlanguage{english}
\begin{quote}
  Psychological safety is a belief that one will not be punished or humiliated for speaking up with ideas, questions, concerns, or mistakes.
\end{quote}
\selectlanguage{ngerman}


\subsection{Unbewusste Risikoabschätzungen}
\index{Risikoabschätzungen}

Psychologische Sicherheit hat dadurch einen großen Einfluss auf kurzfristige \emph{unbewusste Risikoabschätzungen}, die wir in Gruppen jeden Tag treffen:

\begin{quote}
  \glqq Wenn ich jetzt etwas sage oder frage, wie groß ist das Risiko, dass ich dann bestraft werde oder eine für mich unangenehme Reaktion ernte?\grqq
\end{quote}

Dabei fällt die individuelle Abschätzung tendenziell so aus, dass man die für einen selbst unangenehmen Folgen vermeiden will, anstatt potenziell langfristige negative Folgen für das Team oder die Organisation zu vermeiden.


\subsection{Was psychologische Sicherheit \emph{nicht} ist}

Psychologische Sicherheit ist keins von diesen Dingen:

\paragraph{Einfach nur nett sein:} Bei psychologischer Sicherheit geht es (auch) um Aufrichtigkeit. Dies ermöglicht erst den offenen, konstruktiven Austausch auch bei unterschiedlichen Meinungen oder Konflikten.

\paragraph{Ein Persönlichkeitsmerkmal:} Psychologische Sicherheit hängt mit der Arbeitsatmosphäre in einem Team zusammen. Dadurch kann dieselbe Person in unterschiedlichen Teams unterschiedlich viel psychologische Sicherheit erfahren.

\paragraph{Vertrauen:} Psychologische Sicherheit beschreibt die Arbeitsatmosphäre in einem gesamten Team, während Vertrauen beschreibt, wie sicher sich eine Person mit einer anderen Person (oder Institution) fühlt. Mehr dazu ist ab Seite~\pageref{vertrauen-vs-ps} zu finden.

\paragraph{Geringe Leistungsstandards:} Geringe/hohe psychologische Sicherheit und geringe/hohe Leistungsstandards können in jeder Kombination auftreten. Eine hohe psychologische Sicherheit hat allerdings einen sehr großen Einfluss darauf, wie hoch die Leistung eines Teams tatsächlich ist. Mehr dazu ist auf Seite~\pageref{ps-leistung} zu finden.


\subsection{Psychologische Sicherheit messen}
\label{ps-messen}
\index{Messen von psychologischer Sicherheit}

Mit diesem von Amy C.~Edmondson für ihre Dissertation entwickelten Fragebogen kann man die psychologische Sicherheit in einem Team messen:

\begin{enumerate}
  \item Wenn ich in meinem Team einen Fehler mache, wird mir das oft vorgehalten. \emph{(R)}
  \item In meinem Team ist es möglich, Probleme und schwierige Themen anzusprechen.
  \item Mitglieder meines Teams lehnen andere manchmal ab, weil sie anders sind. \emph{(R)}
  \item In meinem Team ist es sicher, Risiken einzugehen.
  \item Es ist schwierig, andere Mitglieder meines Teams um Hilfe zu bitten. \emph{(R)}
  \item Niemand in meinem Team würde absichtlich etwas tun, das meine Leistung untergräbt.
  \item Bei der Arbeit mit meinen Teammitgliedern werden meine einzigartigen Fähigkeiten und Talente geschätzt und eingesetzt.
\end{enumerate}

Die mit \emph{(R)} markierten Fragen sind umgekehrt, d.\,h., sie weisen auf eine niedrige psychologische Sicherheit hin.
