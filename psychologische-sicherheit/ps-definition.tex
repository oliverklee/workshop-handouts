\section{Was ist psychologische Sicherheit?}
\label{ps-definition}
\index{psychologische Sicherheit}

Amy C.~Edmondson \cite{the-fearless-organisation} definiert psychologische Sicherheit so:

\selectlanguage{english}
\begin{quote}
  I have defined psychological safety as the belief that the work environment is safe for interpersonal risk taking. The concept refers to the experience of feeling able to speak up with relevant ideas, question, or concerns.
\end{quote}
\selectlanguage{ngerman}

Karolin Helbig und Minette Norman \cite{psychological-safety-playbook} fassen das noch einmal etwas prägnanter zusammen:

\selectlanguage{english}
\begin{quote}
  Psychological safety is a belief that one will not be punished or humiliated for speaking up with ideas, questions, concerns, or mistakes.
\end{quote}
\selectlanguage{ngerman}


\section{Psychologische Sicherheit und Leistung}
\label{ps-leistung}
\index{Leistung}

Die psychologische Sicherheit in einem Team hat einen extrem großen Einfluss auf die Leistung bzw.~Leistungsfähigkeit eines Teams \cite{high-performing-teams}.

Dabei sind psychologische Sicherheit und Leistungsstandards in einem Team unabhängig voneinander. Nur bei hoher psychologischer Sicherheit und hohen Leistungsstandards kann ein Team zu Bestleistungen aufblühen:~\cite{the-fearless-organisation}

\vspace{1em}

\renewcommand{\arraystretch}{2.0}
\begin{tabular}{|p{17em}|p{10em}|p{11em}|}
\hline
& \fett{Niedrige Standards} & \fett{Hohe Standards}
\\ \hline

\fett{Niedrige psychologische Sicherheit}
& Apathie-Zone
& Angst-Zone
\\ \hline

\fett{Hohe psychologische Sicherheit}
& Komfort-Zone
& Zone des Lernens und der Bestleistung
\\ \hline

\end{tabular}
\renewcommand{\arraystretch}{1.0}
