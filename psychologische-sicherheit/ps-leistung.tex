\section{Psychologische Sicherheit und Leistung}
\label{ps-leistung}
\index{Leistung}

Die psychologische Sicherheit in einem Team hat einen extrem großen Einfluss auf die Leistung bzw.~Leistungsfähigkeit eines Teams \cite{high-performing-teams}.


\subsection{Psychologische Sicherheit vs.~Leistungsstandards}
\index{Leistungsstandards}

Psychologische Sicherheit und Leistungsstandards in einem Team sind unabhängig voneinander. Nur bei hoher psychologischer Sicherheit und hohen Leistungsstandards kann ein Team zu Bestleistungen aufblühen:~\cite{the-fearless-organisation}

\vspace{1em}

\renewcommand{\arraystretch}{2.0}
\begin{tabular}{|p{17em}|p{10em}|p{11em}|}
\hline
& \fett{Niedrige Standards} & \fett{Hohe Standards}
\\ \hline

\fett{Niedrige psychologische Sicherheit}
& Apathie-Zone
& Angst-Zone
\\ \hline

\fett{Hohe psychologische Sicherheit}
& Komfort-Zone
& Zone des Lernens und der Bestleistung
\\ \hline

\end{tabular}
\renewcommand{\arraystretch}{1.0}


\subsection{Die Aufgabe von Führung}

Führung hat (in Bezug auf Leistung) zwei Hauptaufgaben:

\begin{enumerate}
  \item \fett{psychologische Sicherheit} aufbauen, um Lernen zu fördern und vermeidbare Fehler zu vermeiden
  \item hohe \fett{Leistungsstandards} setzen und das Team dazu inspirieren und dabei unterstützen, diese zu erreichen
\end{enumerate}
