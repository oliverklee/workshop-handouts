\section{Die Entdeckung der psychologischen Sicherheit}
\label{ps-entdeckung}
\index{Entdeckung der psychologischen Sicherheit}

Die Forscherin Amy C.~Edmonson \cite{the-fearless-organisation} war an einem Forschungsprojekt beteiligt, in dem sie die Folgen von Teamarbeit auf Fehler in der Arbeit in Krankenhäusern untersuchte. Dabei erhob sie die Fehlerquote von Teams (Anzahl der gemeldeten Fehler pro 1000 Patient\_innen-Tage). Weiterhin befragte sie die Teams bezüglich gegenseitigem Respekt, der Qualität der Zusammenarbeit, deren Zufriedenheit und anderer Punkte.

Dabei kam heraus, dass die gemeldete Fehlerquote und die Qualität der Zusammenarbeit im Team \emph{umgekehrt} miteinander korreliert waren~-- also die Teams mit besserer Zusammenarbeit mehr Fehler gemeldet hatten.

Bei weiteren stillen Beobachtungen kam dann heraus, dass sich die Teams tatsächlich sehr darin unterschieden, ob sie offen über Fehler gesprochen haben.

Dafür wurde später dann der Begriff \emph{psychologische Sicherheit} geprägt.

Die psychologische Sicherheit war bei der Untersuchung auch zwischen unterschiedlichen Teams innerhalb desselben Krankenhauses deutlich unterschiedlich. Das hängt damit zusammen, dass diese sehr von den Führungskräften des einzelnen Teams abhängt.
