\section{Einige bekannte Fälle, in denen mangelnde psychologische Sicherheit eine Rolle gespielt hat}
\label{ps-faelle}


\subsection{Der Abgasskandal bei Volkswagen und anderen Autoherstellern}
\index{Abgasskandal}
\index{Dieselgate}
\index{Volkswagen}

Beim Abgasskandal (\glqq Dieselgate\grqq) hatten mehrere Autohersteller illegale Abschalteinrichtung in der Motorsteuerung ihrer Diesel-Fahrzeuge verwendet, um die von den Vorgesetzten vorgegebenen strengeren Abgaswerte auf dem Prüfstand einzuhalten (aber nicht im realen Betrieb auf der Straße).

Dies hat (neben dem Vertrauensverlust) zum Rückruf von mehreren Millionen Fahrzeugen und Strafen in Millionenhöhe geführt.


\subsection{Nokia}
\index{Nokia}

Nach dem großen Erfolg mit einfachen Handys setze der finnische Handyhersteller Nokia sehr auf Feature-Phones. Durch eine Kultur der Angst in dem Unternehmen wagten es die Angestellten nicht, ihr Bedenken darüber zu äußern, dass Feature-Phones gegenüber Smartphones wenig Chancen haben würden.

Wegen des geringen wirtschaftlichen Erfolgs dieser Strategie folgten Entlassungswellen und Teil-Verkäufe der Firma. Nokia wäre daran beinahe zugrunde gegangen. Nach 2020 konnte Nokia durch ein Rebranding und eine radikale Neuausrichtung gerettet werden.


\subsection{Wells Fargo}
\index{Wells Fargo}

Beim US-amerikanische Finanzdienstleister Wells Fargo bekamen die Angestellten in den frühen 2000ern unrealistische Vorgaben, wie viele Produkte sie verkaufen mussten. Dabei übte die Bank sehr großen Druck auf das Personal aus.

Dies führte dazu, dass die Angestellten den Kund\_innen ohne deren Einverständnis Produkte verkauften oder sie belogen.

Dies führte dazu, dass Wells Fargo insgesamt mehr als 7~Milliarden Dollar Strafe zahlen musste.


\subsection{Columbia-Katastrophe}
\index{Columbia-Katastrophe}

2003 verglühte das Space-Shuttle \emph{Columbia} beim Wiedereintritt in die Erdatmosphäre. Dabei kamen alle Besatzungsmitglieder ums Leben. Ursache war ein technischer Defekt an der äußeren Schaumstoffisolierung.

Ein Ingenieur hatte vorher das Problem festgestellt und wollte es an das zuständige Verteidigungsministerium melden. Sein Vorgesetzter teilte ihm mit, dass \glqq die Ingenieure keine Anfragen an jemanden oberhalb ihrer Hierarchieebene schicken durften\grqq, und leitete die Anfrage nicht weiter.


\subsection{Zusammenstoß zweier Boeing 747}
\index{Boeing 747}
\index{Flugzeugzusammenstoß}

1977 stießen zwei Boeing 747 auf einer Landebahn zusammen. 583 Menschen an Bord starben.

Der Pilot hatte dabei das Flugzeug beschleunigt, obwohl die Freigabe der Flugsicherung noch nicht vorlag. Der Erst Offizier wies den Piloten darauf hin. Der Pilot unterbrach ihn und startete das Flugzeug.

