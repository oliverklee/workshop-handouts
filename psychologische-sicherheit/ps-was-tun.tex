\section{Was tun für mehr psychologische Sicherheit?}
\label{ws-was-tun}

Generell ist psychologische Sicherheit ein kontinuierlicher Prozess~-- ihr seid damit niemals \glqq fertig\grqq.

Dies ist eine ungeordnete Liste von Ideen aus \emph{The Fearless Organization} \cite{the-fearless-organisation}.

\begin{itemize}
  \item die Strategie der Organisation als Hypothese behandeln anstelle als Plan \index{Strategie} \index{Hypothesen}
  \item es zur Routine machen, explizit nach möglichen Problemen zu fragen \index{Probleme} \index{Routinen}
  \item fragen: \glqq War alles so sicher, wie du es gerne hättest?\grqq \index{Sicherheit}
  \item demokratisch/kooperativ führen anstelle von autoritär (zu Führungsstilen siehe Seite~\pageref{fuehrungsstile}) \index{Führungsstile}
  \item eine Kultur der radikalen Ehrlichkeit \cite{radical-honesty} schaffen \index{radikale Ehrlichkeit}
  \item eine Kultur des konstruktiven Feedbacks etablieren (Seite \pageref{feedback-regeln}) und zwischen Feedback, Bitten, Forderungen und Kritik sauber trennen \index{Feedback} (Seite \pageref{feedback-vs-kritik})
  \item Projekte und Aufgaben als Experimente betrachten und explizit die Erlaubnis geben, dass etwas fehlschlagen kann \index{Experimente}
  \item etablieren, dass es okay und gewünscht ist, ein Projekt abzubrechen, sobald klar ist, dass es nicht sinnvoll oder machbar ist \index{Projekte abbrechen}
  \item etablieren, dass niemand das Recht hat, eine kritische Meinung zu haben, ohne sie auch zu äußern
  \item etablieren, dass es okay ist, Fehler zu machen \index{Fehler}
  \item etablieren, dass es nicht okay ist, Fehler zu machen, ohne sie zu analysieren und daraus zu lernen
  \item Fehlschläge feiern \index{Fehlschläge}
  \item regelmäßige 1-zu-1-Gespräche etablieren (siehe Seite~\pageref{1-zu-1}) \index{1-zu-1-Gespräche}
  \item nach Input fragen und dann gut zuhören \index{Input} \index{zuhören}
  \item Fehler zugeben \index{Fehler}
  \item klare, direkte Kommunikation (siehe Seite~\pageref{direkte-kommunikation}) \index{direkte Kommunikation}
  \item Menschen ermutigen, Sicherheitsrisiken und Bedenken anzusprechen
  \item konstruktiv und mit Wertschätzung und Respekt antworten, wenn jemand etwas anspricht \index{Wertschätzung} \index{Respekt}
  \item klare Verstöße gegen Regeln mit Sanktionen belegen \index{Verstöße} \index{Sanktionen}
  \item etablieren, dass kontinuierliches Nachsteuern völlig okay ist und niemand etwas direkt von Anfang an perfekt hinbekommen muss \index{nachsteuern} \index{Perfektion}
  \item etablieren, dass Scheitern ein unvermeidliches Element von Lernen ist \index{scheitern}
  \item sagen: \glqq Vielleicht habe ich etwas übersehen. Ich brauche da eure Mithilfe.\grqq \index{übersehen}
  \item Leuten dafür danken, dass sie etwas ansprechen \index{danken}
  \item Scheitern-Partys feiern \index{Scheitern-Partys}
  \item Strukturen zum Feedback-Sammeln etablieren \index{Feedback sammeln}
\end{itemize}


\subsection{Kraftvolle Formulierungen}

\begin{itemize}
  \item \glqq Ich weiß es nicht.\grqq
  \item \glqq Ich brauche Hilfe.\grqq
  \item \glqq Ich habe einen Fehler gemacht.\grqq
  \item \glqq Es tut mir Leid.\grqq
  \item \glqq Wie kann ich helfen?\grqq
  \item \glqq Mit welchen Problemen hast du zur Zeit zu kämpfen?\grqq
  \item \glqq Was sind deine Bedenken?\grqq
\end{itemize}
