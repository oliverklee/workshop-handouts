\section{Was tun für mehr psychologische Sicherheit?}
\label{ws-was-tun}

Generell ist psychologische Sicherheit ein kontinuierlicher Prozess~-- ihr seid damit niemals \glqq fertig\grqq.

\subsection{Ideen aus \emph{The Fearless Organization}}

Dies ist eine ungeordnete Liste von Ideen aus \emph{The Fearless Organization} \cite{the-fearless-organisation}.

\begin{itemize}
  \item die Strategie der Organisation als Hypothese behandeln anstelle als Plan \index{Strategie} \index{Hypothesen}
  \item es zur Routine machen, explizit nach möglichen Problemen zu fragen \index{Probleme} \index{Routinen}
  \item fragen: \glqq War alles so sicher, wie du es gerne hättest?\grqq \index{Sicherheit}
  \item demokratisch/kooperativ führen anstelle von autoritär (zu Führungsstilen siehe Seite~\pageref{fuehrungsstile}) \index{Führungsstile}
  \item eine Kultur der radikalen Ehrlichkeit \cite{radical-honesty} schaffen \index{radikale Ehrlichkeit}
  \item eine Kultur des konstruktiven Feedbacks etablieren (Seite \pageref{feedback-regeln}) und zwischen Feedback, Bitten, Forderungen und Kritik sauber trennen \index{Feedback} (Seite \pageref{feedback-vs-kritik})
  \item Projekte und Aufgaben als Experimente betrachten und explizit die Erlaubnis geben, dass etwas fehlschlagen kann \index{Experimente}
  \item etablieren, dass es okay und gewünscht ist, ein Projekt abzubrechen, sobald klar ist, dass es nicht sinnvoll oder machbar ist \index{Projekte abbrechen}
  \item etablieren, dass niemand das Recht hat, eine kritische Meinung zu haben, ohne sie auch zu äußern
  \item etablieren, dass es okay ist, Fehler zu machen \index{Fehler}
  \item etablieren, dass es nicht okay ist, Fehler zu machen, ohne sie zu analysieren und daraus zu lernen
  \item Fehlschläge feiern \index{Fehlschläge}
  \item regelmäßige 1-zu-1-Gespräche etablieren (siehe Seite~\pageref{1-zu-1}) \index{1-zu-1-Gespräche}
  \item nach Input fragen und dann gut zuhören \index{Input} \index{zuhören}
  \item Fehler zugeben \index{Fehler}
  \item klare, direkte Kommunikation (siehe Seite~\pageref{direkte-kommunikation}) \index{direkte Kommunikation}
  \item Menschen ermutigen, Sicherheitsrisiken und Bedenken anzusprechen
  \item konstruktiv und mit Wertschätzung und Respekt antworten, wenn jemand etwas anspricht \index{Wertschätzung} \index{Respekt}
  \item klare Verstöße gegen Regeln mit Sanktionen belegen \index{Verstöße} \index{Sanktionen}
  \item etablieren, dass kontinuierliches Nachsteuern völlig okay ist und niemand etwas direkt von Anfang an perfekt hinbekommen muss \index{nachsteuern} \index{Perfektion}
  \item etablieren, dass Scheitern ein unvermeidliches Element von Lernen ist \index{scheitern}
  \item sagen: \glqq Vielleicht habe ich etwas übersehen. Ich brauche da eure Mithilfe.\grqq \index{übersehen}
  \item Leuten dafür danken, dass sie etwas ansprechen \index{danken}
  \item Scheitern-Partys feiern \index{Scheitern-Partys}
  \item Strukturen zum Feedback-Sammeln etablieren \index{Feedback sammeln}
\end{itemize}


\subsubsection{Kraftvolle Formulierungen}

\begin{itemize}
  \item \glqq Ich weiß es nicht.\grqq
  \item \glqq Ich brauche Hilfe.\grqq
  \item \glqq Ich habe einen Fehler gemacht.\grqq \index{Fehler}
  \item \glqq Es tut mir Leid.\grqq
  \item \glqq Wie kann ich helfen?\grqq
  \item \glqq Mit welchen Problemen hast du zur Zeit zu kämpfen?\grqq
  \item \glqq Was sind deine Bedenken?\grqq
\end{itemize}


\subsection{Ideen aus \emph{The Psychological Safety Playbook}}

Diese Vorschläge stammen aus \emph{The Psychological Safety Playbook} \cite{psychological-safety-playbook}.


\subsubsection{Mutig kommunizieren}
\index{mutig kommunizieren}

\begin{itemize}
  \item \glqq Was übersehe ich gerade?\grqq
  \item \glqq Das ist eine mögliche Sichtweise. Welche anderen Sichtweisen habt ihr noch?\grqq
  \item die eigenen Gefühle wahrnehmen und kommunizieren \index{Gefühle}
  \item \glqq Ich weiß es nicht.\grqq
  \item mehr lachen (vor allem über dich selbst) \index{lachen}
\end{itemize}



\subsubsection{Zuhören}
\index{zuhören}

\begin{itemize}
  \item zuhören mit dem Ziel, zu verstehen
  \item beim Zuhören vollständig auf die andere Person fokussieren
  \item aktiv zuhören: aussprechen, was du verstanden hast \index{aktiv zuhören}
  \item Emotionen und Bedürfnisse hören und Spekulationen dazu aussprechen \index{Emotionen}
  \item neugierig sein: \glqq Erzähle mir mehr!\grqq \index{Neugier}
\end{itemize}


\subsubsection{Deine Reaktionen im Griff haben}
\index{Reaktionen}

\begin{itemize}
  \item wenn du dich angegriffen fühlst: kurz Pause machen, tief durchatmen und eine konstruktive Reaktion wählen
  \item dein aktives Vokabular für deine Gefühle erweitern
  \item erkenne deine eigenen Annahmen darüber, wie Dinge sind und was andere Menschen denken \index{Annahmen}
  \item \glqq Danke, dass du das ansprichst.\grqq \index{Dank}
  \item \glqq Ja, und~\ldots\grqq{} oder \glqq Gleichzeitig~\ldots\grqq statt \glqq Ja, aber~\ldots\grqq \index{gleichzeitig}
\end{itemize}


\subsubsection{Risiko und Scheitern feiern}
\index{Risiko}
\index{Scheitern}

\begin{itemize}
  \item \glqq Das ist neu für uns. Deswegen werden wir hier und da scheitern.\grqq
  \item Fehlschläge reframen: \glqq Interessant! Was können wir daraus lernen?\grqq \index{Fehlschläge} \index{Reframing}
  \item unangenehme, schwierige Gefühle zulassen und akzeptieren \index{Gefühle}
  \item ein Lern-Mindset vorleben: Fehler eingestehen und daraus Gelerntes teilen \index{Lern-Mindset}
  \item kontinuierliches Lernen feiern: Post-Mortems ohne Schuldzuweisungen etablieren \index{Post-Mortems} \index{Schuldzuweisungen}
\end{itemize}


\subsubsection{Inklusive Rituale schaffen}

\begin{itemize}
  \item \glqq Inklusions-Helfer\_in\grqq-Rolle in Meetings etablieren, die/der darauf achtet, ob alle beitragen \index{Inklusions-Helfer\_in}
  \item \glqq ausreden-lassen\grqq-Regel einführen \index{ausreden lassen}
  \item Regel: \glqq Niemand redet ein zweites Mal, solange nicht alle einmal geredet haben.\grqq
  \item nach Meetings Feedback zur psychologischen Sicherheit einholen
  \item Dankbarkeit und Wertschätzung explizit ausdrücken (mehr dazu auf Seite~\pageref{wertschaetzung}) \index{Dankbarkeit} \index{Wertschätzung}
\end{itemize}


\subsection{Psychologische Sicherheit bei Remote-Arbeit fördern}
\label{ps-remote}
\index{Remote-Arbeit}

In den Artikeln \cite{remote-psychological-safety} und \cite{hybrid-psychological-safety} haben die Autor\_innen untersucht, welche Auswirkungen Remote- und Hybridarbeit auf die psychologische Sicherheit in Teams hatte, und welche Maßnahmen den Teams dabei geholfen haben.

\paragraph{Pull-Requests/Merge-Requests} für Feedback zu Änderungen am Code

\paragraph{Commitment-Montage und Feier-Freitage:} ein Teammeeting am Montag, bei dem sich das Team auf Ziele committet, und ein Meeting am Freitag, an dem das Team das Erreichte feiert

\paragraph{Postmortems und Retrospektiven} um aus Fehlschlägen zu lernen mit dem Mindset, dass Fehlschläge erlaubt sind

\paragraph{20\,\% der Arbeitszeit für persönliche Projekte reservieren} für Lernen, Weiterentwicklung und Experimente

\paragraph{Demos mit Kund\_innen} für Wertschätzung und damit Teamleads danke sagen können

\paragraph{Ansprechen, dass remote Dinge anders sind} und dass es für euch als Gruppe eine gemeinsame Aufgabe ist, herauszufinden, wie ihr auch remote gut zusammenarbeiten könnt.

\paragraph{Verletzlich sein und kommunizieren} darüber, was ihr als Führungskraft nicht wisst und was auch für euch herausfordernd ist.

\paragraph{Kleine Schritte gehen} beim Teilen von Persönlichem, und dann als Team immer weiter in Richtung von mehr Verletzlichkeit gehen

\paragraph{Gute Beispiele miteinander teilen} dazu, was für euch bei der Remote-Arbeit hilfreich ist

\paragraph{Verletzlichkeit schützen} und einschreiten, wenn Leute es missbrauchen, wenn andere sich verletzlich geben
